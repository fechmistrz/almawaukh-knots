\section{Węzły plastrowe} % (fold)
\label{sec:slice}
% Na podstawie podręcznika Murasugiego
Rozpatrzmy węzeł $K$ w~sferze $S^3$ jako brzegu czterowymiarowej kuli $B^4$.
Punkt wewnętrzny $P$ dysku $D$ w~tej kuli nazywamy osobliwym, jeśli nie można wskazać dla niego żadnego otoczenia $U$, homeomorficznego z~$B^4$, takiego że $\partial U \cap D$ jest węzłem trywialnym w~$\partial U$, 3-sferze.
Jeśli $K$ jest brzegiem dysku $D$ pozbawionego punktów osobliwych, to jest on węzłem plastrowym.

W tej sekcji opiszemy pewną czterowymiarową własność węzłów.
Węzły plastrowe na sferze $S^3$ to takie, które są brzegiem ładnie zanurzonego dysku $D$ w~4-kuli.
Co dokładnie znaczy wyrażenie ,,ładnie zanurzony'' zależy od kontekstu: istnieją węzły gładko albo też topologicznie plastrowe.
Precyzuje to pochodząca najwyraźniej od Foxa (1962) definicja:

\begin{definition}
    \index{węzeł!plastrowy}
    Węzeł $K \subseteq S^3$ nazywamy plastrowym, jeśli istnieje płaski dysk $D$ zawarty w~$B^4$ taki, że $K = \partial D = D \cap S^3$.
    Płaski, czyli: dysk $D$ posiada otoczenie $N$ będące kopią zbioru $D \times I^2$, która spotyka sferę $S^3$ dokładnie w~$\partial D \times I^2$.
\end{definition}

Następujące węzły o~mniej niż jedenastu skrzyżowaniach są plastrowe: $6_1$, $8_{20}$, $8_{8}$, $8_{9}$, $9_{27}$, $9_{41}$, $9_{46}$, $10_{22}$, $10_{35}$, $10_{3}$, $10_{42}$, $10_{48}$, $10_{75}$, $10_{87}$, $10_{99}$, $10_{123}$, $10_{129}$, $10_{137}$, $10_{140}$, $10_{153}$, $10_{155}$, $10_{155}$.
Lista ta powstała na podstawie portalu KnotAtlas.

\begin{proposition} \label{slice_square_det}
    Wyznacznik węzła plastrowego jest kwadratem (Rolfsen 1976, strona 224).
\end{proposition}

Węzły plastrowe (slice), skręcone (twist).

\begin{definition} \label{twist_knots}
    \index{węzeł!skręcony}
    Węzeł powstały przez $n$-krotne półskręcanie domkniętej pętli oraz splecienie końców nazywamy węzłem skręconym.
\end{definition}

Węzły skręcone to dokładnie towarzyszące niewęzłowi w~węzłach satelitarnych, tak zwane whiteheadowskie duble niewęzła.
Wszystkie są odwracalne (ale tylko niewęzeł oraz ósemka są amfichiralne) i~mają liczbę gordyjską $1$, ponieważ wystarczy rozwiązać skrzyżowanie, które plotło końce.
Każdy jest $2$-mostowy i~posiada zerową sygnaturę.
Dalsze własności węzłów skręconych zależą od $n$, ilości półskrętów.
Indeks skrzyżowaniowy wynosi $n + 2$.

\begin{proposition}
    Wielomianowymi niezmiennikami węzłów skręconych są:
    \begin{align*}
    (q+1)V(q) & = \begin{cases}
        1+q^{-2}+q^{-n}-q^{-n-3} & n \mbox{ nieparzyste} \\
        q^{3}+q-q^{3-n}+q^{-n} & n \mbox{ parzyste}
    \end{cases} \\
    2 \nabla (z) & = \begin{cases}
        (n+1) z^{2} + 2 & n \mbox{ nieparzyste} \\
        2 - nz^2 & n \mbox{ parzyste}
    \end{cases}
    \end{align*}
\end{proposition}

%\begin{definition}
    %Węzeł $K$ w~$S^3 = \partial D^4$ jest plastrowy, jeśli ogranicza dysk $\alexander^2$ w~$D^4$, który ma rurowe otoczenie $\alexander^2 \times D^2$, którego przekrojem z~$S^3$ jest rurowe otoczenie $K \times D^2$ dla $K$. \index{Węzeł!plastrowy}
%\end{definition}

Istnieje konkurencyjna definicja węzłów plastrowych.
Dwa sploty $K, L \subseteq S^n$ nazywamy zgodnymi (\emph{concordant}), jeśli istnieje włożenie $f \colon K \times [0,1] \to S^n \times [0,1]$ spełniające dwa warunki: $f(K \times 0) = K \times 0$ oraz $f(K \times 1) = L \times 1$.

\begin{definition} \label{def:slice_knot}
    Węzeł zgodny z~niewęzłem nazywamy plastrowym.
\end{definition}

Zgodność jest relacją równoważności, słabszą od izotopii, ale mocniejszą od homotopii.
W zbiorze jej klas abstrakcji, oznaczanym przez $C^1$, można zadać strukturę grupy abelowej izomorficznej z~$\Z^\infty \oplus (\Z/2)^\infty \oplus (\Z/4)^\infty$.
\index{grupa!zgodności}
Działanie dane jest wzorem $[K] + [L] = [K \shrap L]$; niewęzeł stanowi element neutralny.
Elementem odwrotnym do $[K]$ jest $[-K^*]$.

\begin{proposition}
    Albo wszystkie trzy węzły $K, L, K \shrap L$ są plastrowe, albo co najwyżej jeden z~nich.
\end{proposition}

\index{warunek Foxa-Milnora}
Poniższa własność znana jest w~literaturze jako warunek Foxa-Milnora.

\begin{proposition}
    Wielomian Alexandera węzła plastrowego $K$ jest postaci $f(t) f(1/t)$, gdzie $f(t)$ jest pewnym wielomianem Laurenta nad pierścieniem $\Z$.
\end{proposition}

\begin{theorem}[Casson, Gordon, 1975]
    Niewęzeł oraz węzeł dokerski ($6_1$) są jedynymi skręconymi węzłami plastrowymi.
\end{theorem}

\begin{proof}
    Casson, Gordon: ,,Cobordism of classical knots'', Orsay, 1976.
\end{proof}

\begin{definition}
    \index{węzeł!taśmowy}
    Węzeł $K = f(S^1)$ będący brzegiem singularnego dysku $f \colon D \to S^3$ posiadającego następującą własność: każda przecinająca siebie składowa jest łukiem $A \subseteq f(D^2)$, dla którego $f^{-1}(A)$ składa się z~dwóch łuków w~$D^2$ (jeden z~nich jest wewnętrzny), nazywamy taśmowym.
\end{definition}

Suma spójna dowolnego węzła $K$ oraz jego lustra $mK$ jest taśmowa.
Dowód plastrowatości można znaleźć w~Kawauchi, A Survey of knot theory, strona 155.

\begin{proposition}
    Każdy węzeł taśmowy jest plastrowy.
\end{proposition}

Dawno temu Fox zapytał, czy implikacja odwrotna także jest prawdziwa.

\begin{conjecture}[Fox, 1958]
    Każdy węzeł plastrowy jest taśmowy.
\end{conjecture}

W latach sześćdziesiątych podano kryteria na to, by węzeł nie był węzłem plastrowym (Fox-Milnor 1966, Murasugi 1965).
W szczególności, spośród tych o~co najwyżej siedmiu skrzyżowaniach, tylko $6_1$ jest plastrowy.

% http://citeseerx.ist.psu.edu/viewdoc/download?rep=rep1&type=pdf&doi=10.1.1.212.2610
% http://people.brandeis.edu/~ruberman/drslides/wesleyan.pdf

\begin{proposition}
    Niech $K$ będzie węzłem plastrowym.
    Wtedy $\operatorname{Arf} K = 0$.
\end{proposition}

\begin{proof}
    Z kryterium Foxa-Milnora wynika, że wyznacznik węzła
    \begin{equation}
        \det K = |\alexander(-1)| = f(-1)\cdot f(-1)
    \end{equation}
    to kwadrat.
    Fakt \ref{odd_determinant} mówi nam, iż wyznacznik jest jednocześnie liczbą nieparzystą.
    Wynika stąd przystawanie $\det K \equiv 1 \mod 8$, które w~połączeniu z warunkiem Murasugiego (fakt \ref{arf_murasugi}) daje $\operatorname{Arf} K = 0$.
\end{proof}

\begin{proposition} \label{slice_signature}
    Plastrowe węzły mają zerową sygnaturę.
\end{proposition}

\begin{proof}[Szkic dowodu]
    Ustalmy odwzorowanie $f$, które jest niesingularne, symetryczne i~dwuliniowe, z~przestrzeni $V$ o~wymiarze $2n$ oraz wyznaczoną przez nie formę kwadratową.
    Jeśli znika ona na podprzestrzeni wymiaru $n$, to ma zerową sygnaturę.
    %Praca "Infinite Order Amphicheiral Knots". (Charles Livingston, 2001) -- chyba nie?
    (dowód znaleziony w~podręczniku Lickorisha).
    Patrz też twierdzenie 8.8 z~artykułu \cite{murasugi65}.
\end{proof}

Własności opisane w~faktach \ref{slice_square_det} i~\ref{slice_signature} pozwalają stwierdzić, że 223 spośród 249 węzłów pierwszych o~dziesięciu lub mniej skrzyżowaniach nie jest plastrowych.

Każda macierz Seiferta $V$ (całkowitoliczbowa, kwadratowa, taka że $\det (V - V^t) = 1$), która jest unimodularnie sprzężona: istnieje całkowitoliczbowa macierz $P$ o~wyznaczniku równym $\pm 1$, że
\[
    V = P () P^{-1}
\]
stanowi macierz Seiferta pewnego węzła plastrowego.
Takie węzły nazywamy plastrowymi algebraicznie.
Węzeł $K$ w~$S^3$ jest algebraicznie plastrowy dokładnie wtedy, gdy ogranicza izotropiczną powierzchnię w~kuli $B^4$.
Wytłumaczenie w~podręczniku Kawauchiego.

% Koniec sekcji Węzły plastrowe
