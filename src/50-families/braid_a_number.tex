
\subsection{Liczba warkoczowa} % (fold)
\index{liczba warkoczowa|(}

% DICTIONARY;braid number;liczba warkoczowa
\begin{definition}
\label{def:braid_number}%
    Liczba warkoczowa to minimalna liczba pasm, na których można zapleść warkocz, którego domknięciem jest wyjściowy splot.
\end{definition}

Tylko jeden węzeł ma liczbę warkoczową $1$, jest to niewęzeł.
Dwuwarkoczowe są dokładnie węzły torusowe typu $(2, n)$ dla $|n| \ge 3$.
Węzły spełniające $\braid (K) = 3$ nie zostały jeszcze sklasyfikowane.
Liczba warkoczowa splotu zależy od orientacji ogniw i~trudno wyznacza się w~ogólnym przypadku.

\begin{proposition}
    Węzeł o~$n$ skrzyżowaniach można zapleść na $n - 1$ pasmach: $\crossing K \ge 1 + \braid K$.
\end{proposition}

Powyższe ograniczenie nie jest zbyt użyteczne, równość mamy jedynie dla trójlistnika i~ósemki.
Ohyama pokazał, że jeśli $L$ jest nierozszczepionym\footnote{a może nierozszczepialnym?} splotem, to $\crossing L \ge 2 \braid L - 2$.
Dowód korzysta z grafu Seiferta splotu.
Wśród pierwszych węzłów do 10 skrzyżowań mamy równość w oszacowaniu Ohyamy, poza następującymi wyjątkami: $4_1$, $6_1$, $8_1$, $8_3$, $8_{12}$, $10_1$, $10_3$, $10_{13}$, $10_{35}$, $10_{58}$.

\begin{proposition}[nierówność Mortona-Franksa-Williamsa]
    Niech $L$ będzie zorientowanym splotem, zaś $E$ oraz $e$ największą oraz najmniejszą potęgą $l$ w wielomianie HOMLFLY.
    Wtedy
    \begin{equation}
        \braid L \ge \frac{E-e}{2} + 1.
    \end{equation}
\end{proposition}

Nierówność jest ostra dla wszystkich pierwszych węzłów do 10 skrzyżowań poza $9_{42}$, $9_{49}$, $10_{132}$, $10_{150}$ oraz $10_{156}$.
% mathworld wolfram

Wielomian Alexandera wykrywa czasami węzły, których nie otrzyma się przez domykanie ,,małych'' warkoczy.
Przytoczone tu wyniki pochodzą z pracy \cite{jones85} Jonesa, gdzie nie ma jednak ich dowodów.
Jeśli $|\alexander(i)| > 3$, to węzeł nie jest domknięciem 3-warkocza (wniosek 23).
Ta implikacja jest skuteczna przy 43 z 59 węzłów o mniej niż 10 skrzyżowaniach.
Jeśli zaś spełniona jest nierówność $\alexander (\exp (2\pi i / 5)) > 13/2$, nie jest on domknięciem 4-warkocza (wniosek 24).
Prawdopodobnie nie istnieją podobne warunki dla 5-warkoczy.

Znamy liczby warkoczowe między innymi węzłów torusowych (fakt \ref{cor:torus_braid_number}).

\index{liczba warkoczowa|)}

% Koniec podsekcji Liczba warkoczowa

