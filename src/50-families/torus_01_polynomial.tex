
\subsection{Niezmienniki liczbowe węzłów torusowych}
Podamy teraz wartości całkowitoliczbowych niezmienników dla węzłów torusowych przy założeniu, że $p$ lub $q$ nie jest zerem.
Nietrywialne węzły torusowe są odwracalne, ale mają niezerową sygnaturę, więc nie są achiralne (chyba wiedział o tym Schreier \cite{schreier24}).

\begin{proposition}
\index{okres}%
    Węzeł torusowy $T_{p, q}$ ma okres $|p|$ oraz $|q|$.
\end{proposition}

\begin{proposition}
\index{sygnatura}%
    Niech $p, q > 0$ będą liczbami całkowitymi, zaś $R_2$ oznacza resztę z dzielenia przez dwa.
    Zdefiniujmy funkcję $\sigma(p, q) = - \sigma(T_{p, q})$.
    % TODO: czemu ten minus? oni inaczej skierowane łuki mają?
    Spełnia zależność rekurencyjną
    \begin{equation}
        \sigma(p, q) = \begin{cases}
             q^2 + \sigma(p, p - 2q) - R_2(p)       & \text{jeśli } 2q < p, \\
             q^2 - 1                              & \text{jeśli } 2q = p, \\
             q^2 - \sigma(2q - p, q) + R_2(q) - 2 & \text{jeśli } 2q > p > q, \\
             \frac 12 (q^2 + R_2(q)) - 1                 & \text{jeśli } p = q,
             % czwarte stanowi algebraiczne przekształcenie trzeciego dla p >= q
        \end{cases}
    \end{equation}
    z warunkami brzegowymi: $\sigma(1, q) = 0$, $\sigma(2, q) = q-1$ oraz równość $\sigma(p, q) = \sigma(q, p)$.
\end{proposition}

\begin{proof}[Niedowód]
\index[persons]{Gordon, Cameron}%
\index[persons]{Litherland, Richard}%
\index[persons]{Murasugi, Kunio}%
\index[persons]{Brieskorn, Egbert}%
\index[persons]{Hirzebruch, Friedrich}%
    Gordon, Litherland, Murasugi \cite[tw. 5.2]{litherland81} wspominają, że Brieskorn \cite{brieskorn66} policzył sygnatury pewnych rozmaitości algebraicznych, co wystarcza do znalezienia sygnatury węzłów torusowych.
    Prowadzi to do takiego samego wzoru jak ten znaleziony przez Hirzebrucha \cite{hirzebruch68}.
    Ale to wszystko jest trochę nieporęczne, dlatego używają niezmiennika acyklicznego (o~trywialnych zredukowanych grupach homologii; z~angielskiego \emph{null-homologous}) splotu $L$ w~zorientowanej 3-rozmaitości $M$ w~połączeniu z~jego $m$-krotnym rozgałęzionym nakryciem cyklicznym.
\index{wzór Hirzebrucha}%
\end{proof}

Borodzik niedawno przyjrzał się dokładniej sygnaturom węzłów torusowych.
\index[persons]{Borodzik, Maciej}%
Razem z~Oleszkiewiczem \cite{borodzik10} pokazał, że nie istnieje wymierna funkcja $R(p, q)$, która pokrywałaby się z sygnaturą węzła torusowego $T_{p, q}$ dla wszystkich względnie pierwszych, nieparzystych wartości $p$ oraz $q$.
\index[persons]{Oleszkiewicz, Krzysztof}%

\begin{proposition}
    Niech $p, q$ będą względnie pierwszymi liczbami, zaś $C \in [0, 1)$ stałą taką, że $Cpq$ nie jest liczbą całkowitą.
    Przyjmijmy $z = \exp (2 \pi i C)$ i zdefinujmy pomocnicze funkcje: niech $\{x\} = x - \lfloor x \rfloor$ oznacza część ułamkową, zaś
    \begin{equation}
        \langle x \rangle = \begin{cases}
            0 & \text{dla } x \in \Z \\
            \{x\} - 1/2 & \text{dla } x \not \in \Z
        \end{cases}
    \end{equation}
    funkcję piłę.
    Dalej, określmy sumę Dedekinda
    \begin{equation}
        s(p, q, x) = \sum_{j = 0}^{q-1} \left\langle \frac {j}{q} \right\rangle \left\langle \frac {jp}{q} + x \right\rangle.
    \end{equation}
    (Uwaga! Definicja funkcji $s$ z \cite{borodzik10} zawiera złośliwą literówkę.)
    Przy tych oznaczeniach, sygnatura węzła $(p, q)$-torusowego wyznacza się wzorem
    \begin{align}
        \sigma(z) & = \frac{1}{3pq} \left(p^2 + q^2 + 6 \langle Cpq \rangle^2 - \frac {1}{2} \right)  + 2(C^2 - C) pq + (2-4C) \langle Cpq \rangle + {} \\
        & - 2s(p, q, Cp) - 2s(q, p, Cq) - 2s(p, q, p-pC) - 2s(q, p, q-qC). \nonumber
    \end{align}
\end{proposition}

\begin{corollary}
    Jeśli $p, q$ są nieparzyste i względnie pierwsze, to
    \begin{equation}
        \sigma(T_{p,q}) = \frac{1}{6pq} + \frac{2p}{3q} + \frac{2q}{3p} - \frac{pq}{2} - 4(s(2p, q, 0) + s(2q, p, 0)) - 1.
    \end{equation}
\end{corollary}

\begin{corollary}
    Jeśli $p$ jest nieparzyste, zaś $q > 2$ parzyste, to
    \begin{equation}
        \sigma(T_{p,q}) = - \frac{pq}{2} + 4s(2p, q, 0) - 8s(p, q, 0) + 1.
    \end{equation}
\end{corollary}

Wyznaczenie liczby gordyjskiej było dużo trudniejsze.
Murasugi \cite[s. 150]{murasugi96} proponuje Czytelnikowi pokazanie, że zachodzi równość $\unknotting T_{p, 2} = \frac 12 (|p| - 1)$ oraz nierówność
\begin{equation}
    u(T_{p, q}) \le \frac 12 (p-1)(q-1).
\end{equation}

Milnor \cite[uwaga 10.9]{milnor1968} postawił hipotezę, że znak $\le$ można zamienić na $=$.
\index{hipoteza!Milnora}%
Pierwszy dowód znaleźli Kronheimer, Mrówka \cite{kronheimer93}, \cite{kronheimer95}: panowie pokazali, że jeśli $X$ jest gładką, jednospójną, domkniętą i zorientowaną 4-rozmaitością; posiada nietrywialny wielomianowy niezmiennik Donaldsona oraz wymiar maksymalnej dodatniej podprzestrzeni dla formy przecięć drugiej homologii jest nieparzysty, większy od 2, to genus każdej zorientowanej, gładko zanurzonej powierzchni $F$ (poza dwoma wyjątkami, których nie rozumiemy), spełnia nierówność $2g - 2 \ge F \cdot F$.
Ugh!

Później Rasmussen \cite{rasmussen10} podał inny dowód, który nie wykorzystuje już cechowania (\emph{gauge theory}).
% 04 - ArXiV, 10 - peer reviewed
\index[people]{Rasmussen, Jacob}%

\begin{proposition}
\index{liczba gordyjska}%
\label{prp:torus_unknotting_number}%
    Dla względnie pierwszych $p, q > 0$ mamy
    \begin{equation}
        \unknotting T_{p, q} = \frac 12 (p - 1)(q - 1),
    \end{equation}
\end{proposition}

\begin{proposition}
    \index{genus}%
    Dla względnie pierwszych $p, q > 0$ mamy
    \begin{equation}
        \genus T_{p, q} = \frac 12 (p - 1)(q - 1),
    \end{equation}
\end{proposition}

\begin{proof}
    Wyznacznik macierzy Seiferta węzła torusowego jest niezerowy, więc genus równa się stopniu wielomianu Alexandera.
    Patrz też \cite[s. 149]{murasugi96}.
\end{proof}

\begin{proposition}
\index{liczba mostowa}%
    $\bridge T_{p, q} = \min \{|p|, |q|\}$
\end{proposition}

Według Murasugiego \cite[s. 150]{murasugi96} dowód znalazł Schubert \cite{schubert54}.
\index[persons]{Schubert, Horst}%

\begin{corollary}
\index{indeks warkoczowy}%
\label{cor:torus_braid_number}%
    Niech $p, q \neq 0$ będą liczbami całkowitymi.
    Wtedy $\braid T_{p, q} = \min \{|p|, |q|\}$.
\end{corollary}

\begin{proof}
    Niech $K$ będzie węzłem torusowym typu $(p,q)$ z~minimalnym przedstawieniem jako warkocz $\beta$.
    Z konstrukcji domknięcia (czyli dołączenia rozłącznych półokręgów) wynika,
    że diagram $K$ ma dokładnie $b(K)$ lokalnych maksimów.
    Definicja liczby mostowej orzeka, iż $\bridge K \le \braid K$.
    Bez straty ogólności niech $p > q > 0$.
    Skoro węzeł $K$ powstaje z~$q$-warkocza $(\sigma_{q-1} \ldots \sigma_2\sigma_1)^p$,
    indeks $b(K)$ nie przekracza $q = br(K)$.
\end{proof}

\begin{proposition}
\index{indeks skrzyżowaniowy}%
    Mamy $\crossing T_{p, q} = |pq| - \max\{|p|, |q|\}$.
\end{proposition}
    
\begin{proof}
\index[persons]{Murasugi, Kunio}%
    Murasugi \cite[s. 255]{murasugi91} zauważył, że jeśli $\gamma$ jest jednorodnym $n$-warkoczem, gdzie $n = \braid(\widehat{\gamma})$, zaś $d_i$ to suma wykładników przy $\sigma_i$ w $\gamma$, to $\crossing \widehat{\gamma} = \sum_{i=1}^{n-1} |d_i|$.
    Ale splot torusowy $T_{p, q}$, gdzie $2 \le p \le q$, można przedstawić jako domknięcie jednorodnego $p$-warkocza $(\sigma_1 \sigma_2 \cdot \ldots \cdot \sigma_{p-1})^q$, co kończy natychmiast dowód.
\end{proof}

