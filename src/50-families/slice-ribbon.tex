
\subsection{Węzły taśmowe}
\index{węzeł!taśmowy|(}
\begin{definition}
    Węzeł $K = f(S^1)$ będący brzegiem osobliwego dysku $f \colon D \to S^3$ posiadającego następującą własność: każda przecinająca siebie składowa jest łukiem $A \subseteq f(D^2)$, dla którego $f^{-1}(A)$ składa się z~dwóch łuków w~$D^2$ (jeden z~nich jest wewnętrzny), nazywamy taśmowym.
\end{definition}

Jak pisze Kawauchi, mamy oczywiste wynikanie:

\begin{proposition}
\index{węzeł!plastrowy}%
    Każdy węzeł taśmowy jest plastrowy.
\end{proposition}

Dawno temu Fox zapytał, czy implikacja odwrotna jest prawdziwa (\cite[problem 1.33]{kirby78}):
\index{człowiek!Fox, Ralph}%

\begin{conjecture}[slice-ribbon problem]
    \index{hipoteza!plastrowo-taśmowa}
    Czy każdy węzeł plastrowy jest taśmowy?
\end{conjecture}

Wprawdzie Lisca pokazał prawdziwość hipotezy dla węzłów dwumostowych \cite{lisca07},
\index{człowiek!Lisca, ?}%
% korzystając ze słynnego tw. Donaldsona: that a definite intersection form of a compact, oriented, simply connected, smooth manifold of dimension 4 is diagonalisable
\index{węzeł!dwumostowy}%
zaś Greene oraz Jabuka zrobili to dla precli o trzech pasmach w~\cite{greene11};
\index{człowiek!Greene, ?}%
\index{człowiek!Jabuka, ?}%
\index{precel}%
ale Gompf, Scharlemann i~Thompson zasugerowali w~\cite{gompf10} potencjalny kontrprzykład.
\index{człowiek!Gompf, ?}%
\index{człowiek!Scharlemann, ?}%
\index{człowiek!Thompson, ?}%
\index{rozmaitość szwowa}%
Nie możemy przytoczyć tego kontrprzykładu, gdyż korzysta z~rozmaitości szwowych, opisanych w~\cite[s. 53-59]{kawauchi96}.

Teichner myśli\footnote{Patrz \url{https://mathoverflow.net/a/18154}.} o hipotezie plastrowo-taśmowej jako o~życzeniu, które uprościłoby pewne czterowymiarowe problemy, gdyby było prawdziwe.
\index{człowiek!Teichner, ?}%

\index{węzeł!taśmowy|)}

% koniec podsekcji węzły taśmowe

