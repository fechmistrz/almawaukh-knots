\section{Węzły satelitarne} % (fold)
\label{sec:satellite}
% meridian - południk
% longitude - równoleżnik
Załóżmy, że w dopełnieniu pewnego splotu został zanurzony torus.
Jeżeli jest ściśliwy, to albo równoleżnikl torusa ogranicza dysk w dopełnieniu splotu~i torus jest niezawęźlony, albo południk ogranicza dysk w dopełnieniu splotu i~splot nie przebiega wzdłuż torusa.
Żadna z~tych sytuacji nie jest ciekawa.
Inny zdegenerowany przypadek występuje, gdy torus stanowi rurowe otoczenie jednego z~ogniw splotu.
W przeciwnym razie splot można zbudować z~prostszych obiektów.

Oto formalny opis konstrukcji.
Niech $W$ będzie pełnym torusem.
Dysk zanurzony w $W$, którego brzeg stanowi nieściągalną pętlę w $\partial W$, nazywamy południkowym.
Mówimy, że zamknięta krzywa $\lambda \subseteq W$ jest właściwa, jeżeli przecina wszystkie dyski południkowe.

\begin{definition}[węzeł satelitarny]
    \index{węzeł!satelitarny}
    Niech $P$ będzie splotem zanurzonym w~niezawęźlonym torusie $W$ tak, by co najmniej jedno z~ogniw stanowiło właściwą pętlę w~$W$.
    Niech $C$ będzie węzłem, zaś $V$ jego rurowym otoczeniem.
    Wybierzmy dowolny homeomorfizm $h \colon W \to V$.
    Wtedy splot $S = h(P)$ nazywamy satelitą o~wzorcu $P$ oraz towarzyszu $C$.
\end{definition}

% \begin{definition}
%     Węzeł nazywamy satelitarnym, jeśli zawiera nieściśliwy, nierównoległy do brzegu torus we własnym dopełnieniu.
% \end{definition}

Hoste i inni podejrzewają w~\cite{thistlethwaite98}, że jeśli satelita owija się $m$-krotnie wokół torusa, zaś indeks skrzyżowaniowy towarzysza wynosi $k$, to satelita nie posiada diagramu o~mniej niż $km^2$ skrzyżowaniach.

Ponieważ dla trójlistnika $k = 3$, napotkali się tylko na satelity owijające się $m = 2$ razy podczas tablicowania pierwszych węzłów do 16 skrzyżowań.
Nie spodziewano się żadnego satelity ósemki, gdyż wtedy $k = 4$, zatem każdy satelita miałby co najmniej $4 \cdot 2^2 + 1 = 17$ skrzyżowań: dodatkowe $+1$ jest potrzebne, by nie dostać splotu o~dwóch ogniwach.

Najprostszy satelita ma 13 skrzyżowań.

\begin{example}[swallow-follow torus]
	Klasa węzłów satelitarnych obejmuje węzły złożone.
	W ich przypadku można wskazać pewien szczególny torus nieściśliwy -- połykający pierwszy składnik, a~potem podążający za drugim.
	Świetnie przedstawione jest to na stronie 82 książki \cite{cromwell04} Cromwella.
\end{example}

Schubert pokazał, że zorientowane klasy izotopii węzłów w~$S^3$ tworzą wolny przemienny monoid na przeliczalnie wielu generatorach.
Dowód to uważna analiza nieściśliwych torusów obecnych w~dopełnieniu sumy spójnej.
To doprowadziło go do definicji węzłów satelitarnych i~towarzyszących w~przełomowej pracy \cite{schubert53} oraz zunifikowało teorię 3-rozmaitości z teorią węzłów.
Patrz też \cite{motegi97}.
% czemu akurat Motegi?

Na brzegu torusa $V$ można wprowadzić pewien układ współrzędnych: południk to pętla właściwa w $\partial V$, która ogranicza dysk w $V$, natomiast równoleżnik to pętla w $\partial V$, która spotyka południk raz.
Z~dokładnością do izotopii południk jest jeden, ale równoleżnik nie.
Gdy indeks zaczepienia równoleżnika oraz rdzenia torusa wynosi zero, mówimy, że równoleżnik jest preferowany.

\begin{definition}[dubel Whiteheada]
    \index{dubel Whiteheada}
    Jeżeli $P \subseteq W$ jest skręconym jednokrotnie niewęzłem, to węzeł $S$ nazywamy dublem Whiteheada.
\end{definition}

Każdy węzeł posiada nieskończenie wiele dubli Whiteheada: wystarczy rozciąć torus $V$, skręcić jedną końcówkę i~ponownie zszyć, żaden z~nich nie jest odróżniany od niewęzła przez wielomian Alexandera.

Wyróżnia się pewien szczególny homeomorfizm $h$, który przenosi południk i preferowany równoleżnik $W$ na południk i preferowany równoleżnik $V$.
Nazywamy go wiernym.
% faithful
O dublu względem wiernego homeomorfizmu mówimy, że jest nieskręcony.

\begin{definition}[węzeł kablowy]
	\index{węzeł!kablowy}
	Niech $h \colon W \to V$ będzie wiernym homeomorfizmem, zaś $P$ węzłem $(p, q)$-torusowym.
	Satelitę $S$ nazywamy węzłem $(p, q)$-kablowym albo krótko kablem.
\end{definition}

\begin{proposition}
    Każdy kabel wyznacza jednoznacznie węzeł, z~którego powstał.
\end{proposition}

\begin{proof}
    Wniosek 2 z~pracy \cite{feustel78} Feustela, Whittena pokazuje, że na podstawie kabla można wyznaczyć parametry węzła torusowego $K'_{p,q}$ oraz topologię dopełnienia oryginalnego węzła.
    Wiemy jednak z~twierdzenia Gordona-Lueckego, że różne węzły mają różne dopełnienia.
\end{proof}

Niewęzeł nie ma nietrywialnych węzłów towarzyszących.

\begin{definition}
	Towarzysza $C$ nietrywialnego splotu nazywamy właściwym, jeśli nie jest niewęzłem i~nie jest ogniwem tego splotu.
\end{definition}

Sploty bez właściwych towarzyszy określa się zazwyczaj terminem ,,atoroidalny''.
Patrz też diagram przedstawiony w {cromwell04} na stronie 83.

\begin{proposition}
	Duble nietrywialnych węzłów oraz kable są pierwsze.
\end{proposition}

\begin{proof}
	Jest to wniosek z twierdzenia 4.4.1 w~\cite{cromwell04}: jeśli wzorzec jest niewęzłem lub węzłem pierwszym, to każdy właściwy satelita jest pierwszy.
\end{proof}

Niektóre węzły przedstawiają się jako satelity w~dokładnie jeden sposób, inne nie.
Rok 1979 przyniósł amerykańską pracę \cite{jaco79} oraz niemiecką książkę \cite{johannson79}, gdzie niezależnie od siebie opisano jednoznaczny rozkład, nazywany teraz rozkładem Jaco-Shalena-Johannsona:

\begin{proposition}
	Niech $M$ będzie nierozkładalną, orientowalną, domkniętą 3-rozmaitością.
	Istnieje wtedy jedyna z dokładnością do izotopii minimalna rodzina rozłącnzie zanurzonych nieściśliwych torusów tak, że każda składowa 3-rozmaitości powstałej przez rozcinanie wzdluż torusów jest atoroidalna lub włóknistą przestrzenią Seiferta ($S^1$-wiązką nad dwuwymiarowym orbifoldem).
\end{proposition}

Jest on związany z operacją splatania (ang. \emph{splicing}), będącej uogólnieniem budowania satelitów.
Hipotezę o jedyności rozkładu wysnuł wcześniej Waldhausen.

% Koniec sekcji Węzły satelitarne
