Przyjęta przez nas definicja węzła czy splotu jest bardzo ogólna.
O tak określonych obiektach można udowodnić niewiele twierdzeń.
W dalszym ciągu tego rozdziału będziemy rozpatrywali rozmaite klasy splotów.
Jest rzeczą jasną, że im węższa klasa, tym więcej twierdzeń o jej elementach można udowodnić.
Nakładane przez nas ograniczenia będą różnego charakteru.
Zaczniemy od warkoczy oraz supłów, które stanowią cegiełki do budowy splotów.
Z pojęciem supła mocno związane są sploty dwumostowe.
Potem poznamy precle, sploty złożone z wielu warkoczy połączonych ze sobą oraz bardzo symetryczne węzły Lissajous.
Później zbadamy węzły torusowe, klasę zrozumianą jako jedną z pierwszych, uogólnienie węzłów złożonych: węzły satelitarne i opowiemy krótko o węzłach hiperbolicznych.
Na koniec przytoczymy kilka wyników czterowymiarowej teorii węzłów plastrowych i taśmowych.