\section{Węzły Lissajous} % (fold)
\label{sec:lissajous}

\begin{definition}
	Węzłem Lissajous nazywamy węzeł zadany parametrycznie:
	\[
		x = \cos(n_xt + \varphi_x) \quad
		y = \cos(n_yt + \varphi_y) \quad
		z = \cos(n_zt + \varphi_z),
	\]
	gdzie $n_x, n_y, n_z$ to stałe całkowite, zaś $\varphi_x, \varphi_y, \varphi_z$ rzeczywiste.
\end{definition}

Węzeł nie może posiadać samoprzecięć, dlatego żadna z wielkości $n_i\varphi_j-n_j\varphi_i$, dla różnych indeksów $i, j$ nie może być krotnością $\pi$.
Bez straty ogólności możemy założyć, że $\varphi_z = 0$.
Dodatkowo stałe $n_x, n_y, n_z$ muszą być parami względnie pierwsze.

Przykładem są $5_2$ (dla $n_x = 3$, $n_y = 2$, $n_z = 7$, $\varphi_x = 7/10$, $\varphi_y = 2/10$), $6_1$, $7_4$, $8_{15}$, $10_1$, $10_{35}$, $10_{58}$, suma prawego i lewego trójlistnika, suma dwóch kopii $5_2$.
Istnieje nieskończenie wiele węzłów Lissajous (\cite{lamm97}).
Każdy skręcony węzeł z zerowym niezmiennikiem Arfa jest Lissajous.

Węzły Lissajous są bardzo symetryczne.
Jeśli wszystkie stałe $n_x, n_y, n_z$ są nieparzyste, węzeł jest silnie dodatnio achiralny.
Jeśli jedna z nich, na przykład $n_x$, jest parzysta, to półobrót wokół osi $x$ odwzorowuje węzeł w siebie.
To nakłada ograniczenia na wielomian Alexandera.
W przypadku nieparzystym (parzystym), $\Delta(t)$ jest kwadratem w pierścieniu $\Z[t]$ (w $\Z/2[t]$).
Dowód ostatniego stwierdzenia zawierają prace \cite{hartley79} oraz \cite{murasugi71}.

\begin{corollary}
	Trójlistnik, ósemka, węzły torusowe oraz dwumostowe węzły włókniste nie są węzłami Lissajous.
\end{corollary}

% Koniec sekcji Węzły Lissajous