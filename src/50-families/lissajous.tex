
\section{Węzły Lissajous}
% szkielet: praca Lamma
\index{węzeł!Lissajous|(}%
Węzły Lissajous zdefiniowali Bogle, Hearst, Jones i Stoiłow \cite{bogle94} jako węzły, których pewien diagram jest krzywą Lissajous.
\index[persons]{Bogle, Miles}%
\index[persons]{Hearst, John}%
\index[persons]{Jones, Vaughan}%
\index[persons]{Stoiłow, Luben}%

\begin{definition}[węzeł Lissajous]
    Węzeł zadany parametrycznie:
    \begin{equation}
        x = \cos(n_xt + \varphi_x) \quad
        y = \cos(n_yt + \varphi_y) \quad
        z = \cos(n_zt + \varphi_z),
    \end{equation}
    gdzie $n_x, n_y, n_z$ (,,częstotliwości'') to stałe całkowite, zaś $\varphi_x, \varphi_y, \varphi_z$ (,,fazy'') są rzeczywiste, nazywamy węzłem Lissajous.
\end{definition}

Węzeł nie może posiadać samoprzecięć, dlatego żadna z~wielkości $n_i\varphi_j-n_j\varphi_i$, dla różnych indeksów $i, j$ nie może być krotnością $\pi$.
Bez straty ogólności możemy założyć, że $\varphi_z = 0$.
Dodatkowo stałe $n_x, n_y, n_z$ muszą być parami względnie pierwsze.

Wiele węzłów jest węzłami Lissajous:

% TODO: przepisać wszystkie!

\begin{example}
    Dla $n_x = 3$, $n_y = 2$, $n_z = 7$, $\varphi_x = 7/10$, $\varphi_y = 2/10$ mamy węzeł $5_2$.
\end{example}

\begin{example}
    Węzły pierwsze $6_1$, $7_4$, $8_{15}$, $10_1$, $10_{35}$, $10_{58}$ są węzłami Lissajous.
\end{example}

\begin{example}
    Suma prawego i~lewego trójlistnika, suma dwóch kopii $5_2$ są węzłami Lissajous.
\end{example}

Jest wiele węzłów Lissajous:

\begin{proposition}
    Istnieje nieskończenie wiele węzłów Lissajous.
\end{proposition}

\begin{proof}
    Niech $a, b > 1$ będą względnie pierwsze.
    Lamm w \cite{lamm97} pokazał dużo ogólniejszy wynik.
    Mianowicie: węzeł Lissajous o~częstotliwościach $n_x = a$, $n_y = b$, $n_z = 2ab-a-b$ oraz fazach:
    \begin{equation}
        \varphi_x = \frac{2n_x-1}{n_z} \pi, \quad
        \varphi_y = \frac{\pi}{n_z}, \quad
        \varphi_z = 0
    \end{equation}
    posiada sygnaturę $\sigma = a+b-ab-1$ i genus $g = -\sigma/2$.
\index{sygnatura}%
\index{genus}%
\end{proof}

Węzły Lissajous są bardzo symetryczne.

\begin{proposition}
\index{węzeł!silnie dodatnio achiralny}
    Jeśli wszystkie stałe $n_x, n_y, n_z$ są nieparzyste, węzeł jest silnie dodatnio achiralny.
\end{proposition}

\begin{proposition}
\index{węzeł!okresowy}%
\label{prp:lissajous_two_periodic}%
    Jeśli jedna z częstotliwości jest parzysta, to węzeł jest 2-okresowy.
\end{proposition}

\begin{proof}
    Załóżmy, że częstotliwość $n_x$ jest parzysta.
    Wtedy półobrót wokół osi $x$ odwzorowuje węzeł w~siebie.
\end{proof}

To nakłada ograniczenia na wielomian Alexandera.

\begin{proposition}
\index{wielomian!Alexandera}
\label{prp:lissajous_alexander}%
    Niech $K$ będzie węzłem Lissajous.
    Wtedy $\alexander(x)$, jego wielomian Alexandera, jest kwadratem w~pierścieniu $(\Z/2\Z)[x]$.
\end{proposition}

\begin{proof}
    Rozpatrzmy dwa przypadki.

    Jeśli wszystkie częstotliwości są nieparzyste, węzeł jest silnie dodatnio achiralny.
    Wtedy wielomian $\alexander(x)$ jest kwadratem w pierścieniu $\Z[x]$ (pokazano to w \cite{hartley79}).

    W przeciwnym razie możemy powołać się na fakt \ref{prp:lissajous_two_periodic} oraz \ref{prp:murasugi_periodic} (z $n=2^1$ oraz $\lambda = 1$).
    Dostaniemy równość
    \begin{equation}
\label{eqn:lissajous_squared}%
        \alexander_K(t) \equiv \alexander^2_{J}(t) \mod 2,
    \end{equation}
    która kończy dowód.
\end{proof}

Warto zwrócić uwagę, że bycie silnie dodatnio achiralnym jest bardzo restrykcyjnym warunkiem.
Spośród pierwszych węzłów o~co najwyżej 12 skrzyżowaniach, tylko trzy go spełnia: 10a103 ($10_{99}$), 10a121 ($10_{123}$) oraz 12a427.
Tylko pierwszy z nich jest na pewno węzłem Lissajous (stan na 2018).

\begin{example}
    Trójlistnik oraz ósemka nie są węzłami Lissajous.
\end{example}

Wygodnie jest przeformułować warunek z faktu \ref{prp:lissajous_alexander} do następującej postaci.
Wielomian $\alexander(t) = A_0 + A_1(t+1/t) + \ldots + A_n(t^n + 1/t^n)$ jest kwadratem modulo $2$ wtedy i tylko wtedy, gdy współczynniki $A_{2k+1}$ są parzyste dla $k \ge 0$.

\begin{corollary}
\index{niezmiennik!Arfa}%
    Niech $K$ będzie węzłem Lissajous, wtedy niezmiennik $\operatorname{Arf} K = 0$ znika.
\end{corollary}

\begin{proof}
    Niech $\alexander(t) = A_0 + A_1(t+1/t) + \ldots + A_n(t^n + 1/t^n)$ będzie  wielomianem Alexandera.
    Wtedy $\alexander_K(1) = \pm 1$, zatem $0 = \pm 1 + A_0 + 2A_1 + \ldots + 2A_n$.
    Możemy teraz odjąć to od $\alexander_K(-1) = A_0 - 2A_1 \pm \ldots$, by uzyskać $\alexander_K(-1) = \pm 1 - 4A_1 - 4A_3 - \ldots$.
    Razem z wygodnym przeformułowaniem daje to $\alexander_K(-1) \equiv \pm 1 \mod 8$, co w połączeniu z~\ref{prp:arf_murasugi} kończy dowód.
\end{proof}

\begin{corollary}
\index{węzeł!włóknisty}%
\label{cor:lissajous_fibered}%
    Włókniste węzły o nieparzystym genusie nie są węzłami Lissajous.
\end{corollary}

%=% Lamm
\begin{proof}
    Wynika to z~naszego wygodnego przeformułowania oraz stwierdzenia 8.33 w książce \cite{burde14}: jeśli $K$ jest włóknistym węzłem o genusie $g$, to wielomian Alexandera $\alexander_K$ jest stopnia $2g$ i~ma wiodący współczynnik $\pm 1$.
\end{proof}

\begin{corollary}
\index{węzeł!dwumostowy}%
\label{cor:lissajous_twobridge}%
    Niech $K$ będzie dwumostowym węzłem Lissajous.
    Wtedy $\alexander_K(t) \equiv 1 \mod 2$.
\end{corollary}

\begin{proof}
    Nietrywialny węzeł o dwóch mostach nie może być silnie dodatnio achiralny (patrz \cite{hartley79}).
    Jeśli jest węzłem Lissajous, jedna z jego częstotliwości okazuje się być parzysta.
    Wtedy jego faktor jest trywialny i~kongruencja \ref{eqn:lissajous_squared} daje $\alexander_K(t) \equiv 1 \mod 2$.
\end{proof}

%=% Lamm
\begin{corollary}
\index{węzeł!włóknisty}%
    Dwumostowe węzły włókniste nie są węzłami Lissajous.
\end{corollary}

Można skorzystać z tego samego argumentu, co w dowodzie wniosku \ref{cor:lissajous_fibered} oraz \ref{cor:lissajous_twobridge}.

%=% Lamm
\begin{proposition}
    Niech $p, q$ będą względnie pierwszymi liczbami różnymi od $0, \pm 1$, zaś $K$ węzłem.
    Wtedy $(p, q)$-kabel węzła $K$ nie jest węzłem Lissajous.
\end{proposition}

Lamm pisze, że obiekt ten -- $(p, q)$-kabel -- zdefiniowano w książce Eisenbuda/Neumanna ,,Three-dimensional link theory and invariants of plane curve singularities''.
\index[persons]{Neumann, Walter}%
\index[persons]{Eisenbud, David}%
Celowo pomijamy ją w bibliografii, prawdopodobnie chodzi o zwykłe węzły satelitarne.

\begin{proof}
\index[persons]{Seifert, Herbert}%
    Niech $L$ będzie $(p, q)$-kablem węzła $K$.
    Seifert pokazał w \cite{seifert50}, że
    \begin{equation}
        \alexander_L(t) = \frac{(t^{pq}-1)(t-1)}{(t^p-1)(t^q-1)} \alexander_K(t^p),
    \end{equation}
    przy czym wyjątkowo wielomian $\alexander$ nie jest symetryczny w $t$ oraz $1/t$, tylko unormowany.
    Dwa najwyższe współczynniki w wielomianie ,,ułamkowym'' to $\pm 1$, a skoro $|p| > 1$, dwa najwyższe współczynniki $\alexander_L$ to także $\pm 1$.
    ,,Wygodne sformułowanie'' kończy dowód.
\end{proof}

\begin{corollary}
\index{węzeł!torusowy}%
    Nietrywialne węzły torusowe nie są węzłami Lissajous.
\end{corollary}

\begin{proof}
    Węzły torusowe to kable niewęzła.
\end{proof}

\begin{corollary}
    Nietrywialne węzły algebraiczne nie są węzłami Lissajous.
\end{corollary}

Co to są węzły algebraiczne?
Nie wiem.

\begin{proof}
    Wynika to z ogólniejszego stwierdzenia: jeśli spełniony jest warunek $|p_n|, |q_n| > 1$, to iterowane węzły torusowe typu $((p_n, q_n), (p_1, q_1))$ nie są węzłami Lissajous.
\end{proof}

Podamy teraz pierwszy warunek wystarczający, by być węzłem Lissajous.
Znaleźli go Hoste z Zirbel w~\cite{zirbel06}:
\index[persons]{Hoste, Jim}%
\index[persons]{Zirbel, Laura}%

\begin{proposition}
\index{niezmiennik!Arfa}%
\index{węzeł!skręcony}%
    Niech $K$ będzie węzłem skręconym.
    Następujące warunki są równoważne: $K$ jest węzłem Lissajous; niezmiennik Arfa węzła $K$ znika.
\end{proposition}

Węzły bilardowe to zamknięte trajektorie kuli, która zostaje wystrzelona z jednej ze ścian sześcianu, odbija się pod takim samym kątem, pod jakim pada na ściany.
\index{węzeł!bilardowy}%
Jones, Przytycki pokazali w~\cite{jones98}, że węzły bilardowe to dokładnie węzły Lissajous i~zadali pytanie, czy każdy węzeł można zrealizować jako trajektorię kuli w~jakimś wielościanie.
\index[persons]{Jones, Vaughan}%
\index[persons]{Przytycki, Józef}%

Odpowiedź jest pozytywna.
Koseleff, Pecker w~\cite{koseleff14} korzystając z~twierdzenia Manturowa
\index{twierdzenie!Manturowa (o kwazitorycznych warkoczach)}%
(każdy splot jest domknięciem kwazitorycznego warkocza)
pokazują, że każdy węzeł ma diagram, który jest wielokątem gwiaździstym.
Użyte zostało twierdzenie Kroneckera z~1884 roku: jeśli liczby $\theta_0 = 1, \theta_1, \ldots, \theta_k$ są liniowo niezależne nad ciałem $\Q$, to zbiór punktów $(\lfloor n\theta_i \rfloor_{i=0}^k)_{n=0}^\infty$ leży gęsto w kostce jednostkowej.

Lamm, Obermeyer dowiedli w 1999, że węzły bilardowe wewnątrz walca są taśmowe albo okresowe, więc w walcu nie można zrealizować każdego węzła.
\index{węzeł!okresowy}%
\index{węzeł!taśmowy}%
Lamm postawił hipotezę, że jest to możliwe w eliptycznym walcu.
Pozytywnej odpowiedzi ponownie udzielił niedawno Pecker w \cite{pecker12}.

\index{węzeł!Lissajous|)}

% Koniec sekcji Węzły Lissajous

