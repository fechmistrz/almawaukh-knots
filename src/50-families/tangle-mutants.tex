\subsection{Mutanty i mutacje}
\label{sec:mutant}
Na zakończenie wspomnimy o~mutacjach.

\begin{definition}[mutacja]
    % \labelnotinuse{def:mutacja}
    \index{mutacja}
    Półobrót supła względem osi poziomej, pionowej albo też prostopadłej do płaszczyzny, w~jakiej leży diagram, nazywamy mutacją.
    W razie potrzeby zmieniamy orientację supła na przeciwną.
\end{definition}

\begin{tobedone}
rysunek z pracy ,,Tabulating and distinguishing mutants''
\end{tobedone}

\begin{definition}[mutant]
    \label{def:mutant}
    \index{mutant}
    Niech $K$ będzie węzłem.
    Węzeł, który powstaje przez wykonanie ciągu mutacji na węźle $K$, nazywamy mutantem węzła $K$.
\end{definition}

Mutacja węzła o~co najwyżej dziesięciu skrzyżowaniach nie zmienia jego klasy abstrakcji.
Najprostszą, a zarazem najsłynniejszą parą różnych od siebie mutantów stanowią węzeł Conwaya $11n_{34}$ oraz Kinoshity-Terasakiego $11n_{42}$.
\index{węzeł!Conwaya}
\index{węzeł!Kinoshity-Terasakiego}

\begin{tobedone}
rysunek
\end{tobedone}

Conway zauważył podczas klasyfikacji niealternujących węzłów, że tylko one posiadają trywialny wielomian Alexandera.
Mają też taki sam wielomian Jonesa,
\begin{equation}
    \jones(t) = t^{6} -2t^5 +2t^4 -2t^3 +t^2 +2t^{-1} -2t^{-2} +2t^{-3} -t^{-4}.
\end{equation}
Kinoshita, Terasaki zdefiniowali nieskończoną rodzinę węzłów o trywialnym wielomianie Alexandera, której pierwszym wyrazem jest węzeł $11n_{42}$ w~\cite{kinoshita57}.
Dowód tego, że $11n_{34}$ oraz $11n_{42}$ są różne, jako pierwszy podał prawdopodobnie Riley w~1971 roku \cite{riley71}: wykorzystał on homomorfizmy z~grupy węzła w~$PSL(2, 7)$.
Genusy, odpowiednio: $3$ i~$2$, wyznaczył Gabai piętnaście lat później w~\cite{gabai86}, używał foliacji.

Mutanty nie dają się łatwo odróżniać niezmiennikami, niestety dowód tego faktu jest równie trudny.
% Poziom zaawansowania tej książki nie pozwala przedstawić szczegółów dowodu, dlatego wymienimy tylko odnośniki do literatury: 

\begin{proposition}
    Mutacja węzła nie zmienia wielomianu Alexandera.
    \index{wielomian!Alexandera}
\end{proposition}

\begin{proof}
    Stojmenow, Tanaka piszą w \cite{tanaka09}, że to proste ćwiczenie teorii kłębiastej, oraz że rozumowanie łatwo przenosi się na odkryte później wielomiany Jonesa, HOMFLY, BLM/Ho, Kauffmana.
\end{proof}

\begin{proposition}
    Mutacja węzła nie zmienia kablowego wielomianu Jonesa.
    \index{wielomian!Jonesa}
\end{proposition}

\begin{proof}
    \cite{menasco91}.
\end{proof}

\begin{proposition}
    Mutacja węzła nie zmienia 2-kablowego wielomianu HOMFLY.
    \index{wielomian!HOMFLY}
\end{proposition}

\begin{proof}
    \cite{przytycki89}.
\end{proof}

\begin{proposition}
    Mutacja węzła nie zmienia kablowego wielomianu Kauffmana.
    \index{wielomian!kablowy}
    \index{wielomian!Kauffmana}
\end{proposition}

\begin{proof}
    \cite{lipson87}.
\end{proof}

% z tanaka09
Wzór kablowy \cite[tw. 6.15]{lickorish97} pokazuje, że wielomian Alexandera nie odróżnia satelitów zmutowanych węzłów.
Wynik rozszerzono do wielomianów HOMFLY oraz Kauffmana dla 2-satelitów zmutowanych węzłów \cite{lipson87}.
Wielomian HOMFLY radzi sobie lepiej z 3-kablami (często odróżnia je, w tym parę węzłów Kinoshity-Terasakiego/Conwaya).
\index{węzeł!Conwaya}
\index{węzeł!Kinoshity-Terasakiego}
Natomiast Morton, Traczyk \cite{traczyk88} pokazali, że wielomian Jonesa wcale nie odróżnia satelitów mutantów.

\begin{proposition}
    Mutacja węzła nie zmienia wielomianu BLM/Ho.
    \index{wielomian!BLM/Ho}
\end{proposition}

\begin{proof}
    \cite{tanaka09}, choć nie wiem gdzie dokładnie.
\end{proof}

\begin{proposition}
    Mutacja węzła nie zmienia sygnatury Tristrama-Levine'a.
    \index{sygnatura}
    \index{sygnatura!Tristrama-Levine'a}
\end{proposition}

\begin{proof}
    \cite{cooper99}.
\end{proof}

\begin{proposition}
    \label{mutants_the_same_volume}
    Mutacja węzła nie zmienia symplicjalnej objętości Gromowa.
    \index{objętość!Gromowa}
\end{proposition}

\begin{proof}
    \cite{ruberman87}.
    % Ruberman [42] showed that mutants have equal volume in all hyperbolic pieces of the JSJ decomposition.
\end{proof}

\begin{proposition}
    Mutacja węzła nie zmienia instanton homologii Floera.
    \index{homologia!Floera}
\end{proposition}

\begin{proof}
    \cite{ruberman99}.
\end{proof}

\begin{proposition}
    Mutacja węzła nie zmienia niezmienników Wittena.
    \index{niezmiennik!Wittena}
\end{proposition}

\begin{proof}
    \cite{rong94}.
\end{proof}

\begin{proposition}
    Mutacja węzła nie zmienia niezmienników Cassona.
    \index{niezmiennik!Cassona}
\end{proposition}

\begin{proof}
    \cite{kirk89}.
\end{proof}

Warto przytoczyć teraz obserwację 3.8.2 z \cite[s. 43]{kawauchi96}: jeśli sploty $L_1, L_2$ są mutantami, to podwójne przestrzenie nakryciowe nad $S^3$ rozgałęzione odpowiednio wzdłuż $L_1$ oraz $L_2$ są homeomorficzne z zachowaniem orientacji.
\index{przestrzeń nakryciowa}
Co więcej, macierze Seiferta mutantów są $S$-równoważne.
\index{macierz!Seiferta}
To tłumaczy czemu większość niezmienników nie radzi sobie z odróżnianiem mutantów.
% Viro: Two-fold branched coverings of three-sphere

Niedawno Stojmenow podjął się systematycznie szukania mutantów wśród węzłów o~mniej niż 19 skrzyżowaniach (praca \cite{stoimenow10} z~2010 roku).
Początkowo pracował sam, badając pewne subtelne przykłady postanowił uwikłać w swój projekt Toshifumiego Tanakę, a później także Daniela Mateię.
Praca \cite{stoimenow10} jest kontynuacją artykułu, który napisali wspólnymi siłami.

I tak na stronie 531 można przeczytać, że ,,niezmienniki Wasiljewa co najwyżej 8. stopnia nie rozróżniają mutantów węzłów \cite{chmutov94}, ja tego nie widzę.
\index{niezmiennik!Wasiljewa}
Mniej więcej sześć lat później wynik poprawił J. Murakami (nie mylić z H. Murakamim!) do 10. stopnia w~niezindeksowanej pracy \cite{murakami99}.
W międzyczasie Cromwell, Morton znaleźli niezmiennik stopnia 11., który odróżnia węzły Conwaya oraz Kinoshity-Terasakiego; patrz \cite{cromwell96}.
% czy Murakami potwierdził wynik Cromwella, Mortona?

Mutant węzła złożonego także jest złożony, co więcej istnieje bijekcja między czynnikami w ich rozkładach na węzły pierwsze \cite{ruberman87}.
Dzięki temu możemy bez straty ogólności założyć, że badamy tylko węzły pierwsze, niestety wciąż nie jest znana ogólna procedura pozwalająca wyliczyć wszystkie mutanty danego węzła.

Zbiór problemów niskowymiarowej topologii opublikowany przez Kirby'iego \cite{kirby78} zawiera następujące pytanie:
\begin{conjecture}[problem 1.91]
    Niech $K$ będzie prostym\footnote{simple} węzłem bez orientacji.
    Czy istnieją węzły niebędące mutantami $K$, których nie można odróżnić od $K$ wielomianem Jonesa oraz wszystkimi jego satelitami?
\end{conjecture}

Stojmenow pisze, że tak: pierwszą chronologicznie parą jest $14_{41721}$, $14_{42125}$, dowód tego faktu opiera się na wzorze fuzyjnym Masbauma-Vogela odkrytym w pracy \cite{masbaum94}.
% fusion formula
Choć wzór ten zastosowany do konkretnej pary węzłów sprawia zazwyczaj trudności rachunkowe, to jest wystarczającym narzędziem, by rozszerzyć konstrukcję do ogólnego wyniku:

\begin{proposition}
    Istnieje nieskończenie wiele par prostych węzłów hiperbolicznych o tych samych kolorowych wielomianach Jonesa, które nie są swoimi mutantami.
\end{proposition}

\begin{proof}
    \cite{tanaka09}.
    Lorem ipsum dolor sit amet, consectetur adipiscing elit.
    Mauris condimentum leo sit amet venenatis elementum.
    Sed congue ligula ut lorem dictum eleifend.
    Maecenas sem elit, sodales ac massa sed, semper dapibus sem.
    Maecenas vel porta mi.
    Donec venenatis tellus nec tellus rhoncus malesuada.
    Nam gravida eu nisi pretium eleifend.
    Nulla faucibus tincidunt nunc in semper.
    Cras ut blandit magna, eget blandit turpis.
    Vivamus aliquam quam id velit iaculis mattis.
    Ut a posuere orci.
    Vestibulum ante ipsum primis in faucibus orci luctus et ultrices posuere cubilia curae; Vestibulum in ligula quis ex efficitur cursus.
\end{proof}

Zaraz po rewolucji, jaką w latach 80. wywołała relacja kłębiasta, Ewing napisał z~Millettem komputerowy program w~języku C, który wyjątkowo szybko znajdował wielomiany HOMFLY oraz Kauffmana zadanego węzła.
Nawet dziś program ten jest w stanie uporać się z węzłami, z którymi nie radzą sobie inne narzędzia.
Autorzy nie wiedzieli wtedy, że ktoś jeszcze będzie z nich korzystać w przyszłości, dlatego poczynili w kodzie liczne optymalizacje dla stacji roboczej Sun, jaką wtedy dysponowali.
Dzisiaj okazuje się, że dla węzłów o większej liczbie skrzyżowań program często kończy swoje działanie zrzutem pamięci, wpada w pętlę bez wyjścia albo zwraca niepoprawny wynik (składniki wielomianu Kauffmana są postaci $a^m z^n$, gdzie $m + n$ jest nieparzyste).
Stojmenow korzystał z tych programów podczas tablicowania mutantów.
Jak postępował?
\begin{enumerate}
    \item podzielił węzły na grupy o tej samej objętości, wielomianie Jonesa oraz Alexandera;
    \item w każdej z grup szukał ciągu mutacji pomiędzy diagramami minimalnymi;
    \item tam, gdzie nie udało się znaleźć mutantów, liczył 2-kablowy wielomian HOMFLY;
    \item jeśli wielomian był taki sam, szukał ciągu mutacji między nieminimalnymi diagramami do 18 skrzyżowań;
    \item wreszcie pozostałe grupy zostały potraktowane reprezentacjami grupy podstawowej dwukrotnego nakrycia.
\end{enumerate}

Podsumowanie jego pracy zawiera tabela:

\begin{table}[h]
    \centering
    \begin{tabular}{lccccc} \toprule
        skrzyżowania & 11 & 12 & 13  & 14   & 15    \\ \midrule
        pary         & 16 & 70 & 703 & 3917 & 24884 \\
        trójki       &    & 5  & 38  & 233  & 1000  \\
        czwórki      &    &    & 32  & 262  & 2909  \\
        szóstki      &    &    & 1   & 17   & 172   \\
        ósemki       &    &    &     & 6    & 84    \\
        łącznie      & 16 & 75 & 774 & 4435 & 29049 \\
        \bottomrule
        \hline
    \end{tabular}
    \caption{Liczba grup mutantów wśród pierwszych węzłów do 15 skrzyżowań}
\end{table}

% W swojej pracy poruszył jeszcze temat symetrycznych mutantów, my nie poświęcimy temu zagadnieniu miejsca.
Reszta wyłożonego materiału nie pochodzi już z pracy \cite{stoimenow10}.

\begin{proposition}
    Niech $D$ będzie alternującym diagramem.
    Wtedy każdy mutant $D$ też jest alternujący.
\end{proposition}

\begin{conjecture}
    Mutacja nie zmienia liczby gordyjskiej.
\end{conjecture}

Jak czytamy w \cite[problem 1.69]{kirby78}, przypuszczenie to jest bardzo trudne do udowodnienia: wynika z niego inna stara hipoteza teorii węzłów, że liczba gordyjska splotów jest addytywna.
Przytoczymy tylko dwa częściowe wyniki.
Najpierw Rolfsen zauważył, że jedynym mutantem niewęzła jest sam niewęzeł \cite{rolfsen93}.
Dekadę później Gordon, Luecke pokazali, iż klasa węzłów $1$-gordyjskich jest zamknięta na przeprowadzanie mutacji \cite{gordon06}.

\begin{proposition}
    Niech $m, n$ będą nieujemnymi liczbami całkowitymi.
    Wtedy istnieje węzeł $K$ o genusie plastrowym równym $m$, którego pewien mutant ma genus plastrowy równy $n$.
\end{proposition}

Stanowi to uogólnienie obserwacji Livingstona \cite{livingston83}, że istnieją mutanty o~różnym genusie plastrowym.

\begin{proof}
    Kim, Livingston w \cite{kim05}.
\end{proof}

