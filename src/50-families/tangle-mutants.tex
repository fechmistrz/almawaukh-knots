\subsection{Mutanty i mutacje}
\label{sec:mutant}
Na zakończenie wspomnimy o~mutacjach.
\begin{definition}[mutacja]
    \label{def:mutant}
    \index{mutacja/mutant}
    Półobrót supła względem osi poziomej, pionowej albo też prostopadłej do płaszczyzny, w~jakiej leży diagram, nazywamy mutacją, zaś otrzymany tak splot -- mutantem.
    W razie potrzeby zmieniamy orientację supła na przeciwną.
\end{definition}

Mutacja węzła o~co najwyżej dziesięciu skrzyżowaniach nie zmienia jego klasy abstrakcji.
Najsłynniejszą parę mutantów stanowią węzeł Conwaya $11n_{34}$ oraz Kinoshity-Terasakiego $11n_{42}$.
\index{węzeł!Conwaya}
\index{węzeł!Kinoshity-Terasakiego}
Conway zauważył podczas klasyfikacji niealternujących węzłów, że tylko one posiadają trywialny wielomian Alexandera.
Mają też taki sam wielomian Jonesa,
\begin{equation}
    \jones(t) = t^{6} -2t^5 +2t^4 -2t^3 +t^2 +2t^{-1} -2t^{-2} +2t^{-3} -t^{-4}.
\end{equation}
Kinoshita, Terasaki zdefiniowali nieskończoną rodzinę węzłów o trywialnym wielomianie Alexandera, której pierwszym wyrazem jest węzeł $11n_{42}$ (w~\cite{kinoshita57}).
Dowód tego, że $11n_{34}$ oraz $11n_{42}$ są różne, jako pierwszy podał prawdopodobnie Riley w~1971 roku \cite{riley71}: wykorzystał on homomorfizmy z~grupy węzła w~$PSL(2, 7)$.
Genusy, odpowiednio: $3$ i~$2$, wyznaczył Gabai piętnaście lat później w~\cite{gabai86}, używał foliacji.

Niedawno Stojmenow podjął się systematycznie znalezienia mutantów wśród węzłów o~mniej niż 19 skrzyżowaniach (praca \cite{stoimenow10} z~2010 roku).
Twierdzi, że w~\cite{chmutov94} pokazano, że ,,niezmienniki Wasiljewa stopnia co najwyżej 8. nie rozróżniają mutantów''; ja tego nie widzę.
\index{niezmiennik!Wasiljewa}
Wynik poprawił Murakami sześć lat później do 10. stopnia w~dostępnej na swojej stronie internetowej pracy ,,Finite type invariants detecting the mutant knots''.
Potwierdził też, że pewien niezmiennik stopnia 11. używany przez Mortona i~Cromwella odróżnia węzeł Conwaya od węzła Kinoshity-Terasakiego.

Mutanty nie dają się łatwo odróżniać niezmiennikami.

\begin{proposition}
    Mutacja węzłów nie zmienia następujących niezmienników:
    kablowego wielomianu Jonesa, % menasco91
    2-kablowego wielomianu HOMFLY, % przytycki89
    kablowego wielomianu Kauffmana, % lipson87
    \index{wielomian!kablowy}
    sygnatury Tristrama-Levine'a, % cooper99
    \index{sygnatura}
    symplicjalnej objętości Gromowa, % ruberman87
    \index{objętość!Gromowa}
    instanton homologii Floera, % ruberman99
    \index{homologia!Floera}
    niezmienników Wittena % rong94
    \index{niezmiennik!Wittena}
    ani Cassona. % kirk89
    \index{niezmiennik!Cassona}
\end{proposition}

\begin{proof}
    Prace \cite{menasco91}, \cite{przytycki89}, \cite{lipson87}, \cite{cooper99}, \cite{ruberman87}, \cite{ruberman99}, \cite{rong94} oraz \cite{kirk89}.
\end{proof}

Jeśli wyjściowy diagram był alternujący, to mutant też jest alternujący.
Istnieje podejrzenie, że mutacja nie zmienia liczby gordyjskiej.
Gordon i~Luecke w~2006 pokazali to dla klasy węzłów $1$-gordyjskich (\cite{gordon06}), dużo wcześniej wiedzieliśmy tylko, że jedynym mutantem niewęzła jest niewęzeł (Rolfsen w~\cite{rolfsen93}?).

\begin{proposition}
    Niech $m, n$ będą nieujemnymi liczbami całkowitymi.
    Wtedy istnieje węzeł $K$ o genusie plastrowym równym $m$, którego pewien mutant ma genus plastrowy równy $n$.
\end{proposition}

\begin{proof}
    Kim, Livingston w \cite{kim05}.
    Wcześniej Livingston pokazał istnienie mutantów o~różnym genusie plastrowym (\cite{livingston83}).
\end{proof}

\begin{tobedone}
The following observation gives evidence of difficulty in distinguishing between a link and its Conway mutant (cf. [Viro 1977]):
Proposition 3.8.2 If L' is a Conway mutant of a link L, then the double covering spaces over 53 with branch sets Land L' are orientation-preservingly homeomor- phic.

We also observe here another fact in [Cooper 1982] which you can understand after you know the facts in Chapter 5. Namely, the Seifert matrices of any knot and its Conway mutant are S-equivalent, so that their Alexander polynomials and their signatures are equal, respectively.
\end{tobedone}
