
\subsection{Warkocze a sploty}
% DICTIONARY;closure of ...;domknięcie ...;warkocz
Każdy warkocz można domknąć do węzła, łącząc ze sobą punkty $(x_i, 1)$ oraz $(x_i, 0)$ łamanymi, których rzuty do płaszczyzny diagramu nie przecinają się.
\index{warkocz!domknięcie warkocza}%
Nie wiadomo, kto wymyślił operację domykania warkoczy, ale była ona z pewnością znana Alexanderowi: rozpatrywano je jeszcze przed samymi warkoczami.
\index[persons]{Alexander, James}%
% TODO: rysunek w TikZ, jak się domyka.

\begin{theorem}[Alexander, 1923]
\index{twierdzenie!Alexandera}%
\label{thm:alexander}%
    Każdy splot powstaje przez domknięcie pewnego warkocza.
\end{theorem}

\begin{proof}[Niedowód]
    W kolejności chronologicznej:
    najpierw Alexander \cite{alexander23},
\index[persons]{Alexander, James}%
    a~po blisko połowie wieku Morton \cite{mortonhr86},
\index[persons]{Morton, Hugh}%
    Yamada \cite{yamada87} (co daje łatwy do zaimplementowania program komputerowy)
\index[persons]{Yamada, Shuji}%
    i~Vogel \cite{vogel90} (ulepszający algorym Yamady).
\index[persons]{Vogel, Pierre}%
\end{proof}

Manturow \cite{manturov02} pokazał, że od warkocza można wymagać kwazitoryczności (warkocz nazywamy torycznym, jeżeli jest postaci $(\sigma_1 \ldots \sigma_{p-1})^q$ oraz kwazitorycznym, jeżeli powstaje przez odwrócenie niektórych skrzyżowań z~warkocza torycznego).
\index[persons]{Manturow, Wasilij}%
\index{warkocz!toryczny}%
\index{warkocz!kwazitoryczny}%

\begin{theorem}[Markow, 1936]
\index{twierdzenie!Markowa}%
    Sploty powstałe przez domknięcie dwóch warkoczy są takie same wtedy i~tylko wtedy, gdy jeden warkocz powstaje z~drugiego przez ciąg sprzężeń: $z_1 \mapsto z_2 z_1 z_2^{-1}$ oraz procesów Markowa, które zastępują $n$-warkocz $\beta$ przez $(n+1)$-warkocz $\beta\sigma_n^{\pm 1}$.
\end{theorem}

\begin{proof}[Niedowód]
    Kompletny i~godny naśladowania dowód znajduje się w~trudno dostępnej książce Birman \cite{birman74}, więc warto sprawdzić inne, opublikowane później materiały:
\index[persons]{Birman, Joan}%
    Morton \cite{mortonhr86} opisał przepiękną, a~przy tym elementarną ideę ,,nitkowania'',
\index[persons]{Morton, Hugh}%
    potem Traczyk \cite{traczyk98} podał czysto kombinatoryczne, dwuwymiarowe uzasadnienie oparte o~okręgi Seiferta,
\index[persons]{Traczyk, Paweł}%
    wreszcie mamy też nowe geometryczne rozumowanie Birman, Menasco \cite{birman02} wykorzystujące foliacje.
\index[persons]{Menasco, William}%
\index{foliacja}%
\end{proof}

Pierwsze sformułowanie twierdzenia pochodzące od Markowa \cite{markov36} korzystało z~trzech ruchów, jeden z~nich stanowił uogólnienie II ruchu Reidemeistera.
\index[persons]{Markow, Andrej (Марков, Андрей Андреевич)}%
Weinberg \cite{weinberg39} zauważył trzy lata później, że wystarczą dwa ruchy.
\index[persons]{Weinberg, Noah (Вайнберг, Ной Моисеевич)}%
% Weinberg = Ной Вайнберг: http://www.mathsoc.spb.ru/history/Odynec_2020.pdf
Lambropoulou, Rourke \cite{lambropoulou97} przedstawili wersję twierdzenia wymagającą tylko jednego ,,L-ruchu''.
\index[persons]{Lambropoulou, Sofia}%
\index[persons]{Rourke, Colin}%

Historia twierdzenia Markowa jest raczej dramatyczna: Markow przedstawił swój dowód ustnie, ale nigdy go nie opublikował, zostawiając to zadanie swojemu uczniowi, Weinbergowi.
Ten jednak został zabity podczas wojny, wkrótce po opublikowaniu pierwszej pracy na temat teorii węzłów i~na dokładny dowód trzeba było czekać na Birman \cite{birman74} blisko 40 lat.
\index[persons]{Birman, Joan}%

Nie znamy wielu twierdzeń o warkoczach i splotach, jakie powstają przez ich domykanie.
Jones \cite{jones85} podał bez dowodu:

\begin{proposition}
    Niech $b \in B_n$ będzie słowem zapisanym na standardowych generatorach.
    Oznaczmy przez $b_+$, $b_-$ nieznakowaną sumę dodatnich, ujemnych wykładników.
    Jeśli $b_+ - 3b_- \ge n$, to domknięcie warkocza $b$ nie jest achiralne.
\index{węzeł!achiralny}%
\end{proposition}

