
\subsection{Węzły okresowe}
\index{węzeł!okresowy|(}%
Można wyróżnić jeszcze jeden rodzaj symetrii.

% DICTIONARY;period;okres;-
% DICTIONARY;periodic;okresowy;węzeł
\begin{definition}
\label{def:period}%
    Węzeł $K$ nazywamy $n$-okresowym, jeśli istnieje obrót $f \colon \R^3 \to \R^3$ o~kąt $2\pi/n$ wokół pewnej prostej $l$, rozłącznej z~węzłem $K$, taki że $f(K) = K$.
\end{definition}

Zamiast obrotów można rozpatrywać dowolne odwzorowania okresowe $f \colon S^3 \to S^3$, których zbiór punktów stałych jest rozłączny z węzłem $K$, homeomorficzny z $S^1$ oraz które trzymają węzeł $K$ w miejscu, ale dostaje się wtedy dokładnie taką samą klasę węzłów.
Czemu?
Wynika to z hipotezy Smitha, otrzymanej z połączenia głębokich teorii dotyczących geometrii i topologii 3-rozmaitości.
\index{hipoteza!Smitha}%
Kawauchi \cite[s. 125]{kawauchi1996} odsyła tu do książki Morgana, Bassa \cite{morgan1984}, gdzie znajdziemy problem, jego historię i rozwiązanie.
\index[persons]{Morgan, John}%
\index[persons]{Bass, Hyman}%

\begin{proposition}
    Zbiór wszystkich okresów jest niezmiennikiem węzłów.
\end{proposition}

Nieodwracalny węzeł $8_{17}$ nie posiada żadnych okresów.
% ćwiczenie 10.1.5 w Kawauchi
Węzeł $5_1$ jest 5-okresowy, co widać na standardowym diagramie, oraz 2-okresowy, tę drugą symetrię można dostrzec na diagramie realizującym liczbę mostową.
Trójlistnik ma dokładnie dwa okresy, $2$ i~$3$.
Ogólniej, jak głosi Kawauchi \cite[ćwiczenie 10.1.9]{kawauchi1996}:

\begin{proposition}
    Jedynymi okresami węzła $(p, q)$-torusowego są dzielniki liczb $p$ oraz $q$.
\end{proposition}

Murasugi podał dwa warunki, które musi spełniać węzeł o~okresie $n = p^r$, gdzie $r$ jest liczbą pierwszą.
Do ich zrozumienia potrzebujemy dwóch definicji.

\begin{definition}
    Niech $f$ będzie obrotem z definicji \ref{def:period}, zaś $p \colon \R^3 \to \R^3/f \simeq \R^3$ rzutem na przestrzeń ilorazową.
% DICTIONARY;quotient;ilorazowy;węzeł
\index{węzeł!ilorazowy}%
    Wtedy $p(K)$ nazywamy \emph{węzłem ilorazowym}, zaś $K$ to jego $n$-krotne nakrycie.    
\end{definition}

Ładny rysunek węzła ilorazowego znaleźliśmy u Kawauchiego \cite[s. 122]{kawauchi1996}.

\begin{definition}
    Niech $K$ będzie zorientowanym węzłem, zaś $l$ zorientowaną półprostą, która nie jest styczna do węzła $K$.
    Wtedy różnicę między liczbą skrzyżowań dodatnich oraz ujemnych wzdłuż półprostej (bez znaku) nazywamy indeksem zaczepienia $\lambda$ węzła $p(K)$.
\end{definition}

\begin{proposition}[warunek Murasugiego]
\index{warunek!Murasugiego}%
\label{prp:murasugi_periodic}%
    Niech $K$ będzie węzłem o~okresie $n = p^r$, gdzie $p$ jest liczbą pierwszą.
    Niech $J$ będzie jego węzłem ilorazowym, z~indeksem zaczepienia $\lambda$.
    Wtedy wielomian $\alexander_J$ jest dzielnikiem wielomianu $\alexander_K$ oraz istnieje pewna całkowita liczba $k$, taka że
    \begin{equation}
        \alexander_K(t) \equiv \pm t^k \alexander_J(n)^n \left(1 + t + t^2 + \ldots + t^{\lambda - 1}\right)^{n-1} \mod p.
    \end{equation}
\end{proposition}

\begin{proof}[Niedowód]
    Mozolne operacje na macierzach, których wyznacznikiem jest wielomian Alexandera, patrz \cite{murasugi1971}.
    Kawauchi przedstawia inny dowód: najpierw dowodzi tego dla węzła torusowego $T_{n, d}$, którego węzłem ilorazowym jest niewęzeł.
    W ogólnym przypadku, korzysta z relacji kłębiastej dla wielomianu Conwaya.
    Szczegóły oraz odsyłacze do dalszych prac znaleźć można w jego przeglądowej publikacji \cite[s. 122-124]{kawauchi1996}.
\end{proof}

Z prac Borodzika (m.in. \cite{grabowski20} napisanej z Grabowskim, Królem i Marchwicką) dowiedzieliśmy się, że podany wyżej warunek Murasugiego jest tylko jednym z wielu ograniczeń, jakie musi spełniać węzeł okresowy.
Autorzy wymieniają:
\begin{itemize}
    \item warunek Murasugiego udoskonalony przez Davisa, Livingstona \cite{davis1991},
\index[persons]{Davis, James}%
\index[persons]{Livingston, Charles}%
    \item kryterium Naika z homologiami rozgałęzionego nakrycia \cite{naik1997},
\index[persons]{Naik, Swatee}%
\index{homologie}%
\index{rozgałęzione nakrycie}%
    \item kryterium Traczyka z wielomianem Jonesa \cite{traczyk1991},
\index{wielomian Jonesa}%
\index[persons]{Traczyk, Paweł}%
    \item kryterium Przytyckiego z wielomianem HOMFLY-PT \cite{przytyckij1989},
\index{wielomian HOMFLY-PT}%
\index[persons]{Przytycki, Józef}%
    \item kryterium Naika z niezmiennikiem Cassona-Gordona,
\index{niezmiennik Cassona-Gordona}%
\index[persons]{Naik, Swatee}%
    \item kryterium Hillmana, Livingstona, Naika ze skręconym wielomianem Alexandera \cite{hillman2006},
\index{wielomian Alexandera!skręcony}%
\index[persons]{Hillman, Jonathan}%
\index[persons]{Livingston, Charles}%
\index[persons]{Naik, Swatee}%
    \item kryterium Jabuki, Naika z homologią Floera \cite{jabuka2016},
\index{homologia Floera}%
\index[persons]{Jabuka, Stanisław}%
\index[persons]{Naik, Swatee}%
    \item kryteria Borodzika, Politarczyka z homologią Chowanowa, \cite{politarczyk2017}, \cite{politarczyk2021},
\index{homologia Chowanowa}%
\index[persons]{Borodzik, Maciej}%
\index[persons]{Politarczyk, Wojciech}%
    \item kryterium Chena \cite{chen2018} z grupą podstawową.
\index{grupa podstawowa}%
\index[persons]{Chen, Haimiao}%
\end{itemize}
% Sakumy nie ma, bo nie został wymieniony ze słowem criterion; być może napisał coś ogólnego? 
\index{kryterium Naika (okresowości)}%

\index{węzeł!okresowy|)}%

% koniec podsekcji Węzły okresowe

