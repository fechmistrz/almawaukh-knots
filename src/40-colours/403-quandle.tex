
\section{Kwandle i wraki}
\index{kwandel|(}%

Sekcja ta powstała częściowo w~oparciu o~notatki autorstwa Bergera, Geriga\footnote{dostępne pod adresem \url{https://scholar.harvard.edu/files/gerig/files/knotnotes.pdf}} oraz Bergera, Flannery'ego i~Sumnichta\footnote{dostępne pod adresem \url{https://github.com/thyrgle/191_Final_Project/blob/master/paper.pdf}}; ale najlepiej zacząć od długiej na trzy strony odpowiedzi Nelsona \cite{nelson2016} na pytanie ,,czym jest kwandel?''.
\index[persons]{Berger, Andrew}%
\index[persons]{Gerig, Chris}%
\index[persons]{Flannery, Brandon}%
\index[persons]{Sumnicht, Christopher}%
\index[persons]{Nelson, Sam}%
Należy przy tym pamiętać, że jest to dość niszowe zagadnienie, na przykład Burde, Zieschang, Heusener \cite[s. 11]{burde2014} poświęcają mu pięć wierszy i~odsyłają Czytelnika do pracy Fenna, Rourkego \cite{fenn1992}.
\index[persons]{Fenn, Roger}%
\index[persons]{Rourke, Colin}%
Dużo otwartych problemów dotyczących kwandli można znaleźć u~Ohtsukiego \cite[s. 455-465]{ohtsuki2002}.

Kwandle pierwszy raz odkryli Conway z Wraithem około 1959 roku, jeszcze jako studenci I stopnia na uniwersytecie w Cambridge, w nieopublikowanej korespondencji między sobą (patrz definicja przed definicją \ref{def:quandle}).
\index[persons]{Conway, John}%
\index[persons]{Wraith, Gaven}%
Ponownie uczyniono to latach 80. XX wieku: Joyce w~1982 jako kwandle, Matwiejow w tym samym roku pod nazwą grupoidy rozdzielne, wreszcie Brieskorn w 1986 jako zbiory automorficzne.
\index[persons]{Joyce, David}%
\index[persons]{Matwiejow, Siergiej (Матвеев, Сергей Владимирович)}%
\index[persons]{Brieskorn, Egbert}%
\index{grupoid rozdzielny|see {kwandel}}%
\index{zbiór automorficzny|see {kwandel}}%
Joyce zapytany o znaczenie słowa \textsc{quandle} odparł \emph{,,I needed a usable word. Distributive algebra had too many syllables. Piffle was already taken. I tried trindle and quagle, but they didn’t seem right, so I went with quandle''}.\footnote{Źródło: blog Johna Baeza \url{https://golem.ph.utexas.edu/category/2015/05/the_origin_of_the_word_quandle.html}}
\index[persons]{Joyce, David}%

Conway nazwał wraki wrakami (\textsc{wracks}), by częściowo zażartować z~nazwiska jego kolegi Wraitha, a częściowo by zaznaczyć, że są one tym, co zostaje z~grupy, w~której zapomniano o~mnożeniu, ale nie sprzęganiu (w~języku angielskim co najmniej od XVI wieku funkcjonuje zwrot ,,\textsc{wrack and ruin}'' oznaczający zniszczenie).
\index[persons]{Conway, John}%
\index[persons]{Wraith, Gaven}%
Obecnie dominuje określenie \textsc{racks}.

\subsection{Kwandle}
Aksjomaty grupy stanowią uogólnienie symetrii, ponieważ składanie symetrii jest łączne, identyczność jest symetrią, funkcja odwrotna do symetrii jest symetrią.
Znajdziemy teraz słabo znaną strukturę algebraiczną, która odzwierciedla ruchy Reidemeistera (oraz aksjomatyzuje własności sprzężeń w grupach, choć jest to dla nas mniej interesujące).
\index{ruch!Reidemeistera}%

Niech $X$ będzie skończonym zbiorem kolorów wyposażonym w~działanie $\triangleright \colon X \times X \to X$.
Będziemy rysować diagramy węzłów tak, że każdemu łukowi przypisany będzie pewien kolor $x \in X$ zgodnie z następującą regułą (patrz rysunek): kiedy łuk o kolorze $x$ przechodzi pod biegnącym w prawo łukiem koloru $y$, staje się łukiem w kolorze $x \triangleright y$.

\begin{comment}
\[
    \LargeMinusCrossingQuandle
\]
\end{comment}

Formalnie:

% DICTIONARY;quandle;kwandel;-
\begin{definition}[kwandel]
\label{def:quandle}%
\index{kwandel}%
    Zbiór $X$ wyposażony w dwuargumentowe działanie $\triangleright$, które dla wszystkich elementów $x, y, z \in X$ spełnia trzy warunki:
    \begin{enumerate}
        \item $x \triangleright x = x$,
        \item odwzorowanie $\beta_y \colon X \to X$ dane wzorem $\beta_y(x) = x \triangleright y$ jest odwracalne,
        \item $(x \triangleright y) \triangleright z = (x \triangleright z) \triangleright (y \triangleright z)$,
    \end{enumerate}
    nazywamy kwandlem.
\end{definition}

Drugi aksjomat nazywa się czasem odwracalnością z prawej strony: znając $x \triangleright y$ oraz $y$ możemy odtworzyć element $x$, jednak znając $x$ być może nie jesteśmy w stanie odtworzyć elementu $y$.
Jedyny element $x$ taki, że $x \triangleright y = z$ nazwijmy $y \triangleleft z$.
To pozwala podać trochę inną definicję kwandli, którą przytaczamy jako ciekawostkę.

\begin{definition}
    Zbiór $X$ z dwuargumentowymi działaniami $\triangleright, \triangleleft$ taki, że dla wszystkich $x, y, z \in X$ zachodzi:
    \begin{align}
    x \triangleleft x & = x \\
    x \triangleright x & = x \\
    (x \triangleleft y) \triangleright x & = y \\
    x \triangleleft (y \triangleright x) & = y \\
     (x \triangleright z) \triangleright (y \triangleright z) & = (x \triangleright y) \triangleright z \\
    (x \triangleleft y) \triangleleft (x \triangleleft z) & = x \triangleleft (y \triangleleft z)
    \end{align}
    nazywamy kwandlem.
\end{definition}

Pokażemy teraz, czemu aksjomaty kwandli można traktować jako przetłumaczenie ruchów Reidemeistera na język algebry.
\index{ruch!Reidemeistera}%

\begin{proposition}[niezmiennik zliczający]
\index{niezmiennik!zliczający}%
    Niech $X$ będzie skończonym kwandlem.
    Liczba etykietowań diagramu elementami kwandla $X$ jest niezmiennikiem węzłów, zwanym niezmiennikiem zliczającym.
\end{proposition}

\begin{proof}
    Musimy pokazać, że etykiety na diagramiem przed każdym ruchem Reidemeistera wyznaczają jednoznacznie układ etykiet po tym ruchu.
    Pierwszy i drugi ruch:
\begin{comment}
    \begin{figure}[H]
        \begin{minipage}[b]{.48\linewidth}
        \[
            \MedLarReidemeisterOneRightQuandleProof
            \stackrel{R_1}{\cong}
            \MedLarReidemeisterOneStraightQuandleProof
        \]
        \end{minipage}
        \begin{minipage}[b]{.48\linewidth}
        \[
            \MedLarReidemeisterTwoQuandleA \cong \MedLarReidemeisterTwoQuandleB
        \]
        \end{minipage}
    \end{figure}
\end{comment}
\noindent
Trzeci ruch:
\begin{comment}
    \[
        \LargeReidemeisterThreeQuandleA \cong \LargeReidemeisterThreeQuandleB \qedhere
    \]
\end{comment}
\noindent
Jak widać, przy trzecim ruchu coś poszło nie tak, ale to się zateguje później.
\end{proof}

Wiele znanych struktur algebraicznych okazuje się być źródłem kwandli:

\begin{example}[kwandel cykliczny/diedralny]
\index{kwandel!cykliczny}%
\index{kwandel!diedralny}%
    Grupa abelowa z działaniem $x \triangleright y = 2y - x$.
    (Nazwa diedralny bierze się stąd, że zbiór odbić w grupie diedralnej stanowi przykład takiego kwandla).
    % wzięte z https://www.ams.org/journals/notices/201604/rnoti-p378.pdf Nelsona
    (Lepiej nie pytać, skąd wziął się przymiotnik cykliczny).
\end{example}

\begin{example}[kwandle sprzężone]
\index{kwandel!sprzężony}%
    Grupa $G$ z działaniem $x \triangleright y = y^{-n} x y^n$ dla każdej naturalnej wartości $n$.
\end{example}

\begin{example}[kwandel Alexandera]
\index{kwandel!Alexandera}%
    Moduł $M$ nad pierścieniem $\Z[t, 1/t]$ wielomianów Laurenta z~działaniem $x \triangleright y = tx + (1-t) y$.
\end{example}

Dionísio, Lopes \cite[s. 1043]{lopes2003} potrzebują skończonych kwandli Alexandera, mają postać $(\Z/n\Z)[t, 1/t] / h(t)$, gdzie $h$ jest pewnym unormowanym wielomianem.

\begin{example}[kwandel symplektyczny]
\index{kwandel!symplektyczny}%
    Przestrzeń liniowa i antysymetryczna forma dwuliniowa $\langle \bullet | \bullet \rangle$ z działaniem $x \triangleright y = x + \langle x | y \rangle \cdot y$.
\end{example}

Żeby opowiedzieć, jak wyglądały poszukiwania małych kwandli, musimy wprowadzić dwie definicje.
(Potem można o nich zapomnieć).

\begin{definition}[homomorfizm kwandli]
    Niech $Q_1, Q_2$ będą kwandlami.
    Odwzorowanie $f \colon Q_1 \to Q_2$ spełniające warunek
    \begin{equation}
        \forall x, y \in Q_1 : f(x \triangleright y) = f(x) \triangleright f(y),
    \end{equation}
    nazywamy homomorfizmem kwandli.
\end{definition}

\begin{definition}[kwandel spójny]
\index{kwandel!spójny}%
    Niech $Q$ będzie kwandlem, zaś $G$ podgrupą wszystkich automorfizmów $Q$ generowaną przez automorfizmy wewnętrzne $\beta_y(x) = x \triangleright y$.
    Jeżeli działanie podgrupy $G$ na $Q$ jest przechodnie, kwandel $Q$ nazywamy spójnym.
\end{definition}

Dionísio, Lopes \cite{lopes2003} znaleźli 10 kwandli Alexandera, które odróżniają prawie każde dwa spośród 249 węzłów pierwszych do 10 skrzyżowań; rozstrzygnięcia brak dla 962 z 30876 par.
\index[persons]{Dionísio, Miguel}%
\index[persons]{Lopes, Pedro}%
Po upływie prawie dekady Vendramin \cite{vendramin2012} wytropił wszystkie 431 kwandli spójnych rzędu 35 lub mniejszego.
\index[persons]{Vendramin, Leandro}%
Clark, Elhamdadi, Saito oraz Yeatman \cite{clark2013} pokazali niedawno zbiór 26 kwandli, które razem odróżniają od siebie wszystkie 2977 zorientowanych węzłów pierwszych o~co najwyżej 12 skrzyżowaniach.
\index[persons]{Clark, William}%
\index[persons]{Elhamdadi, Mohamed}%
\index[persons]{Saito, Masahico}%
\index[persons]{Yeatman, Timothy}%
Największy z~nich jest rzędu 182.

Joyce w swojej rozprawie doktorskiej przypisał każdemu węzłowi $K$ pewien szczególny kwandel $Q(K)$, kwandel podstawowy.
\index[persons]{Joyce, David}%
\index{kwandel!podstawowy}%
Jest to niezmiennik generowany przez łuki diagramu związany relacjami pochodzącymi ze skrzyżowań, tak jak w~definicji grupy węzła.
Wprawdzie znajomość kwandla $Q(K)$ pozwala odtworzyć węzęł $K$ z dokładnością do jego orientacji, ale jak nietrudno się domyślić, ceną za to jest trudność w~jego wyznaczaniu.

Na przykład Niebrzydowski, Przytycki \cite{niebrzydowski2009} pokazali, że
\index[persons]{Niebrzydowski, Maciej}%
\index[persons]{Przytycki, Józef}%

\begin{example}
    Kwandel podstawowy trójlistnika  jest izomorficzny z~pewnym rzutowym pierwotnym podkwandlem\footnote{Cokolwiek to znaczy!} odwzorowań liniowych przestrzeni symplektycznej $\Z \oplus \Z$
\end{example}

Słabszym, choć prostszym (?) w znalezieniu niezmiennikiem jest liczba homomorfizmów z~kwandla danego węzła w pewien wybrany wcześniej kwandel $Q$.
% wiem to z nLab

\subsection{Półki, wraki i wrzeciona}
Aksjomaty grupy można wzmacniać (grupy abelowe) lub osłabiać (monoidy).
Podobnie czyni się z aksjomatami kwandli.
Kwandle inwolutywne odpowiadają węzłom bez orientacji, wraki dobrze opisują węzły obramowane, i tak dalej.
\index{węzeł!niezorientowany}%
% TODO: w zorientowanym dać patrz też do niezorientowanego?
\index{węzeł!obramowany}%

\begin{definition}[kwandel inwolutywny]
\index{kwandel!inwolutywny}%
\index{kei|see {kwandel inwolutywny}}%
    Kwandel $Q$, w którym dla wszystkich $x, y \in Q$ zachodzi $x \triangleleft (x \triangleleft y) = y$, nazywamy inwolutywnym albo kei.
\end{definition}

Kwandle inwolutywne badał jako pierwszy Takasaki (1943).
\index[persons]{Takasaki, Mituhisa}%
Szukał niełącznej struktury, która dobrze opisywałaby odbicia w skończonej geometrii.

% DICTIONARY;shelf;półka;-
\begin{definition}[półka]
\index{półka}%
    Zbiór $X$ wyposażony w dwuargumentowe działanie $\triangleright$ taki, że dla wszystkich elementów $x, y, z \in X$ zachodzi $(x \triangleright y) \triangleright z = (x \triangleright z) \triangleright (y \triangleright z)$, nazywamy półką.
\end{definition}

\begin{example}
\index{warkocz}%
    Niech $B_\infty$ oznacza kogranicę łańcucha $B_0 \to B_1 \to B_2 \to \ldots$, gdzie inkluzja $B_{n} \to B_{n+1}$ zadana jest przez dołączenie niesplątanego z resztą diagramu pasma, zaś $\phi$ będzie jej endomorfizmem posyłającym generator $\sigma_k$ na $\sigma_{k+1}$.
    Zbiór $B_\infty$ z działaniem $a \triangleleft b = a\phi(b)\sigma_1 \phi{a} ^{-1}$ jest półką.\footnote{Źródło: encyklopedia nLab \url{https://ncatlab.org/nlab/show/shelf}. Uwaga: symbole $\triangleleft, \triangleright$ są tam zamienione znaczeniami!}
\end{example}

To nieprzetłumaczalna gra słów: dwie półki (\emph{shelves}), lewa i prawa, które dobrze do siebie pasują, dają stojak (\emph{rack}, czyli dla nas wrak).
Jak napisała Crans \cite[s. 86]{crans2004}: \emph{,,Just as a rack is comprised of two shelves which fit together nicely, a quandle is made up of two spindles''}.
\index[persons]{Crans, Alissa}%

Półka stanowi uogólnienie dwóch obiektów -- wrzecion i~wraków.

% DICTIONARY;spindle;wrzeciono
\begin{definition}[wrzeciono]
\index{wrzeciono}%
    Zbiór $X$ z dwuargumentowym działaniem $\triangleright$ taki, że dla wszystkich elementów $x, y, z \in X$ zachodzi:
    \begin{enumerate}
        \item $x \triangleright x = x$,
        \item $(x \triangleright y) \triangleright z = (x \triangleright z) \triangleright (y \triangleright z)$
    \end{enumerate}
    nazywamy wrzecionem.
\end{definition}

% DICTIONARY;wrack;wrak
\begin{definition}[wrak]
\index{wrak}%
    Zbiór $X$ z dwuargumentowym działaniem $\triangleright$ takim, że dla każdej trójki elementów $x, y, z \in X$ zachodzi:
    \begin{enumerate}
        \item odwzorowanie $\beta_y \colon X \to X$ dane wzorem $\beta_y(x) = x \triangleright y$ jest odwracalne,
        \item $(x \triangleright y) \triangleright z = (x \triangleright z) \triangleright (y \triangleright z)$
    \end{enumerate}
    nazywamy wrakiem.
\end{definition}

\begin{example}
    % https://www1.cmc.edu/pages/faculty/VNelson/quandles.html
    Zbiór $X = \{1, 2, \ldots, n\}$ z działaniem $x \triangleright y = \sigma(x)$, gdzie $\sigma \in S_n$ jest dowolną permutacją.
\end{example}

\begin{example}
    % https://www1.cmc.edu/pages/faculty/VNelson/quandles.html
    Moduł nad pierścieniem
    \begin{equation}
        \frac{\Z[t^{\pm 1}, s]}{(s^2 - (1-t)s)}
    \end{equation}
    z działaniem $x \triangleright y = tx+sy$.
\end{example}

\index{węzeł!obramowany}% framed?
Wraki są naturalnym niezmiennikiem węzłów obramowanych, bo dobrze współgrają z II, III oraz podwójnym I ruchem Reidemeistera, który to nie zmienia spinu diagramu:
\begin{comment}
\[
    \LargeReidemeisterOneLeftRightQuandleProof
    \cong
    \LargeReidemeisterOneStraightQuandleProofRotated
\]
\end{comment}

\index{kwandel|)}

% Koniec sekcji Kwandle i wraki

