
\section{Etykietowanie}
%\index{etykietowanie|(}%

Dotychczas kolorowaliśmy diagramy węzłów liczbami $0, 1, 2, \ldots, n-1$, czyli elementami grupy $\Z/n\Z$, ale nic nie stoi na przeszkodzie, żeby próbować użyć dowolnej innej skończonej grupy.
Etykietowalność to niezmiennik, który uogólnia kolorowalność.
Poznaliśmy go dzięki książce Livingstona \cite[s. 89-99]{livingston1993}, ale poza panem Charlesem mało kto o nim wspomina!

\begin{definition}[etykietowanie]
    Niech $G$ będzie skończenie generowaną grupą, zaś $D$ diagramem zorientowanego węzła $K$.
    Surjekcję ze zbioru włókien diagramu $D$ na zbiór generatorów grupy $G$ taką, że przy każdym skrzyżowaniu spełnione jest równanie $g_Ug_L = g_Rg_U$ (z oznaczeniami jak na rysunku: włókno $g_U$ biegnie górą; włókna $g_L$, $g_R$ po lewej i prawej stronie podskrzyżowania), nazywamy etykietowaniem diagramu $D$ grupą $G$.
    Węzeł, który posiada diagram etykietowalny grupą $G$ też nazywamy etykietowalnym grupą $G$.
\begin{comment}
        \[
            \LargePlusCrossingLabel
        \]
\end{comment}
\end{definition}

Równanie $g_R = g_Ug_Lg_U^{-1}$ mówi, że etykiety włókien wchodzących oraz wychodzących są sprzężone.
Wynika stąd, że wszystkie etykiety pochodzą z~jednej klasy sprzężoności.
Muszą jednocześnie generować całą grupę, dlatego $G$ musi być grupą nieprzemienną lub trywialną.
Etykietowalność jest niezmiennikiem węzłów i~nie zależy od orientacji węzła: jeżeli elementy $g_1, \ldots, g_n$ generują grupę, to ich odwrotności także.

Etykietowania są naturalnym uogólnieniem kolorowań.
Dlaczego?
% Pokazuję i objaśniam - Małgorzata K.
Niech $p \ge 3$ będzie liczbą pierwszą, natomiast $D_p = \langle r, s \mid r^p = s^2 = e, rsr = s \rangle$ grupą diedralną rzędu $2p$.
Elementy tej grupy to $1, r, r^2, \ldots, r^{p-1}, s, sr, \ldots, sr^{p-1}$.
,,Obrót'' $r^k$ jest sprzężony tylko ze swoją odwrotnością (mamy $sr^lr^k(sr^l)^{-1} = sr^ks = r^{-k}$), ale ,,symetrie osiowe'' $sr^k$ tworzą jedną klasę sprzężoności.
Łatwo widać, że dowolne dwie z~nich generują całą grupę $D_p$.

\begin{proposition}
    Węzeł $K$ jest $p$-kolorowalny wtedy i~tylko wtedy, gdy jest $D_p$-etykietowalny.
\end{proposition}

\begin{proof}
    Załóżmy, że $K$ ma $n$ włókien.
    Wiemy już, że każde $D_p$-etykietowanie wykorzystuje tylko elementy $sr^{a_1}, \ldots, sr^{a_n}$ dla $1 \le a_i \le p$.
    Jest ono prawidłowe dokładnie wtedy, gdy analogiczne kolorowanie liczbami $a_1, \ldots, a_n$ jest prawidłowe.
\end{proof}

Pozostało wskazać parę węzłów, których nie odróżniają kolorowania, ale odróżnia pewne etykietowanie.

\begin{example}
    Węzeł $9_{46}$ jest etykietowalny grupą $S_4$, wystarczy jego włóknom dobrze przypisać transpozycje (niech to będzie proste ćwiczenie dla Czytelnika).
    Węzeł $6_1$ nie jest etykietowalny tą samą grupą.
    Wybranie dwóch etykiet przy jednym skrzyżowaniu wymusza etykiety dla wszystkich włókien, ale dwie grupy nie mogą generować grupy $S_4$.
    Węzły te są więc różne.
\end{example}

Węzły $6_1$ i $9_{46}$ mają ten sam wielomian Alexandera, zatem ten sam wyznacznik, więc na mocy faktu \ref{prp:colour_determinant} te same własności kolorujące.
\index{wielomian Alexandera}%
Dużo później odkryliśmy parę jeszcze raz w~\cite[s. 138]{burde2014}, gdzie pokazano, że  $E_2(6_1) = \Z(t)$; $E_2(9_{46}) = (t-2, 2t-1)$.
Niestety nie wiemy, kto pierwszy wpadł na ten pomysł porównania ich ideałów elementarnych $E_2$. 

Etykietowanie jest mocnym narzędziem odróżniającym węzły.
Thistlethwaite w 1985 roku korzystając z niego klasyfikował węzły o~co najwyżej 13 skrzyżowaniach (jest ich, jak ostatecznie się okazało, 12965).
\index[persons]{Thistlethwaite, Morwen}%
Mają one tylko 5639 różnych wielomianów Alexandera, ale etykietowania trzynastoma różnymi grupami pozwoliły zmniejszyć liczbę nierozpoznanych węzłów do około tysiąca.
Wśród nich 30 posiada wielomian Conwaya $1 + 2z^2 + 2z^4$, ale pary rozróżniane wielomianem HOMFLY mają też różne wielomiany Jonesa.

Kolorowania definiowano kiedyś jako surjekcje $\rho \colon \pi \to D_{2n}$ z~grupy podstawowej.
Jak mówi prezentacja Wirtingera, grupa splotu generowana jest przez ścieżki z~punktu bazowego w~$S^3$ do brzegu rurowego otoczenia splotu, wokół południka i~znowu do bazowego punktu.
\index{prezentacja Wirtingera}%
Fox zauważył, że z~surjektywności $\rho$ wynika, iż generatory mapują się na symetrie osiowe $sr^k$.
Ponieważ istnieje wzajemnie jednoznaczna odpowiedniość między generatorami grupy splotu oraz łukami diagramu, każdemu możemy przypisać liczbę całkowitą $k$.
Etykietowania są więc uogólnieniem kolorowań.
Rozumowanie, które przedstawiliśmy, prowadzi do prostej klasyfikacji grup, których można użyć do etykietowania.

\begin{proposition}
    Niech $K$ będzie węzłem, $\pi$ grupą podstawową jego dopełnienia, zaś $G$ dowolną grupą.
    Następujące warunki są równoważne: $K$ jest $G$-etykietowalny; istnieje surjekcja $\pi_1 \to G$.
\end{proposition}

Historycznie, prezentacja Wirtingera była pierwsza, zaś etykietowania odkryto później.
% TODO: ustalić, kiedy później

\begin{proposition}[Perko]
    Niech $K$ będzie węzłem.
    Następujące warunki są równoważne: $K$ jest etykietowalny grupą $S_3$, $K$ jest etykietowalny grupą $S_4$.
\end{proposition}
% To jest w https://faculty.etsu.edu/gardnerr/Knot-Theory/Notes-Livingston/Livingston-Knot-5-2.pdf, ale nie książce Livingston a livingston1993

\begin{proof}
    Dowód zajmuje tylko kilka stron i składa się z dwóch części.
    Po pierwsze $S_4$ ma podgrupę czwórkową Kleina, więc istnieje epimorfizm $S_4 \to S_3$; pozwala to opuszczać epimorfizmy $G \to S_4$ do epimorfizmów $G \to S_3$.
    
    Jak podnosić epimorfizmy $G \to S_3$ do epimorfizmów $G \to S_4$ pokazał Perko \cite{perko1975}.
\end{proof}

Nie znamy innych nietrywialnych faktów dotyczących etykietowań.

\index{etykietowanie|)}%

