Przez $\N, \Z, \Q, \R, \C$ oznaczamy kolejno zbiór liczb naturalnych (z zerem lub bez, jak kto lubi), całkowitych, wymiernych, rzeczywistych i wreszcie zespolonych.
$S^n$ oznacza $n$-sferę (i tak $S^1$ jest okręgiem, zaś $S^2$ powierzchnią trójwymiarowej kuli); $D^n$ to $n$-dysk.
Włożenie $A$ w $B$ to $A \hookrightarrow B$, ich izotopia to $A \cong B$.
Kratka ($\shrap$) oznacza sumę spójną.
Suma rozłączna $\sqcup$ tym różni się od teoriomnogościowej $\cup$, że składniki muszą być rozłączne. Obraz funkcji $f \colon X \to Y$ staramy się oznaczać $f[X]$, nie $f(X)$.
Lustro i rewers węzła $K$ oznaczamy kolejno $\operatorname{m} K$ i $\operatorname{r} K$.

Dla wielomianowych niezmienników stosujemy następujące symbole: $\alexander$, $\conway$, $\jones$, $P$, $F$, $Q$ dla wielomianu Alexandera, Conwaya, Jonesa, HOMFLY-PT, Kauffmana, BLM/Ho.
Wiele innych niezmienników ma krótkie, jednoliterowe oznaczenia.
I tak $\crossing, \braid, \bridge, \unknotting, \linking, \stick$ to kolejno: liczba/indeks skrzyżowaniowy, warkoczowy, mostowy, gordyjska, zaczepienia, patykowa.
$\sigma$ oznacza sygnaturę (Levine'a-Tristrama), $\det$ wyznacznik, $g, \chi$ genus i charakterystykę Eulera, $\operatorname{wr}$ spin, $\operatorname{lk}$ indeks zaczepienia, 
Rzadko, naprawdę rzadko, potrzebujemy czegoś dla liczby $p$-kolorowań, wybór pada zawsze na $\tau_p$.

% $$PSL(2, 7)$ to rzutowa specjalna grupa liniowa nad ciałem $F_7$.

