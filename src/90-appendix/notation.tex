
\begin{itemize}
    \item $\N \subset \Z \subset \Q \subset \R \subset \C$: zbiór liczb naturalnych, całkowitych, wymiernych, rzeczywistych, zespolonych,
    \item $[a, b] = \{x \in \R \mid a \le x \le b\}$: przedział domknięty,
    \item $S^n = \{x \in \R^{n+1} \mid \|x\| = 1\}$, $n$-sfera
    \item $mK, rK$ lustro i rewers węzła,
    \item $\crossing, \braid, \bridge, \unknotting, \linking, \stick$ liczba/indeks skrzyżowaniowy, warkoczowy, mostowy, gordyjska, zaczepienia, patykowa,
    \item $\det$ wyznacznik,
    \item $\sigma$ sygnatura, być może Levine'a-Tristrama,
    \item $g, \chi$ genus i charakterystyka Eulera,
    \item $\#$ suma spójna,
    \item $\tau_p$ liczba $p$-kolorowań,
    \item $\operatorname{span}$ rozpiętość wielomianu,
    \item $\alexander$, $\conway$, $\jones$, $P$, $F$, $Q$ wielomian Alexandera, Conwaya, Jonesa, HOMFLY-PT, Kauffmana, BLM/Ho.
\end{itemize}

% $$PSL(2, 7)$ to rzutowa specjalna grupa liniowa nad ciałem $F_7$.
