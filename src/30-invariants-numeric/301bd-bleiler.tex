
\subsubsection{Przykład Nakanishiego-Bleilera. Hipoteza Bernharda-Jablana}
Najpierw Nakanishi \cite{nakanishi1983}, a potem Bleiler \cite{bleiler1984} odkryli fascynujący przykład wymiernego węzła $10_8$, który jest $2$-gordyjski, ale świadkiem tego nie może być żaden diagram mininalny, ponieważ, co jeszcze bardziej fascynujące, węzeł ten posiada tylko jeden diagram o~dziesięciu skrzyżowaniach oraz liczbie gordyjskiej 3.
\index[persons]{Bleiler, Steven}%
\index[persons]{Nakanishi, Yasutaka}%
\index{węzeł!10-8}%
Wynika stąd, że liczba $\unknotting$ nie musi być osiągana przez diagram minimalny, wbrew powszechnym przypuszczeniom obecnym jeszcze w latach 70.
Praca \cite{bernhard1994} zawiera indukcyjny dowód faktu, że żaden minimalny diagram węzła oznaczanego w notacji Conwaya przez $C(2m+1, 1, 2m)$ nie daje się rozwiązać w $m$ ruchach, ale pewne nieminimalne diagramy dają się.
Przypadek $m = 2$ odpowiada węzłowi $10_8$.

Przykład Bleilera pokazuje, że do szukania liczby gordyjskiej potrzeba wyrafinowanego algorytmu.
Ponieważ odwrócenie jednego ze skrzyżowań na minimalnym diagramie węzła $10_8$ daje $1$-gordyjski węzeł $4_1, 5_1, 6_1$ lub $6_2$, możemy liczyć, że każdy diagram minimalny ma skrzyżowanie, którego odwrócenie zmniejsza liczbę gordyjską.
Dlatego jeszcze w~latach 90. Bernhard \cite{bernhard1994} i Jablan \cite{jablan1998} postawili hipotezę:

\begin{conjecture}[Bernharda-Jablana]
\index[persons]{Bernhard, James}%
\index[persons]{Jablan, Slavik}%
\index{hipoteza!Bernharda-Jablana}%
\label{con:bernhard_jablan}%
    Niech $K$ będzie węzłem z diagramem $D$, który realizuje liczbę gordyjską $\unknotting K$.
    Istnieje wtedy skrzyżowanie, którego odwrócenie daje nowy diagram $D'$ nowego węzła $K'$ o~mniejszej liczbie gordyjskiej: $1 + \unknotting D' = \unknotting D$.
\end{conjecture}

Przypuszczenie to sprawdzono dla węzłów do jedenastu skrzyżowań oraz splotów o dwóch ogniwach do dziewięciu skrzyżowań (Kohn w \cite{kohn1993}?).
\index[persons]{Kohn, Peter}%
Gdyby hipoteza~\ref{con:bernhard_jablan} była prawdziwa dla wszystkich węzłów, mielibyśmy prosty sposób na wyznaczenie liczby $\unknotting K$: weźmy skończenie wiele diagramów minimalnych dla węzła $K$, na każdym z~nich odwracajmy skrzyżowania i~rekursywnie szukajmy liczb gordyjskich prostszych węzłów.
Najmniejsza spośród nich różni się wtedy o~jeden od liczby $\unknotting K$.

Brittenham, Hermiller w artykule \cite{brittenham2021} twierdzą, że hipoteza jest fałszywa.
Kontrprzykład został znaleziony komputerowo, z pomocą programu SnapPy.
\index{program SnapPy}%
\index[persons]{Brittenham, Mark}%
\index[persons]{Hermiller, Susan}%

\begin{example}[Brittenham, Hermiller]
\index{węzeł!12n-288}%
\index{węzeł!12n-491}%
\index{węzeł!12n-501}%
\index{węzeł!13n-3370}%
    Hipoteza Bernharda-Jablana jest fałszywa dla co najmniej jednego spośród czterech węzłów: $12n_{288}$, $12n_{491}$, $12n_{501}$, $13n_{3'370}$.
\end{example}

Bleiler \cite{bleiler1984} postawił problem: czy jeden węzeł może mieć kilka diagramów minimalnych, z~których tylko niektóre są świadkiem $1$-gordyjskości?
Rozwiązanie przyszło z Japonii: według Kanenobu, Murakamiego \cite{kanenobumurakami1986} dzieje się tak m.in. dla węzła $8_{14}$.
\index{węzeł!8-14}%
\index[persons]{Kanenobu, Taizo}%
\index[persons]{Murakami, Hitoshi}%
Stojmenow w~pracy \cite{stoimenow2001} pełnej różnych przykładów wskazał dodatkowo węzły $14_{36'750}$ oraz $14_{36'760}$.
\index{węzeł!14-36750}%
\index{węzeł!14-36760}%
\index[persons]{Stojmenow, Aleksander}%

