
% DICTIONARY;crossing number;indeks skrzyżowaniowy;-
\section{Indeks skrzyżowaniowy}
Opisując notację Alexandera-Briggsa w podsekcji \ref{alexander_briggs_notation} wspomnieliśmy już raz o indeksie skrzyżowaniowym.

\index{indeks skrzyżowaniowy|(}%
\begin{definition}
    Niech $L$ będzie splotem.
    Minimalną liczbę skrzyżowań, jakie muszą się pojawić na diagramie przedstawiającym splot $L$ nazywamy indeksem skrzyżowaniowym i oznaczamy krótko $\crossing L$.
\end{definition}

Pytanie, czy indeks skrzyżowaniowy jest addytywny, to jeden z najstarszych problemów teorii węzłów.

\begin{conjecture}
\index{hipoteza!o indeksie skrzyżowaniowym}%
\index{suma spójna}%
\label{con:crossing_additive}%
    Niech $K_1$ oraz $K_2$ będą węzłami.
    Wtedy $\crossing K_1 + \crossing K_2 = \crossing K_1 \shrap K_2$.
\end{conjecture}

Oto częściowe odpowiedzi.
Jeśli $K_1, K_2$ są alternującymi węzłami o~odpowiednio $c_1, c_2$ skrzyżowaniach, to istnieje alternujący diagram ich sumy $K_1 \shrap K_2$ o~$c_1 + c_2$ skrzyżowaniach.
\index{węzeł!alternujący}%
Kauffman \cite[twierdzenie 2.10]{kauffman1987}, Murasugi \cite[wniosek 6]{murasugi1987} oraz Thistlethwaite \cite[wniosek 1]{thistlethwaite1987} pokazali niezależnie, że diagram ten jest minimalny.
\index[persons]{Kauffman, Louis}%
\index[persons]{Murasugi, Kunio}%
\index[persons]{Thistlethwaite, Morwen}%

% DICTIONARY;adequate;adekwatny;węzeł
Thistlethwaite rozszerzył wynik do tak zwanych węzłów adekwatnych: sam \cite{thistlethwaite1988} albo razem z Lickorishem \cite{lickorish1988} (tak przynajmniej sugeruje Malutin \cite[s. 3]{malyutin2016}, nie widzimy tego).
\index[persons]{Lickorish, William}%
\index[persons]{Malutin, Andriej}%
\index{węzeł!adekwatny}%
Mając diagram węzła $K$, można wygładzić wszystkie skrzyżowania dodatnio i dostać splot złożony z rozłącznych okręgów.
Jeżeli zmiana dowolnego wygładzenia na ujemne sprawia, że liczba ogniw splotu zmniejsza się, diagram nazywamy dodatnio adekwatnym.
Węzeł dodatnio adekwatny to taki, który posiada jakiś dodatnio adekwatny diagram.
Analogicznie definiuje się ujemną adekwatność.
Węzeł, który jest dodatnio oraz ujemnie adekwatny, nazywamy krótko adekwatnym.

Na początku XX wieku Diao \cite{diao2004} oraz Gruber \cite{gruber2003} niezależnie udowodnili hipotezę \ref{con:crossing_additive} dla pewnej szerokiej klasy węzłów, obejmującej wszystkie węzły torusowe, wiele węzłów alternujących oraz jeszcze inne obiekty, których nie chcemy opisywać.
% diao04 -> tw. 3.8
\index[persons]{Diao, Yuanan}%
\index[persons]{Gruber, Hermann}%
\index{węzeł!torusowy}%

Lackenby \cite{lackenby2009} pokazał, że dla pewnej stałej $N \le 152$ zachodzi
\index[persons]{Lackenby, Marc}
\begin{equation}
    \frac 1 N \sum_{i=1}^n \crossing{K_i} \le \crossing \left(\bigshrap_{i=1}^n K_i\right) \le \sum_{i=1}^n \crossing{K_i}.
\end{equation}
(Tylko pierwsza nierówność jest ciekawa).
Jego argumentu wykorzystującego powierzchnie normalne nie można poprawić tak, by otrzymać stałą $N = 1$.
Jednocześnie od 2009 roku nie widać postępu nad hipotezą.

\index{indeks skrzyżowaniowy|)}%

% Koniec podsekcji Indeks skrzyżowaniowy

