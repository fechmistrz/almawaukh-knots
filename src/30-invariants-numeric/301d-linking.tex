
% DICTIONARY;linking number;indeks zaczepienia;-
\section{Indeks zaczepienia}
\index{indeks zaczepienia|(}%
Około 1833 roku Gauß wyraził indeks zaczepienia dwóch węzłów jako pewną (nieciekawą dla nas) całkę, co czyni go najstarszym niezmiennikiem splotów.
\index[persons]{Gauß, Carl}%
Po przeczytaniu \cite{colberg2013} wydaje nam się, że osiągnął to korzystając z praw fizyki: prawa Ampère'a i Biota-Savarta.
My żyjemy w~XXI wieku, wystarczy nam diagramatyczna definicja.
(Ale patrz też: \cite[s. 11]{kawauchi1996}.)
% Erin Colberg - A brief history of knot theory
% TODO: czemu patrz też?

% DICTIONARY;sign;znak;skrzyżowanie
\begin{definition}[znak]
\index{znak skrzyżowania}%
    Liczbę $\pm 1$ przypisaną do skrzyżowania zgodnie z regułą:
\begin{comment}
    \setlength{\intextsep}{4pt plus 2pt minus 2pt}
    \begin{figure}[H]
        \begin{minipage}[b]{.48\linewidth}
            \[
                \sign \left( \MedLarPlusCrossingArrows \right) = +1
            \]
        \end{minipage}
        \begin{minipage}[b]{.48\linewidth}
            \[
                \sign \left( \MedLarMinusCrossingArrows \right) = -1
            \]
        \end{minipage}

    \end{figure}
\end{comment}
\noindent
    nazywamy znakiem skrzyżowania.
\end{definition}

Skrzyżowania dodatnie to takie, w których obrócenie dolnego łuku w prawo daje górny łuk, dlatego czasem nazywa się je także praworęcznymi.
Oczywiście skrzyżowania ujemne nazywamy wtedy leworęcznymi.
\index{skrzyżowanie!dodatnie i ujemne}%
\index{skrzyżowanie!lewo- i prawoskrętne}%

% DICTIONARY;smoothing;wygładzenie;-
\begin{definition}[wygładzenie]
\index{skrzyżowanie!... wygładzenie}%
    Diagramy powstałe przez zmianę biegu łuków danego skrzyżowania zgodnie z poniższymy rysunkami:
\begin{comment}
    {\setlength{\intextsep}{4pt plus 2pt minus 2pt}
    \begin{figure}[H]
        \setlength{\intextsep}{4pt plus 2pt minus 2pt}
        \begin{minipage}[b]{.48\linewidth}
            \[
                \MedLarAlphaSmoothing
            \]
            \subcaption{wygładzenie dodatnie}
        \end{minipage}
        \begin{minipage}[b]{.48\linewidth}
            \[
                \MedLarBetaSmoothing
            \]
            \subcaption{wygładzenie ujemne}
        \end{minipage}
    \end{figure}
    }
\end{comment}
\noindent
    nazywamy wygładzeniem.
    Jeżeli nie zaznaczono inaczej, wygładzamy zgodnie ze znakiem skrzyżowania.
\end{definition}

\begin{definition}[indeks zaczepienia]
    Niech $L = K_1 \sqcup K_2$ będzie splotem o dwóch ogniwach, zaś $D$ jego diagramem.
    Wielkość
    \begin{equation}
        \linking(K_1, K_2) = \frac 12 \sum_i \sign c_i,
    \end{equation}
    gdzie sumowanie rozciąga się na wszystkie skrzyżowania, na których spotykają się łuki różnych ogniw, nazywamy indeksem zaczepienia węzłów $K_1, K_2$.
    Ogólniej, jeśli dany jest splot $L = K_1 \sqcup \ldots \sqcup K_n$ posiadający $n$ ogniw, to jego indeks zaczepienia wyznacza wzór
    \begin{equation}
        \linking(L) = \sum_{i < j} \linking(K_i, K_j).
    \end{equation}
\end{definition}

Zauważmy, że indeks zaczepienia splotu Hopfa wynosi $1$, natomiast splotu Whiteheada $0$.
\index{splot!Hopfa}%
\index{splot!Whiteheada}%
Są zatem różne.
W obydwu przypadkach indeks zaczepienia jest liczbą całkowitą.
Istotnie, na mocy twierdzenia Jordana $\linking$ jest funkcją o całkowitych wartościach.

\begin{proposition}
    Indeks zaczepienia jest dobrze określonym niezmiennikiem zorientowanych splotów.
\end{proposition}

\begin{proof}
    Wielkość $\linking L$ jest sumą znaków pewnych skrzyżowań, zatem na mocy twierdzenia Reidemeistera wystarczy sprawdzić, jaki jest wpływ ruchów Reidemeistera na te składniki:
\begin{comment}
{\setlength{\intextsep}{4pt plus 2pt minus 2pt}
\begin{figure}[H]
\centering
    %
    \begin{minipage}[b]{.3\linewidth}
        \[
            \MedLarReidemeisterOneLeft \cong \MedLarReidemeisterOneStraight
        \]
        \subcaption{ruch $R_1$}
    \end{minipage}
    %
    \begin{minipage}[b]{.3\linewidth}
        \[
            \MedLarReidemeisterTwoLinkingA \cong \MedLarReidemeisterTwoB
        \]
        \subcaption{ruch $R_2$}
    \end{minipage}
    %
    \begin{minipage}[b]{.35\linewidth}
        \[
            \MedLarReidemeisterThreeLinkingA \cong \MedLarReidemeisterThreeLinkingB
        \]
        \subcaption{ruch $R_3$}
    \end{minipage}
\end{figure}
}
\end{comment}
\noindent
    Ὅπερ ἔδει δεῖξαι...
\end{proof}

\index{indeks zaczepienia|)}%

% koniec podsekcji Indeks zaczepienia

