
\subsubsection{Znane wartości}
Cha, Livingston \cite{cha2018} podają, że znamy liczby gordyjskie wszystkich węzłów pierwszych do dziesięciu skrzyżowań poza dziewięcioma wyjątkami: $10_{11}$, $10_{47}$, $10_{51}$, $10_{54}$, $10_{61}$, $10_{76}$, $10_{77}$, $10_{79}$, $10_{100}$ (gdzie nie mamy pewności, czy $\unknotting = 2$, czy $\unknotting = 3$).
\index[persons]{Cha, Jae}%
\index[persons]{Livingston, Charles}%
Kto pierwszy znalazł liczbę gordyjską którego węzła ostaraliśmy się bezbłędnie przepisać z bazy danych KnotInfo\footnote{Patrz \url{https://knotinfo.math.indiana.edu/descriptions/unknotting_number.html}}.
Według KnotInfo oprócz węzłów torusowych, tych wymienionych poniżej oraz 1-gordyjskich, do 10 skrzyżowań mamy jeszcze 2 węzły o~siedmiu skrzyżowaniach, 3 o~ośmiu, 15 o~dziewięciu i~68 o~dziesięciu, których liczba gordyjska zdaje się należeć do folkloru matematycznego.

{
    \setlength{\intextsep}{4pt plus 2pt minus 2pt}
\begin{table}[H]
    \raggedright
    \footnotesize
    \centering
    \begin{tabular}{l|p{100mm}} \toprule
    rok & węzły i odkrywcy ich liczb gordyjskich \\ \midrule
    1982 & $7_{4}$ (Lickorish \cite{lickorish1985}) \\
    1986 & $8_{4}, 8_{6}, 8_{8}, 8_{12}, 9_{5}, 9_{8}, 9_{15}, 9_{17}, 9_{31}$ (Kanenobu, Murakami \cite{kanenobumurakami1986}) \\
    1989 & $9_{25}$ (Kobayashi \cite{kobayashi1989}) \\
    1994 & $10_{8}$ (Adams \cite[s. 62]{adams1994}?) \\
    1998 & $10_{65}, 10_{69}, 10_{89}, 10_{97}, 10_{108}, 10_{163}, 10_{165}$ (Miyazawa \cite{miyazawa1998}), $10_{154}, 10_{161}$ (Tanaka \cite{tanaka1998}) \\
    1999 & $10_{67}$ (Traczyk \cite{traczyk1999}) \\
    2000 & $8_{16}$ (Murakami, Yasuhara \cite{yasuhara2000}) \\
    2002 & $10_{139}, 10_{145}, 10_{152}, 10_{154}, 10_{161}$ (Gibson, Ishikawa \cite{ishikawa2002}) \\
    2004 & $8_{18}, 9_{37}, 9_{40}, 9_{46}, 9_{48}, 9_{49}, 10_{86}, 10_{103}, 10_{105}, 10_{106}, 10_{109}, 10_{121}, 10_{131}$ (Stojmenow \cite{stoimenow2004}; ostatni węzeł zdaje się być jedynym 1-gordyjskim na tej liście!) \\
    2005 & $9_{29}$, $10_{79}$, $10_{81}$, $10_{87}$, $10_{90}$, $10_{93}$, $10_{94}$, $10_{96}$, $10_{148}$, $10_{151}$, $10_{153}$ (Gordon, Luecke \cite{gordon2006}), $8_{10}$, $10_{48}$, $10_{52}$, $10_{54}$, $10_{57}$, $10_{58}$, $10_{64}$, $10_{68}$, $10_{70}$, $10_{77}$, $10_{110}$, $10_{112}$, $10_{116}$, $10_{117}$, $10_{125}$, $10_{126}$, $10_{130}$, $10_{135}$, $10_{138}$, $10_{158}$, $10_{162}$ (Ozsváth, Szabó \cite{szabo2005}), $10_{83}$ (Nakanishi \cite{nakanishi2005}) \\
    2008 & $9_{10}, 9_{13}, 9_{35}, 9_{38}, 10_{53}, 10_{101}, 10_{120}$ (Owens \cite{owens2008}) \\
    \bottomrule
    \hline
    \end{tabular}
% 50 za dużo -< aż do przykład NB
% 25 trochę za dużo  \vspace{-25pt} ^ to samo
% 5 za mało
\end{table}
}
\index[persons]{Lickorish, William}%
\index[persons]{Kanenobu, Taizo}%
\index[persons]{Murakami, Hitoshi}%
\index[persons]{Kobayashi, Tsuyoshi}%
\index[persons]{Adams, Colin}%
\index[persons]{Miyazawa, Yasuyuki}%
\index[persons]{Tanaka, Toshifumi}%
\index[persons]{Traczyk, Paweł}%
\index[persons]{Yasuhara, Akira}%
\index[persons]{Gibson, William}%
\index[persons]{Ishikawa, Masaharu}%
\index[persons]{Stojmenow, Aleksander}%
\index[persons]{Gordon, Cameron}%
\index[persons]{Luecke, John}%
\index[persons]{Nakanishi, Yasutaka}%
\index[persons]{Szabó, Zoltán}%
\index[persons]{Ozsváth, Peter}%
\index[persons]{Owens, Brendan}%
\normalsize
