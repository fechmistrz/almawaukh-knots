\subsection{Indeks skrzyżowaniowy}
\index{indeks!skrzyżowaniowy|(}%
\label{sub:crossing_number}%
Z angielskiego \emph{crossing number}.

\begin{definition}
    Niech $L$ będzie splotem.
    Minimalną liczbę skrzyżowań występujących na diagramach przedstawiających splot $L$ nazywamy indeksem skrzyżowaniowym i~oznaczamy $\crossing L$.
\end{definition}

Pytanie, czy indeks skrzyżowaniowy jest addytywny, to jeden z najstarszych problemów teorii węzłów.

\begin{conjecture}
    \index{hipoteza!o indeksie skrzyżowaniowym}
    \label{con:crossing_additive}
    Niech $K_1$ oraz $K_2$ będą węzłami.
    Wtedy $\crossing K_1 + \crossing K_2 = \crossing K_1 \shrap K_2$.
\end{conjecture}

Oto częściowe odpowiedzi.
Jeśli $K_1, K_2$ są alternującymi węzłami o~odpowiednio $c_1, c_2$ skrzyżowaniach, to istnieje alternujący diagram ich sumy $K_1 \shrap K_2$ o~$c_1 + c_2$ skrzyżowaniach.
\index{węzeł!alternujący}%
Kauffman \cite[twierdzenie 2.10]{kauffman87}, Murasugi \cite[wniosek 6]{murasugi87} oraz Thistlethwaite \cite[wniosek 1]{thistlethwaite87} pokazali niezależnie, że diagram ten jest minimalny.
Ten ostatni rozszerzył wynik do tak zwanych węzłów adekwatnych: sam w \cite{thistlethwaite88} albo z Lickorishem w \cite{lickorish88}.
\index{węzeł!adekwatny}%
Na początku XX wieku Diao \cite{diao04} oraz Gruber \cite{gruber03} niezależnie udowodnili hipotezę \ref{con:crossing_additive} dla pewnej szerokiej klasy węzłów, obejmującej wszystkie węzły torusowe, wiele węzłów alternujących oraz pewne inne obiekty, których nie chcemy opisywać.
\index{węzeł!torusowy}%
Lackenby w~pracy \cite{lackenby09} pokazał, że dla pewnej stałej $N \le 152$ zachodzi
\begin{equation}
    \frac 1 N \sum_{i=1}^n \crossing{K_i} \le \crossing \left(\bigshrap_{i=1}^n K_i\right) \le \sum_{i=1}^n \crossing{K_i}.
\end{equation}
(Tylko pierwsza nierówność jest ciekawa).
Jego argumentu wykorzystującego powierzchnie normalne nie można poprawić tak, by otrzymać stałą $N = 1$.
Jednocześnie od 2009 roku nie widać postępu nad hipotezą.

\index{indeks!skrzyżowaniowy|)}%

% Koniec podsekcji Indeks skrzyżowaniowy
