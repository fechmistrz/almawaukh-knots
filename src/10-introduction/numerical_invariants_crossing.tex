\subsection{Indeks skrzyżowaniowy} % (fold)
\label{sub:crossing_number}
\index{indeks!skrzyżowaniowy}
Z angielskiego \emph{crossing number}.

\begin{definition}
    Niech $L$ będzie splotem.
    Minimalną liczbę skrzyżowań widocznych na diagramie, który przedstawia splot $L$, nazywamy indeksem skrzyżowaniowym i~oznaczamy $\crossing L$.
\end{definition}

Pytanie, czy indeks skrzyżowaniowy jest addytywny, to jeden z najstarszych problemów teorii węzłów.

\begin{conjecture}
    \label{cnj:crossing_additive}
    Niech $K_1$ oraz $K_2$ będą węzłami.
    Wtedy $\crossing K_1 + \crossing K_2 = \crossing K_1 \shrap K_2$.
\end{conjecture}

Oto częściowe odpowiedzi.
Jeśli $K_1, K_2$ są alternującymi węzłami o~odpowiednio $c_1, c_2$ skrzyżowaniach, to istnieje alternujący diagram ich sumy $K_1 \shrap K_2$ o~$c_1 + c_2$ skrzyżowaniach.
Kauffman, Murasugi oraz Thistlethwaite pokazali niezależnie, że diagram ten jest minimalny (patrz na przykład \cite{murasugi87}, wniosek 6).
Thistlethwaite rozszerzył wynik do tak zwanych węzłów adekwatnych w \cite{thistlethwaite88}.
Wreszcie Gruber w \cite{gruber03} udowodnił hipotezę \ref{cnj:crossing_additive} dla węzłów torusowych.
Lackenby w~pracy \cite{lackenby09} pokazał, że dla pewnej stałej $N \le 152$ zachodzi
\begin{equation}
    \frac 1 N \sum_{i=1}^n \crossing K_i \le \crossing \left(\bigshrap_{i=1}^n K_i\right) \le \sum_{i=1}^n \crossing{cr} K_i.
\end{equation}
(Tylko pierwsza nierówność jest ciekawa).
Jego argumentu wykorzystującego powierzchnie normalne nie można poprawić tak, by otrzymać stałą $N = 1$.

% Koniec podsekcji Indeks skrzyżowaniowy
