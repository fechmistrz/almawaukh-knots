
\subsubsection{Jedenaście skrzyżowań}
\color{white}
Na dalszy postęp musieliśmy czekać do lat sześćdziesiątych.
Wtedy to John Conway\footnote{TODO: biografia Conwaya...} \cite{conway1970} znalazł prawie wszystkie pierwsze węzły o~mniej niż 12 skrzyżowaniach oraz sploty o~mniej niż 11 skrzyżowaniach.
\index[persons]{Conway, John}%
Odkrył jeden duplikat oraz jedenaście pominięć w~starych tablicach Little'a (do 11 skrzyżowań), ale sam zgubił cztery węzły.
Naprawienie błędu zajęło chwilę: dwa brakujące węzły zawierała praca magisterska Lombardero z~1968 roku, dwa odkrył Caudron około 1980 roku \cite{caudron1982}.
\index[persons]{Caudron, Alain}%
\index[persons]{Lombardero, ?}%
Conway też popełnił dwa duplikaty: współcześnie bardziej znany to para Perko, węzeł reprezentowany przez dwa diagramy aż do tablic Rolfsena.
\index{para Perko}%
Nazywamy ją tak, ponieważ została dostrzeżona przez Perko \cite{perko1974}, który bezskutecznie próbował odróżnić składniki pary diedralnym indeksem zaczepienia: dla obydwu diagramów wynosi on $32/11$.
\index[persons]{Perko, Kenneth}%
% https://www.researchgate.net/profile/Ken-Perko/publication/299560799_The_History_of_the_Perko_Pair/links/56ff0ba508aea6b77468d550/The-History-of-the-Perko-Pair.pdf both knots yielded a 5-fold dihedral linking number of 32/11
Jako przyczynę tak długiego niezauważenia pary Perko podaje się hipotezę Little'a, że spin minimalnego diagramu jest niezmiennikiem, gdyż błędnie założył, że 2-przejścia oraz flype wystarczają do zmiany dowolnego minimalnego diagramu w~inny.
% DICTIONARY;2-pass move;2-przejście;-
\index{spin}%
\index[persons]{Little, Charles}%
(Diagramy pary Perko mają różny spin).
Zachęcamy w~tym miejscu do nieprzeskakiwania strony \pageref{rolfsens_mistake}, gdzie para Perko pojawia się raz jeszcze.
Mniej znany duplikat wystąpił w~tablicy splotów do 10 skrzyżowań, gdzie \texttt{2.-2.-20.20} jest lustrem \texttt{8*-20:-20}.

Pomimo opisanych wyżej drobnych niepowodzeń, metodę Conwaya (mającą fundamenty w~pomysłach Kirkmana) uznaje się za bardzo dobrą.
Conway potrzebował zaledwie kilku godzin na przeprowadzenie swojej klasyfikacji; a my używamy jej po dziś dzień, na przykład Tuzun, Sikora zweryfikowali dzięki niej hipotezę \ref{con:jones} do 24 skrzyżowań.
\index[persons]{Tuzun, Robert}%
\index[persons]{Sikora, Adam}%

Rękopis \cite{siebenmann1979} (albo \cite{bonahon1989}?) Bonahona, Siebenmanna klasyfikuje węzły algebraiczne.
\index[persons]{Bonahon, Francis}%
\index[persons]{Siebenmann, Laurent}%
Kres ery ręcznych obliczeń nastąpił, gdy z~nielicznymi niealgebraicznymi węzłami do 11 skrzyżowań poradził sobie Perko \cite{perko1980}, \cite{perko1982}.
\index[persons]{Perko, Kenneth}%

% MAKOTO SAKUMA - A SURVEY OF THE IMPACT OF THURSTON’S WORK ON KNOT THEORY
% through hand calculation of homological invariants (in particular linking invariants) of finite branched coverings for those knots that are not covered by Bonahon and Siebenmann’s result described in Subsection 4.1. See [268] for an interesting historical note.

