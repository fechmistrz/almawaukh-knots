
\subsection{Węzły okresowe}
\index{węzeł!okresowy|(}%
Można wyróżnić jeszcze jeden rodzaj symetrii.

% DICTIONARY;period;okres;-
% DICTIONARY;periodic;okresowy;węzeł
\begin{definition}
\label{def:period}%
    Węzeł $K$ nazywamy $n$-okresowym, jeśli istnieje obrót $f \colon \R^3 \to \R^3$ o~kąt $2\pi/n$ wokół pewnej prostej $l$, rozłącznej z~węzłem $K$, taki że $f(K) = K$.
\end{definition}

Zamiast obrotów można rozpatrywać dowolne odwzorowania okresowe $f \colon S^3 \to S^3$, których zbiór punktów stałych jest rozłączny z węzłem $K$, homeomorficzny z $S^1$ oraz które trzymają węzeł $K$ w miejscu, ale dostaje się wtedy dokładnie taką samą klasę węzłów.
Czemu?
Wynika to z hipotezy Smitha, otrzymanej z połączenia głębokich teorii dotyczących geometrii i topologii 3-rozmaitości.
\index{hipoteza!Smitha}%
Kawauchi \cite[s. 125]{kawauchi1996} odsyła tu do książki Morgana, Bassa \cite{morgan1984}, gdzie znajdziemy problem, jego historię i rozwiązanie.
\index[persons]{Morgan, John}%
\index[persons]{Bass, Hyman}%

\begin{proposition}
    Zbiór wszystkich okresów jest niezmiennikiem węzłów.
\end{proposition}

Nieodwracalny węzeł $8_{17}$ nie posiada żadnych okresów.
% ćwiczenie 10.1.5 w Kawauchi
Węzeł $5_1$ jest 5-okresowy, co widać na standardowym diagramie, oraz 2-okresowy, tę drugą symetrię można dostrzec na diagramie realizującym liczbę mostową.
Trójlistnik ma dokładnie dwa okresy, $2$ i~$3$.
Ogólniej, jak głosi Kawauchi \cite[ćwiczenie 10.1.9]{kawauchi1996}:

\begin{proposition}
    Jedynymi okresami węzła $(p, q)$-torusowego są dzielniki liczb $p$ oraz $q$.
\end{proposition}

Z~każdym węzłem okresowym związany jest inny, prostszy węzeł.
Niech $f$ będzie obrotem z definicji \ref{def:period}, zaś $p \colon \R^3 \to \R^3/f \simeq \R^3$ rzutem na przestrzeń ilorazową.
% DICTIONARY;quotient;ilorazowy;węzeł
\index{węzeł!ilorazowy}%
Wtedy $p(K)$ nazywamy \emph{węzłem ilorazowym}, zaś $K$ to jego $n$-krotne nakrycie.

Murasugi podał dwa warunki, które musi spełniać węzeł o~okresie $n = p^r$, gdzie $r$ jest liczbą pierwszą.
Do ich zrozumienia potrzebujemy prostej definicji.
Ustalmy półprostą, która nie jest styczna do węzła $K$, po czym zorientujmy ją oraz węzeł.
Indeksem zaczepienia $\lambda$ węzła $p(K)$ jest różnica między liczbą skrzyżowań dodatnich oraz ujemnych wzdłuż półprostej (bez znaku).

\begin{proposition}[warunek Murasugiego]
\index{warunek!Murasugiego}%
\label{prp:murasugi_periodic}%
    Niech $K$ będzie węzłem o~okresie $n = p^r$, gdzie $p$ jest liczbą pierwszą.
    Niech $J$ będzie jego węzłem ilorazowym, z~indeksem zaczepienia $\lambda$.
    Wtedy wielomian $\alexander_J$ jest dzielnikiem wielomianu $\alexander_K$ oraz istnieje pewna całkowita liczba $k$, taka że
    \begin{equation}
        \alexander_K(t) \equiv \pm t^k \alexander_J(n)^n \left(1 + t + t^2 + \ldots + t^{\lambda - 1}\right)^{n-1} \mod p.
    \end{equation}
\end{proposition}

\begin{proof}
    Mozolne operacje na macierzach, których wyznacznikiem jest wielomian Alexandera, patrz \cite{murasugi1971}.
    Kawauchi przedstawia inny dowód: najpierw dowodzi tego dla węzła torusowego $T_{n, d}$, którego węzłem ilorazowym jest niewęzeł.
    W ogólnym przypadku, korzysta z relacji kłębiastej dla wielomianu Conwaya.
    Szczegóły oraz odsyłacze do dalszych prac znaleźć można w jego przeglądowej publikacji \cite[s. 122-124]{kawauchi1996}.
\end{proof}

Z prac Borodzika dowiedzieliśmy się, że użytecznym narzędziem do badania okresowości węzłów jest kryterium Naika z pracy \cite{naik1997} oraz że można je wzmocnić.
% Na przykład z https://arxiv.org/pdf/1810.03881.pdf się dowiedzieliśmy
\index{kryterium Naika (okresowości)}%

\index{węzeł!okresowy|)}%

% koniec podsekcji Węzły okresowe

