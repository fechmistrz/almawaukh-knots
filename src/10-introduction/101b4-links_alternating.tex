
\subsubsection{Sploty alternujące}

Zazwyczaj do zdefiniowania splotów alternujących potrzebne są najpierw diagramy.
% Zanim opowiemy, jak dotąd przebiegała klasyfikacja węzłów o małej liczbie skrzyżowań, zdefiniujemy klasę splotów ze specjalnymi diagramami.

% DICTIONARY;alternating;alternujący;węzeł
\begin{definition}[splot alternujący]
\label{def:alternating_link}%
\index{węzeł!alternujący}%
    Niech $D$ będzie diagramem splotu $L$.
    Jeżeli podczas poruszania się wzdłuż każdego ogniwa naprzemiennie mijamy podskrzyżowania oraz nadskrzyżowania, to diagram nazywamy alternujący.
    
    Splot $L$ jest alternujący, jeśli posiada alternujący diagram $D$s.
\end{definition}

Około 1961 roku Ralph Fox zapytał \emph{,,What is an alternating knot?''}.
\index[persons]{Fox, Ralph}%
Szukano takiej definicji węzła alternującego, która nie odnosi się bezpośrednio do diagramów, aż w~2015 roku Greene \cite{greene2017} podał geometryczną charakteryzację: nierozszczepialny splot w $S^3$ jest alternujący wtedy i~tylko wtedy, gdy ogranicza dodatnią oraz ujemną określoną powierzchnię rozpinającą.
\index[persons]{Greene, Joshua}%

\begin{remark}[Ralph Hartzler Fox]
    Matematyk amerykański urodzon w Morrisville, Pensylwanii w~1913 roku; zmarł w Filadelfii, tamże w 1973 roku.
    Był promotorem Johna Milnora, Lee Neuwirtha (o~których jeszcze wspomnimy!) i 23 innych osób, o których nie wspomnimy.
    Oprócz tego nadzorował pracę licencjacką Kennetha Perko.
    Zawdzięczamy mu spopularyzowanie $n$-kolorowania na koledżu Haverford w 1956 roku, podanie nowego sposobu na znalezienie wielomianu Alexandera przy użyciu rachunku różniczkowego Foxa oraz niektóre terminy teorii węzłów uzywane po dziś dzień: węzeł plastrowy, węzeł taśmowy, okrąg i powierzchnia Seiferta.
\end{remark}

Nie ma zwartego wzoru na liczbę splotów alternujących, ale wiemy, że rośnie co najmniej wykładniczo względem liczby skrzyżowań:

\begin{proposition}
\index{supeł}%
    Niech $a_n$ oznacza liczbę supłów o~$n$ skrzyżowaniach, które są alternujące oraz pierwsze.
    Wtedy
    \begin{equation}
        a_n \sim \frac{3c_1 \lambda^{n-3/2}}{4\sqrt{\pi n^{5}}},
    \end{equation}
    gdzie zarówno $c_1$, pierwszy współczynnik rozwinięcia Taylora funkcji $\Phi(\eta)$ zdefiniowanej w \cite{sundberg1998}, jak i $\lambda$ są jawnie znanymi stałymi:
    \begin{align}
        c_1 & = \sqrt{\frac{5^7 \cdot (21001 + 371 \sqrt{21001})^3}{2 \cdot 3^{10} \cdot (17 + 3\sqrt{21001})^5}} \\
        \lambda & = \frac {1}{40} (101 + \sqrt{21001})
    \end{align}
    Niech $A_n$ oznacza liczbę pierwszych, alternujących splotów o $n$ skrzyżowaniach.
    Wtedy $A_n \approx \lambda^n$, dokładniej: jeśli $n \ge 3$, to
    \begin{equation}
        \frac{a_{n-1}}{16n - 24} \le A \le \frac{a_n - 1}{2}.
    \end{equation}
\end{proposition}

Węzły pierwsze i~supły pojawiają się odpowiednio w definicjach \ref{def:prime_knot}, \ref{def:tangle}.

\begin{proof}[Niedowód]
\index[persons]{Sundberg, Carl}%
\index[persons]{Thistlethwaite, Morwen}%
    Zamiast przedstawić dowód albo chociaż jego szkic, wymienimy trzy narzędzia użyte przez Sundberga, Thistlethwaite'a \cite{sundberg1998}:
    algebraiczną metodę Conwaya znajdowania splotów,
    wynik Tuttego dotyczącego liczby ukorzenionych $c$-sieci
    oraz (wtedy już udowodnioną) hipotezę Taita.
\index[persons]{Conway, John}%
\index[persons]{Tutte, William}%
\index{hipotezy Taita}%
\end{proof}

\begin{proposition}
    Niech $a_n$ oznacza liczbę supłów o~$n$ skrzyżowaniach, które są alternujące oraz pierwsze.
    Wtedy funkcja tworząca $f(z) = \sum_n a_n z^n$ spełnia równanie
    \begin{equation}
    f(1+z) - f(z)^2 - (1+f(z))q(f(z)) -z - \frac{2z^2}{1-z} = 0,
    \end{equation}
    gdzie $q(z)$ jest pomocniczą funkcją
    \begin{equation}
        q(z) = \frac{2z^2 - 10z - 1 + \sqrt{(1-4z)^3}} {2(z+2)^3} - \frac{2}{1+z} -z + 2.
    \end{equation}
\end{proposition}

Powyższa ciekawostka także pochodzi z cytowanej wcześniej pracy \cite{sundberg1998}.

