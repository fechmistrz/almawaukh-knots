
\subsubsection{Liczba gordyjska jako metryka}
Mając dane dwa węzły $K_0, K_1$, rozpatrzmy wszystkie homotopie
\begin{equation}
    f : [0,1] \times S^1 \to \R^3
\end{equation}
takie, że wszystkie funkcje $f_t$ są zanurzeniami z co najwyżej jednym punktem podwójnym.
Zażądajmy dodatkowo, by styczne do krótkich łuków, które przecinają się w tym punkcie, były od siebie różne.
Odległością gordyjską między węzłami $K_0, K_1$ jest minimalna liczba podwójnych punktów, jakie posiada homotopia $f$.
Twierdzenie C~z~pracy\footnote{To nie jest główne twierdzenie tamże. Autorzy definiuję $\omega$-sygnaturę domknięcia warkocza, a~że sklejenie dwóch 4-rozmaitości z narożnikami (corners) nie odpowiada dodaniu ich sygnatur, to ich funkcja nie jest homomorfizmem. Wspomniany jest wzór Novikowa-Walla, który wyraża różnicę pewnych defektów jako indeks Masłowa i (to jest główne twierdzenie) różnica ta pokrywa się z kocyklem Meyera reprezentacji Burau-Squiera, cokolwiek to znaczy. Pojawia się również jakaś funkcja Rademachera.} Gambaudo, Ghysa \cite{gambaudo05} głosi, że przestrzeń wszystkich węzłów wyposażona w taką metrykę zawiera prawie idealną kopię przestrzeni euklidesowej dowolnego wymiaru.
\index[persons]{Gambaudo, Jean-Marc}%
\index[persons]{Ghys, Étienne}%
Dokładniej:

\begin{proposition}
    Dla każdej liczby całkowitej $n \ge 1$ istnieje funkcja $\xi: \Z^n \to \mathcal{K}$, dodatnie stałe $A, B, C$ oraz norma $\|\cdot\|$ na przestrzeni $\R^n$ takie, że spełniona jest podwójna nierówność
    \begin{equation}
        A\|x-y\|  - B \le d(\xi(x), \xi(y)) \le C\|x-y\|.
    \end{equation}
\end{proposition}

\begin{proof}
    Dowód korzysta z grup warkoczowych, które poznamy w sekcji \ref{sec:braid}.
\end{proof}

