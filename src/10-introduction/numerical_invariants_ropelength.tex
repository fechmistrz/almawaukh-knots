\subsection{Długość sznurowa} % (fold)
\label{sub:ropelength}
Długość sznurowa, z~angielskiego \emph{ropelength}, pochodzi z~fizycznej teorii węzłów, która bierze pod uwagę obiekty wykonane z~nieelastycznych materiałów.

\begin{definition}
    Niech $L$ będzie splotem o długości $l$ oraz grubości $\tau$: posiada rurowe otoczenie bez samoprzecięć z~przekrojem poprzecznym o~promieniu $\tau$.
    Iloraz
    \begin{equation}
        \operatorname{len} L = \frac l \tau
    \end{equation}
    nazywamy długością sznurową splotu.
\end{definition}

Przez wiele lat zastanawiano się, czy można zawiązać węzeł ze sznura o~długości jednej stopy i~promieniu jednego cala lub równoważnie, czy $\operatorname{len} K \le 12$ dla pewnego węzła $K$.
Na początku XXI wieku wiedzieliśmy z \cite{cantarella02}, że najkrótszy węzeł ma długość $10.726$, potem Diao udzielił negatywnej odpowiedzi na to pytanie w \cite{diao03}.
Wreszcie rozumowanie \cite{denne06} oparte o~czterosieczne pokazuje, że długość sznurowa nietrywialnego węzła wynosi co najmniej $15.66$.
Ponieważ eksperymenty komputerowe pokazują, że długość trójlistnika nie przekracza $16.372$, oszacowanie to jest więc dość ostre.

% Węzeł realizujący długość sznurową jest klasy $C^1,1$.

Prowadzono obszerne poszukiwania na temat zależności między długością sznurową i~innymi niezmiennikami.
Mamy na przykład:

\begin{proposition}
    $\operatorname{len} K = \Omega (\operatorname{cr}^{3/4} K)$.
\end{proposition}

Ograniczenie to realizowane jest przez pewne węzły torusowe oraz sploty Hopfa.

\begin{proposition}
    $\operatorname{len} K = O(\operatorname{cr} K \cdot \log^5(\operatorname{cr} K)).$
\end{proposition}

\begin{proof}
    Świeży wynik z \cite{diao19}, którego dowód wykorzystuje kraty liczbowe.
\end{proof}

Wcześniej znaliśmy słabszą równość $\operatorname{len} K = O(\operatorname{cr}^{3/2} K)$ dzięki cyklom Hamiltona w~grafach zanurzonych właśnie w~kratach liczbowych \cite{yu04}.

% Koniec podsekcji Ropelength
