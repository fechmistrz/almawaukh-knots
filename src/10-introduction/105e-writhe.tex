\subsection{Spin}
\index{spin|(}

% DICTIONARY;writhe;spin
\begin{definition}[spin]
    Niech $D$ będzie diagramem zorientowanego splotu.
    Wielkość
    \begin{equation}
        \writhe D = \sum_c \operatorname{sign} c,
    \end{equation}
    gdzie sumowanie przebiega po wszystkich skrzyżowaniach diagramu $D$, nazywamy spinem.
\end{definition}

Co ważne, spin nie jest niezmiennikiem splotów ani węzłów.
Para Perko przedstawia ten sam węzeł z~minimalną liczbą skrzyżowań i~spinem równym siedem lub dziewięć.
\index{para Perko}%
Dzięki temu przez wiele lat nie została dostrzeżona.
Spin jest za to niezmiennikiem węzłów alternujących, mówi o~tym druga hipoteza Taita.
\index{hipoteza Taita}%

\begin{lemma}
    \label{lem:writhe_reidemeister}
    Spin nie zależy od orientacji.
    Tylko I ruch Reidemeistera zmienia spin:
\begin{comment}
    \begin{equation}
        \writhe \left(\MediumReidemeisterIaLeft\right) = \writhe \left(\MediumReidemeisterIb\right) - 1.
    \end{equation}
\end{comment}
    Pozostałe ruchy nie mają na niego wpływu.
\end{lemma}

\index{spin|)}

% Koniec sekcji Spin
