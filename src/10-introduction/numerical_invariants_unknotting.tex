\subsection{Liczba gordyjska} % (fold)

\index{liczba!gordyjska}
Z angielskiego \emph{unknotting number}.

\begin{definition}
    Niech $L$ będzie splotem.
    Minimalną liczbę skrzyżowań, które trzeba odwrócić na pewnym jego diagramie, by dostać niewęzeł, nazywamy liczbą gordyjską i~oznaczamy $\operatorname{u}(L)$.
\end{definition}

Zgodnie z ,,klasyczną'' definicją, między odwracaniem kolejnych skrzyżowań mamy prawo wykonać izotopie otaczające; natomiast zgodnie ze ,,standardową'' definicją, takie izotopie są zabronione.
Obie definicje są równoważne: tłumaczy to książka Adamsa \cite[s. 58]{adams94}.

\begin{lemma}
    \label{lem:unknotting_well_defined}
    W dowolnym rzucie splotu można odwrócić pewne skrzyżowania tak, by uzyskać diagram niesplotu.
\end{lemma}

\begin{proof}
    Bez straty ogólności założę, że diagram przedstawia węzeł.
    Ustalmy zatem diagram węzła i~wybierzmy jakiś początkowy punkt na nim, różny od skrzyżowania wraz z~kierunkiem, wzdłuż którego będziemy przemierzać węzeł.
    Za każdym razem, kiedy odwiedzamy nowe skrzyżowanie, zmieniamy je w~razie potrzeby na takie, przez które przemieszczamy się wzdłuż górnego łuku.
    Skrzyżowań już odwiedzonych nie zmieniamy wcale.

    Teraz wyobraźmy sobie nasz nowy węzeł w~trójwymiarowej przestrzeni $\mathbb R^3$, przy czym oś $z$ skierowana jest z~płaszczyzny, w~której leży diagram, w~naszą stronę.
    Umieśćmy początkowy punkt tak, by jego trzecią współrzędną była $z = 1$.

    Przemierzając węzeł, zmniejszamy stopniowo tę współrzędną, aż osiągniemy wartość $0$ tuż przed punktem, z~którego wyruszyliśmy.
    Połączmy obydwa punkty (początkowy oraz ten, w~którym osiągamy współrzędną $z = 0$) pionowym odcinkiem.
    Zauważmy, że kiedy patrzymy na węzeł w~kierunku osi $z$, nie widzimy żadnych skrzyżowań.

    Oznacza to, że nasza procedura przekształciła początkowy diagram w~diagram niewęzła, co należało okazać.
\end{proof}

W pracy \cite{shimizu14} Shimizu rozpatruje różne operacje, które rozwiązują węzły lub sploty.
Nie będziemy się nimi zajmować, podamy tylko przykład: zamiana pod- i nadskrzyżowań wokół obszaru na diagramie rozwiązuje węzły, ale nie sploty; kontrprzykładem jest splot Hopfa.
Patrz też \cite[s. 141-154]{kawauchi96}.

\begin{tobedone}
    Kawauchi, s. 151: $u(K) \ge \log_3|Q(-1)|$.
\end{tobedone}

\begin{tobedone}
    Kawauchi, s. 151: $u = 1, g = 1$ to duble.
\end{tobedone}

Jeśli odwrócenie pewnych skrzyżowań daje niewęzeł, to odwrócenie pozostałych także.
To daje proste liczby gordyjskiej: $2 \operatorname{u} (K) \le \crossing (K)$.
Nie jest zbyt pomocne, równość zachodzi pięć razy dla pierwszych węzłów do 12 skrzyżowań: $3_{1}$, $5_{1}$, $7_{1}$, $9_{1}$, $11a_{367}$.
Dokładna wartość liczby gordyjskiej jest znana tylko dla niektórych klas węzłów, na przykład torusowych (fakt \ref{prp:torus_unknotting_number}) albo skręconych (definicja \ref{def:twist_knot}).

Dla każdego nietrywialnego splotu istnieje diagram wymagający odwrócenia dowolnie wielu skrzyżowań.
Wcześniej Nakanishi znalazł 2-gordyjski diagram 1-gordyjskiego węzła $6_2$ (\cite{nakanishi83}) oraz udowodnił, że każdy nietrywialny węzeł ma diagram, który nie jest 1-gordyjski (\cite{nakanishi96}).
Dowód zawiera praca \cite{taniyama09} Taniyamy.
Pokazany jest tam jeszcze jeden godny uwagi fakt.
Jeśli liczba gordyjska diagramu $D$ wynosi $\frac 12 (\crossing D - 1)$, co jest maksymalną możliwą wartością zgodnie z~naszym prostym ograniczeniem,
to węzeł jest $(2,p)$-torusowy albo wygląda jak diagram niewęzła po pierwszym ruchu Reidemeistera.

Bleiler odkrył w~\cite{bleiler84} fascynujący przykład wymiernego węzła $10_8$, który jest $2$-gordyjski, ale świadkiem tego nie może być żaden diagram mininalny, ponieważ, co jeszcze bardziej fascynujące, węzeł ten posiada tylko jeden diagram o~dziesięciu skrzyżowaniach oraz liczbie gordyjskiej 3.
Wynika stąd, że liczba $u$ nie musi być osiągana przez diagram minimalny, wbrew powszechnym przypuszczeniom obecnym jeszcze w latach 70.
Praca \cite{bernhard94} opisuje nieskończoną rodzinę węzłów $C_k$, gdzie $C_2 = 10_8$ jest węzłem Bleilera.

Przykład Bleilera pokazuje, że do szukania liczby gordyjskiej potrzeba wyrafinowanego algorytmu.
Ponieważ odwrócenie jednego ze skrzyżowań na minimalnym diagramie węzła $10_8$ daje $1$-gordyjski węzeł $4_1, 5_1, 6_1$ lub $6_2$, możemy liczyć, że każdy diagram minimalny ma skrzyżowanie, którego odwrócenie zmniejsza liczbę gordyjską.
Dlatego jeszcze w~latach 90. postawiono hipotezę:

\begin{conjecture}[Bernharda-Jablana, \cite{bernhard94}, \cite{jablan98}]
    \index{hipoteza!Bernharda-Jablana}
    \label{con:bernhard_jablan}
    Każdy węzeł $K$ posiada diagram $D$ realizujący liczbę gordyjską oraz skrzyżowanie, którego odwrócenie daje nowy węzeł $K'$ z diagramem $D'$ o mniejszej liczbie gordyjskiej: $u(D') < u(D)$.
\end{conjecture}

Zakładając prawdziwość hipotezy \ref{con:bernhard_jablan}, mamy prosty sposób na wyznaczenie liczby $u(K)$: weźmy skończenie wiele diagramów minimalnych dla węzła $K$, na każdym z~nich odwracajmy skrzyżowania i rekursywnie szukajmy liczb gordyjskich prostszych węzłów.
Najmniejsza spośród nich różni się wtedy o~jeden od liczby $u(K)$.

Brittenham, Hermiller w artykule \cite{brittenham17} twierdzą, że hipoteza jest fałszywa, ten nie został jednak jeszcze zrecenzowany.
Prawdziwość sprawdzono natomiast dla węzłów do jedenastu skrzyżowań oraz splotów o dwóch ogniwach do dziewięciu skrzyżowań (Kohn w \cite{kohn93}?).

\begin{example}[Brittenham, Hermiller]
    Hipoteza Bernharda-Jablana jest fałszywa dla co najmniej jednego spośród czterech węzłów: $12n_{288}$, $12n_{491}$, $12n_{501}$, $13n_{3370}$.
\end{example}

Bleiler postawił w~\cite{bleiler84} problem: czy jeden węzeł może mieć kilka diagramów minimalnych, z~których tylko niektóre są świadkiem $1$-gordyjskości?
Rozwiązanie przyszło wkrótce z Japonii: według \cite{kanenobumurakami86} dzieje się tak m.in. dla węzła $8_{14}$.
Stojemenow w~pracy \cite{stoimenow01} pełnej różnych przykładów wskazał dodatkowo węzły $14_{36750}$ oraz $14_{36760}$.

Sploty o liczbie gordyjskiej 1 zasługują na szczególną uwagę.

\begin{proposition}
    Niech $L$ będzie wymiernym splotem 1-gordyjskim.
    Wtedy na minimalnym diagramie $L$ jedno ze skrzyżowań jest rozwiązujące.
\end{proposition}

\begin{proof}
    Kanenobu, Murakami dla węzłów \cite{kanenobumurakami86}, wkrótce po tym Kohn dla splotów \cite{kohn91}.
\end{proof}

Z pracy \cite{kanenobumurakami86} wynika dodatkowo, że liczba gordyjska węzłów $8_{3}$, $8_{4}$, $8_{6}$, $8_{8}$, $8_{12}$, $9_{5}$, $9_{8}$, $9_{15}$, $9_{17}$, $9_{31}$ wynosi dokładnie 2, wcześniej wiedzieliśmy, że jest równa co najwyżej 2.

Coward, Lackenby dowiedli w~\cite{coward11}, że jeśli $K$ jest 1-gordyjski i~o genusie 1, to z~dokładnością do pewnej relacji równoważności, tylko jedna zmiana skrzyżowania rozwiązuje go; chyba że $K$ jest ósemką -- wtedy takie zmiany są dwie.

\begin{proposition}
    \label{prp:unknotting_one_prime}
    Węzły $1$-gordyjskie są pierwsze.
\end{proposition}

Podejrzewał to H. Wendt w~1937 roku, kiedy policzył liczbę gordyjską węzła babskiego używając homologii rozgałęzionego nakrycia cyklicznego.

\begin{proof}[Niedowód]
    W pracy \cite{scharleman85} z~1985 roku M. Scharleman podał dość zawiłe uzasadnienie, w~które zamieszane były grafy planarne.
    Obecnie znamy prostsze dowody, patrz \cite{lackenby97} albo \cite{zhang91}.
\end{proof}

Scharlemann pokazał w \cite[wniosek 1.6]{scharlemann98}, że liczba gordyjska jest podaddytywna, to znaczy zachodzi $u(K_1 \shrap K_2) \le u(K_1) + u(K_2)$.
Stąd oraz z faktu \ref{prp:unknotting_one_prime} wynika, że suma dwóch $1$-gordyjskich węzłów jest $2$-gordyjska, ale od początku teorii węzłów podejrzewano dużo więcej:

\begin{conjecture}
    \index{hipoteza!o liczbie gordyjskiej}
    Niech $K_1, K_2$ będą węzłami.
    Wtedy $u(K_1 \shrap K_2) = u(K_1) + u(K_2)$, czyli liczba gordyjska jest addytywna.
\end{conjecture}

Dotychczas wyznaczono liczbę gordyjską dla prawie wszystkich węzłów pierwszych o~co najwyżej dziesięciu skrzyżowaniach,
Cha, Livingston \cite{cha05} podają następującą listę wyjątków:
$10_{11}$, $10_{47}$, $10_{51}$, $10_{54}$, $10_{61}$, $10_{76}$, $10_{77}$, $10_{79}$, $10_{100}$ (stan na rok 2018).

Borodzik oraz Friedl podali niedawno całkiem mocne ograniczenia na liczbę gordyjską w~pracach \cite{borodzik14} i~\cite{borodzik15}.
Ich narzędziem jest parowanie Blanchfielda.
Poprawiają tam starsze estymaty wynikające z~sygnatury Levine'a-Tristrama, indeksu Nakanishiego oraz przeszkody Lickorisha.
Wśród węzłów o~co najwyżej dwunastu skrzyżowaniach 25 ma liczbę gordyjską równą co najmniej trzy, co trudno pokazać innymi metodami.

Liczbę gordyjską można uogólnić w naturalny sposób do metryki.
Mianowicie mając dane dwa węzły $K_0, K_1$, rozpatrzmy wszystkie homotopie $f : [0,1] \times S^1 \to \R^3$ takie, że wszystkie funkcje $f_t$ są zanurzeniami z co najwyżej jednym punktem podwójnym.
Zażądajmy dodatkowo, by styczne do krótkich łuków, które przecinają się w tym punkcie, były od siebie różne.
Odległością gordyjską między węzłami $K_0, K_1$ jest minimalna liczba podwójnych punktów, jakie posiada homotopia $f$.
Twierdzenie C~z~pracy \cite{gambaudo05} głosi, że zawiera ona prawie idealną kopię przestrzeni euklidesowej dowolnego wymiaru.
Dokładniej:

\begin{proposition}
    Dla każdej liczby całkowitej $n \ge 1$ istnieje funkcja $\xi: \Z^n \to \mathcal{K}$, dodatnie stałe $A, B, C$ oraz norma $\|\cdot\|$ na przestrzeni $\R^n$ takie, że spełniona jest podwójna nierówność
    \begin{equation}
        A\|x-y\|  - B \le d(\xi(x), \xi(y)) \le C\|x-y\|.
    \end{equation}
\end{proposition}

\begin{proof}
    Dowód korzysta z grup warkoczowych, które poznamy w sekcji \ref{sec:braid}.
\end{proof}

% Koniec podsekcji Liczba gordyjska
