
% DICTIONARY;bridge number;liczba mostowa;-
\subsection{Liczba mostowa}
\index{liczba mostowa|(}%

Wprowadzona w~1954 przez Schuberta.

\begin{definition}
    Niech $D$ będzie diagramem węzła.
    Liczbę mostów, czyli długich łuków, które biegną tylko przez nadskrzyżowania, nazywamy liczbą mostową diagramu $D$.
    Minimalną liczbę mostową wśród wszystkich diagramów $D$ węzła $K$, $\bridge K$, nazywamy liczbą mostową węzła $K$.
\end{definition}

Węzły $n$-mostowe rozkładają się na sumę dwóch wymiernych $n$-supłów.
\index{supeł!wymierny}%

\begin{proposition}
\label{prp:bridge_additive}%
    Niech $K_1, K_2$ będą węzłami.
    Wtedy $\bridge (K_1) + \bridge(K_2) = \bridge(K_1 \# K_2) + 1$.
\end{proposition}

\begin{proof}[Niedowód]
    Schubert pokazał to blisko pół wieku temu w~\cite{schubert54}.
    Nowszy dowód pochodzi od Schultens, w~artykule \cite{schultens03} skorzystała z~foliacji na brzegu węzła towarzyszącego satelitarnemu.
    Dokładniejszy opis powyższych prac wykraczałby poza zakres tego opracowania, zostanie więc pominięty.
\end{proof}

Tylko jeden węzeł jest jednomostowy, to niewęzeł.
Kolejne w~hierarchii skomplikowania, czyli dwumostowe, to domknięcia wymiernych supłów.
Węzły trójmostowe pozostają nie do końca zbadane, Japończycy pokazali w~\cite{fukuhama99}, że trzymostowe węzły genusu jeden są preclami.
\index{genus}%
\index{precel}%
\index{węzeł!trzymostowy}%
Hilden, Montesinos, Tejada, Toro \cite{hilden12} klasyfikują wszystkie węzły trzymostowe przy użyciu tak zwanej reprezentacji motylkowej, podobną do wyniku Schuberta z~sekcji \ref{sub:twobridge}.
\index{liczba mostowa!węzeł trzymostowy}%
\index{węzeł!trzymostowy!see {liczba mostowa}}%
\index{reprezentacja motylkowa}%

Murasugi wspomina w rozdziale 4.3 podręcznika \cite{murasugi96} następującą hipotezę, nie podaje jednak wcale, skąd się wzięła:

\begin{conjecture}
\index{hipoteza!mostowo-skrzyżowaniowa}%
    Jeśli $K$ jest węzłem, to $\crossing K \ge 3 \bridge K - 3$, przy czym równość zachodzi dokładnie dla niewęzła, trójlistnika i~sumy spójnej trójlistników.
\end{conjecture}

Należy więc uzupełnić brakujące informacje.
Murasugi w pracy \cite{murasugi88} przypuszcza, że dla splotów o $\mu$ ogniwach zachodzi nierównosć $\crossing L + \mu - 1 \ge 3 \bridge L - 3$, przedstawia jednocześnie dowód jej szczególnego przypadku, dla alternujących splotów algebraicznych.
\index{człowiek!Murasugi, Kunio}%
Hipoteza Murasugiego stanowi uogólnienie dużo starszego problemu pochodzącego od Foxa \cite{fox50}, który zapytał, czy nierówność jest prawdziwa dla węzłów, gdy $\mu = 1$.
\index{człowiek!Fox, Ralph}%

\begin{proposition}
\index{liczba gordyjska}%
\label{no_relation_bridge_unknotting}%
    Nie istnieje bezpośredni związek między liczbą mostową oraz gordyjską.
\end{proposition}

\begin{proof}
    Węzły torusowe $T_{2,n}$ są dwumostowe, a~ich liczba gordyjska nieograniczona.

    Podwojenie węzła (poza specjalnymi przypadkami, jak pokazał Schubert) zwiększa liczbę mostową dwukrotnie; liczba gordyjska takiego podwojenia wynosi $1$.
\end{proof}

Podobnie nie ma zależności między liczbą mostową oraz genusem.
\index{genus}%

\index{liczba mostowa|)}%

% Koniec podsekcji Liczba mostowa

