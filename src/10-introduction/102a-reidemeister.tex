
\subsection{Ruchy Reidemeistera}

Każdy węzeł ma zatem wiele diagramów; mając dane dwa różne chcielibyśmy wiedzieć, czy przedstawiają ten sam węzeł.
Kurt Reidemeister podał proste kryterium rozstrzygające ten problem w~latach dwudziestych ubiegłego wieku.
\index[persons]{Reidemeister, Kurt}%
Najpierw zdefiniujmy trzy lokalne operacje na diagramach.

% DICTIONARY;Reidemeister/Turaev/... move;ruch Reidemeistera/Turajewa/...;-
\begin{definition}[ruchy Reidemeistera]
\index{ruch!Reidemeistera}%
    Trzy gatunki lokalnych deformacji diagramu splotu:
    \begin{figure}[H]
    \centering
    %
    \begin{minipage}[b]{.3\linewidth}
        \[
            \LargeReidemeisterOneLeft \stackrel{R_1}{\cong} \LargeReidemeisterOneStraight
        \]
        \subcaption{ruch $R_1$}
    \end{minipage}
    %
    \begin{minipage}[b]{.3\linewidth}
        \[
            \LargeReidemeisterTwoA \stackrel{R_2}{\cong} \LargeReidemeisterTwoB
        \]
        \subcaption{ruch $R_2$}
    \end{minipage}
    %
    \begin{minipage}[b]{.35\linewidth}
        \[
            \LargeReidemeisterThreeA \stackrel{R_3}{\cong} \LargeReidemeisterThreeB
        \]
        \subcaption{ruch $R_3$}
    \end{minipage}
    \end{figure}
    skręcenie lub rozkręcenie ($R_1$), wsunięcie lub rozsunięcie ($R_2$) oraz przesunięcie łuku przez skrzyżowanie ($R_3$) nazywamy ruchami Reidemeistera.
\end{definition}

Ruch $R_i$ operuje więc na $i$ łukach diagramu.
Reidemeister w~swojej pierwszej pracy przyjął inną kolejność, jego drugi ruch jest naszym pierwszym.
Colberg \cite[s. 6]{colberg13} pisze, że Maxwell znał ruchy Reidemeistera przed Reidemeisterem, ale mimo próśb Taita nigdy nie zgłosił swojego odkrycia w Royal Society of Edinburgh.

\begin{theorem}[Reidemeister, 1927]
\label{thm:reidemeister}%
\index{twierdzenie!Reidemeistera}%
\index[persons]{Reidemeister, Kurt}%
    Niech $D_1, D_2$ będą diagramami dwóch splotów $L_1, L_2$.
    Sploty $L_1, L_2$ są takie same wtedy i tylko wtedy, gdy diagram $D_2$ można otrzymać z $D_1$ wykonując skończony ciąg ruchów Reidemeistera oraz gładkich deformacji łuków, bez zmiany biegu skrzyżowań.
\end{theorem}

Dowód podali niezależnie Reidemeister \cite{reidemeister27} oraz Alexander, Briggs \cite{briggs27}.
\index[persons]{Briggs, Garland}%
\index[persons]{Alexander, James}%
Twierdzenie Reidemeistera jest prawdziwe także dla splotów zorientowanych, wtedy jednak w każdym ruchu trzeba uwzględnić wszystkie możliwe orientacje łuków.

\begin{proof}
\index[persons]{Burde, Gerhard}%
\index[persons]{Murasugi, Kunio}%
\index[persons]{Prasołow, Wiktor}%
\index[persons]{Sosiński, Aleksiej}%
\index[persons]{Zieschang, Heiner}%
    Szkielet dowodu można znaleźć w~książce Burdego i~Zieschanga \cite[s. 9-11]{burde14}.
    Kluczowe pomysły zawiera też \cite[s. 11-12]{prasolov97} Prasołowa i~Sosińskiego.
    Innym przystępnym źródłem jest podręcznik \cite[s. 50-56]{murasugi96} Murasugiego.
\end{proof}

Trace \cite{trace83} zauważył, że dwa diagramy jednego węzła są związane tylko II i III ruchem (ale nie I) wtedy i tylko wtedy, gdy mają ten sam spin oraz indeks punktu względem krzywej (,,winding number'').
\index[persons]{Trace, Bruce}%
Z prac Östlunda \cite{ostlund01}, Manturowa\footnote{Niestety nie wiem, o które strony tej książki chodzi.} \cite{manturov04} oraz Haggego \cite{hagge06} wynika, że dla każdego węzła istnieje para diagramów, do przejścia między którymi trzeba wykorzystać wszystkie trzy ruchy.
\index[persons]{Östlund, Olof}%
\index[persons]{Manturow, Wasilij}%
\index[persons]{Hagge, Tobias}%
% praca Haggego nazywa się "Every Reidemeister move is needed for each knot type" ale nawet w MathSciNecie wspomnieni są Ostlund i Manturow, więc zostawiam. Tekst skopiowany z Wiki
Coward \cite{coward06} zademonstrował, że nawet jeśli wszystkie trzy ruchy są potrzebne, można je wykonywać w specjalnej kolejności: najpierw tylko I ruchy, potem tylko II ruchy, następnie tylko III ruchy i znowu II ruchy.
\index[persons]{Coward, Alexander}%

W praktyce twierdzenia \ref{thm:reidemeister} nie stosuje się bezpośrednio do diagramów splotów.
Mając dane dwa spójne diagramy tego samego splotu trudno znaleźć jest ciąg ruchów przekształcający jeden z nich w drugi.
Załóżmy, że widać na nich odpowiednio $n_1, n_2$ skrzyżowań.
Jak piszą Coward, Lackenby w \cite{coward11}, istnieje funkcja $f \colon \N \times \N \to \N$ taka, że między dwoma diagramami można przejść wykonując co najwyżej $f(n_1, n_2)$ ruchów.
\index[persons]{Coward, Alexander}%
\index[persons]{Lackenby, Marc}%
Wynika to z oczywistego faktu, że istnieje skończenie wiele spójnych diagramów o danej liczbie skrzyżowań oraz twierdzenia Reidemeistera.
Okazuje się jednak, że od funkcji $f$ można żądać, by była obliczalna
(a to jest chyba równoważne istnienia algorytmu rozpoznającego, czy dwa diagramy przedstawiają jeden węzeł)
% http://people.dm.unipi.it/martelli/Cortona/Lackenby.pdf 7 of 90
i faktycznie, główny wynik \cite{coward11} orzeka, że
\begin{equation}
    f(n_1, n_2) = 2^{2^{\ldots^{2^{n_1 + n_2}}}}
\end{equation}
jest taką funkcją.
Piętrowa potęga liczy sobie aż $10^{1000000 (n_1 + n_2)}$ warstw.

Natomiast jeżeli $n_2 = 0$, czyli drugi diagram przedstawia niewęzeł, ,,wystarcza'' $(236n_1)^{11}$ ruchów, przy czym liczba skrzyżowań podczas transformacji nigdy nie przekracza $49c^2$: to świeższy wynik samego Lackenby'ego \cite{lackenby15}, gdzie poprawił wcześniejsze oszacowania Hassa, Lagariasa.

Hayashi \cite{hayashi05} dowiódł, że liczbę ruchów Reidemeistera potrzebnych, by rozszczepić splot można ograniczyć z góry na podstawie indeksu skrzyżowaniowego.
\index[persons]{Hayashi, Chuichiro}%

% DICTIONARY;invariant;niezmiennik;-
Zamiast tego definiuje się niezmienniki, czyli funkcje ze zbioru wszystkich diagramów, które nie zmieniają swojej wartości podczas wykonywania ruchów Reidemeistera.
Kiedy pewien niezmiennik przyjmuje różne wartości na dwóch diagramach, te przedstawiają dwa istotnie różne sploty.
Gdy wartości są te same, nie dostajemy żadnej informacji.
Sploty mogą być równoważne albo nie.
Niezmienniki będą nam stale towarzyszyć w~wędrówce po krainie węzłów.

% koniec sekcji Ruchy Reidemeistera