
\subsection{Ruchy Reidemeistera}

Wiemy już, że węzły mają wiele diagramów.
Mając dane dwa różne diagramy chcielibyśmy wiedzieć, czy przedstawiają ten sam węzeł.
Stosowne narzędzie dostarczył Kurt Reidemeister w~latach dwudziestych ubiegłego wieku.
\index[persons]{Reidemeister, Kurt}%
Zdefiniujmy trzy lokalne operacje na diagramach, a~potem wysłowimy kryterium  Reidemeistera rozstrzygające problem równości węzłów.

% DICTIONARY;Reidemeister/Turaev/... move;ruch Reidemeistera/Turajewa/...;-
\begin{definition}[ruchy Reidemeistera]
\index{ruch!Reidemeistera}%
    Trzy gatunki lokalnych deformacji diagramu splotu:
    \begin{figure}[H]
    \centering
    \begin{minipage}[b]{.22\linewidth}%
        \centering%
        \MedLarReidemeisterOneLeft $\stackrel{R_1}{\cong}$ \MedLarReidemeisterOneStraight%
        \subcaption{ruch $R_1$}%
    \end{minipage}
    \quad\quad\quad
    \begin{minipage}[b]{.2\linewidth}
        \centering
        \MedLarReidemeisterTwoA $\stackrel{R_2}{\cong}$ \MedLarReidemeisterTwoB
        \subcaption{ruch $R_2$}
    \end{minipage}
    \quad\quad\quad
    \begin{minipage}[b]{.32\linewidth}
        \centering
        \MedLarReidemeisterThreeA $\stackrel{R_3}{\cong}$ \MedLarReidemeisterThreeB
        \subcaption{ruch $R_3$}
    \end{minipage}
    \end{figure}
    nazywamy ruchami Reidemeistera.
    Czasami używa się konkretnych nazw:
    \begin{itemize}
        \item skręcenie/rozkręcenie (dla $R_1$),
        \item wsunięcie/rozsunięcie (dla $R_2$) oraz
        \item przesunięcie łuku przez skrzyżowanie (dla $R_3$).
    \end{itemize}
\end{definition}

Reidemeister w~swojej pierwszej pracy przyjął inną kolejność, jego drugi ruch jest naszym pierwszym.
Dzięki temu ruch $R_k$ operuje na $k$ łukach diagramu.
Colberg \cite[s. 6]{colberg2013} pisze, że Maxwell znał ruchy Reidemeistera przed Reidemeisterem, ale mimo próśb Taita nigdy nie zgłosił swojego odkrycia w Royal Society of Edinburgh.
\index[persons]{Tait, Peter}%
\index[persons]{Maxwell, James}%

\begin{theorem}[Reidemeister, 1927]
\label{thm:reidemeister}%
\index{twierdzenie!Reidemeistera}%
\index[persons]{Reidemeister, Kurt}%
    Niech $D_1, D_2$ będą diagramami dwóch splotów $L_1, L_2$.
    Sploty $L_1, L_2$ są takie same wtedy i tylko wtedy, gdy diagram $D_2$ można otrzymać z $D_1$ wykonując skończony ciąg ruchów Reidemeistera oraz gładkich deformacji łuków, bez zmiany biegu skrzyżowań.
\end{theorem}
% https://math.stackexchange.com/questions/4399634/two-knots-k-and-k-prime-are-equivalent-if-and-only-if-their-projections-p
% Reidemeister, and pretty much every other author, has worked with the piecewise-linear case. In a way it doesn't matter which you choose, since there's a theorem that the categories of smooth and PL manifolds are equivalent in some sense. However, it's not so clear how you pass from one setting to the other (or at least I've never gone through the details myself!)

Twierdzenie Reidemeistera jest prawdziwe także dla splotów zorientowanych, ale wtedy trzeba uwzględnić różne orientacje łuków i~nie jest oczywiste, ile spośród $2^1 + 2^2 + 2^3 = 14$ wersji jest potrzebne.
Polyak \cite{polyak2010} pokazał, że minimalny zbiór zorientowanych ruchów składa się na przykład z~dwóch wersji ruchu $R_1$, jednej wersji ruchu $R_2$ i~jednej wersji ruchu $R_3$.
\index[persons]{Polyak, Michael}%

\begin{proof}[Niedowód]
Dowód podali niezależnie Reidemeister \cite{reidemeister1927} oraz Alexander, Briggs \cite{alexander1927}.
\index[persons]{Reidemeister, Kurt}%
\index[persons]{Briggs, Garland}%
\index[persons]{Alexander, James}%
    Szkielet dowodu można znaleźć w~książce Burdego i~Zieschanga \cite[s. 9-11]{burde2014}, ale kluczowe pomysły podają też Prasołow z~Sosińskim \cite[s. 11-12]{prasolov1997}.
\index[persons]{Burde, Gerhard}%
\index[persons]{Zieschang, Heiner}%
\index[persons]{Prasołow, Wiktor (Прасолов, Виктор Васильевич)}%
\index[persons]{Sosiński, Aleksiej (Сосинский, Алексей Брониславович)}%
    Innym przystępnym źródłem jest podręcznik Murasugiego \cite[s. 50-56]{murasugi1996}.
\index[persons]{Murasugi, Kunio}%
\end{proof}

Trace \cite{trace1983} zauważył, że dwa diagramy jednego węzła są związane tylko II i III ruchem (ale nie I) wtedy i tylko wtedy, gdy mają ten sam spin oraz indeksy nawinięcia.
\index[persons]{Trace, Bruce}%
Z prac Östlunda \cite{ostlund2001}, Manturowa \cite[s. ???]{manturov2004} oraz Haggego \cite{hagge2006} wynika, że dla każdego węzła istnieje para diagramów, do przejścia między którymi trzeba wykorzystać wszystkie trzy ruchy.
% TODO: ustalić, które strony w Manturowie
\index[persons]{Östlund, Olof}%
\index[persons]{Manturow, Wasilij}%
\index[persons]{Hagge, Tobias}%
% praca Haggego nazywa się "Every Reidemeister move is needed for each knot type" ale nawet w MathSciNecie wspomnieni są Ostlund i Manturow, więc zostawiam. Tekst skopiowany z Wiki
Coward \cite{coward2006} zademonstrował, że nawet jeśli wszystkie trzy ruchy są potrzebne, można je wykonywać w specjalnej kolejności: najpierw tylko I ruchy, potem tylko II ruchy, następnie tylko III ruchy i~znowu II ruchy.
\index[persons]{Coward, Alexander}%

Do pokazania, że dwa węzły $K_1, K_2$ nie są równoważne, powinniśmy na mocy twierdzenia \ref{thm:reidemeister} udowodnić, że żaden ciąg ruchów Reidemeistera nie przekształca jednego w drugi.
Oczywiście nikt o zdrowych zmysłach tak nie postępuje.
Zamiast tego wprowadza się stosowny niezmiennik, czyli funkcję $f$ określoną na zbiorze wszystkich węzłów (albo splotów, supłów, warkoczy itd.) tak, że jeśli węzły $K_1 \cong K_2$ są równoważne, to $f(K_1) = f(K_2)$.
% DICTIONARY;invariant;niezmiennik;-
Łatwo widać, że jeśli $f(K_1) \neq f(K_2)$, to węzły $K_1, K_2$ nie mogą być równoważne.
Natomiast gdy wartości są te same, nie dostajemy żadnej informacji.

Poznaliśmy jak na razie dwa niezmienniki: liczbę ogniw splotu oraz dopełnienie splotu do przestrzeni, w której jest zanurzony ($\mathbb R^3$ lub $S^3$).
Wiele, chociaż nie wszystkich, innych niezmienników definiuje się nie bezpośrednio na zbiorze węzłów, ale na zbiorze diagramów.
Należy wtedy sprawdzić, że każdy z trzech ruchów Reidemeistera nie ma wpływu na wartość niezmiennika.

Niezmienniki będą nam stale towarzyszyć w~wędrówce po krainie węzłów.

\begin{remark}[Kurt Werner Friedrich Reidemeister]
    ?
\end{remark}

\begin{remark}[James Waddell Alexander]
    ?
\end{remark}

\begin{remark}[Garland Baird Briggs]
    Matematyk amerykański, urodzon w Sebrell, Wirginii w 1894 roku; zmarł w Kolumbii w 1959 roku.
    Niestety nie wiemy za dużo o tym człowieku.
\end{remark}

\color{white}

\subsubsection{Dygresja -- wyniki ilościowe wokół twierdzenia Reidemeistera}
Załóżmy, że na dwóch diagramach tego samego węzła widać odpowiednio $n_1, n_2$ skrzyżowań.
Jak piszą Coward, Lackenby \cite{coward2011}, istnieje funkcja $f \colon \N \times \N \to \N$ taka, że między dwoma diagramami można przejść wykonując co najwyżej $f(n_1, n_2)$ ruchów.
\index[persons]{Coward, Alexander}%
\index[persons]{Lackenby, Marc}%
Wynika to z faktu, że istnieje skończenie wiele spójnych diagramów o danej liczbie skrzyżowań oraz twierdzenia Reidemeistera.
Okazuje się jednak, że od funkcji $f$ można żądać, by była obliczalna
(a to jest chyba równoważne istnienia algorytmu rozpoznającego, czy dwa diagramy przedstawiają jeden węzeł)
% http://people.dm.unipi.it/martelli/Cortona/Lackenby.pdf 7 of 90
i faktycznie, główny wynik \cite{coward2011} orzeka, że
\begin{equation}
    f(n_1, n_2) = 2^{2^{\ldots^{2^{n_1 + n_2}}}}
\end{equation}
jest taką funkcją.
Piętrowa potęga liczy sobie aż $10^{1000000 (n_1 + n_2)}$ warstw.

Natomiast jeżeli $n_2 = 0$, czyli drugi diagram przedstawia niewęzeł, ,,wystarcza'' $(236n_1)^{11}$ ruchów, przy czym liczba skrzyżowań podczas transformacji nigdy nie przekracza $49c^2$: to świeższy wynik samego Lackenby'ego \cite{lackenby2015}, gdzie poprawił wcześniejsze oszacowania Hassa, Lagariasa.
Przykład diagramu niewęzła, do rozwiązania którego nie można tylko usuwać istniejących skrzyżowań, przedstawiają Burde, Zieschang, Heusener \cite[s. 12]{burde2014}.

Hayashi \cite{hayashi2005} dowiódł, że liczbę ruchów Reidemeistera potrzebnych, by rozszczepić splot można ograniczyć z góry na podstawie indeksu skrzyżowaniowego.
\index[persons]{Hayashi, Chuichiro}%

% koniec sekcji Ruchy Reidemeistera

