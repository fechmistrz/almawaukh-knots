\subsection{Hipotezy Taita}
\begin{conjecture}[I hipoteza Taita]
    \index{hipoteza!Taita}
    \label{con:tait_1}
    Zredukowany alternujący diagram splotu ma minimalny indeks skrzyżowaniowy.
\end{conjecture}
% To bardzo ważny rezultat, którego prawdziwość przypuszczał już P. G. Tait w~XIX wieku.
% Nikt nie był w~stanie podać dowodu przed pojawieniem się wielomianu Jonesa.

Najpierw znaleziono dowód korzystający z wielomianu Jonesa: dokonali tego w 1987 roku równocześnie Kauffman \cite{kauffman87}, Murasugi \cite{murasugi87} oraz Thistlethwaite \cite{thistlethwaite87}.
Trzydzieści lat później Greene zaprezentował geometryczne podejście do problemu w \cite{greene17}.

\begin{conjecture}[II hipoteza Taita]

    Achiralny splot alternujący ma zerowy spin.
\end{conjecture}

Pierwsze dowody pochodzą znowu od Kauffmana \cite{kauffman87} oraz Thistlethwaite'a \cite{thistlethwaite87}.

\begin{conjecture}[III hipoteza Taita]

    Niech $D_1, D_2$ będą zredukowanymi alternującymi diagramami zorientowanego pierwszego splotu.
    Wtedy diagram $D_2$ można otrzymać z~$D_1$ korzystając jedynie z~ruchu \emph{flype}.
\end{conjecture}

Trzecią hipotezę udowodnił Menasco wspólnie z~Thistlethwaitem, \cite{menasco91}.
Wynika z~niej, że dwa zredukowane diagramy alternujące tego samego węzła mają ten sam spin: ruch flype nie zmienia spinu (dla niektórych to jest II hipoteza).
Pzedstawimy ze szczegółami dowód pierwszej hipotezy w~sekcji \ref{sub:tait_conjectures} oraz wspomnimy krótko o technikach użytych w dowodach pozostałych dwóch.