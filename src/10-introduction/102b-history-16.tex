
\subsubsection{Szesnaście skrzyżowań}
Około roku 1998 Hoste z~Weeksem oraz niezależnie Thistlethwaite \cite{thistlethwaite1998} znaleźli 1 701 936 pierwszych węzłów do 16 skrzyżowań.
\index[persons]{Hoste, Jim}%
\index[persons]{Thistlethwaite, Morwen}%
\index[persons]{Weeks, Jeff}%
Używali przy tym różnych podejść: Hoste z Weeksem wykorzystywali niezmienniki hiperboliczne oraz program SnapPea; Thistlethwaite natomiast wzbogacił zestaw ruchów Reidemeistera o flype, przejście, 2-przejście, ruch Perko oraz kilka innych ezoterycznych przekształceńtak, żeby poradzić sobie z upartymi parami diagramów.
Wśród pierwszych węzłów do 16 skrzyżowań na 32 wyjątkowe (niehiperboliczne) węzły składa się 12 węzłów torusowych oraz 20 satelitów trójlistnika.
To samo jeszcze raz napiszemy na stronie \pageref{page:nonhyperbolic_below_16}.

