
Chociaż w~świetle definicji \ref{def:knot} węzły są pewnymi regularnymi podzbiorami przestrzeni $\R^3$, z~kombinatorycznego punktu widzenia wygodniej jest rysować je na płaszczyźnie.

% DICTIONARY;oriented;zorientowany;węzeł
\begin{definition}[orientacja]
\index{węzeł!zorientowany}%
\index{orientacja|see {węzeł zorientowany}}%
    Węzeł, w~którym wybrano kierunek, w~którym należy się po nim poruszać, nazywamy zorientowanym.
    Splot nazywamy zorientowanym, jeśli wszystkie jego ogniwa traktowane jako węzły są zorientowane.
\end{definition}

Orientację na diagramie zaznaczamy małą strzałką wskazującą kierunek poruszania się.

% DICTIONARY;shadow;cień;-
\begin{definition}
\index{cień}%
    Rzut węzła $K \subseteq \R^3$ na płaszczyznę nazywamy cieniem.
\end{definition}

% DICTIONARY;crossing;skrzyżowanie;-
\begin{definition}[skrzyżowanie]
\index{skrzyżowanie}%
    Podwójny punkt w cieniu nazywamy skrzyżowaniem.
\end{definition}

% DICTIONARY;diagram;diagram;-
\begin{definition}[diagram]
\index{diagram}%
    Cień razem z~informacją o~tym, jak przebiegają skrzyżowania i pozbawiony katastrof: potrójnych przecięć, stycznych czy dziobów nazywamy diagramem.
    % TODO: Narysować katastrofy
\end{definition}

\begin{definition}[włókno]
\index{włókno}%
    Fragment diagramu, który biegnie między dwoma kolejnymi tunelami, czyli podskrzyżowaniami, nazywamy włóknem.
\end{definition}

\begin{definition}[nić]
\index{nić}%
    Fragment diagramu, który biegnie między dwoma kolejnymi skrzyżowaniami, nazywamy nicią.
\end{definition}

Nici powstają z włókien przez rozcięcie ich przy każdym nadskrzyżowaniu.

\begin{proposition}
    Niech $L$ będzie splotem.
    Jego diagramy tworzą otwarty i~gęsty podzbiór wszystkich rzutów.
\end{proposition}

Kawauchi \cite[s. 7]{kawauchi96} wspomina w tym miejscu podręcznik Crowella, Foxa \cite[s. 7]{crowell63}.
To samo jest na przykład u Burdego, Zieschanga, Heusenera \cite[s. 10]{burde14}.

\begin{proof}
    Rzut splotu na równoległe płaszczyzny jest taki sam, a te można sparametryzować prostymi przechodzącymi przez początek układu współrzędnych, które tworzą przestrzeń rzutową $\R \mathbb P^2$.
    Niech $S$ będzie zbiorem prostych, które dają złe rzuty.
    Wystarczy pokazać jego nigdziegęstość.
    Okazuje się, że $S$ jest też jednowymiarowy.
\end{proof}

\begin{corollary}
    Każdy splot posiada diagram.
\end{corollary}

