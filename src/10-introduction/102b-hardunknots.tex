
\section{Wykrywanie niewęzła}
Jednym z pierwszych dużych problemów teorii węzłów było poszukiwanie odpowiedzi na pytanie, kiedy diagram przedstawia niewęzeł.
\index{niewęzeł}%
Stosowny algorytm wykrywający niewęzły podał Haken \cite{haken1961}, ale długo nikt nie podjął się jego implementacji.
\index[persons]{Haken, Wolfgang}%
Epple pisze \emph{,,this algorithm was extremely impractical''}, w recenzji z MathSciNet proponuje, żeby przed przeczytaniem pełnej niepotrzebnych dygresji pracy Hakena poznać artykuł \cite{schubert1961} Schuberta.
\index[persons]{Epple, Moritz}%
\index[persons]{Schubert, Horst}%
W życie pomysły Hakena udało się wdrożyć Burtonowi, Budneyowi oraz Petterssonowi w~komputerowym programie Regina\footnote{\url{https://regina-normal.github.io/}.} na przełomie tysiącleci.

\index[persons]{Burton, Benjamin}%
\index[persons]{Budney, Ryan}%
\index[persons]{Pettersson, William}%
%=% https://mathscinet.ams.org/mathscinet-getitem?mr=141107
% DICTIONARY;incompressible;nieściśliwy;-

Burton, Rubinstein i~Tillman \cite{burton2012} pokazali, jak sprawdzać, w~czasie wykładniczym czy powierzchnia normalna na striangulowanej 3-rozmaitości jest (nie)ściśliwa.
\index[persons]{Rubinstein, Joachim}%
\index[persons]{Tillman, Stephan}%
To wystarczyło do udzielenia negatywnej odpowiedzi na pytanie Thurstona: \emph{,,czy przestrzeń Seiferta-Webera jest rozmaitością Hakena?''}, a zatem wykraczającego poza poziom naszego skromnego dzieła.
\index[persons]{Thurston, William}%
\index{przestrzeń!Seiferta-Webera}%
\index{rozmaitość!Hakena}%

SnapPea\footnote{\url{http://geometrygames.org/SnapPea/index.html}.} to inny popularny wśród niskowymiarowych topologów program pozwalający badać hiperboliczne 3-rozmaitości, patrz sekcja \ref{sec:hyperbolic}.

Wiadomo, że genus oraz zredukowana kohomologia Chowanowa wykrywa niewęzły (fakty \ref{prp:genus_detects_unknot}, \ref{khovanov_detects_unknot}) i nie wiadomo, czy wielomian Jonesa to robi (hipoteza \ref{con:jones}).
\index{genus}%
\index{homologia!Chowanowa}%
\index{wielomian!Jonesa}%
Od dawna wiadomo, że wielomian Alexandera nie wykrywa niewęzła (fakt \ref{alexander_no_detects_unknot}).
\index{wielomian!Alexandera}%
W lutym 2021 Lackenby ogłosił nowy algorytm rozpoznający niewęzły w~quasiwielomianowym czasie.
% i nie zrobił tego w THE EFFICIENT CERTIFICATION OF KNOTTEDNESS AND THURSTON NORM, bo to wyszło na arxiv w 2016
\index[persons]{Lackenby, Marc}%

Wyśmienitym punktem wyjścia do poszukiwań trudnych niewęzłów jest praca \cite{schleimer2021}, dzieło Burtona, Changa, Löfflera, de Mesmaya, Marii, Schleimera, Sedgwicka oraz Spreera.
\index[persons]{Burton, Benjamin}%
\index[persons]{Chang, Hsien-Chih}%
\index[persons]{Mesmay, Arnaud@de Mesmay, Arnaud}%
\index[persons]{Löffler, Maarten}%
\index[persons]{Maria, Clément}%
\index[persons]{Schleimer, Saul}%
\index[persons]{Sedgwick, Eric}%
\index[persons]{Spreer, Jonatha}%
Cytuje ona artykuł Lackenby'a \cite{lackenby2015}, gdzie poznaliśmy stary (z 1934 roku!) przykład Goeritza \cite{goeritz1934} diagramu niewęzła o~11 skrzyżowaniach, który można zmienić w~zwykły diagram niewęzła tylko zwiększając po drodze liczbę skrzyżowań.
% 9781470454999 s. 5
\index[persons]{Goeritz, Lebrecht}%
\index{niewęzeł!Goeritza}%
Na kolejne teksty przyszło poczekać ponad pół wieku.
Autorzy przywołują jeszcze klasyczny przykład Freedmana, He, Wanga \cite{freedman1994}; ale też podstępne niewęzły Hakena, Ochiai, Thistlethwaite'a oraz mocno doświadczalną pracę Petronio, Zanellatiego \cite{zanellati2016}.
\index[persons]{Freedman, Michael}%
\index[persons]{He, Zheng-Xu}%
\index[persons]{Wang, Zhenghan}%
\index{niewęzeł!Freedmana}%
\index[persons]{Petronio, Carlo}%
\index[persons]{Zanellati, Adolfo}%

% TODO: sprawdzić, czy w \cite{heinrich2014} nie ma więcej trudnych niewęzłów

My nie potrafimy albo nie chcemy potrafić ładnie rysować, więc pozwolimy sobie pokazać tylko, jak wyglądał wspomniany wcześniej przykład Goeritza.
Na swoim blogu\footnote{\url{https://mickburton.co.uk/2015/06/05/how-do-you-construct-hakens-gordian-knot/}} Burton (inny Burton niż w poprzednim akapicie!) zamieścił wpis pełen rysunków, które tłumaczą, jak powstał niewęzeł Hakena.

\begin{comment}
\begin{figure}[H]
    \centering
    \begin{tikzpicture}[baseline=-0.65ex, scale=0.1]
        \begin{knot}[clip width=5, end tolerance=1pt, flip crossing/.list={1,2,3,4,8,9}]
            % horizontal lines
            \strand[ultra thick] (-40, -10) to (-10, -10);
            \strand[ultra thick] (-40, 10) to (-10, 10);
            \strand[ultra thick] (10, 10) to (30, 10);
            \strand[ultra thick] (10, -10) to (30, -10);
            % 
            \strand[ultra thick] (5-45, -10) [in=left,out=left] to (5-45, 3.33);
            \strand[ultra thick] (5-45, 10) [in=left,out=left] to (5-45, -3.33);
            \strand[ultra thick] (-5-35, -3.33) [in=left,out=right] to (5-35, 3.33);
            \strand[ultra thick] (-5-35, 3.33) [in=left,out=right] to (5-35, -3.33);
            \strand[ultra thick] (-5-25, -3.33) [in=left,out=right] to (5-25, 3.33);
            \strand[ultra thick] (-5-25, 3.33) [in=left,out=right] to (5-25, -3.33);
            \strand[ultra thick] (-5-15, -3.33) [in=left,out=right] to (5-15, 3.33);
            \strand[ultra thick] (-5-15, 3.33) [in=left,out=right] to (5-15, -3.33);
            %
            \strand[ultra thick] (-5+15, -3.33) [in=left,out=right] to (5+15, 3.33);
            \strand[ultra thick] (-5+15, 3.33) [in=left,out=right] to (5+15, -3.33);
            \strand[ultra thick] (-5+25, -3.33) [in=left,out=right] to (5+25, 3.33);
            \strand[ultra thick] (-5+25, 3.33) [in=left,out=right] to (5+25, -3.33);
            \strand[ultra thick] (-5+35, -3.33) [in=right,out=right] to (-5+35, 10);
            \strand[ultra thick] (-5+35, 3.33) [in=right,out=right] to (-5+35, -10);
            %
            \strand[ultra thick] (-5+5, 3.33) [in=left,out=right] to (5+5, 10);
            \strand[ultra thick] (-5+5, 10) [in=left,out=right] to (5+5, 3.33);
            \strand[ultra thick] (-5-5, 3.33) [in=left,out=right] to (5-5, 10);
            \strand[ultra thick] (-5-5, 10) [in=left,out=right] to (5-5, 3.33);
            %
            \strand[ultra thick] (-5+5, -10) [in=left,out=right] to (5+5, -3.33);
            \strand[ultra thick] (-5+5, -3.33) [in=left,out=right] to (5+5, -10);
            \strand[ultra thick] (-5-5, -10) [in=left,out=right] to (5-5, -3.33);
            \strand[ultra thick] (-5-5, -3.33) [in=left,out=right] to (5-5, -10);
        \end{knot}
    \end{tikzpicture}
    \caption{niewęzeł Goeritza}
\end{figure}
\end{comment}

