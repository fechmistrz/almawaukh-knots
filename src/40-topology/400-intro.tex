
W tym rozdziale poznamy niezmienniki wywodzące się z~topologii algebraicznej przy użyciu maszynerii topologii algebraicznej, na tyle, na ile to możliwe.
Zaczniemy od grupy splotu (czyli grupy jego dopełnienia), potem poznamy jej prezentację Wirtingera i~jeszcze raz spotkamy pochodną Foxa.
Następnie odkryjemy powierzchnie Seiferta, jeszcze jedno ,,źródło'' genusu, wyznacznika, sygnatury, niezmiennika Arfa czy przede wszystkim wielomianu Alexandera.
Na koniec powiemy krótko, czym są homologie, w szczególności homologie Chowanowa.

% koniec wstępu do rozdziału 4: topologia

