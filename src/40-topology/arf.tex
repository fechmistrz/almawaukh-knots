\section{Niezmiennik Cahita Arfa} % (fold)
\label{sub:arf}
Niezmiennik Arfa dla węzłów można zdefiniować na kilka sposobów, z~których żaden jest istotnie lepszy od pozostałych.
Pierwszy był Robertello \cite{robertello65}:

\begin{proposition}[Robertello, 1965]
    Niech
    \begin{equation}
        \alexander (t)=c_{0}+c_{1}t+\cdots +c_{n}t^{n}+\cdots +c_{0}t^{2n}
    \end{equation}
    będzie wielomianem Alexandera.
    Wtedy niezmiennik Arfa to $c_{n-1}+c_{n-3}+\cdots +c_{r}\mod 2$, gdzie $r = 0$ dla nieparzystych $n$, $r = 1$ w~przeciwnym razie.
\end{proposition}

Nieco później Murasugi \cite{murasugi69} zauważył, że warunek można uprościć:

\begin{proposition}[Murasugi, 1969]
    \label{arf_murasugi}
    $\operatorname{Arf}(K) = 0$ wtedy i~tylko wtedy, gdy $\alexander_K(-1) \equiv \pm 1 \mod 8$.
\end{proposition}

Louis Kauffman zaproponował inne podejście w~1983 roku, z wykorzystaniem diagramów.
Dwa węzły nazwiemy równoważnymi przez przejścia, jeśli są związane skończenie wieloma ,,przejściami'' \cite[s. 143]{kauffman83}:
\begin{comment}
\[
    \begin{tikzpicture}[baseline=-0.65ex,scale=0.35]
    \begin{knot}[clip width=7]
        \strand[-latex, thick] (-2.5,-1.0) to (2.5,-1.0);
        \strand[-latex, thick] (2.5,1.0) to (-2.5,1.0);
        \strand[-latex, thick] (-1.0,-2.5) to (-1.0,2.5);
        \strand[-latex, thick] (1.0,2.5) to (1.0,-2.5);
    \end{knot}
    \end{tikzpicture}
    \cong
    \begin{tikzpicture}[baseline=-0.65ex,scale=0.35]
    \begin{knot}[clip width=7, flip crossing/.list={1,2,3,4}]
        \strand[-latex, thick] (-2.5,-1.0) to (2.5,-1.0);
        \strand[-latex, thick] (2.5,1.0) to (-2.5,1.0);
        \strand[-latex, thick] (-1.0,-2.5) to (-1.0,2.5);
        \strand[-latex, thick] (1.0,2.5) to (1.0,-2.5);
    \end{knot}
    \end{tikzpicture}
    \quad\mbox{albo}\quad
    \begin{tikzpicture}[baseline=-0.65ex,scale=0.35]
    \begin{knot}[clip width=7]
        \strand[-latex, thick] (-2.5,-1.0) to (2.5,-1.0);
        \strand[-latex, thick] (2.5,1.0) to (-2.5,1.0);
        \strand[latex-, thick] (-1.0,-2.5) to (-1.0,2.5);
        \strand[latex-, thick] (1.0,2.5) to (1.0,-2.5);
    \end{knot}
    \end{tikzpicture}
    \cong
    \begin{tikzpicture}[baseline=-0.65ex,scale=0.35]
    \begin{knot}[clip width=7, flip crossing/.list={1,2,3,4}]
        \strand[-latex, thick] (-2.5,-1.0) to (2.5,-1.0);
        \strand[-latex, thick] (2.5,1.0) to (-2.5,1.0);
        \strand[latex-, thick] (-1.0,-2.5) to (-1.0,2.5);
        \strand[latex-, thick] (1.0,2.5) to (1.0,-2.5);
    \end{knot}
    \end{tikzpicture}
\]
\end{comment}

\begin{definition}[Kauffman, 1983]
    \index{niezmiennik!Arfa}
    Niech $K$ będzie węzłem.
    Jeśli jest równoważny przez przejścia z niewęzłem, to mówimy że jego niezmiennik Arfa znika: $\operatorname{Arf} K = 0$.
    W przeciwnym razie węzeł $K$ jest równoważny z trójlistnikiem, wtedy kładziemy $\operatorname{Arf} K = 1$.
\end{definition}

Wreszcie Jones zauważył \cite{jones85}, że wielomian $\jones$ także pozwala na określenie niezmiennika Arfa, dzięki zespolonym algebrom Clifforda oraz pracy \cite{lannes85}.
Jest to jedyna definicja, którą łatwo rozszerzyć do splotów.

\begin{proposition}[Jones, 1985]
    $\operatorname{Arf}(K) = \jones_K(i)$.
\end{proposition}

Na zakończenie zostawiliśmy definicję zakorzenioną w topologii algebraicznej.

\begin{proposition}
    Niech $(v_{ij})$ będzie macierzą Seiferta powstałą z~krzywych genusu $g$, które reprezentują bazę pierwszej grupy homologii powierzchni.
    To oznacza, że macierz $V$ wymiaru $2g \times 2g$ ma następującą własność: różnica $V - V^t$ jest symplektyczna.
    Niezmiennik Arfa to
    \begin{equation}
        \sum^g_{i=1}v_{2i-1,2i-1}v_{2i,2i} \pmod 2.
    \end{equation}
\end{proposition}

\begin{proposition}
    Niezmiennik Arfa jest $\shrap$-addytywny.
\end{proposition}

O niezmienniku Arfa usłyszymy jeszcze poznając węzły plastrowe, w~sekcji \ref{sec:slice}.

\begin{tobedone}
https://macsphere.mcmaster.ca/handle/11375/25082 Definition 2.20. Let F be a Seifert surface of a classical knot K. The homology group H 1 (F,Z/2) has a quadratic form q represented by 1 2 (V + V T ) (mod 2). The Arf invariant of K is the Arf invariant of q. The geometric interpretation of ...
% q : H 1 (F,Z/2) → Z/2
is that q(a) measures the number of full twists modulo 2 of the band in a neighborhood of a. For example, the knot in Figure 2.9 has Arf invariant equal to 1. Note that the Arf invariant does not change if we choose a different basis. The Arf invariant has not been extended to virtual knots; see [Chr17] and [FIKM14, Sec.8.2.3]. But it is known not to extend to welded knots, and in Section 4.2 we will see that two knots are p-move equivalent if and only if they have the same Arf invariant.
\end{tobedone}


% OEIS http://oeis.org/A131433, ...1434
% Koniec sekcji Niezmiennik Cahita Arfa
