
\subsection{Macierz Seiferta}
\index{macierz Seiferta|(}

Macierz opisana w tej podsekcji nie jest niezmiennikiem węzłów, ale prowadzi do wielomianu Alexandera (i dwóch nowych niezmienników: sygnatury i niezmiennika Arfa).

Niech $K$ będzie węzłem z diagramem $D$ i powierzchnią Seiferta $S$.

% Murasugi, s. 79
\begin{definition}[graf Seiferta]
\index{graf Seiferta}%
    Graf otrzymany przez ściągnięcie dysków z dowodu faktu \ref{prp:seifert_exists} do punktów oraz skurczenie doklejonych pasków do krawędzi nazywamy grafem Seiferta diagramu $D$.
\end{definition}

Murasugi \cite[s. 79]{murasugi1996} proponuje jako ćwiczenie dowód faktu:

\begin{proposition}
    Każdy graf Seiferta jest dwudzielny i planarny.
\end{proposition}

% Murasugi 82, 83
Skoro graf Seiferta jest planarny, to dzieli sferę $S^2$ na $f$ obszarów.
Można wyznaczyć ich liczbę: skoro $\chi(S^2) = d - b + f = 2$, to $f - 1 = 1 - d + b$, pomijamy obszar nieograniczony.
Brzeg każdego obszaru jest zamkniętą krzywą, z których tworzymy krzywe $x_1, \ldots, x_m$ na powierzchni Seiferta.
Generują one grupę podstawową $\pi_1(S)$.

Niech $S$ będzie powierzchnią Seiferta z wyróżnioną wierzchnią stroną.
Jeśli krzywa $x_i$ biegnie po powierzchni $S$, przez $x_i^*$ oznaczać będziemy dodatnie wypchnięcie: równoległą do $x_i$ krzywą, która biegnie tuż nad nią.
(W razie problemów z wyobrażeniem sobie tego, można wykonać papierowy model powierzchni Seiferta albo popatrzeć na rysunek w \cite[s. 111]{livingston1993} albo zrobić jedno i drugie).

\begin{definition}[macierz Seiferta]
\index{macierz!Seiferta}%
    Przy zachowaniu powyższych oznaczeń, macierz, której wyrazy określa wzór $M_{i,j} = \operatorname{lk}(x_i, x_j^*)$, nazywamy macierzą Seiferta.
\end{definition}

Konstrukcja macierzy Seiferta zależy od wyboru diagramu oraz orientacji krzywych $x_i$, dlatego nie jest niezmiennikiem węzłów.
Stanie się nim, kiedy uwzględnimy jeszcze wpływ ruchów Reidemeistera.

\begin{proposition}
    Kwadratowa macierz $V$ o całkowitych wyrazach jest macierzą Seiferta węzła wtedy i~tylko wtedy, gdy $\det(V - V^t) = 1$.
\end{proposition}

\begin{proof}
    Kawauchi \cite[s. 62]{kawauchi1996} pisze, że wynika to z~klasyfikacji macierzy Seiferta splotów.
\end{proof}

\begin{definition}
\index{macierz Seiferta!S-równoważność}%
\index{S-równoważność (macierzy Seiferta)}%
    Dwie macierze $M_1, M_2$ nazywamy $S$-równoważnymi, jeśli jedną z nich daje się otrzymać z~drugiej poprzez wykonanie skończonego ciągu operacji $\Lambda_1$, $\Lambda_2$ oraz ich odwrotności.

    Operacji $\Lambda_2$ nie można krótko opisać słowami, ale na szczęście można wzorem: niech $*$ będzie dowolną liczbą całkowitą, wtedy
    \begin{equation}
        \Lambda_2 (M) = \begin{bmatrix}
  &   &  & 0 & 0 \\
  & M &  & \vdots & \vdots \\
  &   &  & 0 & 0 \\
* & \dots & * & 0 & 0 \\
0 & \dots & 0 & 1 & 0
\end{bmatrix} \textrm{albo} \begin{bmatrix}
  &   &  & * & 0 \\
  & M &  & \vdots & \vdots \\
  &   &  & * & 0 \\
0 & \dots & 0 & 0 & 1 \\
0 & \dots & 0 & 0 & 0
\end{bmatrix}.
    \end{equation}
    Operacja $\Lambda_1$ polega na zastąpieniu macierzy $M$ przez $PMP^t$, gdzie $P$ jest dowolną odwracalną macierzą o całkowitych wyrazach (wtedy $\det P = \pm 1$).
\end{definition}

Badania powyższej relacji równoważności prowadzili w~latach sześćdziesiątych ubiegłego stulecia Trotter \cite{trotter1962}, Murasugi \cite{murasugi1965} oraz Levine \cite{levine1970}.
\index[persons]{Trotter, Hale}%
\index[persons]{Murasugi, Kunio}%
\index[persons]{Levine, Jerome}%
Litera $S$, jak nietrudno się domyślić, pochodzi od Seiferta.

\begin{proposition}
    Macierz Seiferta modulo $S$-równoważność jest niezmiennikiem splotów.
\end{proposition}

Dowód tego faktu jest elementarny, ale dość długi.
Razem z~ułatwiającymi zrozumienie diagramami można znaleźć go w podręczniku Kawauchiego \cite[s. 64]{kawauchi1996} albo Murasugiego, dlatego pomijamy go i skupimy się na tym, jakie niezmienniki można otrzymać z macierzy Seiferta.

Wyznacznik całej macierzy Seiferta nie jest niezmiennikiem.
Wykonując operację $\Lambda_2$ dostajemy macierz, której ostatnia kolumna albo ostatni wiersz są zerami, więc jej wyznacznik także jest zerem.
Jeśli jednak najpierw dokonamy jej symetryzacji, dostaniemy znany już niezmiennik.

\begin{proposition}
\index{wyznacznik}%
    Niech $M$ będzie macierzą Seiferta węzła $K$.
    Wtedy
    \begin{equation}
        \det K = |\det(M + M^t)|.
    \end{equation}
\end{proposition}

\index{wielomian!Alexandera}%
Przez wprowadzenie dodatkowej zmiennej $t \in \R$, ponownie uogólnimy wyznacznik do wielomianu Alexandera.

\begin{proposition}
    Niech $M$ będzie macierzą Seiferta stopnia $k$ węzła $K$.
    Wtedy
    \begin{equation}
        \alexander_K (t) = t^{-k/2}\det(M - tM^t).
    \end{equation}
\end{proposition}

\index{macierz Seiferta|)}

