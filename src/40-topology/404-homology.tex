
\section{Homologie Chowanowa}

Reszta książki nie zależy od tej sekcji, więc Czytelnik nie dozna większego uszczerbku na wiedzy, jeśli pominie tę i kilka następnych stron (na przykład dlatego, że nie zna kompleksów łańcuchowych, różniczek, grup homologii, i tak dalej).
Homologię Chowanowa nazywa się kategoryfikacją wielomianu Jonesa.
Zanim zagłębimy się w szczegóły, rozpatrzmy prostszy przykład tego procesu.
Niech $X$ będzie przestrzenią topologiczną, wtedy charakterystykę Eulera oraz grupy homologii łączy zależność
\begin{equation}
    \chi(X) = \sum_{n = 0}^{\dim X} (-1)^n \operatorname{rk} H_n(X),
\end{equation}
a przy tym grupy homologii dostarczają więcej informacji, co więcej można o nich myśleć jako (jak o? \texttt{:|}) funktorach.
Homologie są kategoryfikacją charakterystyki Eulera.

\index{homologia!Chowanowa|(}
Niech $L$ będzie splotem, zaś $D$ jego diagramem.
Chowanow \cite{khovanov2000} odkrył, że całkowite współczynniki we wzorze o sumowaniu stanów można zamienić na kompleksy grup abelowych.
Skonstruował rodzinę grup $\mathcal H^{i, j}(D)$ takich, że
\begin{equation}
    K(L, q) = \sum_{i, j} q^j (-1)^i \dim_\Q (\mathcal H^{i, j}(D) \otimes \Q),
\end{equation}
gdzie $K$ jest wersją wielomianu Jonesa.
Niestety opis konstrukcji grup $\mathcal H^{i, j}$ przeładował algebraicznymi szczegółami, dlatego później Bar-Natan \cite{barnatan2002}, Viro \cite{viro2002} przygotowali swoje teksty z~myślą o~topologach.
\index[persons]{Bar-Natan, Dror}%
\index[persons]{Viro, Oleg}%
My opierać się będziemy o~późniejszy artykuł Viro \cite{viro2004}, który został napisany, gdyż \emph{,,Nonetheless, in most of the papers on Khovanov homology, the differences between \cite{barnatan2002} and \cite{viro2002} are taken too seriously. In this paper I discuss the constructions again. I begin with the approach of \cite{viro2002}. (...) Then I identify this construction with the construction of \cite{barnatan2002} and \cite{khovanov2000}''}.
Viro \cite{viro2004} wymienia kilka innych dobrych miejsc, gdzie można zacząć wycieczkę do zrozumienia homologii Chowanowa: Turnera \cite{turner2017} albo Bar-Natana \cite{barnatan2002}, a potem Szumakowicza \cite{shumakovitch2012}, Chowanowa \cite{khovanov2000}.
\index[persons]{Chowanow, Michaił (Хованов, Михаил Гелиевич)}%
\index[persons]{Bar-Natan, Dror}%
\index[persons]{Szumakowicz, Aleksander}%
\index[persons]{Turner, Paul}%

Viro zauważa, że powszechna definicja wielomianu Jonesa sprawia problem dla pustego splotu (którego nigdy wcześniej nie rozpatrywaliśmy).
Mamy:
\begin{equation}
    \jones_\varnothing = \frac{1}{-t^{1/2} - t^{-1/2}},
\end{equation}
a to nie jest wielomian Laurenta jednej zmiennej.
\index{wielomian!Jonesa!powiększony}%
Dlatego definiuje powiększony wielomian Jonesa:
% DICTIONARY;Jones polynomial;wielomian Jonesa;-
% DICTIONARY;augmented;powiększony;wielomian Jonesa
\begin{equation}
    \widetilde{\jones_L}(t) = (-t^{1/2} - t^{-1/2}) \cdot \jones_L(t),
\end{equation}
i mówi, że będzie kategoryfikować powiększony wielomian Jonesa, a właściwie powiększoną klamrę Kauffmana.
Niech $q := -t^{1/2}$.
Dostaje się tak nowy wielomian, nazwijmy go $K$.
Spełnia trzy aksjomaty:
\begin{itemize}
\item (normalizacja) $K(\SmallUnknot) = q + 1/q$;
\item (stabilizacja) $K(L \sqcup \SmallUnknot) = (q + 1/q) K(L)$;
\item (relacja kłębiasta) \begin{equation}
    q^{-2}     K\left( \MediumPlusCrossingArrows \right) -
    q^{2}      K\left( \MediumMinusCrossingArrows \right) =
    (q^{-1}-q) K\left( \MediumJustSmoothing \right).
\end{equation}
\end{itemize}

Stąd widać już, jakie grupy dobrać dla niewęzła:
\begin{equation}
    H^{i,j} = \begin{cases}
        \Z & \textrm{ jeśli } i = 0, j = \pm 1 \\
        0  & \textrm{ w przeciwnym razie}.
    \end{cases}
\end{equation}
Wtedy spełniona jest równość
\begin{equation}
    K(L, q) = \sum_{i, j} (-1)^i q^j \operatorname{rk} H^{i, j} (L).
\end{equation}
Pozostało powtórzyć to dla dowolnego splotu.
Wzór o sumowaniu stanów przybiera postać:
\begin{equation}
    K(L, q) = \sum_s (-1)^{(\writhe D - |s|)/2} q^{(3\writhe D - |s|)/2} (q+1/q)^{|sD|}.
\end{equation}

Reprezentacja ta ma jedną wadę: każdy składnik z prawej strony przyczynia się do różnych jednomianów, zatem ma wpływ na różne grupy (których dopiero szukamy).
,,Surowe'' stany nie są prawdziwym odpowiednikiem sympleksów, jakie spotyka się podczas kategoryfikacji charakterystyki Eulera.
Najprostszym pomysłem, jak to naprawić, jest rozbicie ostatniej potęgi $q + 1/q$.
Zauważmy, że ma tyle czynników, ile wygładzenie diagramu ma składowych.
To motywuje definicję:

\begin{definition}[stan wzbogacony]
\index{stan diagramu!wzbogacony}%
    Stan diagramu $D$ razem z przypisaniem znaku $+$ lub $-$ do każdego okręgu $sD$ nazywamy stanem wzbogaconym.
\end{definition}

Dla ustalonego wzbogaconego stanu $S$ diagramu $D$ oznaczmy przez $\tau(S)$ sumę znaków przypisanych do okręgów\footnote{Oznaczenie wzięte z pracy Viro, żywimy nadzieję, że nikt nie weźmie $\tau$ za liczbę kolorowań z rodziału drugiego. Poza tym, Viro pisze $\sigma(s)$ zamiast naszego $|s|$ oraz $|s|$ zamiast naszego $|sD|$. Ostrożność wskazana.}.
Wtedy
\begin{equation}
    q^{(3 \writhe D - |s|)/2} (q + 1/q)^{|sD|} = \sum_{S/s} q^{(3 \writhe D - |s| + 2 \tau(S))/2},
\end{equation}
gdzie sumowanie odbywa się po wszystkich stanach $S$ wzbogacających stan $s$.
Niech
\begin{equation}
    j(S) := \frac 12 (3 \writhe D - |s| + 2 \tau(S)).
\end{equation}
Dobrnęliśmy do
\begin{equation}
    K(L, q) = \sum_S (-1)^{(\writhe D - |s|)/2} q^{j(S)},
\end{equation}
tym razem sumujemy po wszystkich wzbogaconych stanach diagramu $D$.

Potrzebujemy jeszcze trochę nowych obiektów.
Niech $C(D)$ oznacza wolną abelową grupę generowaną przez wzbogacone stany diagramu $D$, a $C^j(D)$ będzie jej podgrupą generowaną przez wzbogacone stany $S$ takie, że $j(S) = j$.

Czyni to $C(D)$ wolną grupą abelową z $\Z$-gradacją:
\begin{equation}
    C(D) = \bigoplus_{j \in \Z} C^j (D).
\end{equation}

Dla ustalonego stanu wzbogaconego $S$, niech $i(S) = (\writhe D - |s|)/2$.
Określmy ostatnią podgrupę, $C^{i,j}(D) \le C^j(S)$ generowaną przez wzbudzone stany $S$, dla których $i(S) = i$.
Dostajemy wreszcie
\begin{equation}
    K(L, q) = \sum_{j = -\infty}^\infty q^j \sum_{i = -\infty}^\infty (-1)^i \operatorname{rk} C^{i, j}(D).
\end{equation}

Teraz ,,wystarczy'' zdefiniować funkcję $d \colon C^{i, j} \to C^{i+1, j}$ oraz sprawdzić, że $d^2 = 0$, czyli że $d$ jest różniczką.
I to byłby już koniec, ale nam brakuje sił, by przybliżyć konstrukcję.
Różniczka pozwala przejść z grup $C^{i,j}$ do grup homologii.
Pewne wyjaśnienia znaleźć można w~\cite[s. 42]{przytycki2015}, gdzie podano przepis wymagający tylko ponumerowania skrzyżowań.
\index[persons]{Przytycki, Józef}%

Topolog izraelski Bar-Natan \cite{barnatan2007} podał algorytm\footnote{Źródło: komentarze pod postem \url{https://mathoverflow.net/a/232267}} o~złożoności obliczeniowej $O(\exp(c \sqrt n))$ liczący homologie Chowanowa, gdzie $c$ jest pewną stałą, zaś $n$ liczbą skrzyżowań na diagramie.
\index[persons]{Bar-Natan, Dror}%
Nie możemy liczyć na istotne przyspieszenie:
znalezienie przybliżenia wielomianu Jonesa jest w ogólności problemem \#P-trudnym (Kuperberg \cite{kuperberg2015}, Vertigan \cite{vertigan2005}),
\index[persons]{Kuperberg, Greg}%
\index[persons]{Vertigan, Dirk}%
ale trywialnym przy znanych homologiach.
(Patrz też fakt \ref{prp:jones_at_roots_of_unity}).
% TODO: Gukov, Halverson, Ruehle, Sulkowski: Learning to unknot
% [31] = Hass J, Lagarias J C and Pippenger N 1999 The computational complexity of knot and link problems J. ACM 46 185 proved that the unknotting problem, i.e. the decision problem whether a given knot K is actually an unknot, is in complexity class NP
% [32] = Kuperberg G 2014 Knottedness is in np, modulo GRH Adv. Math. 256 493: assuming GRH, unknot recognition problem is in coNP, this assumption was later relaxed in [33] = Lackenby M 2017 The efficient certification of knottedness and thurston norm (arXiv:1604.00290)

Kronheimer, Mrówka \cite{kronheimer2011} pokazali:
\index[persons]{Kronheimer, Peter}%
\index[persons]{Mrówka, Tomasz}%

\begin{proposition}
\label{khovanov_detects_unknot}%
    Zredukowana kohomologia Chowanowa wykrywa niewęzeł.
\end{proposition}

\begin{proof}[Niedowód]
% DICTIONARY;sutured;szwowa;rozmaitość
\index{rozmaitość!szwowa}%
    Dowód składa się z~dwóch kroków.
    W~pierwszym panowie pokazują, że istnieje ciąg spektralny zaczynający się od zredukowanej kohomologii Chowanowa, po którym następuje koniec: homologia zdefiniowana osobliwymi instantonami.
    Potem dowodzą, że ta homologia jest izomorficzna z~instantonową homologią Floera szwowego dopełnienia węzła, o~której wiadomo, że wykrywa niewęzeł.
\index{homologia!Floera}%
\end{proof}

\index{homologia!Chowanowa|)}

% Koniec sekcji Homologie

