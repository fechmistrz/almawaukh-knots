
\section{Homologie}

Ta sekcja wymaga znajomości przynajmniej homologii: kompleksów łańcuchowych, różniczek, grup homologii.
Można się tego nauczyć z~każdego podręcznika topologii algebraicznej.

\subsection{Homologie Chowanowa}
\index{homologia!Chowanowa|(}
Viro napisał w 2004 piękną pracę \cite{viro04}, by objaśnić homologię Chowanowa używając tak mało algebry, jak to tylko możliwe.
Jest przyjazna dla początkujących, więc na niej opiera się reszta tej podsekcji\footnote{Viro wymienia kilka innych artykułów, z których można czerpać wiedzę. \emph{,,A good place to start: \cite{barnatan02} followed by \cite{shumakovitch12}, \cite{khovanov00}. Another possible starting point: \cite{turner17}.''}}.
Będziemy pracować ze stanami Kauffmana i obramowanymi węzłami, ponieważ autor sugeruje, że to bardziej naturalne.
Oryginalna praca Chowanowa to \cite{khovanov00}.

Niech $L$ będzie splotem, zaś $D$ jego diagramem.
Chowanow skonstruował rodzinę grup $\mathcal H^{i, j}(D)$ takich, że
\begin{equation}
    K(L, q) = \sum_{i, j} q^j (-1)^i \dim_\Q (\mathcal H^{i, j}(D) \otimes \Q),
\end{equation}
gdzie $K$ jest wersją wielomianu Jonesa.
Grupy $\mathcal H^{i, j}$ są u~niego grupami homologii pewnych kompleksów łańcuchowych.
Ich konstrukcja była przeładowana algebraicznymi szczegółami, później Bar-Natan \cite{barnatan02}, Viro podali jej warianty z~myślą o~topologach.
% Viro - Remarks on definition of Khovanov homology, arXiv

Homologię Chowanowa nazywa się kategoryfikacją wielomianu Jonesa.
Zanim zagłębimy się w szczegóły, rozpatrzmy prostszy przykład tego procesu.
Niech $X$ będzie przestrzenią topologiczną, wtedy charakterystykę Eulera oraz grupy homologii łączy zależność
\begin{equation}
    \chi(X) = \sum_{n = 0}^{\dim X} (-1)^n \operatorname{rk} H_n(X),
\end{equation}
a przy tym grupy homologii dostarczają więcej informacji, co więcej można o nich myśleć jako funktorach.
Homologie są kategoryfikacją charakterystyki Eulera.

Kategoryfikacja wielomianu Jonesa polega na zastąpieniu jakoś jego współczynników przez ciąg grup abelowych.
Wzór o sumowaniu stanów przypomina ostatnią równość, brakuje tylko przedstawienia składników po prawej stronie jako alternująca suma rang grup.

Viro zauważa, że powszechna definicja wielomianu Jonesa sprawia problem dla pustego splotu (którego nigdy wcześniej nie rozpatrywaliśmy).
Mamy:
\begin{equation}
    \jones_\varnothing = \frac{1}{-t^{1/2} - t^{-1/2}},
\end{equation}
a to nie jest wielomian Laurenta jednej zmiennej.
\index{wielomian Jonesa!powiększony}%
Dlatego definiuje powiększony wielomian Jonesa:
\begin{equation}
    \widetilde{\jones_L}(t) = (-t^{1/2} - t^{-1/2}) \cdot \jones_L(t),
\end{equation}
i mówi, że będzie kategoryfikować powiększony wielomian Jonesa, a właściwie powiększoną klamrę Kauffmana.

Jak pisze dalej, pewne drobne trudności techniczne mogły skłonić Chowanowa do pozbycia się ułamkowych potęg przez zmianę zmiennej w~powiększonym wielomianie Jonesa: niech $q := -t^{1/2}$.
Dostaje się tak nowy wielomian, nazwijmy go $K$.
Spełnia trzy aksjomaty:
\begin{itemize}
\item (normalizacja) $K(\SmallUnknot) = q + 1/q$;
\item (stabilizacja) $K(L \sqcup \SmallUnknot) = (q + 1/q) K(L)$;
\item (relacja kłębiasta) \begin{equation}
    q^{-2}     K\left( \MediumPlusCrossingArrows \right) -
    q^{2}      K\left( \MediumMinusCrossingArrows \right) =
    (q^{-1}-q) K\left( \MediumJustSmoothing \right).
\end{equation}
\end{itemize}

Stąd widać już, jakie grupy dobrać dla niewęzła:
\begin{equation}
    H^{i,j} = \begin{cases}
        \Z & \textrm{ jeśli } i = 0, j = \pm 1 \\
        0  & \textrm{ w przeciwnym razie}.
    \end{cases}
\end{equation}
Wtedy spełniona jest równość
\begin{equation}
    K(L, q) = \sum_{i, j} (-1)^i q^j \operatorname{rk} H^{i, j} (L).
\end{equation}
Pozostało powtórzyć to dla dowolnego splotu.
Wzór o sumowaniu stanów przybiera postać:
\begin{equation}
    K(L, q) = \sum_s (-1)^{(\writhe D - |s|)/2} q^{(3\writhe D - |s|)/2} (q+1/q)^{|sD|}.
\end{equation}

Reprezentacja ta ma jedną wadę: każdy składnik z prawej strony przyczynia się do różnych jednomianów, zatem ma wpływ na różne grupy (których dopiero szukamy).
,,Surowe'' stany nie są prawdziwym odpowiednikiem sympleksów, jakie spotyka się podczas kategoryfikacji charakterystyki Eulera.
Najprostszym pomysłem, jak to naprawić, jest rozbicie ostatniej potęgi $q + 1/q$.
Zauważmy, że ma tyle czynników, ile wygładzenie diagramu ma składowych.
To motywuje definicję:

\begin{definition}[stan wzbogacony]
\index{stan!wzbogacony}%
    Stan diagramu $D$ razem z przypisaniem znaku $+$ lub $-$ do każdego okręgu $sD$ nazywamy stanem wzbogaconym.
\end{definition}

Dla ustalonego wzbogaconego stanu $S$ diagramu $D$ oznaczmy przez $\tau(S)$ sumę znaków przypisanych do okręgów\footnote{Oznaczenie wzięte z pracy Viro, żywimy nadzieję, że nikt nie weźmie $\tau$ za liczbę kolorowań z rodziału drugiego. Poza tym, Viro pisze $\sigma(s)$ zamiast naszego $|s|$ oraz $|s|$ zamiast naszego $|sD|$. Ostrożność wskazana.}.
Wtedy
\begin{equation}
    q^{(3 \writhe D - |s|)/2} (q + 1/q)^{|sD|} = \sum_{S/s} q^{(3 \writhe D - |s| + 2 \tau(S))/2},
\end{equation}
gdzie sumowanie odbywa się po wszystkich stanach $S$ wzbogacających stan $s$.
Niech
\begin{equation}
    j(S) := \frac 12 (3 \writhe D - |s| + 2 \tau(S)).
\end{equation}
Dobrnęliśmy do
\begin{equation}
    K(L, q) = \sum_S (-1)^{(\writhe D - |s|)/2} q^{j(S)},
\end{equation}
tym razem sumujemy po wszystkich wzbogaconych stanach diagramu $D$.

Potrzebujemy jeszcze trochę nowych obiektów.
Niech $C(D)$ oznacza wolną abelową grupę generowaną przez wzbogacone stany diagramu $D$, a $C^j(D)$ będzie jej podgrupą generowaną przez wzbogacone stany $S$ takie, że $j(S) = j$.

Czyni to $C(D)$ wolną grupą abelową z $\Z$-gradacją:
\begin{equation}
    C(D) = \bigoplus_{j \in \Z} C^j (D).
\end{equation}

Dla ustalonego stanu wzbogaconego $S$, niech $i(S) = (\writhe D - |s|)/2$.
Określmy ostatnią podgrupę, $C^{i,j}(D) \le C^j(S)$ generowaną przez wzbudzone stany $S$, dla których $i(S) = i$.
Dostajemy wreszcie
\begin{equation}
    K(L, q) = \sum_{j = -\infty}^\infty q^j \sum_{i = -\infty}^\infty (-1)^i \operatorname{rk} C^{i, j}(D).
\end{equation}

Teraz ,,wystarczy'' zdefiniować funkcję $d$ i sprawdzić, że jest różniczką, to znaczy że $d^2 = 0$.
Tak też robi Viro, nam brakuje sił, by przybliżyć konstrukcję.
To już koniec -- różniczka pozwala przejść z grup $C^{i,j}$ do grup homologii.
Pewne wyjaśnienia znaleźć można w~\cite[s. 42]{przytycki15}, gdzie podano przepis wymagający właściwie tylko ponumerowania skrzyżowań.

Bar-Natan, topolog izraelski, napisał program liczący homologie Chowanowa szybko \cite{barnatan07}, przy czym szybko oznacza: chyba\footnote{Źródło: komentarze pod postem \url{https://mathoverflow.net/a/232267}} w~czasie $O(\exp(c \sqrt n))$, dla diagramu o~$n$ skrzyżowaniach.
Nie możemy liczyć na istotne przyspieszenie:
znalezienie przybliżenia wielomianu Jonesa jest problemem \#P-trudnym (\cite{kuperberg15}, \cite{vertigan05}),
a przy znanych homologiach -- trywialnym.
(Ale patrz też fakt \ref{prp:jones_at_roots_of_unity}).

Kronheimer, Mrówka \cite{kronheimer11} pokazali:

\begin{proposition}
\label{khovanov_detects_unknot}%
    Zredukowana kohomologia Chowanowa wykrywa niewęzeł.
\end{proposition}

\begin{proof}
    Dowód składa się z dwóch kroków.
    W pierwszym panowie pokazują, że istnieje ciąg spektralny zaczynający się od zredukowanej kohomologii Chowanowa, po którym następuje koniec: homologia zdefiniowana osobliwymi instantonami.

% DICTIONARY;sutured manifold;rozmaitość szwowa;-
\index{rozmaitość szwowa}
    Potem dowodzą, że ta homologia jest izomorficzna z instantonową homologią Floera szwowego dopełnienia węzła, o której wiadomo, że wykrywa niewęzeł.
\end{proof}

\index{homologia!Chowanowa|)}

\subsection{Homologia Floera}
Do zrobienia...

% Koniec sekcji Homologie

