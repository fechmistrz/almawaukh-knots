
\section{Niezmiennik Arfa}
\index{niezmiennik!Arfa|(}

Cahit Arf wprowadził w 1941 roku pewien niezmiennik nieosobliwych form kwadratowych nad ciałem charakterystyki dwu.
Zrobił to między innymi po to, by sklasyfikować takie formy kwadratowe.
My poznamy wariant niezmiennika Arfy dla węzłów.

Niech $(v_{i,j})$ będzie macierzą Seiferta węzła $K$ o genusie $g$.
Wtedy jej wymiary wynoszą $2g \times 2g$ i macierz $V-V^t$ jest symplektyczna.

\begin{definition}
    Zachowując powyższe oznaczenia, niezmiennik Arfa to
    \begin{equation}
        \sum^g_{i=1}v_{2i-1,2i-1}v_{2i,2i} \pmod 2.
    \end{equation}
    % Przyjmuje on dwie wartości: 0, 1
\end{definition}

Niezmiennik Arfa dla węzłów można zdefiniować na kilka sposobów, z~których żaden nie jest istotnie lepszy od pozostałych.
Pierwszy był pomysł Robertello \cite{robertello65}:
\index{człowiek!Robertello, Raymond}%

\begin{proposition}[Robertello, 1965]
    Niech $K$ będzie węzłem, zaś
    \begin{equation}
        \alexander_K(t)=c_{0}+c_{1}t+\cdots +c_{n}t^{n}+\cdots +c_{0}t^{2n}
    \end{equation}
    jego wielomianem Alexandera.
    Wtedy niezmiennik Arfa to $c_{n-1}+c_{n-3}+\cdots +c_{r} \mod 2$, gdzie $r = 0$ dla nieparzystych $n$, $r = 1$ w~przeciwnym razie.
\end{proposition}

Nieco później Murasugi \cite{murasugi69} zauważył, że warunek można uprościć:

\begin{proposition}[Murasugi, 1969]
    \label{prp:arf_murasugi}
    Niech $K$ będzie węzłem.
    Wtedy $\operatorname{Arf} K = 0$ wtedy i~tylko wtedy, gdy $\alexander_K(-1) \equiv \pm 1 \mod 8$.
\end{proposition}

Kauffman zaproponował inne podejście z wykorzystaniem diagramów.
\index{człowiek!Kauffman, Louis}%
Dwa węzły nazwiemy równoważnymi przez przejścia, jeśli są związane skończenie wieloma ,,przejściami'' \cite[s. 143]{kauffman83}:
\begin{comment}
\[
    \LargeTwoPassMoveA \cong \LargeTwoPassMoveB
    \quad\mbox{albo}\quad
    \LargeTwoPassMoveC \cong \LargeTwoPassMoveD
\]
\end{comment}

\begin{definition}[Kauffman, 1983]
    Każdy węzeł $K$ jest równoważny przez przejścia albo z niewęzłem (wtedy mówimy, że $\operatorname{Arf} K = 0$), albo z trójlistnikiem (wtedy, że $\operatorname{Arf} K = 1$).
\end{definition}

Wreszcie Jones zauważył \cite[tw. 19]{jones85}, że dzięki zespolonym algebrom Clifforda oraz pracy \cite{lannes85} niezmiennik Arfa jest specjalną wartością wielomianu Jonesa.
Jest to jedyna definicja, którą łatwo rozszerzyć do splotów.

\begin{proposition}[Jones, 1985]
\label{prp:arf_jones}%
    $\operatorname{Arf}(K) = \jones_K(i)$.
\end{proposition}

\begin{corollary}
    Niezmiennik Arfa jest $\shrap$-addytywny (modulo 2).
\end{corollary}

\begin{proof}
    Wynika to z faktu \ref{prp:arf_jones} oraz \ref{prp:jones_multiplicative_2}, ale bezpośredni dowód też istnieje. % gdzie?
\end{proof}

O niezmienniku Arfa usłyszymy jeszcze poznając węzły plastrowe.

\index{niezmiennik!Arfa|)}

% Koniec sekcji Niezmiennik Arfa

