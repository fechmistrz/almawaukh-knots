
\subsection{Węzły rozwłóknione}
\index{węzeł!włóknisty|see {węzeł rozwłókniony}}%
\index{węzeł!rozwłókniony|(}%
Wspomnijmy jeszcze krótko o~specjalnym rodzaju węzłów i splotów (patrz \cite[s. 49-50]{kawauchi96}).

% DICTIONARY;fibered;rozwłókniony, włóknisty;-
\begin{definition}
    Niech $L \subseteq S^3$ będzie splotem.
    Jeśli istnieje rodzina $F_t$ powierzchni Seiferta dla splotu $K$ sparametryzowana przez $t \in S^1$ taka, że $F_t \cap F_s = K$ dla $t \neq s$, to splot $K$ nazywamy rozwłóknionym albo włóknistym.
\end{definition}

\index{splot!Neuwirtha}%
Dawniej nazywano je splotami Neuwirtha, gdyż ten pokazał w~swojej pracy dyplomowej z~1959 roku, że można je scharakteryzować jako sploty, których komutant grupy podstawowej jest skończenie generowany, lub równoważnie, wolny.

\begin{example}
    Niewęzeł, trójlistnik $3_1$, ósemka $4_1$, $5_{1}$, $6_{2}$, $6_{3}$, $7_{1}$, $7_{6}$, $7_{7}$, $8_{2}$, $8_{5}$, $8_{7}$, $8_{9}$, $8_{10}$, $8_{12}$, $8_{16}$..$8_{21}$, splot Hopfa oraz wszystkie węzły torusowe są rozwłóknione.
\end{example}

(Jeśli węzeł pierwszy o co najwyżej ośmiu skrzyżowaniach nie został wymieniony w tym przykładzie, to nie jest rozwłókniony).
Rozkład liczby węzłów rozwłóknionych wśród węzłów pierwszych wygląda następująco:

\renewcommand*{\arraystretch}{1.4}
\footnotesize
\begin{longtable}{lcccccccccc}
    \hline
    \textbf{skrzyżowania} & 3 & 4 & 5 & 6 & 7 & 8 & 9 & 10 &  11 &  12 \\ \hline \endhead
    węzły pierwsze & 1 & 1 & 2 & 3 & 7 & 21 & 49 & 165 & 552 & 2176 \\
    rozwłóknione węzły pierwsze & 1 & 1 & 1 & 1 & 3 & 12 & 23 & 74 & 256 & 873 \\
    \hline
\end{longtable}
% ZWERYFIKOWANO: funkcja count_fibered
\normalsize

Lwia część analizy węzłów o 12 skrzyżowaniach została wykonana przez Stojmenowa i~Hirasawę, jak podaje baza danych KnotInfo \cite{knotinfo22}.
% źródło: https://knotinfo.math.indiana.edu/descriptions/fibered.html
\index[persons]{Hirasawa, Mikami}%
\index[persons]{Stojmenow, Aleksander}%

\begin{proposition}
\index{wielomian!Alexandera}%
    Pierwszy i~ostatni współczynnik wielomianu Alexandera węzła rozwłóknionego to $\pm 1$.
\end{proposition}

% Kryterium to jest wystarczające dla węzłów pierwszych o co najwyżej 10 skrzyżowaniach oraz alternujących, ale znany jest przykład niewłóknistego węzła o 21 skrzyżowaniach, którego wielomian Alexandera ma postać $t^4 - t^3 + t^2 - t +1$.
% TODO: ustalić, czemu tak dużo skrzyżowań (z której książki ten fakt?). Sam wynik wydaje się być folklorem, tzn. nie wiadomo kto pierwszy to pokazał.

Powyższy warunek konieczny jest też dostateczny dla węzłów pierwszych o co najwyżej 10 skrzyżowaniach.
Trzy węzły pierwsze o 11 skrzyżowaniach: $11n_{34}$, $11n_{42}$, $11n_{73}$ oraz osiemnaście o 12 skrzyżowaniach są świadkami, że wynikanie w dwie strony nie zachodzi.
% ZWERYFIKOWANO: funkcja alexander_fibered
Trzynaście z tych osiemnastu: 12n57, 12n210, 12n214, 12n258, 12n279, 12n382, 12n394, 12n464, 12n483, 12n535, 12n650, 12n801 ma wielomian Alexandera, którego stopień jest dwukrotnością 3-genusu (cztery znalazł Mikami Hirasawa, pozostałe Stefan Friedl oraz Taehee Kim).
% źródło: https://knotinfo.math.indiana.edu/descriptions/fibered.html
\index[persons]{Friedl, Stefan}%
\index[persons]{Taehee, Kim}%
\index[persons]{Hirasawa, Mikami}%
Wiemy to, ponownie, dzięki bazie KnotInfo \cite{knotinfo22}.

\begin{example}
\index{węzeł!skręcony}%
    Niech $K$ będzie węzłem skręconym z $n$ półskrętami.
    Wtedy $K$ nie jest rozwłókniony (chyba że $n = 1$), ponieważ jego wielomianem Alexandera jest
\index{wielomian!Alexandera}%
    \begin{equation}
        \alexander_n(t) = n \cdot \left(t + \frac 1 t \right) - (2n+1).
    \end{equation}
\end{example}

\begin{corollary}
% TODO: węzeł dokerski do indeksu?
    $2$-skręcony węzeł $6_1$ (węzeł dokerski) nie jest rozwłókniony.
\end{corollary}

Rolfsen \cite[s. 326]{rolfsen76} podaje jako ćwiczenie w swojej książce:

\begin{proposition}
\index{suma spójna}%
    Rodzina węzłów rozwłóknionych jest zamknięta na branie sum spójnych.
\end{proposition}

Dzięki książce Kawauchiego \cite[s. 84]{kawauchi96} jesteśmy świadom jeszcze, że Stallings ma swoje twierdzenie o~rozwłóknienieach dla zwartych 3-rozmaitości ze specjalnym przypadkiem:

\begin{proposition}
    Niech $y$ będzie jedynym epimorfizmem grupy splotu $L$ na nieskończoną grupę cykliczną $\langle t \rangle$, który posyła każdy południk na $t$. % each meridian
    Wtedy splot $L$ jest rozwłókniony wtedy i tylko wtedy, gdy jądro $\ker y$ jest skończenie generowalne.
\end{proposition}

\index{węzeł!rozwłókniony|)}%

% koniec podsekcji Węzły rozwłóknione
