
\subsection{Pochodna Foxa}
\index{pochodna Foxa|(}%
Istnieje alternatywna prezentacja grupy węzła, która pochodzi od Dehna, gdzie zamiast etykietować łuki, przypisuje się różne litery czterem częściom płaszczyzny, które są rozcinane przez skrzyżowanie.
\index[persons]{Dehn, Max}%
Pomijamy tę prezentację dla oszczędności miejsca.
Klasycznie, jak na przykład w~\cite{crowell63}, macierz, a~co za tym idzie, także wielomian Alexandera wprowadza się przy użyciu prezentacji Wirtingera i~pochodnej Foxa.
Oryginalna praca Alexandera była jednak bliższa duchem pomysłom Dehna.
\index[persons]{Alexander, James}%

% DICTIONARY;Fox derivative;pochodna Foxa;-
\begin{definition}[pochodna Foxa]
\label{def:fox_derivative}%
    Niech $G$ będzie wolną grupą generowaną przez (niekoniecznie skończony) podzbiór $\{g_i\}_{i \in I}$.
    Odwzorowanie $\partial/\partial g_i \colon G \to \Z G$ spełniające trzy aksjomaty:
    \begin{align}
        \frac{\partial}{\partial g_i} (e) & = 0 \\
        \frac{\partial}{\partial g_i} (g_j) & = \delta_{ij} \\
        \forall u, v \in G : \frac{\partial}{\partial g_i} (uv) & = \frac{\partial}{\partial g_i}(u) + u \frac{\partial}{\partial g_i} (w),
    \end{align}
    gdzie $\delta_{ij}$ oznacza deltę Kroneckera, nazywamy pochodną cząstkową Foxa.
\end{definition}

Fox opublikował w~Annals of Mathematics cykl pięciu artykułów \cite{fox53}, \cite{fox54}, \cite{fox56}, \cite{fox58}, \cite{fox60} poświęconych wolnemu rachunkowi różniczkowemu.
\index[persons]{Fox, Ralph}%
Definicja \ref{def:fox_derivative} jest tylko małym wycinkiem tego cyklu.
nLab wspomina jeszcze o ,,\emph{a nice introduction in} \cite{crowell63}'', podręczniku Crowella, Foxa.

Ustalmy prezentację grupy węzła z $n$ relacjami (słowami) $w_1, \ldots, w_n$ nad $n$-literowym alfabetem $x_1, \ldots, x_n$.
Zdefiniujmy następnie macierz Jacobiego wymiaru $n \times n$, elementami której są pochodne Foxa słów $w_i$ względem zmiennych $x_j$:
\index{macierz Jacobiego}%
\begin{equation}
    J = \left(\frac{\partial w_i}{\partial x_j}\right).
\end{equation}

Wykreślmy z macierzy $J$ najpierw jedną kolumnę oraz jeden wiersz z tej macierzy, po czym podstawmy za wszystkie litery zmienną $t$ i policzmy wyznacznik.
Otrzymaliśmy znowu wielomian Alexandera!
\index{wielomian!Alexandera}%

\index{pochodna Foxa|)}%

% koniec podsekcji pochodna Foxa

