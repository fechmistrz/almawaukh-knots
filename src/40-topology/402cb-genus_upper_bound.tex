
\subsubsection{Ograniczanie genusu z góry}
Z góry genus ograniczony jest przez kilka klasycznych niezmienników numerycznych.
Zanim to pokażemy, przytoczymy techniczny lemat udowodniony przez Yamadę \cite{yamada1987}.
\index[persons]{Yamada, Shuji}%

\begin{proposition}
\index{indeks warkoczowy}%
\label{prp:seifert_circles_braid}%
    Niech $L$ będzie splotem, zaś $\operatorname{s} L$ minimalną liczbą okręgów Seiferta, które dostajemy ze wszystkich możliwych diagramów splotu $L$.
    Wtedy $\operatorname{s} L = \braid L$ jest równe indeksowi warkoczowemu.
\end{proposition}

Powyższe stwierdzenie występuje bez dowodu (bez?) w \cite[s. 17]{kawauchi1996}.
Zmyślny dowód Yamady polega na zmianie rzutu tak, by nie zmienić liczby okręgów Seiferta.

\begin{proposition}
    Niech $L$ będzie splotem.
    Wtedy $\crossing L - \braid L - \operatorname{\mu} L + 2 \ge 2 \genus L$.
\end{proposition}

\begin{proof}
    Ustalmy minimalny diagram $D$ dla splotu $L$ i zastosujmy do niego algorytm Seiferta.
    Dostaniemy tak $s$ okręgów Seiferta oraz powierzchnię o genusie $g$.
    Fakt~\ref{prp:seifert_euler_characteristics} mówiący, że $\chi = s - c$, można przekształcić do
    \begin{equation}
        g = \frac{c + 2 - s - \mu(K)}{2}.
    \end{equation}
    Z~minimalności diagramu wynika, że $c = \crossing L$.
    Fakt~\ref{prp:seifert_circles_braid} mówi, że $s \ge \braid L$.
    Nierówność $g \ge \genus L$ wynika z~definicji genusu.
    Z powyższych rozważań wynika, że
    \begin{equation}
        \crossing L + 2 \ge 2 \genus L + \braid L + \operatorname{\mu} L,
    \end{equation}
    a to jest równoważnie nierówności, której prawdziwości dowodzimy.
\end{proof}

\begin{corollary}
\index{indeks skrzyżowaniowy}%
\label{cor:crossing_genus}%
    Niech $K$ będzie węzłem.
    Wtedy $\crossing K \ge 2 \genus K$.
\end{corollary}

% koniec psot

