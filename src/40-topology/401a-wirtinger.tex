
\subsection{Prezentacja Wirtingera}
\index{prezentacja Wirtingera|(}
Wiemy więc już trochę o~nowym niezmienniku, ale nie umiemy go jeszcze wyznaczać.
Jak zauważył Wilhelm Wirtinger około roku 1905, a więc jeszcze przed narodzinami teorii węzłów, grupa węzła zawsze posiada pewną specjalną prezentację, nazwaną na jego cześć prezentacją Wirtingera.
Jest to skończona prezentacja, w~której wszystkie relacje są postaci $w g_i w^{-1} = g_j$, gdzie $w$ to pewne słowo na generatorach, $g_1, \ldots, g_k$.
Przedstawimy ją zaraz ze względu na użyteczność w~rachunkach, dowodząc jednocześnie jej istnienia.

\begin{proposition}
    Grupa każdego węzła posiada prezentację Wirtingera.
\end{proposition}

\begin{proof}
    Oto zarys konstruktywnego dowodu.
    Przedstawiony algorytm jest bardzo wygodnym sposobem na wyznaczenie grupy węzła.
    Niech $K$ będzie węzłem z~diagramem o~$n$ łukach i~$m$ skrzyżowaniach.
    Wtedy
    \begin{equation}
        \pi_1(K) \cong \langle a_1, \ldots, a_n \mid r_1, \ldots, r_m\rangle,
    \end{equation}
    gdzie $a_i$ to włókna diagramu, zaś $r_x$ to relacje Wirtingera: $a_ia_ja_i^{-1}a_k^{-1}=1$,
\begin{comment}
    \begin{figure}[H]
    \begin{minipage}[b]{.48\linewidth}
        \[
            \LargeWirtingerRelationA
        \]
    \end{minipage}
    \begin{minipage}[b]{.48\linewidth}
        \[
            \LargeWirtingerRelationB
        \]
    \end{minipage}
    \end{figure}
\end{comment}
    w~których łuk $a_i$ biegnie górą, zaś $a_j$ leży po jego lewej stronie.
\end{proof}

\begin{comment}
\begin{figure}[H]
    \begin{minipage}[b]{.48\linewidth}
        \[
            \HugeWirtingerPlus
        \]
        \subcaption{skrzyżowanie dodatnie: $x_j = x_k x_{j+1} x_k^{-1}$}
    \end{minipage}
    \begin{minipage}[b]{.48\linewidth}
        \[
            \HugeWirtingerMinus
        \]
        \subcaption{skrzyżowanie ujemne: $x_j = x_k^{-1} x_{j+1} x_k$}
    \end{minipage}
\end{figure}
\end{comment}

\begin{corollary}
    Niech $G$ będzie grupą węzła.
    Wtedy jej abelianizacją jest $G^{\operatorname{ab}} = \Z$.
\end{corollary}

\begin{proof}
    Relacja $a_ia_ja_i^{-1}a_k^{-1}=1$ po przejściu do abelianizacji przyjmuje postać $a_j = a_k$.
    Oznacza to, że etykieta łuku nie zmienia się podczas przejścia pod każdym skrzyżowaniem, zatem wszystkie etykiety są takie same.

    Można też zauważyć, że abelianizacją grupy podstawowej węzła jest pierwsza grupa homologii okręgu, czyli $\Z$.
\end{proof}

Michel Kervaire w \cite{kervaire65} pokazał, jakie własności musi posiadać grupa węzła (i~wiemy o~tym, bo przeczytaliśmy książkę Kawauchiego, w tym \cite[tw. 14.1.1]{kawauchi96}):
\index{człowiek!Kervaire, Michel}%

\begin{proposition}
    Niech $G$ będzie grupą węzła $S^n \subseteq S^{n+2}$.
    Wtedy:
    \begin{enumerate}[leftmargin=*]
        \itemsep0em
        \item grupa $G$ jest skończenie prezentowana,
        \item abelianizacja $G/G'$ jest nieskończoną grupą cykliczną,
        \item druga grupa homologii $H_2(G) = 0$ jest trywialna,
        \item istnieje element $x \in G$ zwany południkiem taki, że $G$ jest najmniejszą podgrupą normalną $G$, która zawiera $x$.
    \end{enumerate}
\end{proposition}

Wyżej wymienione warunki konieczne są także wystarczające, jeżeli $n \ge 3$, jednakże problem pełnej charakteryzacji w~czwartym wymiarze jest otwarty.
Warunki 2. i 3. wynikają z~dualności Alexandera, zaś 1. i 4. stanowią przeformułowanie prezentacji Wirtingera.

\index{prezentacja Wirtingera|)}

% koniec podsekcji Prezentacja Wirtingera

