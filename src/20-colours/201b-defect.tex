
\subsection{Defekt}
\index{defekt|(}%
\begin{definition}[defekt]
    Wymiar jądra macierzy kolorującej modulo $p$, defekt, nazywamy defektem węzła.
\end{definition}

\begin{proposition}
\label{no_relation_defects}%
    Defekty modulo różne liczby pierwsze są niezależne od siebie.
\end{proposition}

Patrz \cite[s. 145]{livingston93}.

\begin{proof}[Niedowód]
    Na przykład suma spójna $k$ trójlistników i~$j$ węzłów $T_{2,5}$ posiada defekt $k$ modulo $3$ oraz $j$ modulo $5$.
    Podobne przykłady istnieją dla innych zbiorów liczb pierwszych.
\end{proof}

Defekt także jest niezmiennikiem, choć rzadziej używanym.
Węzeł o~defekcie $n$ modulo $p$ posiada $p(p^n-1)$ kolorowań $p$ kolorami.
Węzły $8_{18}$ oraz $9_{24}$ mają ten sam wyznacznik, $45$.
Ich defekty modulo $3$ to $1$ i~$2$, zatem są różne.
\index{defekt|)}%

