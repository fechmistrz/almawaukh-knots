
\section{Kwandle i wraki}
\index{kwandel|(}%

Sekcja ta powstała częściowo w~oparciu o~notatki autorstwa Bergera, Geriga\footnote{dostępne pod adresem \url{https://scholar.harvard.edu/files/gerig/files/knotnotes.pdf}} oraz Bergera, Flannery'ego i~Sumnichta\footnote{dostępne pod adresem \url{https://github.com/thyrgle/191_Final_Project/blob/master/paper.pdf}}.
\index[persons]{Berger, Andrew}%
\index[persons]{Gerig, Chris}%
\index[persons]{Flannery, Brandon}%
\index[persons]{Sumnicht, Christopher}%
Kwandle to struktura algebraiczna podobna do grupy.
Aksjomaty grupy stanowią uogólnienie symetrii, ponieważ składanie symetrii jest łączne, identyczność jest symetrią, funkcja odwrotna do symetrii jest symetrią.
Aksjomaty kwandli odzwierciedlać będą ruchy Reidemeistera.
\index{ruch!Reidemeistera}%

David Joyce zapytany o znaczenie słowa \emph{quandle} odpowiedział: \emph{,,I needed a usable word. Distributive algebra had too many syllables. Piffle was already taken. I tried trindle and quagle, but they didn’t seem right, so I went with quandle''}.
\index[persons]{Joyce, David}%

% DICTIONARY;quandle;kwandel;-
\begin{definition}[kwandel]
\index{kwandel}%
    Zbiór $X$ wyposażony w dwuargumentowe działanie $\triangleright$, które dla wszystkich elementów $x, y, z \in X$ spełnia trzy warunki:
    \begin{enumerate}
        \item $x \triangleright x = x$,
        \item odwzorowanie $\beta_y \colon X \to X$ dane wzorem $\beta_y(x) = x \triangleright y$ jest odwracalne,
        \item $(x \triangleright y) \triangleright z = (x \triangleright z) \triangleright (y \triangleright z)$,
    \end{enumerate}
    nazywamy kwandlem.
\end{definition}

Kwandle można rozpatrywać jako samodzielne konstrukcje algebraiczne.
My pokażemy, że są naturalnym niezmiennikiem węzłów.

Niech $X$ będzie skończonym kwandlem, zaś $K$ węzłem.
Elementy $x \in X$ będą dla nas kolorami, którymi oznaczymy długie łuki na diagramie węzła $K$.
Gdy trzy kolory spotykają się przy jednym skrzyżowaniu, definiujemy funkcję $\triangleright \colon X \times X \to X$, jak na rysunku.
To znaczy: kiedy łuk o kolorze $x$ przechodzi pod łukiem koloru $y$, staje się łukiem w kolorze $x \triangleright y$.
\begin{comment}
\[
    \LargeMinusCrossingQuandle
\]
\end{comment}

Ta definicja pochodzi z~nieopublikowanej korespondencji między Johnem Conwayem i~Gavenem Wraithem, którzy w 1959 byli studentami I stopnia na uniwersytecie w Cambridge.
\index[persons]{Conway, John}%
\index[persons]{Wraith, Gaven}%
% index: uniwersytet w Cambridge? a są inne?
Ponownie odkryto ją w latach 80. XX wieku: Joyce w 1982 po raz pierwszy nazwał te obiekty kwandlami, Matwiejow w tym samym roku jako grupoidy rozdzielne, Brieskorn w 1986 jako zbiory automorficzne.
\index[persons]{Joyce, David}%
\index[persons]{Matwiejow, Siergiej}%
% https://ru.wikipedia.org/wiki/Матвеев,_Сергей_Владимирович
\index[persons]{Brieskorn, Egbert}%
\index{grupoid rozdzielny|see {kwandel}}%
\index{zbiór automorficzny|see {kwandel}}%

Drugi aksjomat nazywa się czasem odwracalnością z prawej strony: znając $x \triangleright y$ oraz $y$ możemy odtworzyć element $x$, jednak znając $x$ być może nie jesteśmy w stanie odtworzyć elementu $y$.
Jedyny element $x$ taki, że $x \triangleright y = z$ nazwijmy $y \triangleleft z$.
To pozwala podać trochę inną definicję kwandli, my nie będziemy jej nigdy używać:

\begin{definition}
    Zbiór $X$ z dwuargumentowymi działaniami $\triangleright, \triangleleft$ taki, że dla wszystkich $x, y, z \in X$ zachodzi:
    \begin{align}
    x \triangleleft x & = x \\
    x \triangleright x & = x \\
    (x \triangleleft y) \triangleright x & = y \\
    x \triangleleft (y \triangleright x) & = y \\
     (x \triangleright z) \triangleright (y \triangleright z) & = (x \triangleright y) \triangleright z \\
    (x \triangleleft y) \triangleleft (x \triangleleft z) & = x \triangleleft (y \triangleleft z)
    \end{align}
    nazywamy kwandlem.
\end{definition}

Teraz możemy przetłumaczyć ruchy Reidemeistera w aksjomaty kwandli.
\index{ruch!Reidemeistera}%

\begin{proposition}
\index{niezmiennik!zliczający}%
    Niech $X$ będzie skończonym kwandlem.
    Liczba etykietowań diagramu elementami kwandla $X$ jest niezmiennikiem węzłów, zwanym niezmiennikiem zliczającym.
\end{proposition}

\begin{proof}
    Musimy pokazać, że etykiety na diagramiem przed każdym ruchem Reidemeistera wyznaczają jednoznacznie układ etykiet po tym ruchu.
    Pierwszy i drugi ruch:
\begin{comment}
    \begin{figure}[H]
        \begin{minipage}[b]{.48\linewidth}
        \[
            \LargeReidemeisterOneRightQuandleProof
            \stackrel{R_1}{\cong}
            \LargeReidemeisterOneStraightQuandleProof
        \]
        \end{minipage}
        \begin{minipage}[b]{.48\linewidth}
        \[
            \LargeReidemeisterTwoQuandleA \cong \LargeReidemeisterTwoQuandleB
        \]
        \end{minipage}
    \end{figure}
\end{comment}
    Trzeci ruch:
\begin{comment}
    \[
        \LargeReidemeisterThreeQuandleA \cong \LargeReidemeisterThreeQuandleB \qedhere
    \]
\end{comment}
\end{proof}

Homomorfizmy definiujemy standardowo, przez analogię do grup:

\begin{definition}
    Niech $Q_1, Q_2$ będą kwandlami.
    Odwzorowanie $f \colon Q_1 \to Q_2$ spełniające warunek
    \begin{equation}
        \forall x, y \in Q_1 : f(x \triangleright y) = f(x) \triangleright f(y),
    \end{equation}
    nazywamy homomorfizmem.
\end{definition}

Wiele znanych struktur algebraicznych okazuje się być źródłem kwandli.

\begin{example}[kwandel cykliczny/diedralny]
\index{kwandel!cykliczny}%
\index{kwandel!diedralny}%
    Grupa abelowa z działaniem $x \triangleright y = 2y - x$.
\end{example}

\begin{example}[kwandle sprzężone]
\index{kwandel!sprzężony}%
    Grupa z działaniem $x \triangleright y = y^{-n} x y^n$ dla każdego $n \in \N$.
\end{example}

\begin{example}[kwandel Alexandera]
\index{kwandel!Alexandera}%
    Moduł nad pierścieniem $\Z[t, 1/t]$ wielomianów Laurenta z~działaniem $x \triangleright y = tx + (1-t) y$.
\end{example}

\begin{example}[kwandel symplektyczny]
\index{kwandel!symplektyczny}%
    Przestrzeń liniowa i antysymetryczna forma dwuliniowa $\langle \cdot | \cdot \rangle$ z działaniem $x \triangleright y = x + \langle x | y \rangle y$.
\end{example}

Joyce w swojej rozprawie doktorskiej przypisał każdemu węzłowi $K$ pewien szczególny kwandel $Q(K)$, kwandel podstawowy.
\index[persons]{Joyce, David}%
\index{kwandel!podstawowy}%
Definicja tego obiektu jest dość zawiła: łuki diagramu są generatorami, zaś skrzyżowania odpowiadają za relacje.
Joyce pokazał, że kwandel $Q(K)$ wyznacza węzeł $K$ jednoznacznie z dokładnością do orientacji.
Nie czyni to jednak nowego niezmiennika użytecznym, gdyż wyznaczenie go nawet w najprostszych przypadkach stanowi trudność.
Niebrzydowski, Przytycki \cite{niebrzydowski09} pokazali na przykład w 2009, że
\index[persons]{Niebrzydowski, Maciej}%
\index[persons]{Przytycki, Józef}%
% We prove that the fundamental quandle of the trefoil knot is isomorphic to the projective primitive subquandle of transvections of the symplectic space Z ⊕ Z. The last quandle can be identified with the Dehn quandle of the torus and the cord quandle on a 2-sphere with four punctures.

\begin{example}
    Kwandel podstawowy trójlistnika  jest izomorficzny z~pewnym rzutowym pierwotnym podkwandlem\footnote{Cokolwiek to znaczy!} odwzorowań liniowych przestrzeni symplektycznej $\Z \oplus \Z$
\end{example}

Aksjomaty grupy można wzmacniać (grupy abelowe) lub osłabiać (monoidy).
Podobnie czyni się z aksjomatami kwandli.
Kwandle inwolutywne odpowiadają węzłom bez orientacji, wraki dobrze opisują węzły obramowane, i tak dalej.
\index{węzeł!niezorientowany}%
% TODO: w zorientowanym dać patrz też do niezorientowanego?
\index{węzeł!obramowany}%

\begin{definition}[kwandel inwolutywny]
\index{kwandel!inwolutywny}%
\index{kei|see {kwandel inwolutywny}}%
    Kwandel $Q$, w którym dla wszystkich $x, y \in Q$ zachodzi $x \triangleleft (x \triangleleft y) = y$, nazywamy inwolutywnym albo kei.
\end{definition}

Kwandle inwolutywne badał jako pierwszy Mituhisa Takasaki (1943).
\index[persons]{Takasaki, Mituhisa}%
Szukał niełącznej struktury, która dobrze opisywałaby odbicia w skończonej geometrii.

% DICTIONARY;shelf;półka;-
\begin{definition}[półka]
\index{półka}%
    Zbiór $X$ wyposażony w dwuargumentowe działanie $\triangleright$ taki, że dla wszystkich elementów $x, y, z \in X$ zachodzi $(x \triangleright y) \triangleright z = (x \triangleright z) \triangleright (y \triangleright z)$, nazywamy półką.
\end{definition}

\begin{example}
\index{warkocz}%
% TODO: warkocz czy grupa warkoczy?
    Niech\footnote{Patrz \url{http://nlab-pages.s3.us-east-2.amazonaws.com/nlab/show/shelf\#infinite_braid_group}.} $B_\infty$ oznacza grupę wszystkich warkoczy, zaś $\phi$ będzie jej endomorfizmem posyłającym generator $\sigma_k$ na $\sigma_{k+1}$.
    Zbiór $B_\infty$ z działaniem $a \triangleleft b = a\phi(b)\sigma_1 \phi{a} ^{-1}$ jest półką.
\end{example}

To nieprzetłumaczalna gra słów: dwie półki (\emph{shelves}), lewa i prawa, które dobrze do siebie pasują, dają stojak (\emph{rack}, czyli dla nas wrak).
Jak napisała Crans w pracy \cite[s. 86]{crans04}: \emph{,,Just as a rack is comprised of two shelves which fit together nicely, a quandle is made up of two spindles''}.
\index[persons]{Crans, Alissa}%

Półka stanowi uogólnienie dwóch obiektów -- wraków i wrzecion.

% DICTIONARY;wrack;wrak
\begin{definition}[wrak]
\index{wrak}%
    Zbiór $X$ z dwuargumentowym działaniem $\triangleright$ takim, że dla każdej trójki elementów $x, y, z \in X$ zachodzi:
    \begin{enumerate}
        \item odwzorowanie $\beta_y \colon X \to X$ dane wzorem $\beta_y(x) = x \triangleright y$ jest odwracalne,
        \item $(x \triangleright y) \triangleright z = (x \triangleright z) \triangleright (y \triangleright z)$
    \end{enumerate}
    nazywamy wrakiem.
\end{definition}

\begin{example}
    % https://www1.cmc.edu/pages/faculty/VNelson/quandles.html
    Zbiór $X = \{1, 2, \ldots, n\}$ i permutacja $\sigma \in S_n$ z działaniem $x \triangleright y = \sigma(x)$.
\end{example}

\begin{example}
    % https://www1.cmc.edu/pages/faculty/VNelson/quandles.html
    Moduł nad pierścieniem $\Z[t^{\pm 1}, s]/(s^2 - (1-t)s)$ z działaniem $x \triangleright y = tx+sy$.
\end{example}

\index{węzeł!obramowany}% framed?
Wraki są naturalnym niezmiennikiem węzłów obramowanych, bo dobrze współgrają z II, III oraz podwójnym I ruchem Reidemeistera, który to nie zmienia spinu diagramu:
\begin{comment}
\[
    \LargeReidemeisterOneLeftRightQuandleProof
    \cong
    \LargeReidemeisterOneStraightQuandleProofRotated
\]
\end{comment}

% DICTIONARY;spindle;wrzeciono
\begin{definition}[wrzeciono]
\index{wrzeciono}%
    Zbiór $X$ z dwuargumentowym działaniem $\triangleright$ taki, że dla wszystkich elementów $x, y, z \in X$ zachodzi:
    \begin{enumerate}
        \item $x \triangleright x = x$,
        \item $(x \triangleright y) \triangleright z = (x \triangleright z) \triangleright (y \triangleright z)$
    \end{enumerate}
    nazywamy wrzecionem.
\end{definition}

Zatem kwandle to wraki, które są też wrzecionami.
Muszę w~tym miejscu wtrącić uwagę językową.
Conway nazwał wraki wrakami (\emph{wracks}), by częściowo zażartować z~nazwiska jego kolegi Gavina Wraitha, a częściowo by zaznaczyć, że są one tym, co zostaje z~grupy, w~której zapomniano o~mnożeniu, ale nie sprzęganiu (w~języku angielskim co najmniej od XVI wieku funkcjonuje zwrot ,,wrack and ruin'' oznaczający zniszczenie).
\index[persons]{Conway, John}%
\index[persons]{Wraith, Gaven}%
Obecnie dominuje określenie \emph{racks}.

Jak wyglądały poszukiwania małych kwandli?
Dionísio, Lopes \cite{lopes03} znaleźli 10 kwandli Alexandera, które odróżniają 249 węzłów pierwszych do 10 skrzyżowań.
\index[persons]{Dionísio, Miguel}%
\index[persons]{Lopes, Pedro}%
Po upływie prawie dekady Vendramin \cite{vendramin12} wytropił wszystkie 431 kwandli spójnych rzędu 35 lub mniejszego.
\index[persons]{Vendramin, Leandro}%
Clark, Elhamdadi, Saito oraz Yeatman \cite{clark13} pokazali niedawno zbiór 26 kwandli, które razem odróżniają od siebie wszystkie 2977 zorientowanych węzłów pierwszych o~co najwyżej 12 skrzyżowaniach.
\index[persons]{Clark, William}%
\index[persons]{Elhamdadi, Mohamed}%
\index[persons]{Saito, Masahico}%
\index[persons]{Yeatman, Timothy}%
Największy z~nich jest rzędu 182.

\begin{definition}[kwandel spójny]
\index{kwandel!spójny}%
    Niech $Q$ będzie kwandlem.
    W grupie wszystkich automorfizmów $Q$ można wyróżnić grupę generowaną przez automorfizmy wewnętrzne $\beta_y(x) = x \triangleright y$.
    Jeżeli działanie tej podgrupy na $Q$ jest przechodnie, kwandel nazywamy spójnym.
\end{definition}

Dużo otwartych problemów dotyczących kwandli można znaleźć w \cite[s. 455-465]{ohtsuki02}.

\index{kwandel|)}

% Koniec sekcji Kwandle i wraki

