
\subsection{Macierz Goeritza}
\index{macierz!Goeritza|(}%
Istnieje jeszcze jedna kombinatoryczna metoda badania węzłów, która prowadzi między innymi do pojęcia wyznacznika.
Tuż przed wojną Lebrecht Goeritz pokazał, jak diagram węzła wyznacza specjalną formę kwadratową.
\index[persons]{Goeritz, Lebrecht}%
Nieco później Trotter zmodyfikował jego pomysł, by sygnatura formy stanowiła niezmiennik splotów.
\index[persons]{Trotter, Hale}%
Gordon, Litherland, Murasugi ujednolicili dwa wyżej wymienione podejścia w~pracy \cite{litherland81}.
\index[persons]{Gordon, Cameron}%
\index[persons]{Litherland, Richard}%
\index[persons]{Murasugi, Kunio}%
My opiszemy krótko macierz Goeritza.

Ustalmy uszachowiony diagram $D$ dla splotu $L$.
\index{uszachowienie}%
Oznaczmy białe obszary $0, 1, \ldots, m$, przy czym $0$ jest obszarem nieograniczonym.
Przydzielmy skrzyżowaniom znaki:
\begin{comment}
\begin{figure}[H]
    \begin{minipage}[b]{.48\linewidth}
    \centering
    \LargePlusCrossingChessboard
    \end{minipage}
    %
    \begin{minipage}[b]{.48\linewidth}
    \centering
    \LargeMinusCrossingChessboard
    \end{minipage}
\end{figure}
\end{comment}

\begin{definition}
    Macierz Goeritza powstaje przez skreślenie z~macierzy $G_+$ jednego wiersza oraz jednej kolumny:
    \begin{equation}
        G_+=\begin{pmatrix}
        G_{00} & \cdots & G_{0m} \\
        \vdots & \ddots & \vdots \\
        G_{m0} & \cdots & G_{mm}
        \end{pmatrix},
    \end{equation}
    gdzie jeśli $i\neq j$, to $G_{ij}$ jest sumą znaków skrzyżowań przyległych do $i$ oraz $j$.
    Dla $i = j$, $G_{ii}$ jest minus sumą znaków skrzyżowań wokół $j$-tego obszaru.
\end{definition}

Macierz $G_+$ posiada dwie własności pozwalające wykryć proste błędy rachunkowe: jest symetryczna, a~jej kolumny i~wiersze sumują się do zera.
Jest przy tym zazwyczaj mniejsza od macierzy kolorującej.

\begin{proposition}
    Niech $K$ będzie ustalonym węzłem, $G$ jego macierzą Goeritza, zaś $A$ macierzą kolorującą.
    Z~dokładnością do znaku, obie macierze mają ten sam wyznacznik: $\det G = \pm \det A$.
\end{proposition}

Nie możemy niestety podać dowodu tego faktu, wymaga bowiem znajomości topologii algebraicznej, której wolelibyśmy nie zakładać.
Macierz Goeritza nie jest niezmiennikiem splotów.
Mamy jednak:

\begin{proposition}
    Niech $D_1, D_2$ będą dwoma diagramami ustalonego splotu.
    Wtedy macierz Goeritza $G_2$ można otrzymać z macierzy $G_1$ w skończonej liczbie kroków:
    \begin{itemize}
        \item zamiany macierzy $G$ na $PGP^{-1}$, gdzie $P$ i~$P^{-1}$ mają całkowite wyrazy
        \item dopisania lub skreślenia $\pm 1$ na końcu przekątnej (dla węzłów) albo $-1, 0, 1$ (dla splotów).
    \end{itemize}
\end{proposition}

\begin{proof}
\index[persons]{Goeritz, Lebrecht}%
\index[persons]{Kneser, Martin}%
\index[persons]{Kyle, Roger}%
\index[persons]{Puppe, Dieter}%
    Pierwszy był oczywiście piszący po niemiecku Goeritz \cite{goeritz33}.

    Dwie dekady później opublikowana została jeszcze jedna praca po niemiecku Knesera, Puppego \cite{kneser53}.
    Główny wynik tamże zależy od mało znanego twierdzenia Meyera o liczbie klas form kwadratowych w genusie (?) i równie tajemniczego twierdzenia Eichlera o spinorowych reprezentacjach grupy ortogonalnej (??).
    % Meyer: Vierteljschr. Naturforsch. Ges. Zürich 36, 241–250 (1891)
    % Eichler: MR0051875

    Kyle napisał prawie w tym samym czasie \cite{kyle54} i nie zrobił tego po niemiecku.
    % znalazłem te dowody w https://arxiv.org/pdf/0909.1118.pdf
\end{proof}

\index{macierz!Goeritza|)}

