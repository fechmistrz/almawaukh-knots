
\subsection{Macierz Goeritza}
\index{macierz!Goeritza|(}%
Tuż przed wojną Lebrecht Goeritz \cite{goeritz33} pokazał, że każdy diagram węzła wyznacza formę kwadratową.
Macierz tej formy nie jest niezmiennikiem węzłów, ale jej symbole Hassego oraz wyznacznik (bez znaku) już tak.
% https://encyclopediaofmath.org/wiki/Hasse_invariant
\index[persons]{Goeritz, Lebrecht}%
Ani Kawauchi \cite{kawauchi96}, ani Murasugi \cite{murasugi96} nie wspominają explicite odkrycia Goerita; robią to Burde, Zieschang, Heusener \cite[s. 258-262]{burde14}, za co bardzo im dziękujemy.

Trotter zmodyfikował pomysły Goeritza i otrzymał inną formę kwadratową, sygnatura formy jest niezmiennikiem splotów.
\index[persons]{Trotter, Hale}%
Sygnaturę (razem z dalszymi uogólnieniami) opisujemy w~późniejszej podsekcji \ref{sub:signature}, teraz skupiając się na macierzy Goeritza.

\begin{definition}
    Ustalmy uszachowiony diagram $D$ dla splotu $L$.
    \index{uszachowienie}%
    Oznaczmy białe obszary $0, 1, \ldots, m$, przy czym $0$ jest obszarem nieograniczonym.
    Przydzielmy skrzyżowaniom znaki:
    \begin{comment}
    \begin{figure}[H]
        \begin{minipage}[b]{.48\linewidth}
        \centering
        \LargePlusCrossingChessboard
        \end{minipage}
        %
        \begin{minipage}[b]{.48\linewidth}
        \centering
        \LargeMinusCrossingChessboard
        \end{minipage}
    \end{figure}
    \end{comment}
    Macierz Goeritza powstaje przez skreślenie z~macierzy $G_+$ jednego wiersza oraz jednej kolumny:
    \begin{equation}
        G_+=\begin{pmatrix}
        G_{00} & \cdots & G_{0m} \\
        \vdots & \ddots & \vdots \\
        G_{m0} & \cdots & G_{mm}
        \end{pmatrix},
    \end{equation}
    gdzie jeśli $i\neq j$, to $G_{ij}$ jest sumą znaków skrzyżowań przyległych do $i$ oraz $j$.
    Dla $i = j$, $G_{ii}$ jest minus sumą znaków skrzyżowań wokół $j$-tego obszaru.
\end{definition}

Macierz $G_+$ posiada dwie własności pozwalające wykryć proste błędy rachunkowe: jest symetryczna, a~jej kolumny i~wiersze sumują się do zera.
Jest przy tym zazwyczaj mniejsza od macierzy kolorującej.

\begin{proposition}
    Niech $K$ będzie węzłem.
    Ustalmy diagram $D$ i oznaczmy przez $G, A$ kolejno macierz Goeritza oraz macierz kolorującą tego diagramu.
    Wtedy  $\det G = \pm \det A$.
\end{proposition}

\begin{proof}[Niedowód]
    Notatki z~kursu MX4540 w~Aberdeen, gdzie pierwszy raz przeczytaliśmy o tej równości, sugerują, że dowód wymaga znajomości topologii algebraicznej, a tego wolelibyśmy uniknąć.
    Chodzi prawdopodobnie o dowód Lickorisha \cite[s. 99]{lickorish97}.
    Ale potem znaleźliśmy artykuł Kolaya \cite{kolay19}, który używa tylko algebry liniowej.
\end{proof}    

Macierz Goeritza nie jest niezmiennikiem splotów.
Mamy jednak:

\begin{proposition}
    Niech $D_1, D_2$ będą dwoma diagramami ustalonego splotu.
    Wtedy macierz Goeritza $G_2$ można otrzymać z macierzy $G_1$ w skończonej liczbie kroków:
    \begin{itemize}
        \item zamiany macierzy $G$ na $PGP^{-1}$, gdzie $P$ i~$P^{-1}$ mają całkowite wyrazy
        \item dopisania lub skreślenia $\pm 1$ na końcu przekątnej (dla węzłów) albo $-1, 0, 1$ (dla splotów).
    \end{itemize}
\end{proposition}

\begin{proof}[Niedowód]
\index[persons]{Goeritz, Lebrecht}%
\index[persons]{Kneser, Martin}%
\index[persons]{Kyle, Roger}%
\index[persons]{Puppe, Dieter}%
    Pierwszy był oczywiście piszący po niemiecku Goeritz \cite{goeritz33}.

    Dwie dekady później opublikowana została jeszcze jedna praca po niemiecku Knesera, Puppego \cite{kneser53}.
    Główny wynik tamże zależy od mało znanego twierdzenia Meyera o liczbie klas form kwadratowych w genusie (?) i równie tajemniczego twierdzenia Eichlera o spinorowych reprezentacjach grupy ortogonalnej (??).
    % Meyer: Vierteljschr. Naturforsch. Ges. Zürich 36, 241–250 (1891)
    % Eichler: MR0051875

    Kyle napisał prawie w tym samym czasie \cite{kyle54} i nie zrobił tego po niemiecku.
    % znalazłem te dowody w https://arxiv.org/pdf/0909.1118.pdf
\end{proof}

\index{macierz!Goeritza|)}

