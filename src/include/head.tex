\newcommand{\N}{\mathbb N}
\newcommand{\Z}{\mathbb Z}
\newcommand{\Q}{\mathbb Q}
\newcommand{\R}{\mathbb R}
\newcommand{\C}{\mathbb C}

\newcommand{\shrap}{\mathbin{\#}}
\DeclareMathOperator*{\bigshrap}{\#}

\newcommand{\bracket}[1]{\left\langle{#1}\right\rangle}

\newcommand{\alexander}{\Delta}
\newcommand{\conway}{\nabla}
\newcommand{\jones}{V}
% span?

\newcommand{\braid}{\operatorname{b}}
\newcommand{\bridge}{\operatorname{br}}
\newcommand{\crossing}{\operatorname{cr}}
\newcommand{\genus}{\operatorname{g}}
\newcommand{\linking}{\operatorname{lk}}
\newcommand{\ropelength}{\operatorname{len}}
\newcommand{\sign}{\operatorname{sgn}}
\newcommand{\stick}{\operatorname{s}}
\newcommand{\unknotting}{\operatorname{u}}
\newcommand{\volume}{\operatorname{vol}}
\newcommand{\writhe}{\operatorname{wr}}

\newcommand{\inversedcurvearrowright}{\rotatebox[origin=c]{180}{$\curvearrowleft$}}
\newcommand{\inversedcurvearrowleft}{\rotatebox[origin=c]{180}{$\curvearrowright$}}

\usepackage{comment}
\includecomment{comment}
\usepackage{makeidx}
\usepackage{enumitem}
\usepackage{booktabs}
\usepackage{longtable}
\usepackage[table]{xcolor}
\usepackage[colorinlistoftodos,prependcaption]{todonotes}
\usepackage{tikz}
\usetikzlibrary{arrows.meta}
\usetikzlibrary{decorations.markings}
\usetikzlibrary{decorations.pathreplacing}
\usetikzlibrary{knots}
\colorlet{darkblue}{blue!80!black}
\newcommand{\MalyNieWezel} {\begin{tikzpicture}[baseline=-0.65ex, scale=0.02]
	\begin{knot}[clip width=5, end tolerance=1pt]
		\strand[semithick] (0,0) circle (5);
	\end{knot}
\end{tikzpicture}}

\newcommand{\NieWezel} {\begin{tikzpicture}[baseline=-0.65ex, scale=0.04]
	\begin{knot}[clip width=5, end tolerance=1pt]
		\strand[semithick] (0,0) circle (5);
	\end{knot}
\end{tikzpicture}}

\tikzset{
	->-/.style={decoration={markings, mark=at position .5 with {\arrow{>}}},postaction={decorate}},
	-<-/.style={decoration={markings, mark=at position .5 with {\arrow{<}}},postaction={decorate}},
	LUK/.style ={
		draw=black,
		line join=miter,
		line cap=butt,
		miter limit=4.00,
		line width=0.2 mm
	},
	CIENKILUK/.style ={
		draw=black,
		line join=miter,
		line cap=butt,
		miter limit=4.00,
		line width=0.1 mm
	},
	TEKSTOWY/.style ={
		draw=black,
		line join=round,
		line cap=butt,
		miter limit=20.00,
		line width=0.2 mm
	},
	OBSZAR/.style={
		draw=none,
		fill=white!#1!red
	},
	OBSZAR/.default = 80,
}

% makes compilation faster, breaks few diagrams
% \usetikzlibrary{external}
% \tikzexternalize[prefix=tikz/]

\definecolor{lightgray}{gray}{0.9} % define lightgray
\let\oldtabular\tabular % alternate rowcolors for all tables
\let\endoldtabular\endtabular
\renewenvironment{tabular}
{\rowcolors{2}{white}{lightgray}\oldtabular}
{\endoldtabular}
\let\oldlongtable\longtable % alternate rowcolors for all long-tables
\let\endoldlongtable\endlongtable
\renewenvironment{longtable}
{\rowcolors{2}{white}{lightgray}\oldlongtable}
{\endoldlongtable}

% \author{Leon Suwalski}
% \title{Krótkie wprowadzenie do teorii węzłów (wersja robocza)}
% \date{2018}

\makeindex
\begin{document}
% \maketitle

% 1
\thispagestyle{empty}
{\noindent\fontsize{18pt}{18pt}\selectfont Księgozbiór matemagiczny, tom 61}

\noindent\makebox[\linewidth]{\rule{\textwidth}{1pt}}

\newpage

% 2
\thispagestyle{empty}
\phantom{nothing}
\newpage

% 3
\thispagestyle{empty}
{\noindent\fontsize{18pt}{18pt}\selectfont Casimir Allard}

\noindent\makebox[\linewidth]{\rule{\textwidth}{1pt}}

\vspace{10mm}

{\noindent\fontsize{24pt}{24pt}\selectfont \textbf{Kombinatoryczna\\teoria węzłów}}
\vspace{10mm}

{\noindent\fontsize{14pt}{14pt}\selectfont  Wydanie drugie poprawione}

\newpage

% 4
\thispagestyle{empty}
{\noindent Prof. Casimir Allard\\
Université Bordeaux I\\
351 Cours de la Libération\\
33400 Talence, Francja}
\vspace{5mm}

{\noindent \textbf{Kategorie MSC 2010}\\57M25, 57Q45} \vspace{5mm}

{\noindent \textbf{Tytuł oryginału}\\La théorie combinatoire des næuds} \vspace{5mm}

{\noindent \textbf{Z francuskiego tłumaczyła}\\Julia Mróz} \vspace{5mm}

{\noindent \textbf{Projekt okładki}\\Wulfgang Kot} \vspace{5mm}

{\noindent \textbf{Redaktor}\\Radosław Jagoda} \vspace{5mm}

{\noindent \textbf{Redaktor techniczny}\\Klara Chmiel}\vspace{5mm}

{\noindent \textbf{Korektorzy}\\Jerzy Maślanka, Zuzanna Szpinak}

\vfill

{\noindent Copyleft by Antykwariat Czarnoksięski, Gorzów Wielkopolski 2020. Książka, a także każda jej część, mogą być przedrukowywane oraz w jakikolwiek inny sposób reprodukowane czy powielane mechanicznie, fotooptycznie, zapisywane elektronicznie lub magnetycznie, oraz odczytywane w środkach publicznego przekazu bez pisemnej zgody wydawcy.}

\vspace{5mm}

{\noindent Przygotowano w systemie \TeX, wydrukowano na siarczystym papierze.}

\chapter*{Przedmowa}
Książka wprowadza we współczesną teorię węzłów, najważniejsze metody i obszary tej wciąż niedocenianej, choć żywo rozwijającej się dyscypliny matematycznej.
Tor wykładu wzorowany jest na seminarium z teorii węzłów jakie odbyło się na Wydziale Matematyki Uniwersytetu Wrocławskiego w semestrze zimowym 2013/2014 i dlatego podręcznik nadaje się szczególnie do pierwszego czytania dla studentów wyższych lat studiów matematycznych, natomiast dla młodych pracowników naukowych może okazać się pożytecznym źródłem odsyłaczy do prac, które zainspirują do dalszych badań.
Zwięźle i na tyle, na ile było to możliwe opisuję rozmaite narzędzia stosowane do badania węzłów, splotów, supłów i innych: klasyczne niezmienniki numeryczne i ruchy Reidemeistera, niezmienniki kolorujące i spokrewnione z nimi kwandle, potem diagramatyczny wielomian Jonesa, homologiczny wielomian Alexandera oraz dość świeże niezmienniki typu skończonego.
Ze względu na niepełne zrozumienie maszynerii topologi algebraicznej, rozdział czwarty został napisany trochę niechlujnie, jednakże żywię nadzieję poprawić się w następnym wydaniu trzymanej przez Ciebie książki.
W ostatnim, piątym rozdziale przedstawiam krótko charakterystyczne rodziny: warkocze, dwumostowe sploty, mutanty, precle, sploty torusowe, satelitarne i hiperboliczne, wreszcie węzły plastrowe i taśmowe, obiekt zainteresowań czterowymiarowej topologii.
Dołączam również tablice węzłów pierwszych o małej liczbie skrzyżowań.
Z takim zakresem i ujęciem materiału pozycja jest unikalna nie tylko we francuskiej, ale i w światowej literaturze matematycznej.

Serdecznie dziękuję Ś. G. oraz J. Ś. bez których ten tekst nigdy by nie powstał.\\${}$

\begin{flushright}
Casimir Allard,\\Marsylia, 4 lutego 2019
\end{flushright}

\section*{Przedmowa do wydania drugiego}
Od poprzedniego wydania minął ponad rok i~trochę się w~tym czasie zmieniło.

Dodałem informację o~różnych notacjach dla węzłów, poprawiłem informację na temat płci niektórych osób, napisałem nową sekcję o~macierzy Seiferta, wymieniłem niezmienniki nieodróżniające mutantów.
Pojawiła się klasyfikacja kwandli, nierówność Mortona, homologia Floera i kilka mniejszych matematycznych obiektów.

Niektóre istniejące sekcje przeniosłem w~lepsze mam nadzieję miejsca, niektóre trudne dowody otrzymały swój zarys.
Tam, gdzie to możliwe, podałem rys historyczny, na przykład w~sekcji o~kolorowaniach wspomniałem o~Foxie, który chciał uczynić teorię węzłów prostszą dla swoich studentów.
Na końcu książki umieściłem krótki słowniczek angielsko-polski.

Wreszcie usunąłem znalezione literówki i~uzupełniłem luki, zarówno w~bibliografii jak i~rysunkach.
Pojawiły się też drobne poprawki typograficzne, dla przykładu od teraz wszystkie równania powinny być numerowane.

Żywię nadzieję, że korzystanie z~książki będzie równie (jeśli nie bardziej) przyjemne, co w~przypadku pierwszego jej wydania.

\begin{flushright}
Casimir Allard,\\Marsylia, 19 sierpnia 2020
\end{flushright}

\section*{Przedmowa do wydania trzeciego}
Do napisania...

\begin{flushright}
Casimir Allard,\\Marsylia, ?? ??????? 2022
\end{flushright}

\tableofcontents