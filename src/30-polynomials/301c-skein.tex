
\subsection{Relacja kłębiasta (definicja druga)}

\begin{definition}[relacja kłębiasta]
\label{skein_symbols}%
% TODO: uwspólnić definicje relacji kłębiastych?
\index{relacja kłębiasta}%
    Niech $L$ będzie zorientowanym splotem z ustalonym diagramem oraz skrzyżowaniem.
    Oznaczmy przez $L_+, L_-, L_0$ trzy diagramy splotów, które różnią się jedynie na małym obszarze wokół ustalonego skrzyżowania:
\begin{comment}
    \begin{figure}[H]
        \centering
        \begin{minipage}[b]{.3\linewidth}
            \centering
            \[\LargePlusCrossingArrows\]
            \subcaption{$L_+$}
        \end{minipage}
        \begin{minipage}[b]{.3\linewidth}
            \centering
            \[\LargeMinusCrossingArrows\]
            \subcaption{$L_-$}
        \end{minipage}
        \begin{minipage}[b]{.3\linewidth}
            \centering
            \[\LargeJustSmoothing\]
            \subcaption{$L_0$}
        \end{minipage}
    \end{figure}
\end{comment}
\noindent
Gdy wartości niezmiennika (zazwyczaj zorientowanych) splotów $f$: $f(L_+)$, $f(L_-)$, $f(L_0)$ są związane jakimś wielomianowym równaniem, którego współczynniki nie zależą od wyboru splotu $L$, to mówimy, że niezmiennik $f$ spełnia relację kłębiastą.
\end{definition}

% DICTIONARY;skein;kłąb;-
% DICTIONARY;skein relation;relacja kłębiasta;-
Termin ,,skein'' (kłąb) wprowadził Conway około roku 1970, kontynuując tradycję używania słów, które kojarzą się ze sznurkami.
\index[persons]{Conway, John}%
Po polsku powinniśmy mówić o relacji motkowej, tak jak zaproponował dawno temu Przytycki, ale my nie przepadamy za tym terminem (i nie potrafimy powiedzieć, dlaczego).
Jednocześnie nie znamy się na robieniu na drutach niczym Paula, więc może nie popełniamy zbrodnii mówiąc o~kłębach zamiast o~motkach.

Kiedy pracuje się z niezorientowanymi węzłami, używa się innego (!) zestawu oznaczeń:
\begin{comment}
    \begin{figure}[H]
        \centering
        \begin{minipage}[b]{.23\linewidth}
            \centering
            \[\LargeMinusCrossing\]
            \subcaption{$L_+$}
        \end{minipage}
        \begin{minipage}[b]{.23\linewidth}
            \centering
            \[\LargePlusCrossing\]
            \subcaption{$L_-$}
        \end{minipage}
        \begin{minipage}[b]{.23\linewidth}
            \centering
            \[\LargeAlphaSmoothing\]
            \subcaption{$L_0$}
        \end{minipage}
        \begin{minipage}[b]{.23\linewidth}
            \centering
            \[\LargeBetaSmoothing\]
            \subcaption{$L_\infty$}
        \end{minipage}
    \end{figure}
\end{comment}

\begin{example}
    Wielomian Alexandera każdego węzła jest wyznaczony jednoznacznie przez relację kłębiastą
    \begin{equation}
        \label{eqn:alexander_skein}
        \alexander_{L_+}(t) - \alexander_{L_-}(t) - (t^{1/2} - t^{-1/2}) \alexander_{L_0}(t) = 0
    \end{equation}
    z warunkiem brzegowym $\alexander_{\SmallUnknot}(t) = 1$.
\end{example}

Wzór ten, choć znany był Alexanderowi, nie zyskał przez wiele dekad uwagi matematyków.
\index[persons]{Alexander, James}%
Mogło tak być, gdyż w pracy \cite{alexander1928} znalazł się on na samym końcu, pod nagłówkiem ,,twierdzenia różne''.
Na nowo odkrył go Conway: chcąc szybko liczyć wielomian Alexandera zaproponował, by reparametryzować go wzorem $\alexander(x^2) = \conway(x - 1/x)$.
Spełnia wtedy zależność
\begin{equation}
    \conway_{L_+}(x)- \conway_{L_-}(x) = x \conway_{L_0}(x).
\end{equation}

Relacja kłębiasta wystarcza do wyznaczenia $\alexander_L$ każdego splotu na mocy lematu \ref{lem:unknotting_well_defined}.

\begin{proposition}
\index{splot!rozszczepialny}%
\label{prp:alexander_unlinks}
    Niech $L$ będzie splotem rozszczepialnym.
    Wtedy $\alexander_L(t) \equiv 0$.
\end{proposition}

\begin{proof}
    Skorzystamy z~relacji kłębiastej.
    Niech $L_0$ będzie splotem rozszczepialnym z~dwoma ogniwami.
    Wtedy węzły $L_+$ oraz $L_-$ powstałe przez dodanie skrzyżowania między ogniwami są tego samego typu, zatem
    \begin{equation}
        \alexander_{L_0} = \frac{\alexander_{L_+} - \alexander_{L_-}}{t^{1/2} - t^{-1/2}} = 0,
    \end{equation}
    a to chcieliśmy udowodnić.
\end{proof}

Implikacja w drugą stronę jest fałszywa.
Niech $\sigma_* = \sigma_{2} \sigma_{3}^{-2} \sigma_{2}$.
Domknięcie warkocza $\sigma_{1} \sigma_* \sigma_{1} \sigma_{3} \sigma_* \sigma_{1} \sigma_{3} \sigma_* \sigma_{3}$ nie jest rozszczepialne, ale jego wielomian Alexandera jest zerem.
% TODO: nie potrafię znaleźć, kto pierwszy odkrył takie warkocze
% patrz też https://math.stackexchange.com/questions/3740577/why-the-multivariate-alexander-polynomial-of-l14n63195-is-zero
\index{warkocz}%
Warkocze poznamy w rozdziale piątym.

\begin{corollary}
    Wielomian Alexandera nie odróżnia od siebie niesplotów.
\end{corollary}

Wady tej nie posiada wielomian Jonesa.

% koniec podsekcji Relacja kłębiasta

