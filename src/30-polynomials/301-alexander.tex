
\section{Wielomian Alexandera}
\index{wielomian!Alexandera|(}%
Wielomian Alexandera to najstarszy niezmiennik tego typu, odkryty w 1923 roku \cite{alexander1923}.
Jego pierwsza definicja była czysto topologiczna algebraicznie: niech $K$ będzie węzłem w~3-sferze, zaś $X$ nieskończonym nakryciem cyklicznym jego dopełnienia (otrzymanym przez rozcięcie dopełnienia wzdłuż powierzchni Seiferta).
Na przestrzeni $X$ oraz grupie homologii $H_1(X)$, działa automorfizm $t$, który czyni z~niej moduł nad pierścieniem $\Z[t, t^{-1}]$, i~to skończenie prezentowalny.
Jeśli posiada przedstawienie z~$r$ generatorami i~$s$ relacjami, gdzie $r \le s$, rozpatrzmy ideał generowany przez minory $r \times r$ macierzy prezentacji (jeśli nie, weźmy ideał zerowy).
Alexander pokazał, że ideał ten zawsze jest niezerowy i~główny, generowany przez coś, co teraz nazywamy wielomianem Alexandera.

Nie jest to definicja, z którą praca stanowi przyjemność.
Dlatego podamy prostszy opis, oparty o równania kolorujące.
Później sprawdzimy, jaki wpływ na wielomian mają suma spójna, lustro i~rewers oraz jak Conway odkrył na nowo wielomian Alexandera, jednocześnie przyspieszając jego liczenie.
Na koniec pokażemy, co łączy go ze zdefiniowanymi wcześniej i~później numerycznymi niezmiennikami.


\subsection{Równania kolorujące (definicja pierwsza)}

\begin{definition}
\index{równanie kolorujące}%
    Wielomianowe równanie kolorujące związane ze skrzyżowaniem
\begin{comment}
    \[
        \LargePlusCrossingLabel
    \]
\end{comment}
    splotu zorientowanego to $g + tk - tg - h = 0$.
    Tylko orientacja górnego łuku ma znaczenie.
\end{definition}

Istotnie, wystarczy podstawić tutaj $t = -1$, by otrzymać szczególną definicję \ref{def:colouring_equation}.

\begin{definition}[wielomian Alexandera]
\label{def:alexander_polynomial}%
    Ustalmy diagram $D$ zorientowanego splotu $L$, gdzie nie ma żadnej krzywej zamkniętej (niewęzła $\SmallUnknot$).
    Przypiszmy etykiety $x_0, \ldots, x_m$ do włókien oraz $0, \ldots, m$ do skrzyżowań diagramu $D$.
    Niech $p_{ij}$ będzie współczynnikiem przy włóknie $x_j$ w~wielomianowym równaniu kolorującym nad wierzchołkiem $i$.
    Z macierzy $P=(p_{ij})$ wykreślmy po jednej kolumnie i~wierszu.
    Wyznacznik zmniejszonej macierzy to wielomian Alexandera, oznaczamy go $\alexander_L(t)$.
\end{definition}

Nasz nowy niezmiennik nie jest zwykłym wielomianem, tylko wielomianem Laurenta jednej zmiennej, czyli elementem pierścienia $\Z[t, t^{-1}]$.

\begin{proposition}
    \label{alexander_invariance}
    Wielomian Alexandera z~dokładnością do mnożenia przez jedności:
    \begin{equation}
        f(t) \equiv g(t) \iff \exists m \in \Z: f(t) = \pm t^m g(t)
    \end{equation}
    jest niezmiennikiem zorientowanych splotów.
\end{proposition}

W dowodzie niezmienniczości wyznacznika węzła skorzystaliśmy z~relacji między nim a~grupą kolorującą.
Poprzednie wydania książki zawierały sugestię, że elementarny (czyli taki, który nie korzysta z~teorii modułów) dowód niezmienniczości wielomianu Alexandera nie istnieje.
Sugestia ta była błędna.

Podany dowód dotyczy tylko splotów o spójnym diagramie.
Jeżeli mamy diagram, który nie spełnia tego założenia, to fakt \ref{prp:alexander_unlinks} orzeka, że jego wielomian Alexandera znika.

\begin{proof}
    Ustalmy diagram o~$k$ skrzyżowaniach, który rozcina płaszczyznę na $k+2$ obszarów i~utwórzmy macierz o~wymiarach $k \times k$, której kolumny odpowiadają obszarom, wiersze zaś skrzyżowaniom -- pomijając przy tym dwa sąsiadujące ze sobą obszary -- o~wyrazach ze współczynników równań kolorujących.
    Jej wyznacznik jest wielomianem Alexandera.

    Sąsiadującym ze sobą obszarom przypiszmy kolejne liczby całkowite tak, by obszar leżący po prawej stronie włókna miał niższy indeks.
    Pokażemy najpierw, że skasowanie kolumny indeksu $n$ oraz $n+1$ sprawia, że wyznacznik zmienia się co najwyżej o~czynnik $\pm t^m$ dla pewnego $m$.
    Niech $S_n$ oznacza sumę kolumn indeksu $n$.
    Każdy wiersz macierzy zawiera cztery niezerowe wyrazy: $\pm 1, \pm t$, zatem $\sum_n S_n = 0$.
    Równość ta zachodzi nawet po przemnożeniu kolumny indeksu $n$ przez $t^{-n}$: $\sum_n t^{-n}S_n = 0$, co prowadzi do relacji $\sum_n (t^{-n}-1) S_n = 0$.
    Jeśli więc indeks kolumny $v_j$ wynosi $n$, to $(t^{-n}-1)v_j$ jest kombinacją liniową innych kolumn niezerowego indeksu (ponieważ $t^0 - 1 = 0$).

    Rozpatrzmy macierze $M_{0,j}, M_{0,k}$, gdzie indeksy $j$-tej i~$k$-tej kolumny to odpowiednio $p$ i~$q$.
    Z powyższych rozważań wynika, że $(t^{-q}-1) \alexander_{0,j} = \pm (t^{-p}-1)\alexander_{0,k}$, ale indeksy obszarów są wyznaczone z~dokładnością do stałej addytywnej.
    Biorąc $i$-tą oraz $l$-tą kolumnę, indeksów $r$ oraz $s$, dostaniemy zależności
    \begin{align}
        (t^{r-q}-1) \alexander_{l,j} & = \pm (t^{r-p} - 1)\alexander_{l,i} \\
        (t^{q-s}-1) \alexander_{k,l} & = \pm (t^{q-r} - 1)\alexander_{k,i}
    \end{align}
    co prowadzi do
    \begin{equation}
        \alexander_{l,j} = \pm \frac{t^{q-r}(t^{r-p}-1)}{t^{q-s}-1} \alexander_{k,i}
    \end{equation}
    Położenie $p = r +1$, $s =q+1$ pokazuje, że różny wybór kolumn do skreślenia zmienia wyznacznik macierzy co najwyżej o~czynnik $\pm t^m$.

    Wprowadźmy jeszcze jedną techniczną definicję.
    Dwie kwadratowe macierze będą dla nas równoważne, jeśli można przejść od jednej do drugiej przy użyciu pięciu operacji:
    \begin{enumerate}
        \item przemnożenie wiersza lub kolumny przez $-1$;
        \item zamiana dwóch wierszy lub kolumn miejscami;
        \item dodanie jednego wiersza do innego (lub kolumny do innej);
        \item przemnożenie lub podzielenie kolumny przez $t$;
        \item rozszerzenie lub zmniejszenie macierzy o~$1$ na przekątnej i~zera w~innych miejscach.
    \end{enumerate}

    Ruchy Reidemeistera prowadzą do macierzy równoważnych wyjściowym.
    Każda z~tych operacji zmienia wyznacznik macierzy o~czynnik $\pm t^{-m}$, co kończy dowód.
\end{proof}

Zwyczajowo wielomian normalizuje się: bierze reprezentanta, który jest symetryczny w~zmiennych $t$ i $t^{-1}$ oraz przyjmuje w~punkcie $1$ wartość $\alexander_L(1) = 1$.
Odwrotnie, dowolny wielomian Laurenta z~całkowitymi współczynnikami o~takich własnościach jest wielomianem Alexandera pewnego węzła:

\begin{proposition}
\label{prp:alexander_hosokawa}%
    Każdy wielomian Laurenta $p(t)$ o~całkowitych współczynnikach taki, że $p(1/t) = p(t)$ i~$p(1) = \pm 1$ jest wielomianem Alexandera pewnego węzła.
\end{proposition}

\begin{proof}[Niedowód]
\index[persons]{Hosokawa, Fujitsugu}%
    Hosokawa w \cite{hosokawa58} udowodnił to dla pomocniczego wielomianu splotów
    \begin{equation}
        \frac{\Delta(t, \ldots, t)}{(1-t)^{\max(0, \mu - 2)}},
    \end{equation}
    gdzie $\mu$ oznacza liczbę ogniw.
    Książka Rolfsena \cite[s. 171-172]{rolfsen76} zawiera natomiast jawną konstrukcję węzła o~danym wielomianie Alexandera.
\end{proof}

\begin{proposition}
\label{alexander_no_detects_unknot}%
    Wielomian Alexandera nie wykrywa niewęzła.
\end{proposition}

\begin{proof}
    $\alexander(11_{471}) = 1$.
\end{proof}

Wszystkie przykłady węzłów pierwszych do 12 skrzyżowań o~trywialnym wielomianie Alexandera to $11_{471} = 11n_{34}$, $11_{473} = 11n_{42}$, $12n_{313}$, $12n_{430}$.
Dzięki pracy Sakumy \cite{sakuma20} dowiedzieliśmy się, że Seifert \cite{seifert35}, Whitehead \cite{whitehead37}, Kinoshita i~Terasaka \cite{kinoshita57} pokazali, jak znaleźć więcej węzłów o trywialnym wielomianie Alexandera.
% In fact, H. Seifert [301], J. H.C. Whitehead [336], and Kinoshita-Terasaka [175] gave systematic construction of nontrivial knots with trivial Alexander polynomial.
\index[persons]{Kinoshita, Shinichi}%
\index[persons]{Sakuma, Makoto}%
\index[persons]{Seifert, Herbert}%
\index[persons]{Terasaka, Hidetaka}%
\index[persons]{Whitehead, John}%
W~tej książce mamy już jedno źródło całej serii przykładów: fakt \ref{prp:pretzel_alexander}, między innymi $(-3, 5, 7)$ precel jest takim przykładem.

% koniec podsekcji Równania kolorujące




\subsection{Wielomian Alexandera a operacje na węzłach}
Wielomian Alexandera nie odróżnia luster $mL$ i~rewersów $rL$ od wyjściowych splotów $L$:

\begin{proposition}
\index{lustro}%
\index{rewers}%
    Niech $L$ będzie zorientowanym splotem.
    Wtedy $\alexander_{mL}(t) = \alexander_L(1/t) = \alexander_{rL}(t)$.
\end{proposition}

\begin{proof}
    Po odbiciu diagramu względem pionowej prostej skrzyżowanie z~definicji \ref{def:colouring_equation} też się odbija.
    Równanie związane z~nim zmienia się według schematu:
    \begin{equation}
        a + tc - ta - b = 0 \rightleftharpoons a + tb - ta - c = 0
    \end{equation}
    Pierwsze równanie z~$t$ zamienionym na $1/t$ staje się drugim równaniem przemnożonym przez $-1/t$, zatem $\alexander_{mL}(t) = \alexander_L(1/t)$.
    
    Dowód drugiej równości przebiega analogicznie.
\end{proof}

\begin{proposition}
\label{prp:alexander_multiplicative}%
    Niech $K_1, K_2$ będą zorientowanymi węzłami.
    Wtedy
    \begin{equation}
        \alexander_{K_1 \shrap K_2}(t) \equiv \alexander_{K_1}(t) \alexander_{K_2}(t).
    \end{equation}
\end{proposition}

\begin{proof}
    Wybierzmy poniższe diagramy dla węzłów $K_1$ oraz $K$:
\begin{comment}
    \[\begin{tikzpicture}[baseline=-0.65ex, scale=0.07]
    %\useasboundingbox (-5, -5) rectangle (5,5);
    \begin{knot}[clip width=7, end tolerance=1pt]
        \strand[thick] (-70, -20) rectangle (-30, 20);
        \strand[thick] (30, -20) rectangle ( 70, 20);
        \strand[thick] (-10, -10) [in=right, out=left] to (-25, 10);
        \strand[thick,-latex] (-25, 10) to (-30, 10);
        \strand[thick] (-30,-10) [in=left, out=right] to (-25, -10) to (-10, 10);
        \strand[thick] (-10, 10) [in=up, out=right] to (-5, 0) [in=right, out=down] to (-10, -10);

        % prawe strzalki
        \strand[thick] (30, 10) [in=right, out=left] to (25, 10) to (10, -10);
        \strand[thick,latex-] (30, -10) [in=right, out=left] to (25, -10) to (10, 10);
        \strand[thick] (10, 10) [in=up, out=left] to (5, 0) [in=left, out=down] to (10, -10);

        \node[darkblue] at (-50,10) [below] {$x_1,\ldots,x_{m-1}$};
        \node[red] at (-50,-10) [above] {$1,\ldots,m$};

        \node[darkblue] at (50,10) [below] {$y_1,\ldots,y_{n-1}$};
        \node[red] at (50,-10) [above] {$1,\ldots,n$};

        \node[darkblue] at (-20,-10)[below] {$x_m$};
        \node[darkblue] at (-10, 10)[above] {$x_0$};
        \node[darkblue] at (25,-10)[below] {$y_n$};
        \node[darkblue] at (10,10)[above] {$y_0$};
        \node[red] at ( 20,  0)[right]{$0$};
        \node[red] at (-20,  0)[left]{$0$};
    \end{knot}
    \end{tikzpicture}\]
\end{comment}
    Niech $M_1$ oraz $M_2$ oznaczają macierze otrzymane z~wielomianowych równań kolorujących dla $K_1$ oraz $K_2$ przez skreślenie pierwszej kolumny i~pierwszego wiersza.
    Wtedy $\alexander_{K_1}(t) = \det A$ oraz $\alexander_{K_2}(t) = \det B$.
    
    Poniższy diagram przedstawia sumę $K_1 \shrap K_2$:
\begin{comment}
    \[\begin{tikzpicture}[baseline=-0.65ex, scale=0.07]
        %\useasboundingbox (-5, -5) rectangle (5,5);
        \begin{knot}[clip width=5, end tolerance=1pt]
            \strand[thick] (-70, -20) rectangle (-30, 20);
            \strand[thick] (30, -20) rectangle ( 70, 20);
            \strand[thick] (-10, -10) [in=right, out=left] to (-25, 10);
            \strand[thick,-latex] (-25, 10) to (-30, 10);
            \strand[thick] (-30,-10) [in=left, out=right] to (-25, -10) to (-10, 10);
            \strand[thick] (-10, -10) to (10, -10);
            \strand[thick] (-10, 10) to (10, 10);

            % prawe strzalki
            \strand[thick] (30, 10) [in=right, out=left] to (25, 10) to (10, -10);
            \strand[thick,latex-] (30, -10) [in=right, out=left] to (25, -10) to (10, 10);

            \node[darkblue] at (-50,10) [below] {$x_1,\ldots,x_{m-1}$};
            \node[red] at (-50,-10) [above] {$1,\ldots,m$};

            \node[darkblue] at (50,10) [below] {$y_1,\ldots,y_{n-1}$};
            \node[red] at (50,-10) [above] {$1,\ldots,n$};

            \node[darkblue] at (-20,-10)[below] {$x_m$};
            \node[darkblue] at (0, 10)[above] {$z$};
            \node[darkblue] at (25,-10)[below] {$y_n$};
            \node[darkblue] at (0, -10)[below] {$x_0 = y_0$};
            \node[red] at ( 20,  0)[right]{$\zeta$};
            \node[red] at (-20,  0)[left]{$0$};
        \end{knot}
    \end{tikzpicture}\]
\end{comment}

    Ponumerujemy teraz łuki i skrzyżowania na nowo tak, by znalezienie wyznacznika nie wymagało dużo liczenia.
    Przyjmijmy kolejność dla łuków sumy od $x_0$, $x_1, \ldots, x_m$ przez $y_1, \ldots, y_n$ do $z$; zaś skrzyżowania ustawmy od $0, 1, \ldots, m$ (pochodzące od $K_1$) przez $1, \ldots, n$ (pochodzące od $K_2$) do $\zeta$.
    Wielomianowe równanie kolorujące dla $K_1 \shrap K_2$ nad wszystkimi skrzyżowaniami poza ostatnim są takie same, jak ich odpowiedniki przed dodaniem do siebie węzłów.
    (Jeżeli to stwierdzenie jest nieoczywiste, warto wykonać rysunek dla $K_1 = K_2 = 3_1$ i przeprowadzić stosowne rachunki).
    Natomiast nad skrzyżowaniem $\zeta$ stosowne równanie grzmi $(1-t)x_0 + tz - y_n = 0$.

    Z otrzymanej dużej macierzy skreślmy ponownie pierwszą kolumnę oraz pierwszy wiersz.
    Aby zakończyć dowód, musimy obliczyć jej wyznacznik.
    Mamy:
    \begin{align*}
        \alexander_{K_1 \shrap K_2}(t) & = \det \left(\begin{array}{cccc}
            M_1 &     &        & \\
                & M_2 &        & \\
                &     & \ddots & \\
                &     & -1     & t
    \end{array}\right) = \pm t \alexander_{K_1}(t) \alexander_{K_2}(t),
    \end{align*}
    gdzie ostatnia równość wynika z rozwinięcia Laplace'a wyznacznika względem ostatniej kolumny.
\end{proof}

Można zadać sobie pytanie, czy z równości $\Delta_{K_1}(t) = \Delta_{K_2}(t) \Delta_{K_3}(t)$ wynika, że $K_1$ jest sumą $K_2 \shrap K_3$ albo chociaż czy $K_1$ jest złożony.
Odpowiedź będzie negatywna, jak pokazuje przykład:
\begin{align}
    \Delta_{8_{17}}(t) & = 3(t^2 +t^{-2}) -8(t+t^{-1}) + 11 \\
    & = (t+t^{-1} + 1) (3(t+t^{-1}) - 5) \\
    & = \Delta_{3_1}(t) \Delta_{7_2}(t),
\end{align}
ponieważ wszystkie występujące w niej węzły są pierwsze.

% koniec podsekcji Wielomian Alexandera a operacje na węzłach




\subsection{Relacja kłębiasta (definicja druga)}

\begin{definition}[relacja kłębiasta]
% TODO: uwspólnić definicje relacji kłębiastych?
\index{relacja kłębiasta}%
    Niech $L$ będzie zorientowanym splotem z ustalonym diagramem oraz skrzyżowaniem.
    Oznaczmy przez $L_+, L_-, L_0$ trzy diagramy splotów, które różnią się jedynie na małym obszarze wokół ustalonego skrzyżowania:
\begin{comment}
    \begin{figure}[H]
        \centering
        \begin{minipage}[b]{.3\linewidth}
            \centering
            \[\LargePlusCrossingArrows\]
            \subcaption{$L_+$}
        \end{minipage}
        \begin{minipage}[b]{.3\linewidth}
            \centering
            \[\LargeMinusCrossingArrows\]
            \subcaption{$L_-$}
        \end{minipage}
        \begin{minipage}[b]{.3\linewidth}
            \centering
            \[\LargeJustSmoothing\]
            \subcaption{$L_0$}
        \end{minipage}
    \end{figure}
\end{comment}
    Mówimy, że niezmiennik zorientowanych splotów $f$ spełnia relację kłębiastą, jeżeli wartości $f(L_+)$, $f(L_-)$ i $f(L_0)$ są związane pewnym wielomianowym równaniem, niezależnie od wyboru splotu $L$.
\end{definition}

% DICTIONARY;skein;kłąb;-
% DICTIONARY;skein relation;relacja kłębiasta;-
Termin ,,skein'' (kłąb) wprowadził Conway około roku 1970, kontynuując tradycję używania słów, które kojarzą się ze sznurkami.
\index[persons]{Conway, John}%
Czasami mówi się o relacji motkowej, my nie zamierzamy używać tego synonimu.

\begin{definition}
    Niech $L$ będzie zorientowanym splotem.
    Wielomian Laurenta $\alexander_L(t) \in \Z[t^{\pm 1/2}]$, który spełnia relację kłębiastą
    \begin{equation}
        \alexander_{L_+}(t) - \alexander_{L_-}(t) - (t^{1/2} - t^{-1/2}) \alexander_{L_0}(t) = 0
    \end{equation}
    z warunkiem brzegowym $\alexander_{\SmallUnknot}(t) = 1$, nazywamy wielomianem Alexandera.
\end{definition}

Wzór ten, choć znany był Alexanderowi, nie zyskał przez wiele dekad uwagi matematyków.
\index[persons]{Alexander, James}%
Mogło tak być, gdyż w pracy \cite{alexander28} znalazł się on na samym końcu, pod nagłówkiem ,,twierdzenia różne''.
Na nowo odkrył go Conway: chcąc szybko liczyć wielomian Alexandera zaproponował, by reparametryzować go wzorem $\alexander(x^2) = \conway(x - 1/x)$.
Spełnia wtedy zależność
\begin{equation}
    \conway_{L_+}(x)- \conway_{L_-}(x) = x \conway_{L_0}(x).
\end{equation}

Relacja kłębiasta wystarcza do wyznaczenia $\alexander_L$ każdego splotu na mocy lematu \ref{lem:unknotting_well_defined}.

\begin{proposition}
\index{splot!rozszczepialny}%
\label{prp:alexander_unlinks}
    Niech $L$ będzie splotem rozszczepialnym.
    Wtedy $\alexander_L(t) \equiv 0$.
\end{proposition}

\begin{proof}
    Skorzystamy z~relacji kłębiastej.
    Niech $L_0$ będzie splotem rozsczepialnym z~dwoma ogniwami.
    Wtedy węzły $L_+$ oraz $L_-$ powstałe przez dodanie skrzyżowania między ogniwami są tego samego typu, zatem
    \begin{equation}
        \alexander_{L_0} = \frac{\alexander_{L_+} - \alexander_{L_-}}{t^{1/2} - t^{-1/2}} = 0,
    \end{equation}
    a to chcieliśmy udowodnić.
\end{proof}

Implikacja w drugą stronę jest fałszywa.
Niech $\sigma_* = \sigma_{2} \sigma_{3}^{-2} \sigma_{2}$.
Domknięcie warkocza $\sigma_{1} \sigma_* \sigma_{1} \sigma_{3} \sigma_* \sigma_{1} \sigma_{3} \sigma_* \sigma_{3}$ nie jest rozszczepialne, ale jego wielomian Alexandera jest zerem.
% TODO: nie potrafię znaleźć, kto pierwszy odkrył takie warkocze
% patrz też https://math.stackexchange.com/questions/3740577/why-the-multivariate-alexander-polynomial-of-l14n63195-is-zero
\index{warkocz}%
Warkocze poznamy w rozdziale piątym.

\begin{corollary}
    Wielomian Alexandera nie odróżnia od siebie niesplotów.
\end{corollary}

Wady tej nie posiada wielomian Jonesa.

% koniec podsekcji Relacja kłębiasta




\subsection{Wielomian Alexandera a niezmienniki numeryczne}
\begin{proposition}
\label{prp:alexander_determinant}%
    Niech $L$ będzie zorientowanym splotem.
    Wtedy $|\alexander_L(-1)| = \det L$.
\end{proposition}

\begin{proof}
    Wystarczy porównać definicję dla $\alexander_L$ (\ref{def:alexander_polynomial}) oraz $\det L$ (\ref{def:determinant}).
\end{proof}

\begin{proposition}
    Wielomian Alexandera zadaje ograniczenie na indeks skrzyżowaniowy:
    \begin{equation}
        \operatorname{span} \alexander_K(t) < \crossing K.
    \end{equation}
\end{proposition}

Być może istnieje bezpośredni dowód tej nierówności, ale jedyne uzasadnienie, jakie znam, opiera się na fakcie \ref{prp:alexander_genus} oraz wniosku \ref{cor:crossing_genus} -- czyli własnościach genusu.
\index{genus}%

\begin{proposition}
\index{węzeł!alternujący}%
    Tylko skończenie wiele węzłów alternujących może mieć ten sam wielomian Alexandera.
\end{proposition}

\begin{proof}
    Załóżmy nie wprost, że istnieje nieskończony ciąg $K_n$ węzłów alternujących o~tym samym wielomianie Alexandera $\alexander_K(t)$.
    Wszystkie jego wyrazy mają ten sam wyznacznik, ponieważ $\det K_n = |\alexander_K(-1)|$.
    Z faktu \ref{prp:bankwitz} wynika, że indeks skrzyżowaniowy węzłów $K_n$ jest wspólnie ograniczony: $c_k \le \det K_n = \det K$.
    To prowadzi do sprzeczności: węzłów o~danym indeksie skrzyżowaniowym jest tylko skończenie wiele.
\end{proof}

Wielomian Alexandera nie nakłada żadnych ograniczeń na liczbę gordyjską, wspominamy o~tym w dowodzie faktu \ref{balanced_iff_four_conditions}.

% koniec podsekcji Wielomian Alexandera a niezmienniki numeryczne




\subsection{Pochodna Foxa (definicja trzecia)}
\index{pochodna Foxa|(}%
Pojęcie grupy węzła oraz jej prezentacji wprowadzamy później, więc warto wrócić tutaj dopiero przy drugim lub następnym czytaniu!
Są dwa konkurencyjne podejścia do prezentowania grupy węzła.
Zgodnie z pomysłem pochodzącym jeszcze od Dehna, można przypisywać różne litery czterem częściom płaszczyzny, które są wycinane przez łuki skrzyżowania.
\index[persons]{Dehn, Max}%
Między innymi pierwsza praca Alexandera była bliska takiemu postępowaniu, ale dla oszczędności miejsca, nie opiszemy go wcale.
\index[persons]{Alexander, James}%

Alternatywne rozwiązanie każe etykietować nie obszary płaszczyzny, tylko łuki diagramu.
Klasyczne podręczniki teorii węzłów, takie jak \cite{crowell63}, macierz, a~co za tym idzie, także wielomian Alexandera wprowadzają właśnie w ten sposób: przy użyciu prezentacji Wirtingera i~pochodnej Foxa.
Jak sugeruje tytuł podsekcji, opiszemy teraz ten sposób.

Zakręcone!

% DICTIONARY;Fox derivative;pochodna Foxa;-
\begin{definition}[pochodna Foxa]
\label{def:fox_derivative}%
    Niech $G$ będzie wolną grupą generowaną przez (niekoniecznie skończony) podzbiór $\{g_i\}_{i \in I}$.
    Odwzorowanie $\partial/\partial g_i \colon G \to \Z G$ spełniające trzy aksjomaty:
    \begin{align}
        \frac{\partial}{\partial g_i} (e) & = 0 \\
        \frac{\partial}{\partial g_i} (g_j) & = \delta_{ij} \\
        \forall u, v \in G : \frac{\partial}{\partial g_i} (uv) & = \frac{\partial}{\partial g_i}(u) + u \frac{\partial}{\partial g_i} (w),
    \end{align}
    gdzie $\delta_{ij}$ oznacza deltę Kroneckera, nazywamy pochodną cząstkową Foxa.
\end{definition}

Fox opublikował w~Annals of Mathematics cykl pięciu artykułów \cite{fox53}, \cite{fox54}, \cite{fox56}, \cite{fox58}, \cite{fox60} poświęconych wolnemu rachunkowi różniczkowemu.
\index[persons]{Fox, Ralph}%
Definicja \ref{def:fox_derivative} jest tylko małym wycinkiem tego cyklu.
Strona nLab wspomina jeszcze o ,,\emph{a nice introduction in} \cite{crowell63}'', podręczniku Crowella, Foxa.

Ustalmy grupę $G$ oraz jej prezentację $\langle X | R \rangle = F/N$, gdzie $F = \langle X \rangle$ jest wolną grupą abelową, zaś $N$ to domknięcie normalne relacji $R$.
Mamy wtedy kanoniczny rzut $\varphi \colon F \to G$.
Niech $G^{ab} = G/[G, G]$ oznacza abelianizację, wtedy funkcja $\varphi^{ab} \colon F \to G^{ab}$ jest dobrze określona.
Ponieważ nie prowadzi to do zamieszania, tych samych liter będziemy używać także do funkcji $\varphi \colon \Z F \to \Z G$.
Definiujemy teraz macierz Jacobiego wymiaru $n \times n$:
\index{macierz Jacobiego}%
\begin{equation}
    J = \left(\varphi \left(\frac{\partial r_i}{\partial x_j}\right) \right).
\end{equation}
oraz macierz $J^{ab}$, która jest obrazem $J$ nad $\Z G^{ab}$.
W pierścieniu $\Z G^{ab}$ wyróżnia się ideały generowane przez minory (wyznaczniki podmacierzy) rozmiaru $i \times i$ w $J^{ab}$.
Ciąg $D_1, D_2, \ldots$ tych ideałów jest z dokładnością do jakichś technicznych szczegółów niezmiennikiem, to znaczy nie zależy od prezentacji.

W szczególnym przypadku, kiedy $G$ jest grupą węzła, relacje $r_i$ pochodzą z prezentacji Wirtingera, zaś grupa $G^{ab}$ jest nieskończona, cykliczna.
Niech $t$ oznacza jej generator; wtedy najwyższy niezerowy ideał $D_i$ jest główny.
Generator tego ideału nazywamy wielomianem Alexandera.
\index{wielomian!Alexandera}%

Rachunki są trochę prostsze niż wydają się być.
Wystarczy najpierw wykreślić z macierzy $J$ jedną kolumnę oraz jeden wiersz, po czym podstawić za wszystkie litery zmienną $t$ i policzyć wyznacznik.
Otrzymaliśmy znowu wielomian Alexandera!

\index{pochodna Foxa|)}%

% koniec podsekcji pochodna Foxa




\subsection{Różniaste różności}
Istnieje odmiana wielomianu Alexandera, która liczy sobie tyle zmiennych, ile ogniw posiada splot (nie opisywaliśmy jej i~nie zamierzamy).
Klasyfikację \ref{prp:alexander_hosokawa} można częściowo uogólnić: Torres \cite{torres53} znalazł dwie geometryczne własności, nazwane później warunkami Torresa.
\index[persons]{Torres, Guillermo}%
\index{warunek!Torresa}%
Są warunkami koniecznymi, ale nie wystarczającymi, jak odkrył ponad ćwierć wieku później Hillman \cite{hillman81}: wielomian
\index[persons]{Hillman, Jonathan}%
\begin{equation}
    D(x,y) = \frac{(1 - x^6y^6)(x - 1 + 1/x) - 2(1 - x^5y^5)(1 - x)(1 - y)}{1-xy}
\end{equation}
spełnia warunki Torresa, ale nie jest wielomianem Alexandera żadnego splotu.

Fox \cite{fox62} podejrzewał, że
\index[persons]{Fox, Ralph}%
\begin{conjecture}
\index{hipoteza!trapezoidalna}%
    Ciąg współczynników wielomianu Alexandera węzła alternującego jest unimodalny.
\end{conjecture}
Dowód podano dla węzłów algebraicznych (Murasugi \cite{murasugi85}) oraz genusu dwa (Ozsváth i~Szabó \cite{ozsvath03}).
\index[persons]{Murasugi, Kunio}%
\index[persons]{Ozsváth, Peter}%
\index[persons]{Szabó, Zoltán}%
Hipoteza w~ogólnym przypadku pozostaje otwarta.

Wiemy natomiast, że kolejne współczynniki wielomianu Conwaya węzła alternującego są przeciwnych znaków i niezerowe (\cite[s. 242]{murasugi96} odsyła do pracy tego samego autora, \cite{murasugi59}).
Wynika stąd, że prawie wszystkie węzły równoległe\footnote{Parallel knots, nie zamierzamy używać tego terminu poza tym zdaniem}, w tym: $(p, q)$-torusowe dla $p > q > 2$ oraz kablowe\footnote{Schlauchknoten.} nie są alternujące.
\index{węzeł!równoległy}%
% DICTIONARY;parallel;równoległy;węzeł

% koniec podsekcji Różniaste różności



\index{wielomian!Alexandera|)}%

