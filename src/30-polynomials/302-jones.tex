
\section{Rewolucyjny wielomian Jonesa}
\index{wielomian!Jonesa|(}%
Jones odkrył wielomian określany teraz jego nazwiskiem podczas badań algebr operatorów: zauważył, że przypominają one warkocze.
Znalazł reprezentację grupy warkoczy w~swojej algebrze i~złożył ją ze śladem na niej, pierwsza definicja była więc czysto algebraiczna.

Rezultat Jonesa był przełomowy, ale my odłożymy przytoczenie szczegółów na później, najpierw zaś pokażemy, jak można uzyskać jego wielomian w~prostszy sposób: metodami kombinatorycznymi, pojawi się też relacja kłębiasta.

Na koniec zaprezentujemy, jak wielomian Jonesa umożliwia dowód hipotez Taita.


\subsection{Algebra Temperleya-Lieba i ślad Markowa}

% https://en.wikipedia.org/wiki/Jones_polynomial
Jones otrzymał swój wielomian jako efekt uboczny badań nad algebrami operatorów: wziął ,,ślad'' pewnej reprezentacji warkoczy w~algebrę, która miała ważne znaczenie w~mechanice statystycznej.
\index[persons]{Jones, Vaughan}%
Jego praca \cite{jones1985} jest bardzo zwięzła, nam oświecenie przyniosła dopiero książka Kauffmana \cite[s. 85-103]{kauffman1991} oraz strona ,,Aharonov–Jones–Landau algorithm'' z angielskiej Wikipedii.
\index{algorytm!Aharonova-Jonesa-Landaua}%
Dziękujemy jej kilkunastu autorom.

% książka Kauffmana
\begin{definition}[algebra Temperleya-Lieba]
\label{def:temperley_lieb}%
\index{algebra!Temperleya-Lieba}%
    Niech $R$ będzie przemiennym pierścieniem, w~którym ustalono element $\delta \in R$.
    Wtedy $R$-algebrę $TL_n(\delta)$ generowaną przez elementy $e_1, \ldots, e_{n-1}$, które związane są relacjami
    \begin{align}
        U_i^2 & = \delta U_i, \\
        U_i U_{i \pm 1} U_i & = U_i, \\
        U_i U_j & = U_j U_i
    \end{align}
    dla $|i-j| \ge 2$, nazywamy algebrą Temperleya-Lieba.
\end{definition}

$TL_n(\tau)$ daje się przedstawić przy użyciu diagramów: prostokątów, których przeciwległe boki zawierają po $n$ punktów połączonych w~pary tak, by uniknąć samoprzecięć.
Mnożenie elementów algebry odpowiada sklejaniu dwóch diagramów, przy czym każdą zamkniętą pętlę zamieniamy na dodatkowy czynnik $\delta$.
To prawie są warkocze.

\begin{comment}
  \begin{figure}[H]
    \centering
    \begin{minipage}[b]{.19\linewidth}
        \[\MedLarTemperleyA\]
        \subcaption{$1$}
    \end{minipage}
    \begin{minipage}[b]{.19\linewidth}
        \[\MedLarTemperleyC\]
        \subcaption{$U_1$}
    \end{minipage}
    \begin{minipage}[b]{.19\linewidth}
        \[\MedLarTemperleyB\]
        \subcaption{$U_2$}
    \end{minipage}
    \begin{minipage}[b]{.19\linewidth}
        \[\MedLarTemperleyD\]
        \subcaption{$U_1U_2$}
    \end{minipage}
    \begin{minipage}[b]{.19\linewidth}
        \[\MedLarTemperleyE\]
        \subcaption{$U_2U_1$}
    \end{minipage}
    \caption{Diagramatyczne przedstawienie elementów algebry $TL_3(\delta)$}
    \end{figure}
\end{comment}

Mamy bojowe zadanie dla Czytelnika: narysować sześć diagramów odpowiadających trzem relacjom z~definicji \ref{def:temperley_lieb}.

Dla ustalonej liczby zespolonej $A$, niech funkcja $\rho_A \colon B_n \to TL_n(\delta)$ będzie zadana na generatorach grupy warkoczy: $\rho_A(\sigma_i) = AU_i + A^{-1}$.
Bezpośredni rachunek pokazuje, że jeśli $\delta = -A^2-A^{-2}$, to $\rho_A$ jest reprezentacją.

% książka Kauffmana
\begin{definition}[algebra Jonesa]
\index{algebra!Jonesa}%
    Niech $R$ będzie przemiennym pierścieniem, zaś $\tau$ skalarem, który komutuje ze wszystkimi innymi elementami.
    Wtedy $R$-algebrę $TL_n(\tau)$ generowaną przez elementy $e_1, \ldots, e_{n-1}$, związanymi relacjami
    \begin{align}
        e_i^2 & = e_i, \\
        e_i e_{i \pm 1} e_i & = \tau e_i, \\
        e_i e_j & = e_j e_i
    \end{align}
    dla $|i-j| \ge 2$, nazywamy algebrą Jonesa.
\end{definition}

Kauffman pisze, że algebry są ze sobą związane: jeśli ustalimy swoją ulubioną algebrę Temperleya-Lieba, a potem położymy $e_i = \delta^{-1} U_i$ oraz $\tau = \delta^{-2}$, to aksjomaty Jonesa są spełnione.

% artykuł na Wiki
\begin{definition}[ślad Markowa]
\index{sZZZlad@ślad Markowa}%
    Niech $K$ będzie elementem algebry Temperleya-Lieba $TL_n(\delta)$, iloczynem generatorów $e_1, \ldots, e_{n-1}$, którego domknięcie rozpada się na $m$ składowych spójności.
    Wielkość
    \begin{equation}
        \trace K = \delta^{m-n}
    \end{equation}
    nazywamy śladem Markowa elementu $K$.
\end{definition}

Ślad Markowa przedłuża się liniowo do całej algebry Temperleya-Lieba i jest prawdziwym śladem: spełnia warunki $\trace (1) = 1$ oraz $\trace (T_1T_2) = \trace (T_2T_1)$.
Ma też dodatkową własność, że jeśli $w$ jest słowem na literach $e_1, e_2, \ldots, e_{i-1}$, to $\trace(we_i) = \delta^{-1} \trace (w)$.

Jones skorzystał z twierdzeń Alexandera i Markowa dotyczących warkoczy, złożył ze sobą funkcje $\trace$ oraz $\rho_A$ oraz znormalizował wynik.
\index{twierdzenie!Alexandera}%
\index{twierdzenie!Markowa}%
Dostał tak niezmiennik splotów $L$ domknięć warkoczy $B$:
\begin{equation}
    V_L(A^{-4}) = (-A)^{3 \writhe D} \delta^{n-1} (\trace \circ \rho_A)(B).
\end{equation}

% Koniec podsekcji Oryginalna praca Jonesa



\subsection{Definicja kombinatoryczna -- klamra Kauffmana}
\index{klamra Kauffmana|(}
Klamra Kauffmana to wielomian Laurenta jednej zmiennej zdefiniowany w pracy \cite{kauffman87} z 1987 roku, oparty na ruchach Reidemeistera.
Dzięki swojej prostocie mógł być odkryty na początku XX wieku, nim jeszcze maszyneria teorii węzłów została rozwinięta.

Poszukujemy niezmiennika dla splotów o~kilku prostych własnościach.
Przede wszystkim żądamy, by niewęzłowi przypisany był wielomian $1$: $\bracket{\LittleUnknot} = 1$.
Po drugie chcemy wyznaczać nawiasy znając je dla prostszych splotów, co zapiszemy symbolicznie:
\begin{equation}
    \bracket{\MediumRightCrossing} = A \bracket{\MediumRightSmoothing} + B \bracket{\MediumLeftSmoothing}
\end{equation}
Zależy nam też na tym, by móc dodać do splotu trywialną składową: $\langle L \cup \LittleUnknot \rangle = C \langle L \rangle$.
Prosty rachunek pokazuje wpływ drugiego ruchu Reidemeistera na klamrę:
\begin{equation}
    \bracket{\MediumReidemeisterIIaHorizontal}
    = (A^2 + ABC + B^2) \bracket{\MediumRightSmoothing} + BA \bracket{\MediumLeftSmoothing}
    \stackrel{?}{=} \bracket{\MediumLeftSmoothing}.
\end{equation}

Aby zachodziła ostatnia równość wystarczy przyjąć $B = A^{-1}$, co wymusza na nas wartość trzeciego parametru: $C = -A^2 - A^{-2}$.
W ten sposób odkryliśmy definicję.

\begin{definition}[klamra Kauffmana]
    \index{klamra!Kauffmana}
    \label{def:kauffman_bracket}
    Wielomian Laurenta $\bracket{D}$ dla diagramu splotu $D$ zmiennej $A$,
    który jest niezmienniczy ze względu na gładkie deformacje diagramu,
    a~przy tym spełnia trzy poniższe aksjomaty:
    \begin{align}
        \bracket{\MediumUnknot} & = 1
        \label{eqn:kauffman_axiom_1}%
        \\
        \bracket{\MediumRightCrossing} & = 
        A \bracket{\MediumRightSmoothing} + 
        A^{-1} \bracket{\MediumLeftSmoothing}
        \label{eqn:kauffman_axiom_2}%
        \\
        \bracket{D \sqcup \MediumUnknot} & = 
        (-A^{-2} - A^2) \bracket{D}
        \label{eqn:kauffman_axiom_3}%
    \end{align}
    nazywamy klamrą Kauffmana.
\end{definition}

Drugi aksjomat jest wariacją na temat relacji kłębiastej.

\begin{lemma}
    Klamra Kauffmana każdego diagramu wyznacza się w~skończenie wielu krokach.
\end{lemma}

\begin{proof}
    Najprościej dowieść tego indukcyjnie, ze względu na liczbę skrzyżowań na diagramie splotu.
    Baza indukcji to przypadek zero skrzyżowań, czyli niesplotów.
    Zauważmy, że ostatni (i później pierwszy) aksjomat pozwala wyznaczyć wartość klamry Kauffmana dla każdego niesplotu w tylu krokach, ile ogniw ma niesplot.

    Pozostał krok indukcyjny.
    Załóżmy, że wyznaczyliśmy już wartości klamry dla każdego diagramu o $n$ skrzyżowaniach i chcemy ją obliczyć dla kolejnego splotu z diagramem o~$n + 1$ skrzyżowaniach.
    Pozwala na to drugi aksjomat, usuwający jedno ze skrzyżowań.
\end{proof}

Przedstawimy teraz wpływ ruchów Reidemeistera na nasz nowy wielomian.

\begin{lemma}
    Drugi i~trzeci ruch Reidemeistera nie ma wpływu na klamrę Kauffmana,
    pierwszy ruch zmienia ją zgodnie z~regułą:
    \begin{equation}
        \bracket{\MediumReidemeisterIaLeft} = -A^{-3} \bracket{\,\MediumReidemeisterIb\,}.
    \end{equation}
\end{lemma}

\begin{proof}
Pierwszy ruch Reidemeistera:
\begin{comment}
\begin{align*}
    \bracket{\MediumReidemeisterIaLeft} & \stackrel{K2}{=} A 
    \bracket{\MediumReidemeisterIaLeftFirstSmoothering} + 
    A^{-1} \bracket{\MediumReidemeisterIaLeftSecondSmoothering} \\ & \stackrel{K3}{=} 
    A \bracket{\MediumReidemeisterIb} + 
    A^{-1}(-A^{-2}-A^2) \bracket{\MediumReidemeisterIb} = 
    -A^{-3}\bracket{\MediumReidemeisterIb}
\end{align*}
\end{comment}

Dla drugiego ruchu:
\begin{comment}
\begin{align*}
    \bracket{\MediumReidemeisterIIa} & \stackrel{K2}{=}
    A \bracket{\MediumReidemeisterIIaSmoothed} +
    A^{-1} \bracket{\MediumLeftCrossingLow} \\ & \stackrel{K1}{=}
    -A^{-2} \bracket{\MediumLeftSmoothing} + 
    A^{-1} \bracket{\MediumLeftCrossingLow} \\ & \stackrel{K2}{=}
    -A^{-2} \bracket{\MediumLeftSmoothing} +
    A^{-1}A \bracket{\MediumRightSmoothing} + 
    A^{-1}A^{-1} \bracket{\MediumLeftSmoothing} \\ & = 
    \bracket{\MediumRightSmoothing}
\end{align*}
\end{comment}

Dla trzeciego ruchu:
\begin{comment}
\begin{align*}
\bracket{\MediumReidemeisterIIIa} & \stackrel{K2}{=}
A \bracket{\MediumReidemeisterIIIb} +
A^{-1} \bracket{\MediumReidemeisterIIIc} \stackrel{R2}{=} 
A \bracket{\MediumReidemeisterIIId} +
A^{-1} \bracket{\MediumReidemeisterIIIe} \\ & \stackrel{R2}{=}
A \bracket{\MediumReidemeisterIIIbFlipped} +
A^{-1} \bracket{\MediumReidemeisterIIIcFlipped} \stackrel{K2}{=} 
\bracket{\MediumReidemeisterIIIaFlipped},
\end{align*}
korzystaliśmy tu z~własności drugiego ruchu.
\end{comment}
\end{proof}

\begin{corollary}
    Rozpiętość klamry Kauffmana jest niezmiennikiem węzłów.
\end{corollary}

Klamra Kauffmana nie jest niezmiennikiem węzłów ze względu na I ruch Reidemeistera.
Jeżeli przypomnimy sobie, że na mocy lematu \ref{lem:writhe_reidemeister} spin także nie jest niezmiennikiem węzłów, odkryjemy ,,trik Kauffmana'': niedoskonałości tych dwóch obiektów znoszą się wzajemnie.
\index{trik Kauffmana}
\index{spin}

\begin{definition}
\index{wielomian!Jonesa}%
\label{def:jones_polynomial}%
    Niech $L$ będzie zorientowanym splotem.
    Wielomian Laurenta $\jones(L) \in \Z[t^{\pm 1/2}]$ określony przez
    \begin{equation}
        \jones(L)=\left[(-A)^{-3w(D)} \bracket{D}\right]_{t^{1/2}=A^{-2}},
    \end{equation}
    gdzie $D$ to dowolny diagram dla $L$, nazywamy wielomianem Jonesa.
\end{definition}

Sama klamra odegrała ważną rolę podczas unifikacji wielomianu Jonesa oraz innych niezmienników kwantowych.
W szczególności pozwoliła na uogólnienie go do niezmiennika 3-rozmaitości.

\begin{proposition}
    Wielomian Jonesa jest niezmiennikiem zorientowanych splotów.
\end{proposition}

\begin{proof}
    %Skorzystamy z~tego, że indeks zaczepienia jest niezmiennikiem.
    Wystarczy pokazać niezmienniczość $(-A)^{-3w(D)}\langle D\rangle$ na ruchy Reidemeistera.
    
    Niech
    \begin{equation}
        D_1 = \MediumReidemeisterIaLeft,
        \quad\quad\quad
        D_2 = \MediumReidemeisterIb
    \end{equation}
    Jak zauważyliśmy już wcześniej, II i III ruch nie zmienia ani spinu, ani klamry Kauffmana.
    Pozostało sprawdzić I ruch.
    Mamy:
    \begin{equation}
        (-A)^{-3 w\left(D_1\right)} \bracket{D_1} =
        (-A)^{-3 w\left(D_2\right) + 3} (-A)^{-3}\bracket{D_2} =
        (-A)^{-3 w\left(D_2\right)} \bracket{D_2},
    \end{equation}
    co kończy dowód.
\end{proof}

Zazwyczaj, ale nie zawsze, wielomian Jonesa lepiej radzi sobie z odróżnianiem od siebie splotów.
Zaczniemy od wyznaczenia bezpośrednio z definicji, jakie są wielomiany Jonesa niesplotów.
Dla porównania, wielomian Alexandera wszystkich splotów rozszczepialnych jest taki sam (stwierdzenie \ref{prp:alexander_unlinks}).

\begin{proposition}
    \label{prp:jones_trivial_link}
    Wielomianem Jonesa splotu trywialnego o $n$ ogniwach jest
    \begin{equation}
        \jones(K_n) = \left(-\sqrt{t} - \frac{1}{\sqrt {t}}\right)^{n-1}.
    \end{equation}
\end{proposition}

Co więcej, wielomian Jonesa odróżnia od siebie dowolne dwa węzły pierwsze o~co najwyżej 9 skrzyżowaniach.
Dalej występują już kolizje, oto pełna ich lista do 10 skrzyżowań:
$5_{1}$ -- $10_{132}$,
$8_{8}$ -- $10_{129}$,
$8_{16}$ -- $10_{156}$,
$10_{22}$ -- $10_{35}$,
$10_{25}$ -- $10_{56}$,
$10_{40}$ -- $10_{103}$,
$10_{41}$ -- $10_{94}$,
$10_{43}$ -- $10_{91}$,
$10_{59}$ -- $10_{106}$,
$10_{60}$ -- $10_{86}$,
$10_{71}$ -- $10_{104}$,
$10_{73}$ -- $10_{83}$,
$10_{81}$ -- $10_{109}$,
$10_{137}$ -- $10_{155}$.
Jones wiedział, że wielomianowe niezmienniki nie radzą sobie z~odróżnianiem od siebie mutantów, dlatego zapytał w~2000 roku, czy jego wielomian wykrywa niewęzły.
Pozostaje to otwartym problemem do dziś.

\begin{conjecture}
\index{hipoteza!o wielomianie Jonesa i niewęźle}%
\label{con:jones}%
    Niech $K$ będzie węzłem.
    Jeśli $\jones_K(t) \equiv 1$, to $K$ jest niewęzłem.
\end{conjecture}

Hipotezę zweryfikowano komputerowo dla węzłów o~małej liczbie skrzyżowań.
W latach dziewięćdziesiątych Hoste, Thistlethwaite, Weeks zrobili to dla węzłów spełniających $\operatorname{cr} \le 16$.
Wynik poprawiano: Dasbach, Hougardy w~1997 do $\operatorname{cr} = 17$; Yamada w~2000 do $\operatorname{cr} = 18$; wreszcie Tuzun, Sikora w~2016 do $\operatorname{cr} \le 22$.
Patrz \cite[s. 381]{ohtsuki02}.

% TODO: Argumentem przemawiającym za prawdziwością hipotezy jest twierdzenie ,,udowodnione'' przez Jørgena Andersena.
% TODO: \textbf{NIE Pokazał on, że rodzina okablowanych wielomianów Jonesa wykrywa niewęzeł.}
% TODO: Tutaj $n$-okablowanie węzła $K$ to $n$-komponentowy splot $K^n$, który powstaje z~$K$ po zamianie pojedynczej ,,żyły'' na $n$ równoległych żył.

Istnieją sploty o~trywialnym wielomianie Jonesa, jest ich nawet nieskończenie wiele, jak Eliahou, Kauffman i~Thistlethwaite pokazali w~pracy \cite{eliahou03}.

\begin{proposition}
    Niech $k \ge 2$ będzie liczbą naturalną.
    Istnieje nieskończenie wiele splotów pierwszych z $k$ ogniwami, których wielomian Jonesa nie odróżnia od niesplotu z $k$ ogniwami.

    Co więcej, można wymagać, by wszystkie te sploty były satelitami splotu Hopfa.
\index{splot!Hopfa}%
\end{proposition}

Niech $\jones$ będzie wielomianem Jonesa splotu $L$ o~$n$ składowych spójności.
Jego wartości w~niektórych pierwiastkach jedności są związane z~innymi niezmiennikami węzłów.
I tak przyjmując oznaczenie $\omega_k = \exp(2\pi i/k)$ mamy

\begin{proposition}
    \label{prp:jones_at_roots_of_unity}
    $\jones_L(\omega_3) = 1$.
\end{proposition}

\begin{proposition}
    $\jones_L(1) = (-2)^{n-1}$.
\end{proposition}

\begin{proof}
    Jak wkrótce się przekonamy, to proste wnioski z~relacji kłębiastej.
    Explicite wskazał je Jones w \cite[twierdzenie 14, 15]{jones85}.
\end{proof}

\begin{proposition}
    Pochodna w punkcie $t = 1$ znika: $\jones'_L(1) = 0$.
\end{proposition}

\begin{proof}
    Twierdzenie 16 w \cite{jones85}.
\end{proof}

\begin{proposition}
    $V_L(\omega_6) = \pm i^{n-1} \cdot (\sqrt 3i)^r$, gdzie $r$ jest rangą pierwszej grupy homologii podwójnego rozgałęzionego nakrycia $L$ nad $\Z_3$.
\end{proposition}

\begin{proof}
    Znak $\pm$ został wyznaczony przez Lipsona w \cite{lipson86}, praca ta zawiera też odsyłacz do wyprowadzenia reszty wzoru.
\end{proof}

\begin{proposition}
    Liczba trzy-kolorowań splotu $L$ wynosi $3|\jones_L(\omega_6)|^2$.
\end{proposition}

\begin{proof}
    Patrz \cite{przytycki98}.
\end{proof}

\begin{proposition}
    Jeśli $L$ jest właściwym splotem (indeks zaczepienia każdej składowej o~resztę splotu jest parzysty), to $\jones_L(i) = (-\sqrt 2)^{n-1}(-1)^{\operatorname{Arf} L}$.
    W przeciwnym razie $\jones_L(i) = 0$.
\end{proposition}

\begin{proof}
    Równość tę pokazał Murakami w~1986 roku (\cite{murakami86}).
\end{proof}

\begin{proposition}
    Niech $G$ będzie pierwszą grupą homologii podwójnego nakrycia $S^3$ rozgałęzionego nad składowymi.
    Jeśli $G$ jest torsyjna, to $\jones_L(-1) = |G|$.
    W przeciwnym razie $\jones_L(-1) = 0$.
\end{proposition}

\begin{proof}
    ???? % TODO
\end{proof}

Nie jest znana topologiczna interpretacja wielomianu Jonesa (którą posiada wielomian Alexandera) ani charakteryzacja poza warunkami koniecznymi z~pięciu faktów powyżej.

\begin{corollary}
    Niech $K$ będzie węzłem.
    Wtedy
    \begin{align}
        \jones(1) & = 1 \\
        \jones(-1) & = \pm \det K \\
        \jones(i) & = \begin{cases}
            1 & \text{dla } \alexander(-1) \equiv \pm 1 \mod 8 \\
            -1 & \text{w przeciwnym razie.}
        \end{cases}
    \end{align}
\end{corollary}

Poza powyżej opisanymi przypadkami, wartości wielomianu Jonesa nie można znaleźć w~czasie wielomianowym od ilości skrzyżowań na diagramie (jest to problem $\#P$-trudny).

% Czemu wielomian Jonesa jest wielomianem?
% Odpowiedniki wielomianu Jonesa dla węzłów w~3-rozmaitościach innych niż sfera $S^3$ nie są wielomianami, ale funkcjami z~pierwiastków jedności w~zbiór elementów całkowitch\footnote{algebraic integers} (jak podaje J. Roberts).

Dotychczas wyznaczyliśmy wielomian Jonesa jedynie dla splotów trywialnych (fakt \ref{prp:jones_trivial_link}).
Dlaczego?
Chociaż klamra Kauffmana to użyteczne narzędzie podczas dowodzenia różnych teoretycznych własności, niezbyt nadaje się do obliczeń, szczególnie ręcznych.
Na szczęście wtedy z pomocą przychodzi:

\begin{definition}
    Niech $L$ będzie zorientowanym splotem.
    Wielomian Laurenta $\jones_L(t) \in \Z[t^{\pm 1/2}]$, który spełnia relację kłębiastą
\index{relacja kłębiasta}%
    \begin{equation}
        t^{-1} \jones(L_+) - t\jones(L_-) + (t^{-1/2} - t^{1/2}) \jones(L_0) = 0
    \end{equation}
    z warunkiem brzegowym $\jones(\LittleUnknot) = 1$, nazywamy wielomianem Jonesa.
\end{definition}

\begin{proof}
Niech
\begin{equation}
    L_+ = \MediumRightCrossing
    \quad\quad
    L_- = \MediumLeftCrossing
    \quad\quad
    L_0 = \MediumRightSmoothing
    \quad\quad
    L_\infty = \MediumLeftSmoothing
\end{equation}
% TODO: zdefiniować je raz, a dobrze.
% ack -l 'L_\+'
% src/30-polynomials/alexander.tex
% src/30-polynomials/jones-kauffman.tex
% src/30-polynomials/blmho.tex
% src/30-polynomials/homfly.tex
% src/00-meta-latex/diagrams.tex
(oznaczenia te są standardowe i pozwalają oszczędzić trochę miejsca).
Wyraźmy wielomian Jonesa przez klamrę Kauffmana i~spin.
Chcemy pokazać, że
\begin{align}
    & A^{4}(-A)^{-3w(L_+)}\bracket{L_+} \\
    - & A^{-4}(-A)^{-3w(L_-)}\bracket{L_0} \\ 
    + & (A^2-A^{-2})(-A)^{-3w(L_0)}\bracket{L_0} = 0.
\end{align}

Ale $w(L_\pm) = w(L_0)\pm 1$, zatem to jest równoważne z

\begin{equation}
    -A \bracket{L_+} +
    A^{-1} \bracket{L_-} +
    (A^2-A^{-2}) \bracket{L_0} =0.
\end{equation}

Z definicji klamry Kauffmana wnioskujemy, że

\begin{equation}
    \begin{cases}
        \bracket{L_+} = A\bracket{L_0} + A^{-1}\bracket{L_\infty} \\
        \bracket{L_-} = A\bracket{L_\infty} + A^{-1}\bracket{L_0}
    \end{cases}
\end{equation}

Pierwsze równanie przemnóżmy przez $A$, drugie przez $A^{-1}$, a~następnie dodajmy je do siebie.
Wtedy otrzymamy
\begin{equation}
    A\bracket{L_+} - A^{-1}\bracket{L_-} =
    A^2 (\bracket{L_0} - \bracket{L_\infty}),
\end{equation}
quod erat demonstrandum.
\end{proof}

% \subsection{Odwrotności, lustra i~sumy}
Wielomian Jonesa nie wykrywa orientacji splotu.

\begin{proposition}
    Niech $L$ będzie zorientowanym splotem.
    Wtedy $\jones(rL)=\jones(L)$.
\index{rewers}%
\end{proposition}

\begin{proof}
    Aby obliczyć wielomian rewersu, wykorzystujemy te same diagramy kłębiaste,
    jak dla zwykłego, a~przy tym nie zmieniamy znaku żadnego skrzyżowania.
\end{proof}

Ale czasami potrafi odróżnić splot od jego lustra:

\begin{proposition}
    Niech $L$ będzie zorientowanym splotem.
    Wtedy $\jones(mL)(t)=\jones(L)(t^{-1})$.
\index{lustro}%
\end{proposition}

\begin{proof}
    Zauważmy, że diagramy $L_-$ oraz $L_+$ są wzajemnymi lustrami.
    Dlatego każda relacja kłębiasta dla splotu postaci
    \begin{equation}
        t^{-1} \jones(L_+)(t) - t\jones(L_-)(t) + (t^{-1/2} - t^{1/2}) \jones(L_0)(t) = 0
    \end{equation}
    odpowiada pewnej relacji dla lustra splotu:
    \begin{equation}
        -t\jones(L_+)(t) + t^{-1} \jones(L_-)(t) + (t^{-1/2} - t^{1/2}) \jones(L_0)(t) = 0,
    \end{equation}
    co po zamianie zmiennych $t \mapsto t^{-1}$ i przemnożeniu przez $-1$ daje
    \begin{equation}
        -t^{-1} \jones(L_+)(t^{-1}) + t \jones(L_-)(t^{-1}) + (t^{1/2} - t^{-1/2}) \jones(L_0)(t^{-1}) = 0.
    \end{equation}

    Patrz też: Florian Gellert, Kombinatorische Invarianten, strona 12.
\end{proof}

\begin{corollary}
    \label{cor:joines_of_amphicheiral}
    Jeśli $K$ jest węzłem zwierciadlanym, to wielomian $\jones_K$ jest symetryczny.
\index{węzeł!zwierciadlany}
\end{corollary}

Implikacja odwrotna nie zachodzi na mocy wniosku \ref{cor:acheiral_signature}: węzeł $9_{42}$ ma symetryczny wielomian Jonesa, ale niezerową sygnaturę.
\index{sygnatura}%
Poniżej trzynastu skrzyżowań taka sytuacja ma miejsce dla dokładnie czternastu węzłów pierwszych.
% 9_42, 10_125, 11n_19, 11n_24, 11n_82, 12a_0669, 12a_1171, 12a_1179, 12a_1205, 12n_0362, 12n_0506, 12n_0562, 12n_0571, 12n_0821

Równość $\jones(mL)(t)=\jones(L)(t^{-1})$ nie jest spełniona dla trójlistnika, zatem ten nie jest równoważny ze swoim lustrem.
Wcześniej pokazał to z~dużo większym wysiłkiem Dehn, patrz przykład \ref{exm:trefoil_is_chiral}.

\begin{corollary}
    Wielomian Jonesa nie zależy od orientacji węzła.
    Nie jest to prawdą dla splotów.
\end{corollary}

\begin{proof}
    Każdy węzeł ma tylko dwie orientacje, splot może mieć ich $2^n$, gdzie $n$ to liczba składowych.
\end{proof}

\begin{proposition}
    \label{prp:jones_multiplicative_1}
    Niech $L_1, L_2$ będą zorientowanymi splotami.
    Wtedy
    \begin{equation}
        \jones(L_1 \sqcup L_2) = (-t^{1/2} - t^{-1/2}) \jones(L_1) \jones(L_2).
    \end{equation}
\end{proposition}

\begin{proof}
    Wybierzmy diagramy $D_1, D_2$ dla splotów $L_1, L_2$.
    Po podstawieniu $t^{1/2} = A^{-2}$ widzimy, że chcemy pokazać
    \begin{equation}
        (-A)^{-3w(D_1 \sqcup D_2)} \langle D_1 \sqcup D_2 \rangle
        =
        (-A^2 - A^{-2})(-A)^{-3(w(D_1) + w(D_2))} \langle D_1 \rangle \langle D_2 \rangle.
    \end{equation}

    Oczywiście $w(D_1 \sqcup D_2) = w(D_1) + w(D_2)$, więc wystarczy udowodnić, że
    \begin{equation}
        \langle D_1 \sqcup D_2 \rangle = (-A^2 - A^{-2}) \langle D_1 \rangle \langle D_2 \rangle.
    \end{equation}

    Oznaczmy przez $f_1(D_1)$, $f_2(D_1)$ odpowiednio lewą i~prawą stronę ostatniego równania.
    Są to wielomiany Laurenta, które zależą tylko od $D_1$.
    Aksjomaty Kauffmana pozwalają na pokazanie, że obie funkcje mają następujące własności:
    \begin{align}
        f_i(\LittleUnknot)            & = (-A^2 - A^{-2}) \langle D_2 \rangle \\
        f_i(D_1 \sqcup \LittleUnknot) & = (-A^2 - A^{-2}) f_i(D_1) \\
        f_i(\LittleRightCrossing)     & = A f_i(\LittleRightSmoothing) + A^{-1} f_i(\LittleLeftSmoothing).
    \end{align}
    Ponieważ powyższe tożsamości wystarczają do wyznaczenia wartości funkcji $f_i$ dla dowolnego diagramu $D_1$, dochodzimy do wniosku, że $f_1 \equiv f_2$.
    To kończy dowód.
\end{proof}

\begin{proposition}
\label{prp:jones_multiplicative_2}%
\index{relacja kłębiasta}%
    Niech $K_1, K_2$ będą zorientowanymi węzłami.
    Wtedy
    \begin{equation}
        \jones(K_1 \# K_2) = \jones(K_1) \jones(K_2).
    \end{equation}
\end{proposition}

\begin{proof}
    Rozpatrzmy sploty
\begin{comment}
    \[
        \begin{tikzpicture}[baseline=-0.65ex,scale=0.07]
        \begin{knot}[clip width=5, flip crossing/.list={1}]
            \strand[semithick] (-22, -5) rectangle (-12, 5);
            \strand[semithick] (22, -5) rectangle (12, 5);

            \strand[semithick,Latex-] (-12, 3) [in=left, out=right] to (12, -3);
            \strand[semithick,Latex-] (12, 3) [in=right, out=left] to (-12, -3);

            \node at (-17, 0) {$K_1$};
            \node at (17, 0) {$K_2$};
        \end{knot}
        \end{tikzpicture}
        \quad\quad
        \begin{tikzpicture}[baseline=-0.65ex,scale=0.07]
        \begin{knot}[clip width=5]
            \strand[semithick] (-22, -5) rectangle (-12, 5);
            \strand[semithick] (22, -5) rectangle (12, 5);

            \strand[semithick,Latex-] (-12, 3) [in=left, out=right] to (12, -3);
            \strand[semithick,Latex-] (12, 3) [in=right, out=left] to (-12, -3);

            \node at (-17, 0) {$K_1$};
            \node at (17, 0) {$K_2$};
        \end{knot}
        \end{tikzpicture}
        \quad\quad
        \begin{tikzpicture}[baseline=-0.65ex,scale=0.07]
        \begin{knot}[clip width=5]
            \strand[semithick] (-22, -5) rectangle (-12, 5);
            \strand[semithick] (-12, -3) [in=down, out=right] to (-2, 0);
            \strand[semithick,Latex-] (-12, 3) [in=up, out=right] to (-2, 0);

            \strand[semithick] (22, -5) rectangle (12, 5);
            \strand[semithick] (12, -3) [in=down, out=left] to (2, 0);
            \strand[semithick,Latex-] (12, 3) [in=up, out=left] to (2, 0);

            \node at (-17, 0) {$K_1$};
            \node at (17, 0) {$K_2$};
        \end{knot}
        \end{tikzpicture}
    \]
\end{comment}
    Relacja kłębiasta orzeka w tym przypadku, że
    \begin{equation}
        t^{-1} \jones(K_1 \# K_2) - t \jones(K_1 \# K_2) + (t^{-1/2} - t^{1/2}) \jones(K_1 \sqcup K_2) = 0.
    \end{equation}
    Ostatni składnik sumy można rozwinąć na mocy faktu \ref{prp:jones_multiplicative_1}.
    Po uporządkowaniu dostaniemy:
    \begin{equation}
        (t^{-1} - t) \jones(K_1 \# K_2) - (t^{-1} - t) \jones(K_1) \jones(K_2) = 0,
    \end{equation}
    a stąd widać już prawdziwość dowodzonej tezy.
\end{proof}

% Koniec sekcji Relacja kłębiasta
% Koniec podsekcji Wielomian Jonesa

\index{klamra Kauffmana|)}

% Koniec podsekcji Nawias Kauffmana


\subsection{Odróżnianie węzłów i splotów wielomianem Jonesa}
Wielomian Jonesa często (chociaż nie zawsze) odróżnia od siebie sploty lepiej niż wielomian Alexandera.
Na przykład: wielomian Alexandera wszystkich splotów rozszczepialnych jest taki sam (stwierdzenie \ref{prp:alexander_unlinks}), więc nie odróżnia wcale niesplotów.
Dla porównania, wielomian Jonesa odróżnia je wszystkie:

\begin{proposition}
\label{prp:jones_trivial_link}%
    Wielomianem Jonesa splotu trywialnego o $n$ ogniwach jest
    \begin{equation}
        \jones(K_n) = \left(-\sqrt{t} - \frac{1}{\sqrt {t}}\right)^{n-1}.
    \end{equation}
\end{proposition}

Co więcej, wielomian Jonesa odróżnia od siebie dowolne dwa węzły pierwsze o~co najwyżej 9 skrzyżowaniach.
Dalej występują już kolizje, oto pełna ich lista do 10 skrzyżowań:
$5_{1}$ -- $10_{132}$,
$8_{8}$ -- $10_{129}$,
$8_{16}$ -- $10_{156}$,
$10_{22}$ -- $10_{35}$,
$10_{25}$ -- $10_{56}$,
$10_{40}$ -- $10_{103}$,
$10_{41}$ -- $10_{94}$,
$10_{43}$ -- $10_{91}$,
$10_{59}$ -- $10_{106}$,
$10_{60}$ -- $10_{86}$,
$10_{71}$ -- $10_{104}$,
$10_{73}$ -- $10_{83}$,
$10_{81}$ -- $10_{109}$,
$10_{137}$ -- $10_{155}$.
Jones wiedział, że wielomianowe niezmienniki nie radzą sobie z~odróżnianiem od siebie mutantów, dlatego zapytał, czy jego wielomian wykrywa niewęzły.
Pozostaje to otwartym problemem do dziś.

\begin{conjecture}
\index{hipoteza!o wielomianie Jonesa i niewęźle}%
\label{con:jones}%
    Niech $K$ będzie węzłem.
    Jeśli $\jones_K(t) \equiv 1$, to $K$ jest niewęzłem.
\end{conjecture}

Hipotezę zweryfikowano komputerowo dla węzłów o~małej liczbie skrzyżowań.
W latach dziewięćdziesiątych Hoste, Thistlethwaite, Weeks zrobili to przy okazji tablicowania węzłów spełniających $\crossing K \le 16$.
Wynik poprawiano: Dasbach, Hougardy w~1997 do $\crossing K \le 17$; Yamada w~2000 do $\crossing K \le 18$; wreszcie Tuzun, Sikora w~2016 do $\crossing K \le 22$, potem w~2020 do $\crossing K \le 24$.
Patrz kolejno \cite{thistlethwaite98}, \cite{hougardy97}, \cite{yamada00}, \cite{tuzun18}, \cite{tuzun21}, ale też \cite[s. 381]{ohtsuki02}.

% TODO: Argumentem przemawiającym za prawdziwością hipotezy jest twierdzenie ,,udowodnione'' przez Jørgena Andersena.
% TODO: \textbf{NIE Pokazał on, że rodzina okablowanych wielomianów Jonesa wykrywa niewęzeł.}
% TODO: Tutaj $n$-okablowanie węzła $K$ to $n$-komponentowy splot $K^n$, który powstaje z~$K$ po zamianie pojedynczej ,,żyły'' na $n$ równoległych żył.

Istnieją sploty o~trywialnym wielomianie Jonesa.
Thistlethwaite wskazał dwa z~dwoma oraz jeden z~trzema ogniwami w~\cite{thistlethwaite01}.
Jest ich nawet nieskończenie wiele, jak Eliahou, Kauffman i~Thistlethwaite pokazali w~pracy \cite{eliahou03}.

\begin{proposition}
\index{splot!Hopfa}%
    Niech $k \ge 2$ będzie liczbą naturalną.
    Istnieje nieskończenie wiele splotów pierwszych z $k$ ogniwami, których wielomian Jonesa nie odróżnia od niesplotu z $k$ ogniwami.

    Co więcej, można wymagać, by wszystkie te sploty były satelitami splotu Hopfa.
\end{proposition}




\subsection{Relacja kłębiasta}
Dotychczas wyznaczyliśmy wielomian Jonesa jedynie dla splotów trywialnych (fakt \ref{prp:jones_trivial_link}).
Dlaczego?
Chociaż klamra Kauffmana to użyteczne narzędzie podczas dowodzenia różnych teoretycznych własności, niezbyt nadaje się do obliczeń, szczególnie ręcznych.
Na szczęście wtedy z pomocą przychodzi:

\begin{definition}
\index{relacja kłębiasta}%
    Niech $L$ będzie zorientowanym splotem.
    Wielomian Laurenta $\jones_L(t) \in \Z[t^{\pm 1/2}]$, który spełnia relację kłębiastą
    \begin{equation}
        t^{-1} \jones(L_+) - t\jones(L_-) + (t^{-1/2} - t^{1/2}) \jones(L_0) = 0
    \end{equation}
    z warunkiem brzegowym $\jones(\SmallUnknot) = 1$, nazywamy wielomianem Jonesa.
\end{definition}

Symbole $L_+, L_-, L_0$ objaśnione są przy definicji \ref{skein_symbols}.
% TODO: zamienić je na rysunki

\begin{proof}
% TODO: zdefiniować je raz, a dobrze.
% ack -l 'L_\+'
% src/30-polynomials/alexander.tex
% src/30-polynomials/jones-kauffman.tex
% src/30-polynomials/blmho.tex
% src/30-polynomials/homfly.tex
% src/00-meta-latex/diagrams.tex
% TODO: odwrócić kolejność kroków, wtedy dowód zacznie być z-czapy
Wyraźmy wielomian Jonesa przez klamrę Kauffmana i~spin.
Chcemy pokazać, że
\begin{align}
    & A^{4}(-A)^{-3w(L_+)}\bracket{L_+} \\
    - & A^{-4}(-A)^{-3w(L_-)}\bracket{L_0} \\
    + & (A^2-A^{-2})(-A)^{-3w(L_0)}\bracket{L_0} = 0.
\end{align}

Ale $w(L_\pm) = w(L_0)\pm 1$, zatem to jest równoważne z
\begin{equation}
    -A \bracket{L_+} +
    A^{-1} \bracket{L_-} +
    (A^2-A^{-2}) \bracket{L_0} =0.
\end{equation}

Z definicji klamry Kauffmana wnioskujemy, że
\begin{align}
    \bracket{L_+} & = A\bracket{L_0} + A^{-1}\bracket{L_\infty}, \\
    \bracket{L_-} & = A\bracket{L_\infty} + A^{-1}\bracket{L_0}.
\end{align}

Pierwsze równanie przemnóżmy przez $A$, drugie przez $A^{-1}$, a~następnie dodajmy je do siebie.
Wtedy otrzymamy
\begin{equation}
    A\bracket{L_+} - A^{-1}\bracket{L_-} =
    A^2 (\bracket{L_0} - \bracket{L_\infty}),
\end{equation}
quod erat demonstrandum.
\end{proof}




\subsection{Lustra, rewersy. Sumy}
Wielomian Jonesa nie wykrywa orientacji splotu:

\begin{proposition}
\index{rewers}%
    Niech $L$ będzie zorientowanym splotem.
    Wtedy $\jones(rL)=\jones(L)$.
\end{proposition}

\begin{proof}
    Aby obliczyć wielomian rewersu, wykorzystujemy te same diagramy kłębiaste,
    jak dla zwykłego, a~przy tym nie zmieniamy znaku żadnego skrzyżowania.
\end{proof}

Ale czasami potrafi odróżnić splot od jego lustra:

\begin{proposition}
\index{lustro}%
    Niech $L$ będzie zorientowanym splotem.
    Wtedy $\jones(mL)(t)=\jones(L)(t^{-1})$.
\end{proposition}

(Prawie nikt nie chce podać rozumowania tak szczegółowego jak to poniższe, które udało się nam skraść z niedostępnej dziś nigdzie pracy Gellerta \cite[s. 12]{gellert2009}).
To jest też \cite[fakt 10.4.4]{kawauchi1996}.

\begin{proof}
    Zauważmy, że diagramy $L_-$ oraz $L_+$ są wzajemnymi lustrami.
    Dlatego każda relacja kłębiasta dla splotu postaci
    \begin{equation}
        t^{-1} \jones(L_+)(t) - t\jones(L_-)(t) + (t^{-1/2} - t^{1/2}) \jones(L_0)(t) = 0
    \end{equation}
    odpowiada pewnej relacji dla lustra splotu:
    \begin{equation}
        -t\jones(L_+)(t) + t^{-1} \jones(L_-)(t) + (t^{-1/2} - t^{1/2}) \jones(L_0)(t) = 0,
    \end{equation}
    co po zamianie zmiennych $t \mapsto t^{-1}$ i przemnożeniu przez $-1$ daje
    \begin{equation}
        -t^{-1} \jones(L_+)(t^{-1}) + t \jones(L_-)(t^{-1}) + (t^{1/2} - t^{-1/2}) \jones(L_0)(t^{-1}) = 0.
    \end{equation}
\end{proof}

\begin{corollary}
\index{węzeł!zwierciadlany}%
\label{cor:jones_of_amphicheiral}%
    Jeśli $K$ jest węzłem zwierciadlanym, to wielomian $\jones_K$ jest symetryczny.
\end{corollary}

Równość $\conway(z) = \conway(-z)$ zachodzi dla wszystkich węzłów, zwierciadlanych lub nie.

Implikacja odwrotna nie zachodzi na mocy wniosku \ref{cor:acheiral_signature}: węzeł $9_{42}$ ma symetryczny wielomian Jonesa, ale niezerową sygnaturę.
\index{sygnatura}%
Poniżej trzynastu skrzyżowań kontrprzykładów z niezerową sygnaturą jest 14 ($9_{42}$, $10_{125}$, $11n_{19}$, $11n_{24}$, $11n_{82}$, $12a_{669}$, $12a_{1171}$, $12a_{1179}$, $12a_{1205}$, $12n_{362}$, $12n_{506}$, $12n_{562}$, $12n_{571}$, $12n_{821}$), z zerową -- kolejne 28.
% ZWERYFIKOWANO: funkcja symmetric_not_acheiral

\begin{corollary}
\index{węzeł!zwierciadlany}%
    Jeśli $K$ jest węzłem zwierciadlanym, to jego wielomian HOMFLY spełnia równość $P(a, z) = P(1/a, z)$, zaś wielomian Kauffmana spełnia równość $F(a, z) = F(1/a, z)$.
\end{corollary}

Równość $\jones(mL)(t) = \jones(L)(t^{-1})$ nie jest spełniona dla trójlistnika, zatem ten nie jest równoważny ze swoim lustrem.
Wcześniej pokazał to z~dużo większym wysiłkiem Dehn, patrz przykład \ref{exm:trefoil_is_chiral}.
\index[persons]{Dehn, Max}%

\begin{corollary}
    Wielomian Jonesa nie zależy od orientacji węzła.
\end{corollary}

Nie jest to prawdą dla splotów.

\begin{proof}
    Każdy węzeł ma tylko dwie orientacje, splot może mieć ich $2^n$, gdzie $n$ to liczba składowych.
\end{proof}

\begin{proposition}
\index{suma niespójna}%
\label{prp:jones_multiplicative_1}%
    Niech $L_1, L_2$ będą zorientowanymi splotami.
    Wtedy
    \begin{equation}
        \jones(L_1 \sqcup L_2) = (-t^{1/2} - t^{-1/2}) \jones(L_1) \jones(L_2).
    \end{equation}
\end{proposition}

\begin{proof}
    Wybierzmy diagramy $D_1, D_2$ dla splotów $L_1, L_2$.
    Po podstawieniu $t^{1/2} = A^{-2}$ widzimy, że chcemy pokazać
    \begin{equation}
        (-A)^{-3w(D_1 \sqcup D_2)} \langle D_1 \sqcup D_2 \rangle
        =
        (-A^2 - A^{-2})(-A)^{-3(w(D_1) + w(D_2))} \langle D_1 \rangle \langle D_2 \rangle.
    \end{equation}

    Oczywiście $w(D_1 \sqcup D_2) = w(D_1) + w(D_2)$, więc wystarczy udowodnić, że
    \begin{equation}
        \langle D_1 \sqcup D_2 \rangle = (-A^2 - A^{-2}) \langle D_1 \rangle \langle D_2 \rangle.
    \end{equation}

    Oznaczmy przez $f_1(D_1)$, $f_2(D_1)$ odpowiednio lewą i~prawą stronę ostatniego równania.
    Są to wielomiany Laurenta, które zależą tylko od $D_1$.
    Aksjomaty Kauffmana pozwalają na pokazanie, że obie funkcje mają następujące własności:
    \begin{align}
        f_i(\SmallUnknot)            & = (-A^2 - A^{-2}) \langle D_2 \rangle \\
        f_i(D_1 \sqcup \SmallUnknot) & = (-A^2 - A^{-2}) f_i(D_1) \\
        f_i(\LittleRightCrossing)     & = A f_i(\LittleRightSmoothing) + A^{-1} f_i(\LittleLeftSmoothing).
    \end{align}
    Ponieważ powyższe tożsamości wystarczają do wyznaczenia wartości funkcji $f_i$ dla dowolnego diagramu $D_1$, dochodzimy do wniosku, że $f_1 \equiv f_2$.
    To kończy dowód.
\end{proof}

\begin{proposition}
\label{prp:jones_multiplicative_2}%
\index{relacja kłębiasta}%
\index{suma spójna}%
    Niech $K_1, K_2$ będą zorientowanymi węzłami.
    Wtedy
    \begin{equation}
        \jones(K_1 \# K_2) = \jones(K_1) \jones(K_2).
    \end{equation}
\end{proposition}

\begin{proof}
    Rozpatrzmy sploty
\begin{comment}
    \begin{figure}[H]
    \centering
        %
        \begin{minipage}[b]{.3\linewidth}
            \[
                \MediumJonesShrapA
            \]
        \end{minipage}
        %
        \begin{minipage}[b]{.3\linewidth}
            \[
                \MediumJonesShrapB
            \]
        \end{minipage}
        %
        \begin{minipage}[b]{.3\linewidth}
            \[
                \MediumJonesShrapAB
            \]
        \end{minipage}
    \end{figure}
\end{comment}
    Relacja kłębiasta orzeka w tym przypadku, że
    \begin{equation}
        t^{-1} \jones(K_1 \# K_2) - t \jones(K_1 \# K_2) + (t^{-1/2} - t^{1/2}) \jones(K_1 \sqcup K_2) = 0.
    \end{equation}
    Ostatni składnik sumy można rozwinąć na mocy faktu \ref{prp:jones_multiplicative_1}.
    Po uporządkowaniu dostaniemy:
    \begin{equation}
        (t^{-1} - t) \jones(K_1 \# K_2) - (t^{-1} - t) \jones(K_1) \jones(K_2) = 0,
    \end{equation}
    a stąd widać już prawdziwość dowodzonej tezy.
\end{proof}




\subsection{Wartości w pierwiastkach jedności}
Niech $L$ będzie splotem mającym $n$ ogniw, zaś $\jones$ jego wielomianem Jonesa.
Wtedy wartości $\jones(\omega)$ w~niektórych pierwiastkach jedności są związane z~innymi niezmiennikami węzłów.

I tak przyjmując oznaczenie $\omega_k = \exp(2\pi i/k)$ mamy

\begin{proposition}
    \label{prp:jones_at_roots_of_unity}
    $\jones_L(\omega_3) = 1$.
\end{proposition}

\begin{proposition}
    $\jones_L(1) = (-2)^{n-1}$.
\end{proposition}

\begin{proof}
\index[persons]{Jones, Vaughan}%
    Jak wkrótce się przekonamy, to proste wnioski z~relacji kłębiastej.
    Explicite wskazał je Jones \cite[twierdzenie 14, 15]{jones85}.
\end{proof}

\begin{proposition}
    Pochodna w punkcie $t = 1$ znika: $\jones'_L(1) = 0$.
\end{proposition}

\begin{proof}
    Jones \cite[twierdzenie 16]{jones85}.
\end{proof}

\begin{proposition}
    $V_L(\omega_6) = \pm i^{n-1} \cdot (\sqrt 3i)^r$, gdzie $r$ jest rangą pierwszej grupy homologii podwójnego rozgałęzionego nakrycia $L$ nad $\Z/3\Z$.
\end{proposition}

\begin{proof}
\index[persons]{Lipson, Andrew}%
    Znak $\pm$ został wyznaczony przez Lipsona w \cite{lipson86}, praca ta zawiera też odsyłacz do pracy Milletta, Lickorisha \cite{lickorish86}, którzy wyprowadzają resztę wzoru.
\end{proof}

\begin{proposition}
    Liczba trzy-kolorowań splotu $L$ wynosi $3|\jones_L(\omega_6)|^2$.
\end{proposition}

\begin{proof}
\index[persons]{Przytycki, Józef}%
    Przytycki w \cite{przytycki98}.
\end{proof}

\begin{proposition}
    Jeśli $L$ jest właściwym splotem (indeks zaczepienia każdej składowej o~resztę splotu jest parzysty), to $\jones_L(i) = (\sqrt 2)^{n-1}(-1)^{\operatorname{Arf} L}$.
    W przeciwnym razie $\jones_L(i) = 0$.
\end{proposition}

\begin{proof}
\index[persons]{Murakami, Hitoshi}%
    Murakami \cite{murakami86}, ta praca jest wyjątkowo krótka (tylko 3 strony).
\end{proof}

\begin{proposition}
    Niech $H_1$ będzie pierwszą grupą homologii podwójnego nakrycia $S^3$ rozgałęzionego nad składowymi.
    Jeśli $H_1$ jest torsyjna, to $\jones_L(-1) = |H_1|$.
    W przeciwnym razie $\jones_L(-1) = 0$.
\end{proposition}

Nie znamy dowodu tego faktu: Ohtsuki \cite[s. 383]{ohtsuki02} pisze tylko \emph{,,It is known that $|V_L(-1)|$ is equal to the order of $H_1(M_{2,L})$''}. \hfill \texttt{:(}

Nie jest znana topologiczna interpretacja wielomianu Jonesa (którą posiada wielomian Alexandera) ani charakteryzacja poza warunkami koniecznymi z~pięciu faktów powyżej.

\begin{corollary}
    Niech $K$ będzie węzłem.
    Wtedy
    \begin{align}
        \jones(1) & = 1 \\
        \jones(-1) & = \pm \det K \\
        \jones(i) & = \begin{cases}
            1 & \text{dla } \alexander(-1) \equiv \pm 1 \mod 8 \\
            -1 & \text{w przeciwnym razie.}
        \end{cases}
    \end{align}
\end{corollary}

Poza powyżej opisanymi przypadkami, wartości wielomianu Jonesa nie można znaleźć w~czasie wielomianowym od ilości skrzyżowań na diagramie (jest to problem $\#P$-trudny).

% Koniec sekcji Relacja kłębiasta
% Koniec podsekcji Wielomian Jonesa

\index{klamra Kauffmana|)}

% Koniec podsekcji Nawias Kauffmana



\index{wielomian!Jonesa|)}%


\subsection{Hipotezy Taita}
\label{sub:tait_conjectures}%
\index{hipoteza!Taita|(}%
W tej podsekcji definiujemy stan oraz wygładzenie diagramu, następnie wyprowadzamy wzór o~sumowaniu stanów oraz dowodzimy prawdziwości dwóch technicznych lematów.
Pozwoli to oszacować rozpiętość wielomianu Jonesa, skąd bezpośrednio wynika już I hipoteza Taita \ref{con:tait_1}.
Nie odbiega to znacząco od podejścia Kauffmana w~\cite{kauffman1987}.
\index[persons]{Kauffman, Louis}%
Na koniec naszkicujemy krótko dowody pozostałych hipotez.

\begin{definition}
% DICTIONARY;isthmus;przesmyk;-
\index{diagram!zredukowany}%
\index{przesmyk}%
    Wąskie skrzyżowanie między dwiema rozłącznymi częściami diagramu nazywamy przesmykiem.
\begin{comment}
    \[
        \LargeIsthmus
    \]
\end{comment}
    Diagram, na którym nie ma żadnych przesmyków, nazywamy zredukowanym.
\end{definition}

Słowo przesmyk pochodzi z teorii grafów, skrzyżowanie jest przesmykiem dokładnie wtedy, gdy odpowiadająca mu krawędź w grafie węzła jest przesmykiem, czyli jej usunięcie zwiększa liczbę składowych spójności.
Tam używa się także określenia most, które u~nas ma już zarezerwowane inne znaczenie.
Poza tym w tej książce nie ma grafów.

\begin{definition}[diagram spójny]
\index{diagram!spójny}%
    Diagram, którego nie można podzielić na dwie niepuste części niespotykające się na żadnym skrzyżowaniu, nazywamy spójnym.
\end{definition}

Jeżeli diagram nie jest spójny, możemy przesunąć rozłączne ogniwa na siebie przy użyciu II ruchu Reidemeistera.
Hipoteza Taita mówi coś o zredukowanych, spójnych i alternujących diagramach.

\begin{definition}[stan]
\index{stan diagramu}%
    Niech $D$ będzie diagramem splotu.
    Każdą funkcję $s$ ze zbioru skrzyżowań diagramu $D$ o wartościach $\pm 1$ nazywamy stanem.
    Sumę wszystkich wartości $s$ oznaczamy $|s|$.
\end{definition}

\begin{definition}
\index{wygładzenie}%
    Niech $D$ będzie diagramem splotu $L$, zaś $s$ jego stanem.
    Diagram powstały przez wygładzenie wszystkich skrzyżowań zgodnie z~ich stanem oznaczamy $sD$.

    Przez $|sD|$ rozumiemy liczbę zamkniętych, nieprzecinających się krzywych, z których składa się nowy diagram.
\end{definition}

\begin{proposition}[o sumowaniu stanów]
\index{wzór o sumowaniu stanów}%
    Niech $D$ będzie diagramem splotu.
    Wtedy
    \begin{equation}
        \langle D\rangle = \sum_s (-A^2-A^{-2})^{|sD|-1} A^{|s|},
    \end{equation}
    gdzie sumujemy po wszystkich stanach $s$ dla diagramu $D$.
\end{proposition}

Dla wygody wprowadźmy skrót $\langle D \mid s \rangle := (-A^{-2}-A^2)^{|sD|-1}A^{|s|}$.

\begin{proof}
    Oznaczmy prawą stronę dowodzonej równości przez $[D]$.
    Pokażemy, że spełnia ona trzy aksjomaty Kauffmana z~definicji \ref{def:kauffman_bracket}, skąd wynika już, że $[D] = \bracket{D}$.
    % ona $[\SmallUnknot]=1$, $[D\sqcup\SmallUnknot]=(-A^{-2}-A^2) [D]$ oraz $[\SmallMinusCrossing] = A [\SmallAlphaSmoothing] + A^{-1}[\SmallBetaSmoothing]$.

    Pierwszy aksjomat (równość \ref{eqn:kauffman_axiom_1}) mówi o niewęźle $\SmallUnknot$.
    Posiada on tylko jeden stan $s$ dany wzorem $|s| = 0$, zatem $|s\,\SmallUnknot| = 1$ oraz $[D] = (-A^2 - A^{-2})^0 \cdot A^0 = 1$.

    By pokazać, że funkcja $[\,\cdot\,]$ spełnia drugi aksjomat (równość \ref{eqn:kauffman_axiom_2}) zauważmy, że diagramy $D \sqcup \SmallUnknot$ oraz $D$ mają te same skrzyżowania,
    więc możemy utożsamiać stany $s$ dla $D$ ze stanami $u$ dla $D \sqcup \SmallUnknot$.
    Wtedy $|u| = |s|$ oraz $|u(D \sqcup \SmallUnknot)| = |sD| + 1$.
    Zatem
    \begin{align}
        \left[D \sqcup \SmallUnknot\right]
        & = \sum_u (-A^2-A^{-2})^{|u(D\sqcup\SmallUnknot)|-1} A^{|u|} \\
        & = \sum_s (-A^2-A^{-2})^{|sD|} A^{|s|} \\
        & = (-A^2-A^{-2}) [D].
    \end{align}
    Pozostał ostatni aksjomat, równość \ref{eqn:kauffman_axiom_3}.
    Bezpośrednio z definicji mamy
    \begin{equation}
       A\left[\MediumAlphaSmoothing\right]
       = \sum_u(-A^2-A^{-2})^{|u\SmallAlphaSmoothing|-1}A^{|u|+1},
    \end{equation}
    gdzie $u$ przebiega wszystkie stany $\SmallAlphaSmoothing$.
    Ale $\SmallAlphaSmoothing$ to $\SmallMinusCrossing$ z ustalonym skrzyżowaniem $c$ wygładzonym dodatnio, co daje bijekcję między wszystkimi stanami $u$ diagramu $\SmallAlphaSmoothing$ oraz tymi stanami $s$ diagramu $\SmallMinusCrossing$, dla których $s(c) = + 1$.
    Wtedy $|s\SmallMinusCrossing| = |u\SmallAlphaSmoothing|$, $|s| = |u|+1$ oraz
    \begin{align}
        A\left[\MediumAlphaSmoothing\right]
        & = \sum_u (-A^2-A^{-2})^{|u\,\SmallAlphaSmoothing|-1}A^{|u|+1} \\
        \label{eqn:state_sum_right}
        & = \sum_{s(c)=1}(-A^2-A^{-2})^{|s\,\SmallMinusCrossing|-1}A^{|s|},
    \end{align}
    podobne rozumowanie pokazuje, że
    \begin{align}
        \label{eqn:state_sum_left}
        A^{-1}\left[\MediumBetaSmoothing\right]
        & = \sum_{s(c)=-1}(-A^2-A^{-2})^{|s\,\SmallMinusCrossing|-1}A^{|s|}.
    \end{align}
    Dodanie do siebie równań \ref{eqn:state_sum_right} oraz \ref{eqn:state_sum_left} kończy dowód.
    %: $A[\PrawyGladki]+A^{-1}[\LewyGladki] = \sum_s(-A^2-A^{-2})^{|s\,\SmallMinusCrossing|-1}A^{|s|} = [\RightCrossing]$.
\end{proof}
Zbadamy następnie dwa najprostsze stany dowolnego diagramu.

\begin{definition}
    Stan przypisujący wartość $+1$ ($-1$) każdemu skrzyżowaniu oznaczamy $s_+$ ($s_-$).
\end{definition}

Niech $D$ będzie alternującym, zredukowanym diagramem spójnym.
Wtedy wszystkie jego skrzyżowania mają ten sam znak.
Wybierzmy dla niego to uszachowienie, w~którym wszystkie skrzyżowania są dodatnie:
\begin{comment}
\[
    \begin{tikzpicture}[baseline=-0.65ex,scale=0.15]
    \begin{knot}[clip width=5]
        \strand[thick] (-25, 0) to (25, 0);
        \strand[thick] (10*0-15, -5) to (10*0-15, -1);
        \strand[thick] (10*1-15, -5) to (10*1-15, -1);
        \strand[thick] (10*2-15, -5) to (10*2-15, -1);
        \strand[thick] (10*3-15, -5) to (10*3-15, -1);
        \strand[thick] (10*0-15, 1) to (10*0-15, 5);
        \strand[thick] (10*1-15, 1) to (10*1-15, 5);
        \strand[thick] (10*2-15, 1) to (10*2-15, 5);
        \strand[thick] (10*3-15, 1) to (10*3-15, 5);
        \draw[fill=diagramfiller,draw=none] (-25, 0) rectangle (-15, -5);
        \draw[fill=diagramfiller,draw=none] (-15, 0) rectangle (-5, 5);
        \draw[fill=diagramfiller,draw=none] (-5, 0) rectangle (5, -5);
        \draw[fill=diagramfiller,draw=none] (5, 0) rectangle (15, 5);
        \draw[fill=diagramfiller,draw=none] (15, 0) rectangle (25, -5);
        \node[above left] at (-15, 0) {$+1$};
        \node[above left] at (5, 0) {$+1$};
        \node[below left] at (-5, 0) {$+1$};
        \node[below left] at (15, 0) {$+1$};
    \end{knot}
    \end{tikzpicture}
\]
\end{comment}
Nazywamy je uszachowieniem \emph{standardowym}.
\index{uszachowienie!standardowe}%
Porównajmy wygładzenie $s_+D$ z~$s_-D$:
\begin{comment}
\[
    \begin{tikzpicture}[baseline=-0.65ex,scale=0.10]
        \node at (0, 8) {$s_+D$};
        \draw[fill=diagramfiller,draw=none] (-25, -5) rectangle (25, 5);
        \draw[fill=white, draw=none] (-15, -5) [in=left, out=up] to (-12, 0) -- (-8, 0) [in=up, out=right] to (-5, -5);
        \draw[fill=white, draw=none] (5, -5) [in=left, out=up] to (8, 0) -- (12, 0) [in=up, out=right] to (15, -5);
        \draw[fill=white, draw=none] (-5, 5) [in=left, out=down] to (-2, 0) -- (2, 0) [in=down, out=right] to (5, 5);
        \draw[fill=white, draw=none] (-25, 0) -- (-18, 0) [in=down, out=right] to (-15, 5) -- (-25, 5);
        \draw[fill=white, draw=none] ( 25, 0) -- ( 18, 0) [in=down, out=left] to ( 15, 5) -- ( 25, 5);
    \end{tikzpicture}
    \quad
    \begin{tikzpicture}[baseline=-0.65ex,scale=0.10]
        \node at (0, 8) {$s_-D$};
        \draw[fill=diagramfiller, draw=none] (-15, 5) [in=left, out=up] to (-12, 0) -- (-8, 0) [in=up, out=right] to (-5, 5);
        \draw[fill=diagramfiller, draw=none] (5, 5) [in=left, out=up] to (8, 0) -- (12, 0) [in=up, out=right] to (15, 5);
        \draw[fill=diagramfiller, draw=none] (-5, -5) [in=left, out=down] to (-2, 0) -- (2, 0) [in=down, out=right] to (5, -5);
        \draw[fill=diagramfiller, draw=none] (-25, 0) -- (-18, 0) [in=down, out=right] to (-15, -5) -- (-25, -5);
        \draw[fill=diagramfiller, draw=none] ( 25, 0) -- ( 18, 0) [in=down, out=left] to ( 15, -5) -- ( 25, -5);
    \end{tikzpicture}
\]
\end{comment}

Zamknięte krzywe tworzące $s_+D$ są brzegami jasnych obszarów uszachowienia, podczas gdy te tworzące $s_-D$ stanowią brzeg ciemnych obszarów.
Zauważmy jeszcze, że na każdym skrzyżowaniu występują cztery różne ciemne i~jasne obszary: gdyby pewien obszar dotykał tam sam siebie, mielibyśmy do czynienia w~tym miejscu z~przesmykiem, a~założyliśmy przecież, że diagram jest zredukowany.

\begin{lemma}
\label{lem:pretait_lemma_1}%
    Niech $D$ będzie spójnym diagramem splotu o~$n$ skrzyżowaniach.
    Wtedy
    \begin{equation}
        |s_+D| + |s_-D| \le n+2,
    \end{equation}
    z~równością, gdy diagram $D$ jest alternujący i~zredukowany.
\end{lemma}

\begin{proof}
    Skorzystamy z~indukcji względem $n$.
    Łatwo widać prawdziwość lematu dla $n = 0$.
    Załóżmy, że jest on prawdziwy dla wszystkich diagramów o~$n - 1$ skrzyżowaniach, następnie ustalmy diagram $D$ o~$n$ skrzyżowaniach.

    Wybierzmy skrzyżowanie na diagramie $D$.
    Można je wygładzić na dwa sposoby, jeden z~nich daje spójny diagram $D'$.
    Bez straty ogólności przyjmijmy, że dzieje się tak podczas dodatniego wygładzenia.
    Wtedy zachodzi $|s_+D'| = |s_+D|$, ale $|s_-D'| = |s_-D|\pm 1$, gdyż diagram $s_-D'$ powstaje z~$s_-D$ po zastąpieniu pewnej części
    $\SmallAlphaSmoothing$ przez $\SmallBetaSmoothing$.
    To rozrywa jedną krzywą na dwa kawałki lub scala dwie krzywe w~jedną.
    Z~założenia indukcyjnego mamy
    \begin{align}
        |s_+D| + |s_-D|
        & = |s_+D'| + |s_-D'| \pm 1 \\
        \label{inequality_sd1} & \le (n - 1) + 2 \pm 1 \\
        \label{inequality_sd2} & \le n + 2.
    \end{align}

    Załóżmy, że diagram $D$ jest spójny, alternujący i zredukowany.
    Nierówność \ref{inequality_sd1} jest tak naprawdę równością jako powtórzenie założenia indukcynego.
    Z drugiej strony zachodzi $|s_-D'|=|s_-D|-1$, ponieważ przejście od $s_-D$ do $s_-D'$ skleja dwa ciemne obszary, jak na poniższym rysunku:
\begin{comment}
    \[
        \begin{tikzpicture}[baseline=-0.65ex,scale=0.20]
        \begin{knot}[clip width=5]
            \strand[thick] (-5, 0) to (5, 0);
            \strand[thick] (0, -5) to (0, -1);
            \strand[thick] (0, 1) to (0, 5);
            \draw[fill=diagramfiller,draw=none] (-5, -5) rectangle (0, 0);
            \draw[fill=diagramfiller,draw=none] ( 5,  5) rectangle (0, 0);
            \node at (0, -8) {$D$};
        \end{knot}
        \end{tikzpicture}
        \quad
        \begin{tikzpicture}[baseline=-0.65ex,scale=0.20]
            \draw[fill=diagramfiller, draw=none] (-5, 0) -- (-2, 0) [in=up, out=right] to (0, -2) -- (0, -5) -- (-5, -5);
            \draw[fill=diagramfiller, draw=none] (5, 0) -- (2, 0) [in=down, out=left] to (0, 2) -- (0, 5) -- (5, 5);
            \draw[thick] (-5, 0) -- (-2, 0) [in=up, out=right] to (0, -2) -- (0, -5);
            \draw[thick] (5, 0) -- (2, 0) [in=down, out=left] to (0, 2) -- (0, 5);
            \node at (0, -8) {$s_-D$};
        \end{tikzpicture}
        \quad
        \begin{tikzpicture}[baseline=-0.65ex,scale=0.20]
            \draw[fill=diagramfiller, draw=none] (-5, -5) rectangle (5, 5);
            \draw[fill=white, draw=none] (5, 0) -- (2, 0) [in=up, out=left] to (0, -2) -- (0, -5) -- (5, -5);
            \draw[fill=white, draw=none] (-5, 0) -- (-2, 0) [in=down, out=right] to (0, 2) -- (0, 5) -- (-5, 5);
            \draw[thick] (5, 0) -- (2, 0) [in=up, out=left] to (0, -2) -- (0, -5);
            \draw[thick] (-5, 0) -- (-2, 0) [in=down, out=right] to (0, 2) -- (0, 5);
            \node at (0, -8) {$s_-D'$};
        \end{tikzpicture}
    \]
\end{comment}
    Oznacza to, że nierówność \ref{inequality_sd2} także jest równością, quod erat demonstrandum.
\end{proof}

W~dowodzie hipotezy Taita użyjemy rozpiętości wielomianu Jonesa:

\begin{definition}[rozpiętość]
\index{rozpiętość wielomianu}%
    Niech $f$ będzie wielomianem Laurenta jednej zmiennej $X$.
    Różnicę między najwyższą i najniższą potęgą $X$ występującą w $f$,
    \begin{equation}
        \operatorname{span} f = \operatorname{maxdeg} f - \operatorname{mindeg} f,
    \end{equation}
    nazywamy rozpiętością wielomianu $f$.
\end{definition}

Na przykład $\operatorname{span} (-t+3-1/t) = 1 - (-1) = 2$.

 \begin{lemma}
    \label{lem:pretait_lemma_2}
    Niech $D$ będzie diagramem splotu o~$n$ skrzyżowaniach.
    Wtedy
    \begin{align}
        \operatorname{maxdeg} \langle D \rangle & \le n - 2 + 2|s_+D| \\
        \operatorname{mindeg} \langle D \rangle & \ge 2 - n - 2|s_-D|
    \end{align}
    z równością, jeżeli $D$ jest alternujący, zredukowany i~spójny.
\end{lemma}

\begin{proof}
    Ponieważ dowody tych nierówności przebiegają analogicznie, ograniczymy się do pierwszej z nich.
    Dla oszczędności miejsca będziemy pisać $M$ zamiast $\operatorname{maxdeg}$.

    Niech $s$ będzie dowolnym stanem.
    Istnieje wtedy ciąg stanów $s_+ = s_0, s_1, \ldots, s_r = s$, w~którym stan $s_{i+1}$ powstaje z~$s_i$ przez zmianę wartości w jednym ze skrzyżowań z $+1$ na $-1$.
    Skoro diagram $s_{i+1}D$ uzyskujemy z~$s_{i}D$ przez połączenie dwóch zamkniętych krzywych lub podział jednej krzywej na dwie części, mamy: $|s_{i+1}| = |s_i| - 2$ oraz $|s_{i+1}D| = |s_iD| \pm 1$.
    Wnioskujemy stąd, że
    \begin{align}
        M \langle D \mid s_{i+1} \rangle
        & = 2|s_{i+1}D| + |s_{i+1}|-2 \\
        & = (2|s_iD| + |s_i| -2 ) + (\pm 2-2) \\
        & \le M \langle D|s_i\rangle.
    \end{align}
    Powtarzając odpowiednio wiele razy, dostajemy łańcuch nierówności
    \begin{equation}
        M \langle D \mid s \rangle
        =
        M \langle D \mid s_r \rangle
        \le \ldots \le
        M \langle D \mid s_0 \rangle
        =
        M \langle D \mid s_+ \rangle.
    \end{equation}
    Zauważmy, że dla każdego stanu $s$ zachodzi $M \langle D|s \rangle = 2|sD| + |s| - 2$.
    Zatem
    \begin{equation}
        M \langle D \mid s \rangle \le 2 |s_+D| + |s| - 2 = 2|s_+D| + n - 2,
    \end{equation}
    co kończy dowód pierwszej części lematu.

    Załóżmy teraz, że diagram $D$ jest zredukowany, alternujący i~spójny.
    Chcemy pokazać, że stan $s_+$ dominuje nad pozostałymi: żaden inny nie wnosi do sumy wyrazu z~takim samym najwyższym wykładnikiem.
    Czy warunek $s \neq s_+$ implikuje $M\langle D|s\rangle < M\langle D| s_+\rangle$?
    Dzięki łańcuchowi nierówności wystarczy sprawdzić to tylko dla tych stanów $s$, które powstają z~$s_+$ przez zmianę pojedynczej wartości $+1$ na $-1$.
    Ale to już jest oczywiste, gdyż $sD$ otrzymujemy przez sklejenie dwóch jasnych obszarów $s_+ D$.
\end{proof}

Możemy wreszcie zająć się rozpiętością wielomianu Jonesa.

\begin{proposition}
    Niech $L$ będzie zorientowanym splotem o~spójnym diagramie $D$ z~$n$ skrzyżowaniami.
    Wtedy $\operatorname{span} \jones(L) \le n$, z~równością dla zredukowanego i~alternującego $D$.
\end{proposition}

\begin{proof}
    Pokażemy prawdziwość innego, równoważnego stwierdzenia: $\operatorname{span} \langle D\rangle\le 4n$.
    Dwa poprzednie lematy \ref{lem:pretait_lemma_1} oraz \ref{lem:pretait_lemma_2} mówią razem, że
    \begin{align}
        \operatorname{span}\langle D\rangle
        & = M\langle D\rangle - m\langle D\rangle \\
        & \le (2|s_+D|+n-2)+(2|s_-D|+n-2) \\
        & = 2(|s_+D|+|s_- D|)+2n-4 \\
        & \le 2(n+2)+2n-4 \\
        & = 4n. \qedhere
    \end{align}
\end{proof}

\begin{conjecture}[Taita, pierwsza]
\index{hipoteza!Taita!pierwsza}%
    Zredukowany, spójny, alternujący diagram $D$ zorientowanego splotu $L$ realizuje jego indeks skrzyżowaniowy.
\end{conjecture}

\begin{proof}
    Załóżmy nie wprost, że istnieje diagram o~mniejszej liczbie skrzyżowań,
    mielibyśmy $\operatorname{span} (\jones(L)) < n$, co prowadzi do sprzeczności z~równością $\operatorname{span} (\jones(L)) = n$.
\end{proof}

Szukanie wielomianu Jonesa splotu bywa uciążliwe,
jednak czasami możemy oszacować jego rozpiętość korzystając z~następujących nierówności:

\begin{corollary}
    Niech $L$ będzie zorientowanym splotem ze spójnym diagramem $D$, na którym widać $n$ skrzyżowań.
    Wtedy
    \begin{align}
        3w(D) - 2|s_+D| + 2 - n & \le 4 m(\jones(L) \\
        3w(D) + 2|s_-D| + n - 2 & \ge 4 M(\jones(L)),
    \end{align}
    z~równością dla zredukowanego i~alternującego $D$.
\end{corollary}

\begin{proof}
    Wystarczy powołać się na definicję \ref{def:jones_polynomial} oraz lemat \ref{lem:pretait_lemma_2}.
\end{proof}

Jesteśmy gotowi przejść do kolejnej hipotezy Taita, wciąż podążając za pracą Kauffmana \cite{kauffman1987}.
(do napisania)

Dowód trzeciej hipotezy nie zostanie podany.
Korzysta z geometrycznych i algebraicznych technik: pracy Menasco o czterokrotnie przekłutej 2-sferze w dopełnieniu splotu, z własności wielomianów $\jones$ oraz $F$ znalezionych przez Thistlethwaite'a oraz nieściśliwych powierzchnii z~niepołudnikowym brzegiem...

Czwarta hipoteza wynika dla alternujących węzłów z drugiej, ponieważ odbicie lustrzane diagramu zamienia ze sobą dodatnie i ujemne skrzyżowania, a zatem neguje spin.
Założenia, że węzeł jest alternujący nie można pominąć, patrz tekst za przykładem \ref{property_of_eight_knot}.

\begin{proposition}
    Dla każdego nieparzystego $n \ge 15$ istnieje pierwszy, zwierciadlany węzeł $K$, którego indeksem skrzyżowaniowym jest $n$.
\end{proposition}

Dziesięć lat temu Stojmenow zamieścił na portalu arXiv pracę \cite{stoimenow2007}; czekam na publikację w recenzowanym czasopiśmie.

\index{hipoteza!Taita|)}%

% Koniec podsekcji Rozpiętość i~wielomian Jonesa


