
\subsection{Definicja algebraiczna -- algebra Temperleya-Lieba}

% https://en.wikipedia.org/wiki/Jones_polynomial
Jones otrzymał swój wielomian jako efekt uboczny badań nad algebrami operatorów: wziął ,,ślad'' pewnej reprezentacji warkoczy w~algebrę, która miała ważne znaczenie w~mechanice statystycznej.
\index[persons]{Jones, Vaughan}%
Jego praca \cite{jones1985} jest bardzo zwięzła, nam oświecenie przyniosła dopiero książka Kauffmana \cite[s. 85-103]{kauffman1991} oraz strona ,,Aharonov–Jones–Landau algorithm'' z angielskiej Wikipedii.
\index{algorytm!Aharonova-Jonesa-Landaua}%

% książka Kauffmana
\begin{definition}[algebra Temperleya-Lieba]
\index{algebra!Temperleya-Lieba}%
    Niech $R$ będzie przemiennym pierścieniem, w~którym ustalono element $\delta \in R$.
    Wtedy $R$-algebrę $TL_n(\delta)$ generowaną przez elementy $e_1, \ldots, e_{n-1}$, które związane są relacjami
    \begin{align}
        U_i^2 & = \delta U_i, \\
        U_i U_{i \pm 1} U_i & = U_i, \\
        U_i U_j & = U_j U_i
    \end{align}
    dla $|i-j| \ge 2$, nazywamy algebrą Temperleya-Lieba.
\end{definition}

$TL_n(\tau)$ daje się przedstawić przy użyciu diagramów: prostokątów, których przeciwległe boki zawierają po $n$ punktów połączonych w~pary tak, by uniknąć samoprzecięć.
Mnożenie elementów algebry odpowiada sklejaniu dwóch diagramów, przy czym każdą zamkniętą pętlę zamieniamy na dodatkowy czynnik $\delta$.
To prawie są warkocze.

\begin{comment}
  \begin{figure}[H]
    \centering
    \begin{minipage}[b]{.18\linewidth}
        \[\LargeTemperleyA\]
        \subcaption{$1$}
    \end{minipage}
    \begin{minipage}[b]{.18\linewidth}
        \[\LargeTemperleyC\]
        \subcaption{$U_1$}
    \end{minipage}
    \begin{minipage}[b]{.18\linewidth}
        \[\LargeTemperleyB\]
        \subcaption{$U_2$}
    \end{minipage}
    \begin{minipage}[b]{.18\linewidth}
        \[\LargeTemperleyD\]
        \subcaption{$U_1U_2$}
    \end{minipage}
    \begin{minipage}[b]{.18\linewidth}
        \[\LargeTemperleyE\]
        \subcaption{$U_2U_1$}
    \end{minipage}
    \caption{Diagramatyczne przedstawienie elementów algebry $TL_3(\delta)$}
    \end{figure}
\end{comment}

Dla ustalonej liczby zespolonej $A$, niech funkcja $\rho_A \colon B_n \to TL_n(\delta)$ będzie zadana na generatorach grupy warkoczy: $\rho_A(\sigma_i) = AU_i + A^{-1}$.
Bezpośredni rachunek pokazuje, że jeśli $\delta = -A^2-A^{-2}$, to $\rho_A$ jest reprezentacją.

% książka Kauffmana
\begin{definition}[algebra Jonesa]
\index{algebra!Jonesa}%
    Niech $R$ będzie przemiennym pierścieniem, zaś $\tau$ skalarem, który komutuje ze wszystkimi innymi elementami.
    Wtedy $R$-algebrę $TL_n(\tau)$ generowaną przez elementy $e_1, \ldots, e_{n-1}$, związanymi relacjami
    \begin{align}
        e_i^2 & = e_i, \\
        e_i e_{i \pm 1} e_i & = \tau e_i, \\
        e_i e_j & = e_j e_i
    \end{align}
    dla $|i-j| \ge 2$, nazywamy algebrą Jonesa.
\end{definition}

Kauffman pisze, że algebry są ze sobą związane: jeśli ustalimy swoją ulubioną algebrę Temperleya-Lieba, a potem położymy $e_i = \delta^{-1} U_i$ oraz $\tau = \delta^{-2}$, to aksjomaty Jonesa są spełnione.

% artykuł na Wiki
\begin{definition}[ślad Markowa]
\index{sZZZlad@ślad Markowa}%
    Niech $K \in TL_n(\delta)$ będzie elementem algebry Temperleya-Lieba, iloczynem generatorów $e_1, \ldots, e_{n-1}$, którego domknięcie rozpada się na $m$ składowych spójności.
    Wielkość
    \begin{equation}
        \trace K = \delta^{m-n}
    \end{equation}
    nazywamy śladem Markowa elementu $K$.
\end{definition}

Ślad Markowa przedłuża się liniowo do całej algebry Temperleya-Lieba i jest prawdziwym śladem: spełnia warunki $\trace (1) = 1$ oraz $\trace (T_1T_2) = \trace (T_2T_1)$.
Ma też dodatkową własność, że jeśli $w$ jest słowem na literach $e_1, e_2, \ldots, e_{i-1}$, to $\trace(we_i) = \delta^{-1} \trace (w)$.

Jones skorzystał z twierdzeń Alexandera i Markowa dotyczących warkoczy, złożył ze sobą funkcje $\trace$ oraz $\rho_A$ oraz znormalizował wynik.
\index{twierdzenie!Alexandera}%
\index{twierdzenie!Markowa}%
Dostał tak niezmiennik splotów $L$ domknięć warkoczy $B$:
\begin{equation}
    V_L(A^{-4}) = (-A)^{3 \writhe D} \delta^{n-1} (\trace \circ \rho_A)(B).
\end{equation}

% Koniec podsekcji Oryginalna praca Jonesa

