\section{Wielomian Alexandera}
Większość niezmienników, jakie dotąd widzieliśmy, przypisuje każdemu węzłowi lub splotowi pewną liczbę całkowitą: niezmienniki z pierwszego rozdziału, liczba kolorowań, wyznacznik czy też defekt są właśnie takie.
Poznamy teraz wielomian, opisany po raz pierwszy przez Jamesa Waddella Alexandera w~1923 roku \cite{alexander23}.
Przez blisko sześć dekad pozostał on jedynym wielomianowym niezmiennikiem węzłów.
Ze względu na elementarność takiego podejścia, pominiemy teorię homologii, zamiast tego skupiając naszą uwagę na równaniach kolorujących.
Najpierw trzeba je jednak uogólnić.

\begin{definition}
    Wielomianowe równanie kolorujące związane ze skrzyżowaniem
\begin{comment}
    \[\begin{tikzpicture}[baseline=-0.65ex, scale=0.12]
    \useasboundingbox (-5, -5) rectangle (5,5);
    \begin{knot}[clip width=5, end tolerance=1pt, flip crossing/.list={1}]
        \strand[semithick] (-5,5) to (5,-5);
        \strand[semithick,-Latex] (-5,-5) to (5,5);
        \node[darkblue] at (5, 5)[below right] {$a$};
        \node[darkblue] at (5, -5)[above right] {$b$};
        \node[darkblue] at (-5, 5)[below left] {$c$};
    \end{knot}
    \end{tikzpicture}\]
\end{comment}
    splotu zorientowanego to $a + tc - ta - b = 0$.
    Tylko orientacja górnej wiązki ma znaczenie.
\end{definition}

Istotnie, wystarczy podstawić tutaj $t = -1$, by otrzymać mniej ogólną definicję \ref{def:colouring_equation}.

\begin{definition}[wielomian Alexandera]
    \label{def:alexander_polynomial}
    \index{wielomian!Alexandera}
    Niech $L$ będzie zorientowanym splotem z~diagramem bez krzywych zamkniętych.
    Przypiszmy etykiety $x_0, \ldots, x_m$ do włókien oraz $0, \ldots, m$ do skrzyżowań.
    Niech $P_{ij}$ będzie współczynnikiem przy $x_j$ w~wielomianowym równaniu kolorującym nad wierzchołkiem $i$.
    Z macierzy $P=(P_{ij})$ wykreślmy jedną kolumnę i~jeden wiersz.
    Wyznacznik tak otrzymanej macierzy nazywamy wielomianem Alexandera i~oznaczamy $\alexander_L(t)$.
\end{definition}

Nasz nowy niezmiennik nie jest zwykłym wielomianem, tylko wielomianem Laurenta jednej zmiennej, czyli elementem pierścienia $\Z[t, t^{-1}]$.

\begin{proposition}
    \label{alexander_invariance}
    Wielomian Alexandera z~dokładnością do mnożenia przez jedności:
    \begin{equation}
        f(t) \equiv g(t) \iff \exists m \in \Z: f(t) = \pm t^m g(t)
    \end{equation}
    jest niezmiennikiem zorientowanych splotów.
\end{proposition}

W dowodzie niezmienniczości wyznacznika węzła skorzystaliśmy z~relacji między nim a~grupą kolorującą.
Poprzednie wydania książki zawierały sugestię, że elementarny (czyli taki, który nie korzysta z~teorii modułów) dowód niezmienniczości wielomianu Alexandera nie istnieje.
Sugestia ta była błędna, wystarczy użyć alternatywnej definicji.

\begin{proof}
    Ustalmy diagram o~$k$ skrzyżowaniach, który rozcina płaszczyznę na $k+2$ obszarów i~utwórzmy macierz o~wymiarach $k \times k$, której kolumny odpowiadają obszarom, wiersze zaś skrzyżowaniom -- pomijając przy tym dwa sąsiadujące ze sobą obszary -- o~wyrazach ze współczynników równań kolorujących.
    Jej wyznacznik jest wielomianem Alexandera.

    Sąsiadującym ze sobą obszarom przypiszmy kolejne liczby całkowite tak, by obszar leżący po prawej stronie włókna miał niższy indeks.
    Pokażemy najpierw, że skasowanie kolumny indeksu $n$ oraz $n+1$ sprawia, że wyznacznik zmienia się co najwyżej o~czynnik $\pm t^m$ dla pewnego $m$.
    Niech $S_n$ oznacza sumę kolumn indeksu $n$.
    Każdy wiersz macierzy zawiera cztery niezerowe wyrazy: $\pm 1, \pm t$, zatem $\sum_n S_n = 0$.
    Równość ta zachodzi nawet po przemnożeniu kolumny indeksu $n$ przez $t^{-n}$: $\sum_n t^{-n}S_n = 0$, co prowadzi do relacji $\sum_n (t^{-n}-1) S_n = 0$.
    Jeśli więc indeks kolumny $v_j$ wynosi $n$, to $(t^{-n}-1)v_j$ jest kombinacją liniową innych kolumn niezerowego indeksu (ponieważ $t^0 - 1 = 0$).

    Rozpatrzmy macierze $M_{0,j}, M_{0,k}$, gdzie indeksy $j$-tej i~$k$-tej kolumny to odpowiednio $p$ i~$q$.
    Z powyższych rozważań wynika, że $(t^{-q}-1) \alexander_{0,j} = \pm (t^{-p}-1)\alexander_{0,k}$, ale indeksy obszarów są wyznaczone z~dokładnością do stałej addytywnej.
    Biorąc $i$-tą oraz $l$-tą kolumnę, indeksów $r$ oraz $s$, dostaniemy zależności
    \begin{align}
        (t^{r-q}-1) \alexander_{l,j} & = \pm (t^{r-p} - 1)\alexander_{l,i} \\
        (t^{q-s}-1) \alexander_{k,l} & = \pm (t^{q-r} - 1)\alexander_{k,i}
    \end{align}
    co prowadzi do
    \begin{equation}
        \alexander_{l,j} = \pm \frac{t^{q-r}(t^{r-p}-1)}{t^{q-s}-1} \alexander_{k,i}
    \end{equation}
    Położenie $p = r +1$, $s =q+1$ pokazuje, że różny wybór kolumn do skreślenia zmienia wyznacznik macierzy co najwyżej o~czynnik $\pm t^m$.

    Wprowadźmy jeszcze jedną techniczną definicję.
    Dwie kwadratowe macierze będą dla nas równoważne, jeśli można przejść od jednej do drugiej przy użyciu pięciu operacji:
    \begin{enumerate}[leftmargin=*]
    \itemsep0em
        \item przemnożenie wiersza lub kolumny przez $-1$;
        \item zamiana dwóch wierszy lub kolumn miejscami;
        \item dodanie jednego wiersza do innego (lub kolumny do innej);
        \item przemnożenie lub podzielenie kolumny przez $t$;
        \item rozszerzenie lub zmniejszenie macierzy o~$1$ na przekątnej i~zera w~innych miejscach.
    \end{enumerate}

    Ruchy Reidemeistera prowadzą do macierzy równoważnych wyjściowym.
    Każda z~tych operacji zmienia wyznacznik macierzy o~czynnik $\pm t^{-m}$, co kończy dowód.
\end{proof}

Zwyczajowo wielomian normalizuje się: bierze reprezentanta, który jest symetryczny w~zmiennych $t$ i $t^{-1}$ oraz przyjmuje w~punkcie $1$ wartość $\alexander_L(1) = 1$.
Odwrotnie, dowolny wielomian Laurenta z~całkowitymi współczynnikami o~takich własnościach jest wielomianem Alexandera pewnego węzła:

\begin{proposition}
    \label{prp:alexander_hosokawa}
    Każdy wielomian Laurenta $p(t)$ o~całkowitych współczynnikach taki, że $p(1/t) = p(t)$ i~$p(1) = \pm 1$ jest wielomianem Alexandera pewnego węzła.
\end{proposition}

\begin{proof}[Niedowód]
    Hosokawa w \cite{hosokawa58} udowodnił to dla pomocniczego wielomianu splotów
    \begin{equation}
        \frac{\Delta(t, \ldots, t)}{(1-t)^{\max(0, \mu - 2)}},
    \end{equation}
    gdzie $\mu$ oznacza liczbę ogniw.
    Książka \cite{rolfsen76} Rolfsena na stronach 171-172 zawiera natomiast jawną konstrukcję węzła o~danym wielomianie Alexandera.
\end{proof}

Istnieje nieopisana przez nas odmiana wielomianu Alexandera, która liczy sobie tyle zmiennych, ile ogniw posiada splot.
Fakt \ref{prp:alexander_hosokawa} można częściowo uogólnić: Torres znalazł w~\cite{torres53} dwie geometryczne własności tych wielomianów, nazwane później warunkami Torresa.
\index{warunek!Torresa}
Nie są one warunkami wystarczającymi, jak odkrył ponad ćwierć wieku później Hillman \cite{hillman81}: wielomian
\begin{equation}
    D(x,y) = \frac{1}{1-xy} \cdot \left((1 - x^6y^6)(x - 1 + 1/x) - 2(1 - x^5y^5)(1 - x)(1 - y)\right)
\end{equation}
spełnia warunki Torresa, ale nie jest wielomianem Alexandera.

Wielomian Alexandera nie odróżnia luster i~rewersów od wyjściowych węzłów:

\begin{proposition}
    Niech $L$ będzie zorientowanym splotem.
    Wtedy $\alexander_{mL}(t) = \alexander_L(1/t) = \alexander_{rL}(t)$.
\end{proposition}

\begin{proof}
    Po odbiciu diagramu względem pionowej prostej skrzyżowanie z~definicji \ref{def:colouring_equation} również się odbija.
    Równanie związane z~nim zmienia się według schematu:
    \begin{equation}
        a + tc - ta - b = 0 \rightleftharpoons a + tb - ta - c = 0
    \end{equation}
    Pierwsze równanie z~$t$ zamienionym na $1/t$ staje się drugim równaniem przemnożonym przez $-1/t$.
    Dowód drugiej równości przebiega analogicznie.
\end{proof}

Nasz niezmiennik nie wykrywa niewęzła.
Na przykład $11_{471} = 11n_{34}$, $11_{473} = 11n_{42}$ albo $(-3, 5, 7)$-precel posiadają trywialny wielomian Alexandera, zjawisko to nie występuje wśród nietrywialnych węzłów o co najwyżej 10 skrzyżowaniach.

\begin{proposition}
    \label{prp:alexander_determinant}
    Niech $L$ będzie zorientowanym splotem.
    Wtedy $|\alexander_L(-1)| = \det L$.
\end{proposition}

\begin{proof}
    Wystarczy porównać definicję dla $\alexander_L$ (\ref{def:alexander_polynomial}) oraz $\det L$ (\ref{def:determinant}).
\end{proof}

\begin{proposition}
    \label{prp:alexander_multiplicative}
    Niech $K_1, K_2$ będą zorientowanymi węzłami.
    Wtedy
    \begin{equation}
        \alexander_{K_1 \shrap K_2}(t) \equiv \alexander_{K_1}(t) \alexander_{K_2}(t)
    \end{equation}
\end{proposition}

\begin{proof}
    Wybierzmy poniższe diagramy dla węzłów $K_1$ oraz $K$:
\begin{comment}
    \[\begin{tikzpicture}[baseline=-0.65ex, scale=0.07]
    %\useasboundingbox (-5, -5) rectangle (5,5);
    \begin{knot}[clip width=5, end tolerance=1pt]
        \strand[semithick] (-70, -10) rectangle (-30, 10);
        \strand[semithick] ( 30, -10) rectangle ( 70, 10);
        \strand[semithick,Latex-] (-30, 5) .. controls (-22, 5) and (-18, -5) .. (-10, -5);
        \strand[semithick] (-30,-5) .. controls (-22, -5) and (-18, 5) .. (-10,  5);
        \strand[semithick] (-10, 5) [in=up, out=right] to (-5, 0) [in=right, out=down] to (-10, -5);

        % prawe strzalki
        \strand[semithick,-Latex] (30, 5) .. controls (22, 5) and (18, -5) .. (10, -5);
        \strand[semithick] (30,-5) .. controls (22, -5) and (18, 5) .. (10,  5);
        \strand[semithick] (10, 5) [in=up, out=left] to (5, 0) [in=left, out=down] to (10, -5);

        \node[darkblue] at (-50,5) {$x_1,\ldots,x_{m-1}$};
        \node[red] at (-50,-5) {$1,\ldots,m$};

        \node[darkblue] at (50,5) {$y_1,\ldots,y_{n-1}$};
        \node[red] at (50,-5) {$1,\ldots,n$};

        \node[darkblue] at (-30,-5)[below right] {$x_m$};
        \node[darkblue] at (-15,-5)[below] {$x_0$};
        \node[darkblue] at (30,-5)[below left] {$y_n$};
        \node[darkblue] at (15,-5)[below] {$y_0$};
        \node[red] at ( 19.5,  1)[above]{$0$};
        \node[red] at (-19.5,  1)[above]{$0$};
    \end{knot}
    \end{tikzpicture}
\]
\end{comment}
    Niech $A$ oraz $B$ oznaczają macierze otrzymane z~wielomianowych równań kolorujących dla $K_1$ oraz $K_2$ przez skreślenie skrajnie lewej kolumny i~górnego wiersza.
    Wtedy $\alexander_{K_1}(t) = \det A$ oraz $\alexander_{K_2}(t) = \det B$.
    Poniższy diagram przedstawia sumę $K_1 \shrap K_2$:

\begin{comment}
\[\begin{tikzpicture}[baseline=-0.65ex, scale=0.07]
    %\useasboundingbox (-5, -5) rectangle (5,5);
    \begin{knot}[clip width=5, end tolerance=1pt]
        \strand[semithick] (-70, -10) rectangle (-30, 10);
        \strand[semithick] ( 30, -10) rectangle ( 70, 10);
        \strand[semithick,Latex-] (-30, 5) .. controls (-22, 5) and (-18, -5) .. (-10, -5);
        \strand[semithick] (-30,-5) .. controls (-22, -5) and (-18, 5) .. (-10,  5);

        % prawe strzalki
        \strand[semithick] (30, 5) .. controls (22, 5) and (18, -5) .. (10, -5);
        \strand[semithick] (30,-5) .. controls (22, -5) and (18, 5) .. (10,  5);
        \strand[semithick] (10, 5) to (-10, 5);
        \strand[semithick,-Latex] (10, -5) to (-10, -5);

        \node[darkblue] at (-50,5) {$x_1,\ldots,x_{m-1}$};
        \node[red] at (-50,-5) {$1,\ldots,m$};

        \node[darkblue] at (50,5) {$y_1,\ldots,y_{n-1}$};
        \node[red] at (50,-5) {$1,\ldots,n$};

        \node[darkblue] at (-30,-5)[below right] {$x_m$};
        \node[darkblue] at (0,-5)[below] {$x_0 = y_0$};
        \node[darkblue] at (0, 5)[above] {$z$};
        \node[darkblue] at (30,-5)[below left] {$y_n$};
        \node[red] at ( 19.5,  1)[above]{$\zeta$};
        \node[red] at (-19.5,  1)[above]{$0$};
    \end{knot}
    \end{tikzpicture}\]
\end{comment}

    Uporządkujmy łuki na diagramie jako $x_0 = y_0$, $x_1, \ldots, x_m$, $y_1, \ldots, y_n$, $z$; skrzyżowania: $0, 1, \ldots, m$ (z $K_1$), $1, \ldots, n$ (z $K_2$), $\zeta$.
    Wielomianowe równanie kolorujące dla $K_1 \shrap K_2$ nad skrzyżowaniami $1, \ldots, m$ ($1, \ldots, n$) są takie same, jak przed dodaniem do siebie węzłów.
    Nad skrzyżowaniem $\zeta$ równanie orzeka, że $(1-t)y_0+t z-y_n=0$.

    Wynika stąd, że $\alexander_{K_1 \shrap K_2}(t)$ jest wyznacznikiem macierzy
    \begin{align*}
        M &= \left(\begin{array}{cc|cc|c}
            & & & & \\
            \multicolumn{2}{c|}{\smash{\raisebox{.5\normalbaselineskip}{$A$}}} & & \\
            \hline \\[-\normalbaselineskip]
            & & & & \\
            & & \multicolumn{2}{c|}{\smash{\raisebox{.5\normalbaselineskip}{$B$}}}\\ \hline
            & & & -1 & t
    \end{array}\right)
    \end{align*}

    Skreśliliśmy lewą kolumnę oraz górny wiersz.
    Zatem $\alexander_{K_1 \shrap K_2}(t) = t^?\alexander_{K_1}(t) \alexander_{K_2}(t)$, jeśli nie pomyliliśmy się w~obliczeniach.
\end{proof}

\begin{proposition}
    Wielomian Alexandera zadaje ograniczenie na indeks skrzyżowaniowy $c$:
    \begin{equation}
        \deg \alexander_K(t) < c(K).
    \end{equation}
\end{proposition}

Być może istnieje bezpośredni dowód tej nierówności, ale jedyne uzasadnienie, jakie znam, opiera się na fakcie \ref{prp:alexander_genus} oraz wniosku \ref{cor:crossing_genus}.

\begin{proposition}
    Tylko skończenie wiele węzłów alternujących może mieć ten sam wielomian Alexandera.
\end{proposition}

\begin{proof}
    Załóżmy nie wprost, że istnieje nieskończony ciąg $K_n$ węzłów alternujących o~tym samym wielomianie Alexandera $\alexander_K(t)$.
    Wszystkie jego wyrazy mają ten sam wyznacznik, ponieważ $\det K_n = |\alexander_K(-1)|$.
    Z faktu \ref{prp:bankwitz} wynika, że indeks skrzyżowaniowy węzłów $K_n$ jest wspólnie ograniczony: $c_k \le \det K_n = \det K$.
    To prowadzi do sprzeczności: węzłów o~danym indeksie skrzyżowaniowym jest tylko skończenie wiele.
\end{proof}

Przedstawimy teraz kilka innych definicji, które prowadzą do tego samego niezmiennika.

\begin{definition}[relacja kłębiasta]
    Niech $L$ będzie zorientowanym splotem z ustalonym diagramem oraz skrzyżowaniem.
    Oznaczmy przez $L_+, L_-, L_0$ trzy diagramy splotów, które różnią się jedynie na małym obszarze wokół ustalonego skrzyżowania:
\begin{comment}
    \[
        \skeinplus \quad\quad\quad\quad
        \skeinminus \quad\quad\quad\quad
        \skeinzero
    \]
\end{comment}
    Mówimy, że niezmiennik zorientowanych splotów $f$ spełnia relację kłębiastą, jeżeli wartości $f(L_+)$, $f(L_-)$ i $f(L_0)$ są związane pewnym wielomianowym równaniem, niezależnie od wyboru splotu $L$.
\end{definition}

Termin ,,skein'' (kłąb) wprowadził Conway około roku 1970, kontynuując tradycję używania słów, które kojarzą się ze sznurkami.

\begin{definition}
    Niech $L$ będzie zorientowanym splotem.
    Wielomian Laurenta $\alexander_L(t) \in \Z[t^{\pm 1/2}]$, który spełnia relację kłębiastą
    \begin{equation}
        \alexander_{L_+}(t) - \alexander_{L_-}(t) - (t^{1/2} - t^{-1/2}) \alexander_{L_0}(t) = 0
    \end{equation}
    z warunkiem brzegowym $\alexander_{\LittleUnknot}(t) = 1$, nazywamy wielomianem Alexandera.
\end{definition}

Wzór ten, choć znany był Alexanderowi, nie zyskał przez wiele dekad uwagi matematyków.
Mogło tak być, gdyż w pracy \cite{alexander28} znalazł się on na samym końcu, pod nagłówkiem ,,twierdzenia różne''.
Na nowo odkrył go Conway: chcąc szybko liczyć wielomian Alexandera zaproponował, by reparametryzować go wzorem $\alexander(x^2) = \conway(x - 1/x)$.
Spełnia wtedy zależność
\begin{equation}
    \conway_{L_+}(x)- \conway_{L_-}(x) = x \conway_{L_0}(x).
\end{equation}

Relacja kłębiasta wystarcza do wyznaczenia $\alexander_L$ każdego splotu na mocy lematu \ref{lem:unknotting_well_defined}.
Dzięki niej wiemy też, że wielomian Alexandera nie odróżnia od siebie niesplotów.
Wady tej nie posiada wielomian Jonesa.

\begin{proposition}
    \label{prp:alexander_unlinks}
    Niech $L$ będzie splotem rozszczepialnym.
    Wtedy $\alexander_L(t) \equiv 0$.
\end{proposition}

\begin{proof}
    Skorzystamy z~relacji kłębiastej.
    Niech $L_0$ będzie splotem rozsczepialnym z~dwoma ogniwami.
    Wtedy węzły $L_+$ oraz $L_-$ powstałe przez dodanie skrzyżowania między ogniwami są tego samego typu, zatem
    \begin{equation}
        \alexander_{L_0} = \frac{\alexander_{L_+} - \alexander_{L_-}}{t^{1/2} - t^{-1/2}} = 0,
    \end{equation}
    a to chcieliśmy udowodnić.
\end{proof}

Implikacja w drugą stronę jest fałszywa.
Niech $\sigma_* = \sigma_{2} \sigma_{3}^{-2} \sigma_{2}$.
Domknięcie warkocza $\sigma_{1} \sigma_* \sigma_{1} \sigma_{3} \sigma_* \sigma_{1} \sigma_{3} \sigma_* \sigma_{3}$ nie jest rozszczepialne, ale jego wielomian Alexandera jest zerem.
\index{warkocz}
Warkocze poznamy w rozdziale piątym.

% Wielomian Jonesa (zdefiniowany w~kolejnej sekcji) spełnia podobną równość.
% Ich istnienie może nasunąć przypuszczenie, że dają się wspólnie uogólnić.
% Tak jest w~rzeczywistości -- mocniejszym niezmiennikiem okazał się wielomian HOMFLY-PT.

Murasugi podejrzewał, że w~przypadku węzłów alternujących ciąg współczynników jest unimodalny.
Dowód podano dla węzłów algebraicznych (Murasugi \cite{murasugi85}) oraz genusu dwa (Ozsvath i~Szabo w~\cite{ozsvath03}).
Hipoteza w~ogólnym przypadku pozostaje otwarta.
% Wiemy natomiast, że jeśli węzeł jest alternujący, to kolejne współczynniki wielomianu Conwaya $\alexander$ są przeciwnym znaków? (Murasugi, 242)

% Remark (M. Hutchings) There does exist a~categorification of the Alexander polynomial, or more precisely of ∆K(t)/(1 − t)2, where ∆K(t) denotes the (symmetrized) Alexander polynomial of the knot K. It is a~kind of Seiberg-Witten Floer homology of the three-manifold obtained by zero surgery on K.
% One can regard it as Z×Z/2Z graded, although in fact the column whose Euler characteristic gives the coefficient of tk is relatively Z/2kZ graded.

Kondo pokazał w \cite{kondo79}, że dla każdego węzła można znaleźć inny, 1-gordyjski węzeł o tym samym wielomianie Alexandera.
Wynika stąd, że nie ma związku między wielomianem $\Delta$ oraz liczbą gordyjską.

Na sam koniec pozostawiliśmy najstarszą definicję wielomianu Alexandera.
Niech $K$ będzie węzłem w~3-sferze, zaś $X$ nieskończonym nakryciem cyklicznym jego dopełnienia.
Można je otrzymać rozcinając dopełnienie wzdłuż powierzchni Seiferta.
Na $X$, a~przez to także na grupie homologii $H_1(X)$, działa automorfizm $t$, który czyni z~niej moduł nad pierścieniem $\Z[t, t^{-1}]$, i~to skończenie prezentowalny.
Jeśli posiada przedstawienie z~$r$ generatorami i~$s$ relacjami, gdzie $r \le s$, rozpatrzmy ideał generowany przez minory $r \times r$ macierzy prezentacji (jeśli nie, weźmy ideał zerowy).
Alexander pokazał, że ideał ten zawsze jest niezerowy i~główny.

\begin{tobedone}
    Definicja przy użyciu różniczek Foxa:
    https://math.berkeley.edu/~hutching/teach/215b-2004/yu.pdf
\end{tobedone}
