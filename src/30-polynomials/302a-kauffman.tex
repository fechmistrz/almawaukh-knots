
\subsection{Definicja kombinatoryczna -- klamra Kauffmana}
\index{klamra Kauffmana|(}
Klamra Kauffmana to wielomian Laurenta jednej zmiennej zdefiniowany w pracy \cite{kauffman87} z 1987 roku, oparty na ruchach Reidemeistera.
\index{człowiek!Kauffman, Louis}%
Dzięki swojej prostocie mógł być odkryty na początku XX wieku, nim jeszcze maszyneria teorii węzłów została rozwinięta.

Poszukujemy niezmiennika dla splotów o~kilku prostych własnościach.
Przede wszystkim żądamy, by niewęzłowi przypisany był wielomian $1$: $\bracket{\SmallUnknot} = 1$.
Po drugie chcemy wyznaczać nawiasy znając je dla prostszych splotów, co zapiszemy symbolicznie:
\begin{comment}
\begin{equation}
    \bracket{\MediumMinusCrossing} = A \bracket{\MediumAlphaSmoothing} + B \bracket{\MediumBetaSmoothing}
\end{equation}
\end{comment}
Zależy nam też na tym, by móc dodać do splotu trywialną składową: $\langle L \cup \SmallUnknot \rangle = C \langle L \rangle$.
Prosty rachunek pokazuje wpływ drugiego ruchu Reidemeistera na klamrę:
\begin{comment}
\begin{equation}
    \bracket{\MediumKauffmanReidemeisterTwoA}
    = (A^2 + ABC + B^2) \bracket{\MediumBetaSmoothing} + BA \bracket{\MediumAlphaSmoothing}
    \stackrel{?}{=} \bracket{\MediumAlphaSmoothing}.
\end{equation}
\end{comment}

Aby zachodziła ostatnia równość wystarczy przyjąć $B = A^{-1}$, co wymusza na nas wartość trzeciego parametru: $C = -A^2 - A^{-2}$.
W ten sposób odkryliśmy definicję.

\begin{definition}[klamra Kauffmana]
    \label{def:kauffman_bracket}
    Wielomian Laurenta $\bracket{D}$ dla diagramu splotu $D$ zmiennej $A$,
    który jest niezmienniczy ze względu na gładkie deformacje diagramu,
    a~przy tym spełnia trzy poniższe aksjomaty:
\begin{comment}
    \begin{align}
        \bracket{\MediumUnknot} & = 1
        \label{eqn:kauffman_axiom_1}%
        \\
        \bracket{\MediumMinusCrossing} & =
        A \bracket{\MediumAlphaSmoothing} +
        A^{-1} \bracket{\MediumBetaSmoothing}
        \label{eqn:kauffman_axiom_2}%
        \\
        \bracket{D \sqcup \MediumUnknot} & =
        (-A^{-2} - A^2) \bracket{D}
        \label{eqn:kauffman_axiom_3}%
    \end{align}
\end{comment}
    nazywamy klamrą Kauffmana.
\end{definition}

Drugi aksjomat jest wariacją na temat relacji kłębiastej.

\begin{lemma}
    Klamra Kauffmana każdego diagramu wyznacza się w~skończenie wielu krokach.
\end{lemma}

\begin{proof}
    Najprościej dowieść tego indukcyjnie, ze względu na liczbę skrzyżowań na diagramie splotu.
    Baza indukcji to przypadek zero skrzyżowań, czyli niesplotów.
    Zauważmy, że ostatni (i później pierwszy) aksjomat pozwala wyznaczyć wartość klamry Kauffmana dla każdego niesplotu w tylu krokach, ile ogniw ma niesplot.

    Pozostał krok indukcyjny.
    Załóżmy, że wyznaczyliśmy już wartości klamry dla każdego diagramu o $n$ skrzyżowaniach i chcemy ją obliczyć dla kolejnego splotu z diagramem o~$n + 1$ skrzyżowaniach.
    Pozwala na to drugi aksjomat, usuwający jedno ze skrzyżowań.
\end{proof}

Przedstawimy teraz wpływ ruchów Reidemeistera na nasz nowy wielomian.

\begin{lemma}
    Drugi i~trzeci ruch Reidemeistera nie ma wpływu na klamrę Kauffmana,
    pierwszy ruch zmienia ją zgodnie z~regułą:
\begin{comment}
    \begin{equation}
        \bracket{\MediumReidemeisterOneLeft} = -A^{-3} \bracket{\,\MediumReidemeisterOneStraight\,}.
    \end{equation}
\end{comment}
\end{lemma}

\begin{proof}
Pierwszy ruch Reidemeistera:
\begin{comment}
\begin{align}
    \bracket{\MediumReidemeisterOneLeft} & \stackrel{K2}{=} A
    \bracket{\MediumReidemeisterOneSmoothA} +
    A^{-1} \bracket{\MediumReidemeisterOneSmoothB} \\ & \stackrel{K3}{=}
    A \bracket{\MediumReidemeisterOneStraight} +
    A^{-1}(-A^{-2}-A^2) \bracket{\MediumReidemeisterOneStraight} =
    -A^{-3}\bracket{\MediumReidemeisterOneStraight}
\end{align}
\end{comment}

Dla drugiego ruchu:
\begin{comment}
\begin{align}
    \bracket{\MediumKauffmanReidemeisterTwoA} & \stackrel{K2}{=}
    A \bracket{\MediumKauffmanReidemeisterTwoB} +
    A^{-1} \bracket{\MediumKauffmanReidemeisterTwoC} \\ & \stackrel{K1}{=}
    -A^{-2} \bracket{\MediumBetaSmoothing} +
    A^{-1} \bracket{\MediumKauffmanReidemeisterTwoC} \\ & \stackrel{K2}{=}
    -A^{-2} \bracket{\MediumBetaSmoothing} +
    A^{-1}A \bracket{\MediumAlphaSmoothing} +
    A^{-1}A^{-1} \bracket{\MediumBetaSmoothing} \\ & =
    \bracket{\MediumAlphaSmoothing}
\end{align}
\end{comment}

Dla trzeciego ruchu:
\begin{comment}
\begin{align}
\bracket{\MediumKauffmanReidemeisterThreeA} & \stackrel{K2}{=}
A \bracket{\MediumKauffmanReidemeisterThreeB} +
A^{-1} \bracket{\MediumKauffmanReidemeisterThreeC} \stackrel{R2}{=}
A \bracket{\MediumKauffmanReidemeisterThreeD} +
A^{-1} \bracket{\MediumKauffmanReidemeisterThreeE} \\ & \stackrel{R2}{=}
A \bracket{\MediumKauffmanReidemeisterThreeFlippedB} +
A^{-1} \bracket{\MediumKauffmanReidemeisterThreeFlippedC} \stackrel{K2}{=}
\bracket{\MediumKauffmanReidemeisterThreeFlippedA},
\end{align}
\end{comment}
korzystaliśmy tu z~własności drugiego ruchu.
\end{proof}

\begin{corollary}
    Rozpiętość klamry Kauffmana jest niezmiennikiem węzłów.
\end{corollary}

Pierwszy ruch Reidemeistera jest jedynym, co powstrzymuje klamrę Kauffmana przed byciem niezmiennikiem węzłów.
Jeżeli przypomnimy sobie, że na mocy lematu \ref{lem:writhe_reidemeister} spin nie jest niezmiennikiem z tego samego powodu, odkryjemy ,,trik Kauffmana'': niedoskonałości tych dwóch obiektów znoszą się wzajemnie.
\index{trik Kauffmana}%
\index{spin}%

\begin{definition}
\label{def:jones_polynomial}%
    Niech $L$ będzie zorientowanym splotem.
    Wielomian Laurenta $\jones(L) \in \Z[t^{\pm 1/2}]$ określony przez
    \begin{equation}
        \jones(L)=\left[(-A)^{-3w(D)} \bracket{D}\right]_{t^{1/2}=A^{-2}},
    \end{equation}
    gdzie $D$ to dowolny diagram dla $L$, nazywamy wielomianem Jonesa.
\end{definition}

Sama klamra odegrała ważną rolę podczas unifikacji wielomianu Jonesa oraz innych niezmienników kwantowych.
W szczególności pozwoliła na uogólnienie go do niezmiennika 3-rozmaitości.

\begin{proposition}
    Wielomian Jonesa jest niezmiennikiem zorientowanych splotów.
\end{proposition}

\begin{proof}
    %Skorzystamy z~tego, że indeks zaczepienia jest niezmiennikiem.
    Pokażemy niezmienniczość wyrażenia $(-A)^{-3w(D)}\langle D\rangle$ na ruchy Reidemeistera.

    Niech
\begin{comment}
    \begin{equation}
        D_1 = \LargeReidemeisterOneLeft,
        \quad\quad\quad
        D_2 = \LargeReidemeisterOneStraight
    \end{equation}
\end{comment}
    Jak zauważyliśmy już wcześniej, II i III ruch nie zmienia ani spinu, ani klamry Kauffmana.
    Pozostało sprawdzić I ruch.
    Mamy:
    \begin{equation}
        (-A)^{-3 w\left(D_1\right)} \bracket{D_1} =
        (-A)^{-3 w\left(D_2\right) + 3} (-A)^{-3}\bracket{D_2} =
        (-A)^{-3 w\left(D_2\right)} \bracket{D_2},
    \end{equation}
    co kończy dowód.
\end{proof}

