
Wszystkie poznane dotąd niezmienniki (poza długością sznurową) przyjmowały całkowite wartości.
Teraz poszerzymy skrzynkę z~narzędziami o~klasyczne wielomiany Alexandera, Jonesa, HOMFLY; ale też późniejsze: BLM/Ho, Kauffmana oraz niezmienniki skończonego typu.
Co ciekawe, wielomiany te wywodzą się z~różnych działów matematyki: wielomian $\alexander$ Alexandera z~homologii pewnej przestrzeni nakrywającej, $\jones$ Jonesa: z~algebr von Neumanna.
HOMFLY (albo raczej HOMFLY-PT) to ich naturalne uogólnienie.

Atrakcyjnym wprowadzeniem jest przygotowana przez matematyków niemieckich (a~przez to dostępna tylko w~ich języku) praca \cite{gellert09}.
Pierwotnymi artykułami były \cite{alexander28}, \cite{jones85} oraz \cite{homfly85}, wszystkie należą do przełomowych w~kombinatorycznej teorii węzłów.

