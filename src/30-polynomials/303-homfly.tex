\section{Wielomian HOMFLY}
\index{wielomian!HOMFLY|(}

Po tym, jak Jones przedstawił światu swój wielomian w~1984 roku, matematycy
zaczęli szukać jego uogólnienia zależnego nie od jednej, lecz dwóch zmiennych.
Pierwszym takim węzłowym niezmiennikiem okazał się wielomian HOMFLY.
Jego nazwa pochodzi od sześciu odkrywców: Hoste, Ocneanu, Millett, Freyd, Lickorish, Yetter z~pracy \cite{homfly85}.

Dwa lata później i~niezależnie od nich, Przytycki z~Traczykiem otrzymał ten sam obiekt w~\cite{przytycki87}.
Ich nazwiska czasami są pomijane, gdyż polska poczta była spowolniona w tamtym czasie przez ruch solidarnościowy, pisze o tym Michael McDaniel w artykule ,,Knots to You''.
W przeciwnym razie używa się skrótu HOMFLY-PT.

% To compute the HOMFLY-PT polynomial, one starts from an oriented link diagram? and uses the following rules:

% PP is an isotopy invariant (thus, unchanged by Reidemeister moves).

% P(unknot)=1P(\text{unknot}) = 1

% Let L +L_+, L −L_-, and L 0L_0 be links which are the same except for one part where they differ according to the diagrams below. Then, depending on the choice of variables:
% l⋅P(L +)+l −1⋅P(L −)+m⋅P(L 0)=0l \cdot P(L_+) + l^{-1} \cdot P(L_-) + m \cdot P(L_0) = 0.
% a⋅P(L +)−a −1⋅P(L −)=z⋅P(L 0)a \cdot P(L_+) - a^{-1} \cdot P(L_-) = z \cdot P(L_0). (Sometimes ν\nu is used instead of aa)
% α −1⋅P(L +)−α⋅P(L −)=z⋅P(L 0)\alpha^{-1} \cdot P(L_+) - \alpha \cdot P(L_-) = z \cdot P(L_0).
% Using three variables: x⋅P(L +)+y⋅P(L −)+z⋅P(L 0)=0x \cdot P(L_+) + y \cdot P(L_-) + z \cdot P(L_0) = 0.

\begin{definition}[wielomian HOMFLY]
\index{relacja kłębiasta}%
\label{def:homfly}%
    Niech $L$ będzie zorientowanym diagramem splotu.
    Wielomian Laurenta $P$ dwóch zmiennych, który spełnia relację kłębiastą
    \begin{equation}
        l \cdot P_{L_+} + l^{-1} \cdot P_{L_-} + m \cdot P_{L_0} = 0
    \end{equation}
    z warunkiem brzegowym $P({\SmallUnknot}) \equiv 1$ oraz jest niezmienniczy na izotopie i~trzy ruchy Reidemeistera, nazywamy wielomianem HOMFLY.
\end{definition}

Litery $l, m$ pochodzą od nazwisk odkrywców, Lickorisha i Milletta.
To dość mylące, ale niektórzy autorzy używają innej, równoważnej relacji z~innymi zmiennymi.

\begin{definition}[HOMFLY inaczej]
    $a \cdot P_{L_+} - a^{-1} \cdot P_{L_-} = z \cdot P_{L_0}$.
\end{definition}

Czasami zamiast $a$ pisze się $v$.

\begin{definition}[HOMFLY jeszcze inaczej]
    $\alpha^{-1} \cdot P_{L_+} - \alpha \cdot P_{L_-} = z \cdot P_{L_0}$.
\end{definition}

\begin{definition}[HOMFLY jednorodny]
    $x \cdot S_{L_+} - y \cdot S_{L_-} - z \cdot S_{L_0} = 0$.
\end{definition}

Jednorodny wielomian HOMFLY można znaleźć na przykład w~\cite[s. 10]{millett88} albo \cite[s. 231]{murasugi96}.
Murasugi podaje tam jako ćwiczenie równości:
\begin{align}
    P(\alpha, z) & = S(\alpha^{-1}, \alpha, z) \\
    S(x, y, z) & = P(\sqrt{y/x}, z/\sqrt{xy}).
\end{align}

% Jednorodny wielomian HOMFLY niesie taką samą informację jak jego wersja o~dwóch zmiennych.
% Istotnie, jeśli podstawimy $z = 1$, dostaniemy wielomian dwóch zmiennych.
% W~drugą stronę wystarczy każdy wyraz przemnożyć przez pewną potęgę $z$, by ten był stopnia zero.
% Wygodniej jest jednak nałożyć inny warunek, $xy=1$, co prowadzi do definicji \ref{def:homfly}.
% Ja to pisałem chyba z jakiejś polskiej książki, której nie mogę już znaleźć...

Relacja kłębiasta pozwala wywnioskować własności wielomianu sumy spójnej i~prostej.
Potrzebować będziemy lematu:

\begin{lemma}
    \label{lem:homfly_unlinks}
    Wielomianem HOMFLY dla niesplotu o~$n$ składowych jest
    \begin{equation}
        P(U_n) = \left(-\frac{l+1/l}{m}\right)^{n-1}.
    \end{equation}
\end{lemma}

Można potraktować to jako proste ćwiczenie z~indukcji matematycznej, pozostawiam je uwadze Czytelnika.

\begin{proposition}
    Niech $L_1, L_2$ będą splotami.
    Wtedy
    \begin{align}
        P(L_1 \sqcup L_2) & = - \frac{l + 1/l}{m} \cdot P(L_1) P(L_2) \\
        P(L_1 \shrap L_2) & = P(L_1) P(L_2).
    \end{align}
\end{proposition}

Dowód przebiega analogicznie do dowodu ,,multiplikatywności'' wielomianów Alexandera (fakty \ref{prp:alexander_multiplicative}) oraz Jonesa (fakt \ref{prp:jones_multiplicative_1}, \ref{prp:jones_multiplicative_2}).
Patrz \url{https://math.stackexchange.com/a/399405}.

\begin{proof}
    Każdy splot $L$ jest kombinacją liniową (o współczynnikach będących wielomianami w~$m$ oraz $l$) niesplotów $U_k$ o~różnej liczbie składowych $k$.
    Mamy więc
    \begin{equation}
        L_1 = \sum_{k=1}^n a_k U_k, \quad
        L_2 = \sum_{k=1}^n b_k U_k.
    \end{equation}

    Skorzystamy w~tym miejscu z~lematu \ref{lem:homfly_unlinks}.
    Wynika z~niego, że $P(U_i)P(U_j) = P(U_{i+j-1})$ i~bezpośrednio
    \begin{equation}
        P(L_1)P(L_2) = \sum_{k=1}^{2n} \sum_{i=1}^{k-1} a_i b_{k-i}P(U_{k-1}).
    \end{equation}

    Pozostało spojrzeć na sumę spójną diagramów  $L_1 \shrap L_2$.
    Jeśli usuniemy teraz wszystkie skrzyżowania z~diagramu $L_1$ relacją kłębiastą, dostaniemy $L_1 \# L_2 = \sum_{k=1}^n a_k (U_{k-1} \cup L_2)$, gdyż jedna z~niezawęźlonych składowych zostanie wchłonięta do diagramu diagramu $L_2$.
    Rozwijamy dalej i~otrzymujemy
    \begin{align}
        L_1 \# L_2
        & = \sum_{k=1}^n a_k \left(U_{k-1} \cup \sum_{k=1}^n b_k U_k\right) \\
        & = \sum_{k=1}^{2n} \sum_{i=1}^{k-1} a_i b_{k-i} U_{i-1} \cup U_{k-i} \\
        & = \sum_{k=1}^{2n} \sum_{i=1}^{k-1} a_i b_{k-i} U_{k-1},
    \end{align}
    co kończy dowód.
\end{proof}

Nie wiemy jeszcze, czy definicja \ref{def:homfly} pozwala wyliczyć wielomian w~skończenie wielu krokach. %, ani czy wynik jest jednoznaczny.
Zauważmy, że wyznaczenie nawiasu Kauffmana było prostsze, ponieważ w~każdym kroku liczba skrzyżowań ulegała zmniejszeniu.
Teraz powołamy się na lemat \ref{lem:unknotting_well_defined}.
% istnienie trudno pokazać, jednoznaczność przebiega tak samo jak dla wielomianu Jonesa

\begin{proposition}
    Wielomian HOMFLY można wyznaczyć w~skończenie wielu krokach.
\end{proposition}

\begin{proof}
    Niech $L$ będzie splotem, którego wielomian HOMFLY próbujemy wyznaczyć.
    Ustalmy jego dowolny diagram i~wybierzmy jedno ze skrzyżowań, które należy odwrócić, by uzyskać niesplot.
    Takie skrzyżowanie istnieje na mocy lematu \ref{lem:unknotting_well_defined}.

    Początkowy diagram odpowiada $L_+$ lub $L_-$, relacja kłębiasta pozwala na uzyskanie wielomianu wyjściowego splotu zależnego od wielomianu splotu z
    diagramem, na którym jest mniej skrzyżowań oraz splotu, który jest ,,jedno skrzyżowanie bliżej'' niesplotu.

    Powtarzając tę procedurę dojdziemy w~pewnym momencie do splotów trywialnych, gdzie powołujemy się na wniosek \ref{lem:homfly_unlinks}.
\end{proof}

Porównajmy wielomian HOMFLY ze znanymi już wielomianami Alexandera $\alexander$ oraz Jonesa $\jones$.
Żaden z~nich nie jest mocniejszy od drugiego, istnieją pary węzłów rozróżnialne przez dokładnie jeden.
Na przykład $4_1$ oraz $11n_{19}$ mają różne tylko wielomiany Alexandera, zaś $8_{10}$ oraz $10_{143}$ różne tylko wielomiany Jonesa.

\begin{proposition}
\label{homfly_stronger}
    Wielomian HOMFLY jest mocniejszy niż wielomiany Jonesa i~Alexandera jednocześnie.
\end{proposition}

\begin{proof}
    Podstawiając $l = it^{-1}$, $m = i(t^{-1/2} - t^{1/2})$ w~definicji dostajemy relację kłębiastą dla wielomianu Jonesa.
    Wielomian Alexandera otrzymamy kładąc $l = i$, $m = i(t^{-1/2} - t^{1/2})$.
    Stąd wynika, że wielomian HOMFLY jest co najmniej tak samo mocny jak $\alexander$ oraz $\jones$.

    Wielomian HOMFLY odróżnia $11_{388}$ od swojego odbicia, wielomiany Jonesa i~Conwaya nie, więc jest istotnie mocniejszy.
\end{proof}

Wielomian HOMFLY jest niezmiennikiem splotów zorientowanych, ale możemy go użyć także do porównywania obiektów bez orientacji.
Jeśli wielomiany dwóch splotów, niezależnie od orientacji ich ogniw, są zawsze różne, to same sploty też są różne.
W~\cite{dunfield10} znaleziono diagram pewnego węzła o~75 skrzyżowaniach, którego kablowe mutacje zachowują wielomiany Jonesa oraz Alexandera, ale nie zachowują wielomianu HOMFLY.
Wiadomo, że jeśli istnieje prostszy przykład, to ma co najmniej 13 skrzyżowań.

Pomimo swej mocy nie jest jednak niezmiennikiem zupełnym, nigdy nie odróżnia od siebie mutantów (to fakt \ref{mutants_and_homfly}).
Konkretne przykłady kolizji, tym razem nie mutantów, to $5_1$ oraz $10_{132}$, $8_{8}$ oraz $10_{129}$, $8_{16}$ oraz $10_{156}$ albo $10_{25}$ oraz $10_{56}$.
Co bardziej dramatyczne, Kanenobu wskazał przy użyciu elementarnych metod (!) w~pracy \cite{kanenobu86} z~1986:

\begin{proposition}
    \index{genus}
    \index{liczba mostowa}
    \index{węzeł!hiperboliczny}
    \index{węzeł!taśmowy}
    \index{węzeł!włóknisty}
    Istnieje przeliczalna rodzina węzłów, której wszystkie elementy są hiperboliczne, włókniste, taśmowe, trzymostowe, o~genusie 2 oraz nieodróżnialne od siebie wielomianem HOMFLY.
\end{proposition}

\begin{proof}[Niedowód]
    Rozpatrzmy rodzinę węzłów $K(p, q)$, gdzie całkowite liczby $p, q$ oznaczają ile razy wykonano pełny skręt wewnątrz prostokąta:
\begin{comment}
    \[
        \begin{tikzpicture}[baseline=-0.65ex,scale=0.08]
        \begin{knot}[clip width=5, flip crossing/.list={1,3,6,7}]
        \strand [thick] (0, 25)
            [in=left, out=right] to (20, 25)
            [in=up, out=right] to (25, 5)
            [in=down, out=down] to (15, 5)
            [in=right, out=up] to (10, 20)
            [in=up, out=left] to (5, 15)
            [in=up, out=down] to (5, 5)
            [in=left, out=down] to (15, -10)
        ;
        \strand [thick] (15, -10)
            [in=down, out=right] to (20, 10)
            [in=up, out=up] to (30, 10)
            [in=right, out=down] to (20, -25)
            [in=down, out=left] to (5, -20)
            [in=down, out=up] to (5, -10)
        ;
        \strand [thick] (5, -10)
            [in=down, out=up] to (15, 0)
            [in=right, out=up] to (0, 5)
            [in=up, out=left] to (-15, 0)
            [in=up, out=down] to (-5, -10)
            ;
        \strand [thick] (-15, -10)
            [in=down, out=left] to (-20, 10)
            [in=up, out=up] to (-30, 10)
            [in=left, out=down] to (-20, -25)
            [in=down, out=right] to (-5, -20)
            [in=down, out=up] to (-5, -10)
        ;
        \strand [thick] (0, 25)
            [in=right, out=left] to (-20, 25)
            [in=up, out=left] to (-25, 5)
            [in=down, out=down] to (-15, 5)
            [in=left, out=up] to (-10, 20)
            [in=up, out=right] to (-5, 15)
            [in=up, out=down] to (-5, 5)
            [in=right, out=down] to (-15, -10)
        ;
        \draw[fill=blue!10!white,thick] (-5, 10) rectangle (5, 15);
        \draw[fill=blue!10!white,thick] (-5, -17.5) rectangle (5, -12.5);
        \node at (0, 12.5) {$p$};
        \node at (0, -15) {$q$};
        \end{knot}
        \end{tikzpicture}
    \]
\end{comment}
    I tak zachowując oznaczenia z pracy \cite{kanenobu86} prawdą jest, że:
    \begin{enumerate}
        \item $K(p,q) \approx K(q,p)$ (lemat 1),
        \item węzły $K(p,q)$ i $K(p', q')$ mają ten sam moduł Alexandera wtedy i tylko wtedy, gdy $|p-q| = |p'-q'|$ (lemat 2),
        \item węzły $K(p,q)$ i $K(p', q')$ mają ten sam wielomian Jonesa (HOMFLY) wtedy i tylko wtedy, gdy $p+q = p'+q'$ (lemat 3).
    \end{enumerate}
    Wynika stąd, że węzły $K(p, q)$, gdzie $p \ge q$, są parami różne.
    Dalej, węzeł $K(p,q)$ jest amfichiralny wtedy i tylko wtedy, gdy $p + q = 0$, zawsze trzymostowy (lemat 4) i hiperboliczny dokładnie wtedy, gdy $(p, q) \neq (0, 0)$.
    Autor twierdzi, że dowód włóknistości, taśmowości i~bycia genusu $2$ zamieścił wcześniej w~\cite{kanenobu81}, ale to jest po francusku.
\end{proof}

Zaletą wielomianu HOMFLY jest to, że często wykrywa chiralność (węzeł chiralny nie jest równoważny swemu lustrzanemu odbiciu), ale nie odróżnia enancjomerów węzłów $9_{42}$, $10_{48}$, $10_{71}$, $10_{91}$, $10_{104}$, oraz $10_{125}$.
Wśród węzłów do 10 skrzyżowań dokładnie dwa opierają się testom chiralności opartym na niezmiennikach HOMFLY oraz Kauffmana: $9_{42}$ oraz $10_{71}$.
Do jej wykrycia potrzebna jest na przykład $SU(2)$-teoria Cherna-Simonsa, wyjaśniona w~kolejnej wersji tego skryptu.

% A POLYNOMIAL INVARIANT OF ORIENTED LINKS W. B. R. LICKORISH and KENNETH C. MILLET
% Example 16 - książka

% TODO: Dla każdego splotu $L$ o~$k$ składowych, $P_L(x,y) - 1$ jest krotnością $x+y-1$, zaś $P_L(x,y) + (-1)^K$ jest krotnością $x +y + 1$, zatem wielomian HOMFLY nigdy nie jest zerem. Uwaga: to jest inna parametryzacja!

\index{wielomian!HOMFLY|)}

% Koniec sekcji Wielomian HOMFLY
