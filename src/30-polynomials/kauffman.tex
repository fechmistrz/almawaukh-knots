\section{Wielomian Kauffmana} % (fold)

Mniej więcej w~tym samym czasie, gdy odkryto wielomian BLM/Ho, Kauffman zaproponował sposób, dzięki któremu ten wielomian można uogólnić do odróżniającego lustra.
% Dopuszczał stosowanie tylko drugiego i~trzeciego ruchu Reidemeistera.
Wielomianu Kauffmana nie należy mylić z~klamrą Kauffmana!

\begin{definition}[wielomian Kauffmana]
    \index{wielomian!Kauffmana}
    Niech $K$ będzie zorientowanym splotem, zaś $D$ ustalonym diagramem o~spinie $\writhe D$.
    Istnieje wielomian $L(K)$ wyznaczony przez relację kłębiastą
    \begin{equation}
        L(\LittleLeftCrossing) + L(\LittleRightCrossing) = z L(\LittleLeftSmoothing) + z L(\LittleRightSmoothing)
    \end{equation}
    z warunkiem brzegowym $L(\LittleUnknot) = 1$, niezmienniczy względem II i III ruchu Reidemeistera oraz taki, że $L(s) = a^{-1} L(s_r) = a L(s_l)$, gdzie $s$ jest pojedynczym węzłem, zaś $s_r, s_l$ to jego obrazy względem dwóch wariantów I ruchu Reidemeistera.
    Wtedy wielomian dwóch zmiennych
    \begin{equation}
        % F_L(a, z) = a^{-w(L)} \langle |L| \rangle,
        F_K(a, z) = a^{-\writhe D} L(D),
    \end{equation}
    nazywamy wielomianem Kauffmana.
\end{definition}

\begin{tobedone}
$s_r, s_l$: patrz Stoimenow, Tabulating and distinguishing mutants.
\end{tobedone}

Jego związki z~wielomianem HOMFLY pozostają nieznane.
Wiemy natomiast, że

\begin{proposition}
    Wielomian Kauffmana uogólnia wielomian BLM/Ho, zgodnie z podstawieniem
    \begin{equation}
        Q(x) = F(1, x).
    \end{equation}
\end{proposition}

\begin{proposition}
    Wielomian Kauffmana uogólnia wielomian Jonesa, zgodnie z podstawieniem
    \begin{equation}
        \jones(t)=F(-t^{-3/4},t^{-1/4}+t^{1/4}).
    \end{equation}
\end{proposition}

% Koniec sekcji Wielomian Kauffmana
