\section{Wielomian Kauffmana} % (fold)
\label{sec:kauffman_polynomial}
Mniej więcej w~tym samym czasie, gdy odkryto wielomian BLM/Ho, Kauffman zaproponował sposób, dzięki któremu ten wielomian można uogólnić do odróżniającego lustra.
Dopuszczał stosowanie tylko drugiego i~trzeciego ruchu Reidemeistera.

\index{wielomian!Kauffmana}
Wielomianu Kauffmana nie należy mylić z~nawiasem Kauffmana!
Jest to półzorientowany wielomian dwóch zmiennych dany wzorem
\[
	F_L(a, z) = a^{-w(L)} \langle |L| \rangle,
\]
\todo[inline]{Co oznacza $z$?}
gdzie $w$ to writhe zorientowanego diagramu splotu, zaś $|L|$ to $L$ bez orientacji.
Stanowi uogólnienie wielomianu BLM/Ho: $F(1, x) = Q(x)$ oraz wielomianu Jonesa:
\[
	V(t)=F(-t^{-3/4},t^{-1/4}+t^{1/4}).
\]

Jego związki z~wielomianem HOMFLY pozostają nieznane.
% Koniec sekcji Wielomian Kauffmana
