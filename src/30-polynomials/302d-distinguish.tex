
\subsection{Odróżnianie węzłów i splotów wielomianem Jonesa}
Wielomian Jonesa często (chociaż nie zawsze) odróżnia od siebie sploty lepiej niż wielomian Alexandera.
Na przykład: wielomian Alexandera wszystkich splotów rozszczepialnych jest taki sam (stwierdzenie \ref{prp:alexander_unlinks}), więc nie odróżnia wcale niesplotów.
Dla porównania, wielomian Jonesa odróżnia je wszystkie:

\begin{proposition}
\label{prp:jones_trivial_link}%
    Wielomianem Jonesa splotu trywialnego o $n$ ogniwach jest
    \begin{equation}
        \jones(K_n) = \left(-\sqrt{t} - \frac{1}{\sqrt {t}}\right)^{n-1}.
    \end{equation}
\end{proposition}

Co więcej, wielomian Jonesa odróżnia od siebie dowolne dwa węzły pierwsze o~co najwyżej 9 skrzyżowaniach.
Dalej występują już kolizje, oto pełna ich lista do 10 skrzyżowań:
$5_{1}$ -- $10_{132}$,
$8_{8}$ -- $10_{129}$,
$8_{16}$ -- $10_{156}$,
$10_{22}$ -- $10_{35}$,
$10_{25}$ -- $10_{56}$,
$10_{40}$ -- $10_{103}$,
$10_{41}$ -- $10_{94}$,
$10_{43}$ -- $10_{91}$,
$10_{59}$ -- $10_{106}$,
$10_{60}$ -- $10_{86}$,
$10_{71}$ -- $10_{104}$,
$10_{73}$ -- $10_{83}$,
$10_{81}$ -- $10_{109}$,
$10_{137}$ -- $10_{155}$.
Jones wiedział, że wielomianowe niezmienniki nie radzą sobie z~odróżnianiem od siebie mutantów, dlatego zapytał, czy jego wielomian wykrywa niewęzły.
Pozostaje to otwartym problemem do dziś.

\begin{conjecture}
\index{hipoteza!o wielomianie Jonesa i niewęźle}%
\label{con:jones}%
    Niech $K$ będzie węzłem.
    Jeśli $\jones_K(t) \equiv 1$, to $K$ jest niewęzłem.
\end{conjecture}

Hipotezę zweryfikowano komputerowo dla węzłów o~małej liczbie skrzyżowań.
W latach dziewięćdziesiątych Hoste, Thistlethwaite, Weeks zrobili to przy okazji tablicowania węzłów spełniających $\crossing K \le 16$.
Wynik poprawiano: Dasbach, Hougardy w~1997 do $\crossing K \le 17$; Yamada w~2000 do $\crossing K \le 18$; wreszcie Tuzun, Sikora w~2016 do $\crossing K \le 22$, potem w~2020 do $\crossing K \le 24$.
Patrz kolejno \cite{thistlethwaite98}, \cite{hougardy97}, \cite{yamada00}, \cite{tuzun18}, \cite{tuzun21}, ale też \cite[s. 381]{ohtsuki02}.
\index[persons]{Dasbach, Oliver}%
\index[persons]{Hoste, Jim}%
\index[persons]{Hougardy, Stefan}%
\index[persons]{Sikora, Adam}%
\index[persons]{Thistlethwaite, Morwen}%
\index[persons]{Tuzun, Robert}%
\index[persons]{Weeks, Jeff}%
\index[persons]{Yamada, Shuji}%

% TODO: Argumentem przemawiającym za prawdziwością hipotezy jest twierdzenie ,,udowodnione'' przez Jørgena Andersena.
% TODO: \textbf{NIE Pokazał on, że rodzina okablowanych wielomianów Jonesa wykrywa niewęzeł.}
% TODO: Tutaj $n$-okablowanie węzła $K$ to $n$-komponentowy splot $K^n$, który powstaje z~$K$ po zamianie pojedynczej ,,żyły'' na $n$ równoległych żył.

Istnieją sploty o~trywialnym wielomianie Jonesa.
Thistlethwaite wskazał dwa z~dwoma oraz jeden z~trzema ogniwami w~\cite{thistlethwaite01}.
Jest ich nawet nieskończenie wiele, jak Eliahou, Kauffman i~Thistlethwaite pokazali w~pracy \cite{eliahou03}.

\begin{proposition}
\index{splot!Hopfa}%
    Niech $k \ge 2$ będzie liczbą naturalną.
    Istnieje nieskończenie wiele splotów pierwszych z $k$ ogniwami, których wielomian Jonesa nie odróżnia od niesplotu z $k$ ogniwami.

    Co więcej, można wymagać, by wszystkie te sploty były satelitami splotu Hopfa.
\end{proposition}

