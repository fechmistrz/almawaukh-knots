\subsection{Oryginalna praca Jonesa} % (fold)
\label{sub:jones_paper}
Jones otrzymał swój wielomian jako efekt uboczny badań nad algebrami operatorowymi: wziął ślad pewnej reprezentacji warkoczy w~algebrę, która miała ważne znaczenie w~mechanice statystycznej.

% Reszta rozdziału powstała częściowo w~oparciu o~notatki autorstwa Andrew Bergera oraz Chrisa Geriga\footnote{dostępne pod adresem \url{https://math.berkeley.edu/~cgerig/notes}}.
% Połączymy warkocze z~wielomianem Jonesa i~zobaczymy, jak definiowano ten ostatni przed Kauffmanem.

\begin{definition}[algebra Temperleya-Lieba]
    Niech $R$ będzie przemiennym pierścieniem, w~którym ustalono element $\delta \in R$.
    Algebrą Temperleya-Lieba nazywamy $R$-algebrę $TL_n(\delta)$ generowaną przez elementy $e_1, \ldots, e_{n-1}$, które związane są relacjami (dla $|i-j| \ge 2$):
    \begin{align*}
        e_i^2 & = \delta e_i \\
        e_i e_{i \pm 1} e_i & = e_i \\
        e_i e_j & = e_j e_i
    \end{align*}
    % Algebra Temperleya-Lieba $A_n$ to wolna addytywna algebra na multiplikatywnych generatorach $e_1, \ldots, e_{n-1}$ traktowana jako $\C[\tau, \tau^{-1}]$-moduł.
    % Zmienna $\tau$ komutuje ze wszystkimi generatorami, generatory zaś spełniają relacje ($j$ jest różne od $i - 1, i, i+1$):
\end{definition}

$TL_n(\delta)$ można przedstawić przy użyciu diagramów: prostokątów, których przeciwległe boki zawierają po $n$ punktów połączonych w~pary tak, by uniknąć samoprzecięć. Mnożenie elementów algebry odpowiada sklejaniu dwóch diagramów, przy czym każdą zamkniętą pętlę zamieniamy na dodatkowy czynnik $\delta$.
(To w~gruncie rzeczy są warkocze).

\begin{definition}[ślad Markowa]
    Niech $K \in TL_n(\delta)$ będzie elementem algebry Temperleya-Lieba będącym iloczynem generatorów\footnote{Czyli element $K$ utożsamia się z~pewnym warkoczem o~$n$ pasmach.} $e_1, \ldots, e_{n-1}$, którego domknięcie rozpada się na $m$ składowych spójności.
    Śladem Markowa elementu $K$ nazywamy wielkość $\operatorname{tr} K = \delta^{m-n}$.
\end{definition}

Ustalmy splot $L$.
Na mocy twierdzenia Alexandera \ref{alex_thm}, splot $L$ jest domknięciem pewnego warkocza o~$n$ pasmach.
Zdefiniujmy reprezentację $\rho$ grupy warkoczy $B_n$ w~algebrę $TL_n$ o~współczynnikach z~pierścienia $\Z[A, 1/A]$ (gdzie $\delta = -A^2 - A^{-2}$): standardowy generator warkoczowy $\sigma_i$ poślijmy na $A \cdot e_i + 1/A \cdot 1$.
Wtedy $\langle K \rangle = \delta^{n-1} \operatorname{tr} \rho (\sigma)$ jest klamrą Kauffmana!

Sploty nie przedstawiają się jednak jako domknięcia warkoczy jednoznacznie, musimy wziąć pod uwagę wpływ ruchów Markowa na złożenie $\operatorname{tr} \circ \rho$.
Pozostawimy uwadze Czytelnika pokazanie, iż złożenie to jest rzeczywiście niezmiennicze.
% By tak było, od funkcji $\operatorname{tr}$ żąda się, żeby spełniała zależność  $\operatorname{tr}(we_i) = \tau \cdot \operatorname{tr}(w)$, jeśli tylko $w \in A_{i-1}$.

Zaletą tego podejścia jest możliwość wyboru algebry, która reprezentuje grupę warkoczy.
% Koniec podsekcji Oryginalna praca Jonesa
