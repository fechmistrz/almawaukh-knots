
\subsection{Niezmienniki, które są skończonego typu}

Polyak, Viro \cite{polyak01} piszą, że jedyny niezmiennik Wasiljewa rzędu 2 to
\index[persons]{Polyak, Michael}%
\index[persons]{Viro, Oleg}%
\begin{equation}
    v_2 = \frac 12 \alexander''(1)
\end{equation}
i~że występuje pod nazwą ,,niezmiennik węzłów Cassona''.
\index{niezmiennik Cassona}%
Modulo $2$ to niezmiennik Arfa.
\index{niezmiennik Arfa}%

\begin{example}
\index{wielomian!Conwaya}%
    Niech $K$ będzie węzłem, zaś $\conway_K(t) = \sum_k \conway_{2k} z^{2k}$ jego wielomianem Conwaya.
    Współczynnik $\conway_{2k}$ indukuje niezmiennik Wasiljewa rzędu dokładnie $2k$.
\end{example}

\begin{proof}
\index[persons]{Czmutow, Siergiej}%
\index[persons]{Bar-Natan, Dror}%
    Jak wspomina Czmutow w \cite{chmutov12}, porównanie relacji kłębiastej dla wielomianu Conwaya z tą dla niezmienników Wasiljewa pokazuje, że wielomian Conwaya osobliwego węzła o~$m$ punktach podwójnych jest podzielny przez $z^m$, co dowodzi już, że $c_{2k}$ jest niezmiennikiem rzędu co najwyżej $2k$.

    Bar-Natan\footnote{Weights of Feynman diagrams and the Vassiliev knot invariants} w 1991 pokazał, że $\conway_{2k}$ jest niezmiennikiem rzędu dokładnie $2k$.
    % TODO: zamienić na prawdziwe cytowanie
\end{proof}

Lin oraz Wang \cite{wang96} w~1994 roku na podstawie niezmienników małych rzędów, to jest $v_2$ oraz $v_3$, wysunęli następującą hipotezę: istnieje uniwersalna stała $C$ taka, że
\index[persons]{Lin, Xiao-Song}%
\index[persons]{Wang, Zhenghan}%
\index{hipoteza!Lin-Wanga}%
\begin{equation}
    |v_k(K)| \le C (\crossing K)^k.
\end{equation}

Hipotezę wkrótce udowodniono, najpierw dla węzłów (Bar-Natan, \cite{barnatan95}), nieco później także dla splotów (Stojmenow, \cite{stoimenow001}).
\index[persons]{Bar-Natan, Dror}%
\index[persons]{Stojmenow, Aleksander}%
Wartość stałej $C$ trudno obliczyć, dlatego Stojmenow \cite[problem 1.17]{ohtsuki02} zaproponował ograniczenie się do przypadku $v_k = \conway_k$.

\begin{conjecture}
    Niech $L$ będzie splotem.
    Wtedy
    \begin{equation}
        |\conway_k(L)| \le \frac{(\crossing L)^k}{2^kk!}.
    \end{equation}
\end{conjecture}

Nierówność jest nietrywialna tylko dla splotu $L$ z~$k+1, k-1, \ldots$ składowymi; trywialna dla $k = 0$, łatwa dla $k=1$ (wtedy $\conway_1$ jest indeksem zaczepienia splotów o~dwóch składowych) oraz udowodniona dla węzłów i~$k=2$ przez Polyaka, Viro w~2001 (\cite{polyak01}).
\index[persons]{Polyak, Michael}%
\index[persons]{Viro, Oleg}%

\begin{example}
\index{wielomian!Jonesa}%
    Niech $\jones_K(t)$ będzie wielomianem Jonesa węzła $K$.
    Dokonajmy podstawienia
    \begin{equation}
        t := e^x = 1 + x + \frac{x^2}{2} + \frac{x^3}{6} + \ldots,
    \end{equation}
    a~następnie rozwińmy wynik w~szereg Taylora:
    \begin{equation}
        \jones_K(e^x) = \sum_{k = 0}^\infty b_k x^k.
    \end{equation}
    Współczynnik $b_{k}$ indukuje niezmiennik Wasiljewa rzędu co najwyżej $k$.
\end{example}

Ten i podobne wyniki dla wielomianów HOMFLY, Kauffmana uzyskała Birman z~Linem w~\cite{birman93}, gdzie znacznie uprościli oryginalne techniki Wasiljewa.
\index[persons]{Birman, Joan}%
\index[persons]{Lin, Xiao-Song}%
Patrz też \cite[s. 56]{chmutov12}.
\index[persons]{Czmutow, Siergiej}%

\begin{example}[s. 311 w \cite{murasugi96}]
    Niech $K$ będzie węzłem, zaś $f(t)$ rozwinięciem Taylora wokół $t = 1$ dla wielomianu Jonesa:
    \begin{equation}
        f(t) = \sum_{k = 0}^\infty c_k (t-1)^k.
    \end{equation}
    Współczynnik $c_{k}$ indukuje niezmiennik Wasiljewa rzędu co najwyżej $k$.
\end{example}

