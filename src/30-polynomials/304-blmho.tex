
\section{Wielomian BLM/Ho}
\index{wielomian!BLM/Ho|(}%
Na przełomie grudnia 1984 oraz stycznia 1985 roku, Nowiński zasugerował uwzględnienie czwartego diagramu w~relacji kłębiastej, poza $L_+$, $L_-$, $L_0$.
Nie udało się uzyskać niezmiennika splotów.
Sukces odnieśli, dla niezorientowanych splotów, mniej więcej rok później Brandt, Lickorish, Millett \cite{brandt1986} oraz Ho (w pracy doktorskiej \cite{ho1986}).
\index[persons]{Brandt, Robert}%
\index[persons]{Ho, Chi-Fai}%
\index[persons]{Lickorish, William}%
\index[persons]{Millett, Kenneth}%
% C. F. Ho zaanonsował swój wynik w Abstracts Amer. Math. Soc. 6 (1985), numer 4, 300, ale chyba nigdy później go nie opublikował?

\begin{definition}[wielomian BLM/Ho]
\index{relacja kłębiasta}%
    Niezmiennik zdefiniowany relacją kłębiastą
    \begin{equation}
        Q_{L_+}(x) + Q_{L_-}(x) = x (Q_{L_0}(x) + Q_{L_\infty}(x)),
    \end{equation}
    z warunkiem początkowym $Q(\SmallUnknot) = 1$ nazywamy wielomianem BLM/Ho.
\end{definition}

Jest multiplikatywny.
Nie odróżnia luster ani mutantów (patrz definicja \ref{def:mutant}) i~potrafi liczyć ogniwa:
\index{lustro}%
\index{mutant}%
\index{ogniwo}%
jeśli jest ich $c$, to najmniejszą potęgą $x$ występującą w~wielomianie $Q_L$ jest $x^{1-c}$.
Jego stopień nie przekracza indeksu skrzyżowaniowego.
\index{indeks skrzyżowaniowy}

W tej samej pracy (\cite{brandt1986}) podano jawne wzory na wartości wielomianu BLM/Ho w~czterech punktach:

\begin{proposition}
    Niech $L$ będzie splotem.
    Wtedy $Q_L(1) = 1$.
\end{proposition}

\begin{proposition}
\index{wyznacznik}%
    Niech $L$ będzie splotem.
    Wtedy $Q_L(2) = (\det L)^2$.
\end{proposition}

\begin{proposition}
    Niech $L$ będzie splotem o $c$ ogniwach.
    Wtedy $Q_L(-2) = (-1)^{c-1}$.
\end{proposition}

\begin{proposition}
\index{homologia}%
\index{nakrycie}%
    Niech $L$ będzie splotem, którego dwukrotne rozgałęzione nakrycie ma  homologię modulo $3$ wymiaru $d$.
    Wtedy $Q_L(-1) = 3^d$.
\end{proposition}

Jones zauważył później, że wielomian $Q$ dostarcza jeszcze jednej informacji na temat homologii:
% wiem to z https://mathscinet.ams.org/mathscinet-getitem?mr=1883911

\begin{proposition}
\index{homologia}%
\index{nakrycie}%
    Niech $L$ będzie splotem, którego dwukrotne rozgałęzione nakrycie ma  homologię modulo $5$ wymiaru $e$.
    Wtedy $Q_L((-1 \pm \sqrt 5)/2) = \sqrt{5}^e$.
\end{proposition}

Co można powiedzieć o innych homologiach tego nakrycia?
Niedawno Stojmenowi \cite{stoimenowa2002} udało się udowodnić, że licząc wartość wielomianu $Q$ w~innym ustalonym punkcie nie można dowiedzieć się więcej o~homologiach modulo $p$.

Miyazawa \cite{miyazawa2019} znalazł nieskończenie wiele węzłów i splotów o trywialnych wielomianach $Q$ (BLM/Ho) oraz $\conway$ (Conwaya), ale różnych wielomianach HOMFLY-PT.
\index[persons]{Miyazawa, Yasuyuki}%
Tylko dwa węzły pierwsze o~co najwyżej 16 skrzyżowaniach mają trywialny wielomian $Q$, są to $16n_{389841}$ oraz $16n_{491778}$.

Kanenobu w~pracy \cite{kanenobu1989} podał prosty test potrafiący czasem wykrywać, które węzły nie są dwumostowe.
\index[persons]{Kanenobu, Taizo}%
\index{węzeł!dwumostowy}
Jak pisze Stojmenow w~\cite{stoimenow2000}: \emph{,,The converse of this criterion turns out not to be true; (...) these examples have been suggested by empirical calculations (...), which nevertheless reveal \ref{eqn:kanenobu_rationality_test} to be a surprisingly powerful test''}.
\index[persons]{Stojmenow, Aleksander}%
Jakość testu można ocenić patrząc na węzły o małej liczbie skrzyżowań.
Po pierwsze, wykrywa wszystkie niewymierne węzły pierwsze co najmniej do 10 skrzyżowań oraz rozstrzyga o wymierności dowolnego pierwszego, alternującego węzła co najmniej do 16 skrzyżowań.
Zawodzi natomiast dla następujących węzłów pierwszych do 15 skrzyżowań: $12_{1879}$, $12_{2037}$, $13_{7750}$, $13_{7960}$, $14_{33787}$, $14_{43535}$, $14_{44370}$, $14_{46672}$, $14_{46862}$, $15_{157719}$, $15_{168643}$, $15_{233158}$, $15_{247180}$.
% tego nie da się łatwo zweryfikować, ponieważ w bazie danych KnotInfo brak BLM/Ho

\begin{proposition}
    Jeśli $L$ jest splotem dwumostowym, to
    \begin{equation}
\label{eqn:kanenobu_rationality_test}%
        (-u-u^{-1}) [Q_L(-u-u^{-1})-1] = 2 (\jones_L(u) \jones_L (u^{-1}) - 1).
    \end{equation}
\end{proposition}

Jednocześnie:

\begin{conjecture}
\index{węzeł!pierwszy}%
    Równość \ref{eqn:kanenobu_rationality_test} nie zachodzi dla żadnego węzła złożonego.
\end{conjecture}

Stojmenow \cite[s. 474]{stoimenow2000} zauważa, że jeśli hipoteza jest fałszywa, to kontrprzykład do niej musi być niealternujący, co wynika z~prac Menasco \cite{menasco1984}, Kidwella \cite{kidwell1987} i~Thistlethwaite'a \cite{thistlethwaite1987}.
\index[persons]{Thistlethwaite, Morwen}%
\index[persons]{Menasco, William}%
\index[persons]{Kidwell, Mark}%
\index[persons]{Stojmenow, Aleksander}%
Praca \cite[s. 477]{stoimenow2000} kończy się postawieniem jeszcze jednego problemu:

\begin{conjecture}
    Niech $K_1, K_2$ będą dwoma węzłami o tym samym wielomianie $F$ Kauffmana.
    Czy możliwe jest, by jeden z nich był węzłem wymiernym, zaś drugi niewymiernym?
\end{conjecture}

Wiadomo, że zarówno wśród węzłów wymiernych, jak i niewymiernych, istnieją węzły nierozróżnialne od siebie przy pomocy wielomianu $F$ Kauffmana.
Opisujemy go w kolejnej sekcji.

\index{wielomian!BLM/Ho|)}%

% Koniec sekcji Wielomian BLM/Ho

