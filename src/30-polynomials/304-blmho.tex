
\section{Wielomian BLM/Ho}
\index{wielomian!BLM/Ho|(}%
Na przełomie grudnia 1984 oraz stycznia 1985 roku, Nowiński zasugerował uwzględnienie czwartego diagramu w~relacji kłębiastej, poza $L_+$, $L_-$, $L_0$.
Nie udało się uzyskać niezmiennika splotów.
Sukces odnieśli, dla niezorientowanych splotów, mniej więcej rok później Brandt, Lickorish, Millett oraz Ho (w pracy \cite{brandt86}).
\index{człowiek!Brandt, ?}%
\index{człowiek!Ho, ?}%
\index{człowiek!Lickorish, ?}%
\index{człowiek!Millett, ?}%
% C. F. Ho zaanonsował swój wynik w Abstracts Amer. Math. Soc. 6 (1985), numer 4, 300, ale chyba nigdy później go nie opublikował?

\begin{definition}[wielomian BLM/Ho]
\index{relacja kłębiasta}%
    Niezmiennik zdefiniowany relacją kłębiastą
    \begin{equation}
        Q_{L_+}(x) + Q_{L_-}(x) = x (Q_{L_0}(x) + Q_{L_\infty}(x)),
    \end{equation}
    z warunkiem początkowym $Q(\SmallUnknot) = 1$ nazywamy wielomianem BLM/Ho.
\end{definition}

Jest multiplikatywny.
Nie odróżnia luster ani mutantów (patrz definicja \ref{def:mutant}) i~potrafi liczyć ogniwa:
\index{lustro}%
\index{mutant}%
\index{ogniwo}%
jeśli jest ich $c$, to najmniejszą potęgą $x$ występującą w~wielomianie $Q_L$ jest $x^{1-c}$.
Jego stopień nie przekracza indeksu skrzyżowaniowego.
\index{indeks skrzyżowaniowy}

W tej samej pracy (\cite{brandt86}) podano jawne wzory na wartości wielomianu BLM/Ho w~czterech punktach:

\begin{proposition}
    Niech $L$ będzie splotem.
    Wtedy $Q_L(1) = 1$.
\end{proposition}

\begin{proposition}
\index{wyznacznik}%
    Niech $L$ będzie splotem.
    Wtedy $Q_L(2) = (\det L)^2$.
\end{proposition}

\begin{proposition}
    Niech $L$ będzie splotem o $c$ ogniwach.
    Wtedy $Q_L(-2) = (-1)^{c-1}$.
\end{proposition}

\begin{proposition}
    Niech $L$ będzie splotem z $d$-wymiarową homologią modulo $3$ dla jego dwukrotnego nakrycia.
    Wtedy $Q_L(-1) = 3^d$.
\end{proposition}

Kanenobu w~pracy \cite{kanenobu89} podał prosty test potrafiący czasem wykrywać, które węzły nie są dwumostowe.
\index{człowiek!Kanenobu, Taizo}%
\index{węzeł!dwumostowy}
Jak pisze Stojmenow w~\cite{stoimenow00}: \emph{,,The converse of this criterion turns out not to be true; (...) these examples have been suggested by empirical calculations (...), which nevertheless reveal \ref{eqn:kanenobu_rationality_test} to be a surprisingly powerful test''}.
\index{człowiek!Stoimenow, Alexander}%
Jakość testu można ocenić patrząc na węzły o małej liczbie skrzyżowań.
Po pierwsze, wykrywa wszystkie niewymierne węzły pierwsze co najmniej do 10 skrzyżowań oraz rozstrzyga o wymierności dowolnego pierwszego, alternującego węzła co najmniej do 16 skrzyżowań.
Zawodzi natomiast dla następujących węzłów pierwszych do 15 skrzyżowań: $12_{1879}$, $12_{2037}$, $13_{7750}$, $13_{7960}$, $14_{33787}$, $14_{43535}$, $14_{44370}$, $14_{46672}$, $14_{46862}$, $15_{157719}$, $15_{168643}$, $15_{233158}$, $15_{247180}$.

\begin{proposition}
    Jeśli $L$ jest węzłem dwumostowym, to
    \begin{equation}
\label{eqn:kanenobu_rationality_test}%
        z Q_L(z) = 2 \jones_L(t) \jones_L (1-2z^{-1}+t^{-1}),
    \end{equation}
    gdzie $z = -t - t^{-1}$.
\end{proposition}

Jednocześnie:

\begin{conjecture}
    Równość \ref{eqn:kanenobu_rationality_test} nie zachodzi dla żadnego węzła złożonego.
\index{węzeł!pierwszy}%
\end{conjecture}

Według pracy \cite{stoimenow00}, hipoteza jest prawdziwa w klasie węzłów alternujących.
Stojmenow sugeruje jeszcze następujący problem.
\index{człowiek!Stoimenow, Alexander}%

\begin{conjecture}
    Niech $K_1, K_2$ będą dwoma węzłami o tym samym wielomianie $F$ Kauffmana.
    Czy możliwe jest, by jeden z nich był węzłem wymiernym, zaś drugi niewymiernym?
\end{conjecture}

Wiadomo, że zarówno wśród węzłów wymiernych, jak i niewymiernych, istnieją węzły nierozróżnialne od siebie przy pomocy wielomianu $F$ Kauffmana.
Opisujemy go w kolejnej sekcji.

\index{wielomian!BLM/Ho|)}%

% Koniec sekcji Wielomian BLM/Ho

