
\subsection{Całka Koncewicza}

Żaden niezmiennik Wasiljewa nie jest zupełny: Ohyama \cite{ohyama95} dla każdej liczby naturalnej $n$ i~każdego węzła $K$ wskazał jawnie nieskończoną rodzinę złożonych węzłów takich, że niezmienniki rzędu co najwyżej $n$ nie odróżniają ich od $K$.
\index[persons]{Ohyama, Yoshiyuki}%
Stanford \cite{stanford96} szybko rozszerzył ten wynik: udowodnił, że dla każdego splotu $L$ istnieje nieskończona rodzina pierwszych, nierozszczepialnych, alternujących splotów nieodróżnialnych takimi niezmiennikami.
\index[persons]{Stanford, Ted}%

Z drugiej strony, Czmutow i inni piszą w \cite{duzhin12}, że sześć niezmienników rzędu co najwyżej 4 wystarcza do odróżnienia dowolnych dwóch węzłów pierwszych do 8 skrzyżowań.
\index[persons]{Czmutow, Siergiej}%
\index[persons]{Dużin, Siergiej}%
\index[persons]{Mostovoy, Jacob}%
Kneissler twierdzi (\cite[wniosek 2.5]{kneissler97}), że niezmienniki rzędu co najwyżej 12 nie odróżniają węzłów od ich odwrotności.
\index[persons]{Kneissler, ?}%
\index{węzeł!odwrotny}%

\index{całka Koncewicza|(}%
W 1993 roku Maxim Koncewicz pokazał, że dla każdego węzła można policzyć pewną całkę (teraz nazywaną całką Koncewicza), która jest niezmiennikiem uniwersalnym: z jej wartości można odtworzyć wszystkie inne niezmienniki skończonego typu.
\index[persons]{Koncewicz, Maxim}%
Bar-Natan w 1995 roku znalazł wartość tej całki dla niewęzła:
\index[persons]{Bar-Naran, Dror}%
\begin{equation}
    I (\SmallUnknot) = \exp \left(\sum_{n=0}^\infty b_{2n} w_{2n}\right),
\end{equation}
gdzie $b_{2n}$ to zmodyfikowane liczby Bernoulliego o funkcji tworzącej
\begin{equation}
    \sum_{n=0}^\infty b_{2n} x^{2n} = \frac 12 \log \frac {e^{x/2} - e^{-x/2}}{x/2},
\end{equation}
zaś $w_{2n}$ to ,,koła'': diagramy okręgu z doczepionymi $2n$ promieniami.
Liniową kombinację należy rozumieć jako element algebry chińskich znaków.
\index{algebra!chińskich znaków}%
Następnie Marché w~2003 roku znalazł wartości całki dla węzłów torusowych (\cite{marche04}).
\index[persons]{Marché, Julien}%
Wygląda na to, że nikt nie odważył się dokonać tego dla innych węzłów (stan na 2019).

\begin{conjecture}
    \label{con:vassilliev}
    Całka Koncewicza jest niezmiennikiem zupełnym.
\end{conjecture}

Całka Koncewicza jest mocniejsza od każdego wielomianowego niezmiennika, jaki dotąd poznaliśmy, a~nie wiemy nawet, czy wielomian Jonesa wykrywa niewęzły (hipoteza \ref{con:jones}).
\index{wielomian Jonesa}%
Czmutow, Dużin wspominają w~dość czytelnie napisanym artykule \cite{chmutov05}, że hipoteza \ref{con:vassilliev} jest prawdziwa dla warkoczy (Kohno \cite{kohno87}) i~splotów sznurkowych (Bar-Natan \cite{barnatandror95}).
\index[persons]{Czmutow, Siergiej}%
\index[persons]{Dużin, Siergiej}%
\index[persons]{Kohno, Toshitake}%
\index[persons]{Bar-Natan, Dror}%
% DICTIONARY;string;sznurkowy;splot
%=% kohno87 nie zawiera nazwiska Koncewicz (!)
\index{splot!sznurkowy}%
\index{warkocz}%

Zbiór problemów Ohtsukiego \cite[s. 398-444]{ohtsuki02} poświęca wiele stron na niezmienniki skończonego typu oraz całkę Koncewicza.

\index{całka Koncewicza|)}%

