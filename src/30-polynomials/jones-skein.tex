\subsection{Relacja kłębiasta} % (fold)
\label{sub:skein}
Dotychczas wyznaczyliśmy wielomian Jonesa jedynie dla trywialnych splotów.
Spowodowane jest to tym, że chociaż nawias Kauffmana jest przydatny przy dowodzeniu różnych własności,
to zupełnie nie nadaje się do obliczeń.
Dużo lepszym narzędziem jest następujące twierdzenie.

\begin{theorem}[relacja kłębiasta] \label{tracheotomia} \index{Relacja kłębiasta}
    Wielomian Jonesa spełnia równość $V(\LittleUnknot) = 1$ oraz relację
    \[
        t^{-1} V(L_+) - tV(L_-) + (t^{-1/2} - t^{1/2}) V(L_0) = 0,
    \]

    gdzie $L_+$, $L_-$, $L_0$ to zorientowane sploty, które różnią się jedynie na małym obszarze:
    \[
        \skeinplus \quad\quad\quad\quad
        \skeinminus \quad\quad\quad\quad
        \skeinzero
    \]
\end{theorem}

\begin{proof}
Wyraźmy wielomian Jonesa przez nawias Kauffmana i spin.
Chcemy pokazać, że
\[
    A^{4}(-A)^{-3w(L_+)}\bracket{\LittleRightCrossing}
    -A^{-4}(-A)^{-3w(L_-)}\bracket{\LittleLeftCrossing}
    +(A^2-A^{-2})(-A)^{-3w(L_0)}\bracket{\LittleRightSmoothing} = 0.
\]

Ale $w(L_\pm)=w(L_0)\pm 1$, zatem to jest równoważne z
\[
    -A\bracket{\LittleRightCrossing} +A^{-1}\bracket{\LittleLeftCrossing} +(A^2-A^{-2})\bracket{\LittleRightSmoothing} =0.
\]
Z definicji nawiasu Kauffmana wnioskujemy, że
$\bracket{\LittleRightCrossing} = A\bracket{\LittleRightSmoothing}+A^{-1}\bracket{\LittleLeftSmoothing}$ i
$\bracket{\LittleLeftCrossing} = A\bracket{\LittleLeftSmoothing}+A^{-1}\bracket{\LittleRightSmoothing}$.
Pierwsze równanie przemnóżmy przez $A$, drugie przez $A^{-1}$, a następnie dodajmy je do siebie.
Wtedy otrzymamy $A\bracket{\LittleRightCrossing}-A^{-1}\bracket{\LittleLeftCrossing} =
A^2\bracket{\LittleRightSmoothing} - A^{-2}\bracket{\LittleRightSmoothing}$.
\end{proof}
% Koniec sekcji Relacja kłębiasta
