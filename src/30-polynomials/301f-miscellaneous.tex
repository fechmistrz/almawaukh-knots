
\subsection{Różniaste różności}
Istnieje odmiana wielomianu Alexandera, która liczy sobie tyle zmiennych, ile ogniw posiada splot (nie opisywaliśmy jej i~nie zamierzamy).
Klasyfikację \ref{prp:alexander_hosokawa} można częściowo uogólnić: Torres \cite{torres53} znalazł dwie geometryczne własności, nazwane później warunkami Torresa.
\index[persons]{Torres, Guillermo}%
\index{warunek!Torresa}%
Są warunkami koniecznymi, ale nie wystarczającymi, jak odkrył ponad ćwierć wieku później Hillman \cite{hillman81}: wielomian
\index[persons]{Hillman, Jonathan}%
\begin{equation}
    D(x,y) = \frac{(1 - x^6y^6)(x - 1 + 1/x) - 2(1 - x^5y^5)(1 - x)(1 - y)}{1-xy}
\end{equation}
spełnia warunki Torresa, ale nie jest wielomianem Alexandera żadnego splotu.

Fox \cite{fox62} podejrzewał, że
\index[persons]{Fox, Ralph}%
\begin{conjecture}
\index{hipoteza!trapezoidalna}%
    Ciąg współczynników wielomianu Alexandera węzła alternującego jest unimodalny.
\end{conjecture}
Dowód podano dla węzłów algebraicznych (Murasugi \cite{murasugi85}) oraz genusu dwa (Ozsváth i~Szabó \cite{ozsvath03}).
\index[persons]{Murasugi, Kunio}%
\index[persons]{Ozsváth, Peter}%
\index[persons]{Szabó, Zoltán}%
Hipoteza w~ogólnym przypadku pozostaje otwarta.

Wiemy natomiast, że kolejne współczynniki wielomianu Conwaya węzła alternującego są przeciwnych znaków i niezerowe (\cite[s. 242]{murasugi96} odsyła do pracy tego samego autora, \cite{murasugi59}).
Wynika stąd, że prawie wszystkie węzły równoległe\footnote{Parallel knots, nie zamierzamy używać tego terminu poza tym zdaniem}, w tym: $(p, q)$-torusowe dla $p > q > 2$ oraz kablowe\footnote{Schlauchknoten.} nie są alternujące.
% DICTIONARY;parallel;równoległy;węzeł

% koniec podsekcji Różniaste różności

