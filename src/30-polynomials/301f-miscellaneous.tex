
\subsection{Różniaste różności}
Istnieje odmiana wielomianu Alexandera, która liczy sobie tyle zmiennych, ile ogniw posiada splot (nie opisywaliśmy jej i~nie zamierzamy).
Klasyfikację \ref{prp:alexander_hosokawa} można częściowo uogólnić: Torres \cite{torres53} znalazł dwie geometryczne własności, nazwane później warunkami Torresa.
\index[persons]{Torres, Guillermo}%
\index{warunek!Torresa}%
Są warunkami koniecznymi, ale nie wystarczającymi, jak odkrył ponad ćwierć wieku później Hillman \cite{hillman81}: wielomian
\index[persons]{Hillman, Jonathan}%
\begin{equation}
    D(x,y) = \frac{(1 - x^6y^6)(x - 1 + 1/x) - 2(1 - x^5y^5)(1 - x)(1 - y)}{1-xy}
\end{equation}
spełnia warunki Torresa, ale nie jest wielomianem Alexandera.

Fox \cite{fox62} podejrzewał, że ciąg współczynników wielomianu Alexandera węzła alternującego jest unimodalny.
\index{hipoteza!trapezoidalna}%
\index[persons]{Fox, Ralph}%
Dowód podano dla węzłów algebraicznych (Murasugi \cite{murasugi85}) oraz genusu dwa (Ozsvath i~Szabo w~\cite{ozsvath03}).
\index[persons]{Murasugi, Kunio}%
\index[persons]{Ozsváth, Peter}%
\index[persons]{Szabó, Zoltán}%

Hipoteza w~ogólnym przypadku pozostaje otwarta.
% Wiemy natomiast, że jeśli węzeł jest alternujący, to kolejne współczynniki wielomianu Conwaya $\alexander$ są przeciwnym znaków? (Murasugi, 242)

% Remark (M. Hutchings) There does exist a~categorification of the Alexander polynomial, or more precisely of ∆K(t)/(1 − t)2, where ∆K(t) denotes the (symmetrized) Alexander polynomial of the knot K. It is a~kind of Seiberg-Witten Floer homology of the three-manifold obtained by zero surgery on K.
% One can regard it as Z×Z/2Z graded, although in fact the column whose Euler characteristic gives the coefficient of tk is relatively Z/2kZ graded.

% koniec podsekcji Różniaste różności

