
\subsection{Równania kolorujące (definicja pierwsza)}

\begin{definition}
\index{równanie kolorujące}%
    Niech $D$ będzie diagramem zorientowanego splotu $L$, którego wszystkim łukom przypisano kolory $0, 1, \ldots, n - 1$.
    Wtedy do każdego skrzyżowania:
\begin{comment}
    \[
        \LargePlusCrossingLabel
    \]
\end{comment}
    można przypisać wielomianowe równanie kolorujące $g + tk - tg - h = 0$.
    Tylko orientacja górnego łuku ma znaczenie.
\end{definition}

Nie jest trudno zgadnać, skąd bierze się nazwa ,,równanie kolorujące'': podstawiając tutaj $t = -1$ otrzymujemy  szczególną definicję \ref{def:colouring_equation}.

\begin{definition}[wielomian Alexandera]
\label{def:alexander_polynomial}%
    Niech $L$ będzie splotem, zaś $D$ takim jego diagramem, na którym nie widać żadnych krzywych zamkniętych (literalnie niewęzła $\SmallUnknot$).
    Niech $x_0, \ldots, x_m$ będą pewnymi symbolami przypisanymi do włókien (a symbole $0, \ldots, m$ do skrzyżowań) diagramu $D$.
    Oznaczmy przez $p_{i,j}$ współczynnik przy włóknie $x_j$ w~wielomianowym równaniu kolorującym nad wierzchołkiem $i$.
    Z macierzy $P=(p_{i,j})$ wykreślmy po jednej kolumnie i~wierszu.
    Wyznacznik zmniejszonej macierzy to wielomian Alexandera, oznaczamy go $\alexander_L(t)$.
\end{definition}

Nasz nowy niezmiennik nie jest zwykłym wielomianem, tylko wielomianem Laurenta jednej zmiennej, czyli elementem pierścienia $\Z[t, t^{-1}]$.

\begin{proposition}
\label{alexander_invariance}%
    Wielomian Alexandera z~dokładnością do mnożenia przez jedności:
    \begin{equation}
        f(t) \equiv g(t) \iff \exists m \in \Z: f(t) = \pm t^m g(t)
    \end{equation}
    jest niezmiennikiem zorientowanych splotów.
\end{proposition}

W dowodzie niezmienniczości wyznacznika węzła skorzystalibyśmy z~relacji między nim a~grupą kolorującą.
Poprzednie wydania książki zawierały sugestię, że elementarny (czyli taki, który nie korzysta z~teorii modułów) dowód niezmienniczości wielomianu Alexandera nie istnieje.
Sugestia ta była błędna.

Podany dowód dotyczy tylko splotów o spójnym diagramie.
Jeżeli mamy diagram, który nie spełnia tego założenia, to fakt \ref{prp:alexander_unlinks} orzeka, że jego wielomian Alexandera znika.

\begin{proof}
    Ustalmy diagram o~$k$ skrzyżowaniach, który rozcina płaszczyznę na $k+2$ obszarów i~utwórzmy macierz o~wymiarach $k \times k$, której kolumny odpowiadają obszarom, wiersze zaś skrzyżowaniom -- pomijając przy tym dwa sąsiadujące ze sobą obszary -- o~wyrazach ze współczynników równań kolorujących.
    Jej wyznacznik jest wielomianem Alexandera.

    Sąsiadującym ze sobą obszarom przypiszmy kolejne liczby całkowite tak, by obszar leżący po prawej stronie włókna miał niższy indeks.
    Pokażemy najpierw, że skasowanie kolumny indeksu $n$ oraz $n+1$ sprawia, że wyznacznik zmienia się co najwyżej o~czynnik $\pm t^m$ dla pewnego $m$.
    Niech $S_n$ oznacza sumę kolumn indeksu $n$.
    Każdy wiersz macierzy zawiera cztery niezerowe wyrazy: $\pm 1, \pm t$, zatem $\sum_n S_n = 0$.
    Równość ta zachodzi nawet po przemnożeniu kolumny indeksu $n$ przez $t^{-n}$: $\sum_n t^{-n}S_n = 0$, co prowadzi do relacji $\sum_n (t^{-n}-1) S_n = 0$.
    Jeśli więc indeks kolumny $v_j$ wynosi $n$, to $(t^{-n}-1)v_j$ jest kombinacją liniową innych kolumn niezerowego indeksu (ponieważ $t^0 - 1 = 0$).

    Rozpatrzmy macierze $M_{0,j}, M_{0,k}$, gdzie indeksy $j$-tej i~$k$-tej kolumny to odpowiednio $p$ i~$q$.
    Z powyższych rozważań wynika, że $(t^{-q}-1) \alexander_{0,j} = \pm (t^{-p}-1)\alexander_{0,k}$, ale indeksy obszarów są wyznaczone z~dokładnością do stałej addytywnej.
    Biorąc $i$-tą oraz $l$-tą kolumnę, indeksów $r$ oraz $s$, dostaniemy zależności
    \begin{align}
        (t^{r-q}-1) \alexander_{l,j} & = \pm (t^{r-p} - 1)\alexander_{l,i} \\
        (t^{q-s}-1) \alexander_{k,l} & = \pm (t^{q-r} - 1)\alexander_{k,i}
    \end{align}
    co prowadzi do
    \begin{equation}
        \alexander_{l,j} = \pm \frac{t^{q-r}(t^{r-p}-1)}{t^{q-s}-1} \alexander_{k,i}
    \end{equation}
    Położenie $p = r +1$, $s =q+1$ pokazuje, że różny wybór kolumn do skreślenia zmienia wyznacznik macierzy co najwyżej o~czynnik $\pm t^m$.

    Wprowadźmy jeszcze jedną techniczną definicję.
    Dwie kwadratowe macierze będą dla nas równoważne, jeśli można przejść od jednej do drugiej przy użyciu pięciu operacji:
    \begin{enumerate}
        \item przemnożenie wiersza lub kolumny przez $-1$;
        \item zamiana dwóch wierszy lub kolumn miejscami;
        \item dodanie jednego wiersza do innego (lub kolumny do innej);
        \item przemnożenie lub podzielenie kolumny przez $t$;
        \item rozszerzenie lub zmniejszenie macierzy o~$1$ na przekątnej i~zera w~innych miejscach.
    \end{enumerate}

    Ruchy Reidemeistera prowadzą do macierzy równoważnych wyjściowym.
    Każda z~tych operacji zmienia wyznacznik macierzy o~czynnik $\pm t^{-m}$, co kończy dowód.
\end{proof}

Zwyczajowo wielomian normalizuje się: bierze reprezentanta, który jest symetryczny w~zmiennych $t$ i $t^{-1}$ oraz przyjmuje w~punkcie $1$ wartość $\alexander_L(1) = 1$.
Odwrotnie, dowolny wielomian Laurenta z~całkowitymi współczynnikami o~takich własnościach jest wielomianem Alexandera pewnego węzła:

\begin{proposition}
\label{prp:alexander_hosokawa}%
    Niech $p(t)$ będzie wielomianem Laurenta nad $\Z$ takim, że $p(t) = p(t^{-1})$ oraz $p(1) = \pm 1$.
    Wtedy istnieje węzeł $K$, dla którego $\alexander_K(t) \equiv p(t)$.
\end{proposition}

\begin{proof}[Niedowód]
\index[persons]{Hosokawa, Fujitsugu}%
    Hosokawa \cite{hosokawa1958} udowodnił to dla pomocniczego wielomianu splotów
    \begin{equation}
        \frac{\Delta(t, \ldots, t)}{(1-t)^{\max(0, \mu - 2)}},
    \end{equation}
    gdzie $\mu$ oznacza liczbę ogniw.
    Książka Rolfsena \cite[s. 171-172]{rolfsen1976} zawiera natomiast jawną konstrukcję węzła o~danym wielomianie Alexandera.
\end{proof}

\begin{example}
\label{alexander_no_detects_unknot}%
    Wielomian Alexandera nie wykrywa niewęzła:
    $\alexander(11_{471}) = 1$.
\end{example}

Wszystkie przykłady węzłów pierwszych do 12 skrzyżowań o~trywialnym wielomianie Alexandera to $11_{471} = 11n_{34}$, $11_{473} = 11n_{42}$, $12n_{313}$, $12n_{430}$.
Dzięki pracy Sakumy \cite{sakuma2020} dowiedzieliśmy się, że Seifert \cite{seifert1935}, Whitehead \cite{whitehead1937}, Kinoshita oraz~Terasaka \cite{kinoshita1957} pokazali, jak znaleźć więcej węzłów o trywialnym wielomianie Alexandera.
\index[persons]{Kinoshita, Shinichi}%
\index[persons]{Sakuma, Makoto}%
\index[persons]{Seifert, Herbert}%
\index[persons]{Terasaka, Hidetaka}%
\index[persons]{Whitehead, John}%
W~tej książce mamy już jedno źródło całej serii przykładów: fakt \ref{prp:pretzel_alexander}.
% między innymi $(-3, 5, 7)$ precel jest takim przykładem.
% \index{precel!(-3,5,7)}%
% (Pisaliśmy też o tym za faktem \ref{trivial_alexander_polynomial}).
% koniec podsekcji Równania kolorujące

