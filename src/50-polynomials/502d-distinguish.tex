
\subsection{Odróżnianie węzłów i splotów wielomianem Jonesa}
Wielomian Jonesa często (chociaż nie zawsze) odróżnia od siebie sploty lepiej niż wielomian Alexandera.
Na przykład: wielomian Alexandera wszystkich splotów rozszczepialnych jest taki sam (stwierdzenie \ref{prp:alexander_unlinks}), więc nie odróżnia niesplotów wcale.
Dla porównania, wielomian Jonesa odróżnia je wszystkie:

\begin{proposition}
\label{prp:jones_trivial_link}%
    Wielomianem Jonesa splotu trywialnego o $n$ ogniwach jest
    \begin{equation}
        \jones(K_n) = \left(-\sqrt{t} - \frac{1}{\sqrt {t}}\right)^{n-1}.
    \end{equation}
\end{proposition}

Co więcej, wielomian Jonesa odróżnia od siebie dowolne dwa węzły pierwsze o~co najwyżej 9 skrzyżowaniach.
Dalej występują już kolizje, oto pełna ich lista do 10 skrzyżowań:

\renewcommand*{\arraystretch}{1.4}
\footnotesize
\begin{longtable}{lcccccccccccccc}
    $K_1$ & \rotatebox{90}{$5_{1}$} & \rotatebox{90}{$8_{8}$} & \rotatebox{90}{$8_{16}$} & \rotatebox{90}{$10_{22}$} & \rotatebox{90}{$10_{25}$} & \rotatebox{90}{$10_{40}$}  & \rotatebox{90}{$10_{41}$}  & \rotatebox{90}{$10_{43}$} & \rotatebox{90}{$10_{59}$} & \rotatebox{90}{$10_{60}$} & \rotatebox{90}{$10_{71}$}  & \rotatebox{90}{$10_{73}$}  & \rotatebox{90}{$10_{81}$} & \rotatebox{90}{$10_{137}$} \\
    $K_2$ & \rotatebox{90}{$10_{132}$} & \rotatebox{90}{$10_{129}$} & \rotatebox{90}{$10_{156}$} & \rotatebox{90}{$10_{35}$} & 
\rotatebox{90}{$10_{56}$} & \rotatebox{90}{$10_{103}$} & \rotatebox{90}{$10_{94}$} & \rotatebox{90}{$10_{91}$} & \rotatebox{90}{$10_{106}$} & \rotatebox{90}{$10_{86}$} & \rotatebox{90}{$10_{104}$\,\,} & \rotatebox{90}{$10_{83}$} & \rotatebox{90}{$10_{109}$} & \rotatebox{90}{$10_{155}$}  \\
    \hline
\end{longtable}
% ZWERYFIKOWANO: funkcja jones_collisions
\normalsize

Jones wiedział, że wielomianowe niezmienniki nie radzą sobie z~odróżnianiem od siebie mutantów, dlatego zapytał, czy jego wielomian wykrywa niewęzły.
Pozostaje to otwartym problemem do dziś (patrz na przykład do zbioru problemów Ohtsukiego \cite[s. 381]{ohtsuki2002}).

\begin{conjecture}
\index{hipoteza!o wielomianie Jonesa i niewęźle}%
\label{con:jones}%
    Niech $K$ będzie węzłem.
    Jeśli $\jones_K(t) \equiv 1$, to $K$ jest niewęzłem.
\end{conjecture}

Co przemawia za prawdziwością hipotezy?
Wiemy, że jest prawdziwa w~wielu szczególnych przypadkach: dla węzłów alternujących (wynika to z I hipotezy Taita \ref{con:tait_1}), adekwatnych (Lickorish, Thistlethwaite \cite{lickorish1988}), póładekwatnych (dużo później Stojmenow \cite{stoimenow2011}), dodatnich (także Stojmenow, chyba w \cite{stoimenow2003}) oraz prawie alternujących: węzeł jest prawie alternujący, jeżeli nie jest alternujący, ale odwrócenie jednego skrzyżowania na pewnym diagramie sprawia, że takim się staje (Lowrance, Spyropoulos \cite{lowrance2017}).
\index{splot!alternujący}%
\index{splot!adekwatny}%
\index{splot!póładekwatny}%
\index{splot!dodatni}%
\index{splot!prawie alternujący}%
\index[persons]{Thistlethwaite, Morwen}%
\index[persons]{Lickorish, William}%
\index[persons]{Stojmenow, Alexander}%
\index[persons]{Lowrance, Adam}%
\index[persons]{Spyropoulos, Dean}%
Homologia Chowanowa stanowi uogólnienie wielomianu Jonesa i wykrywa niewęzeł (patrz fakt \ref{khovanov_detects_unknot}).
\index{homologia!Chowanowa}%

Hipotezę zweryfikowano komputerowo dla węzłów o~małej liczbie skrzyżowań.
W latach dziewięćdziesiątych Hoste, Thistlethwaite, Weeks \cite{thistlethwaite1998} zrobili to podczas tablicowania węzłów spełniających $\crossing K \le 16$.
\index[persons]{Hoste, Jim}%
\index[persons]{Thistlethwaite, Morwen}%
\index[persons]{Weeks, Jeff}%
Wynik poprawiano:
Dasbach, Hougardy \cite{hougardy1997} w~1997 do $\crossing K \le 17$; 
\index[persons]{Dasbach, Oliver}%
\index[persons]{Hougardy, Stefan}%
Yamada \cite{yamada1900} w~2000 do $\crossing K \le 18$;
\index[persons]{Yamada, Shuji}%
wreszcie Tuzun, Sikora \cite{tuzun2018} w~2016 do $\crossing K \le 22$,
\index[persons]{Sikora, Adam}%
\index[persons]{Tuzun, Robert}%
potem (w \cite{tuzun2021}) w~2020 do $\crossing K \le 24$.
Ośmiordzeniowe procesory Intel Xeon L5520 z Uniwersytetu w~Buffalo potrzebowały na to łącznie 41,8 lat pracy.

Co przemawia za nieprawdziwością hipotezy?
Najpierw Thistlethwaite w~\cite{thistlethwaite2001} wskazał dwa sploty z~dwoma oraz jeden z~trzema ogniwami, których wielomian Jonesa nie odróżnia od niesplotów o~takiej samej liczbie ogniw.
\index[persons]{Thistlethwaite, Morwen}%
Wynik bardzo szybko bardzo poprawiono: Eliahou, Kauffman i~Thistlethwaite w~pracy \cite{eliahou2003} znaleźli nieskończoną rodzinę splotów o~tej samej własości.
\index[persons]{Eliahou, Shalom}%
\index[persons]{Kauffman, Louis}%
Kauffman znalazł dużo wcześniej nietrywialny węzeł wirtualny, którego wielomian Jonesa jest trywialny.

\begin{proposition}
\index{splot!Hopfa}%
\index{węzeł!satelitarny}%
    Niech $k \ge 2$ będzie liczbą naturalną.
    Istnieje nieskończenie wiele pierwszych satelitów splotu Hopfa z $k$ ogniwami, których wielomian Jonesa nie odróżnia od niesplotu z $k$ ogniwami.
\end{proposition}

