
\subsection{Niezmienniki, które nie są skończonego typu}
Pójdźmy teraz w~ślad za Murasugim i~zdefiniujmy nieskończoną rodzinę węzłów osobliwych $K[p, q]$, gdzie $p$ jest liczbą wierzchołków singularnych, zaś $|q|$ liczbą klasycznych skrzyżowań.
Jeśli $q < 0$, wszystkie klasyczne skrzyżowania odwracamy:
\begin{figure}[H]
    \centering
\begin{comment}
    \begin{tikzpicture}[baseline=-0.65ex, scale=0.1]
    \begin{knot}[clip width=5, end tolerance=1pt, flip crossing/.list={2}]
        % left part
        \draw[thick] (5, 0) [in=-60, out=-120] to (-5, 0) [in=60, out=120] to (-15, 0) [in=-60, out=-120] to (-25, 0) [in=60, out=120] to (-35, 0) [in=180, out=-120] to (-35, -10);
        \draw[thick] (5, 0) [in=60, out=120] to (-5, 0) [in=-60, out=-120] to (-15, 0) [in=60, out=120] to (-25, 0) [in=-60, out=-120] to (-35, 0) [in=-180, out=120] to (-35, 10);
        % right part
        \strand[thick] (5, 0) [in=120, out=60] to (15, 0) [in=-120, out=-60] to (25, 0) [in=120, out=60] to (35, 0) [in=0, out=-60] to (35, -10);
        \strand[thick] (5, 0) [in=-120, out=-60] to (15, 0) [in=120, out=60] to (25, 0) [in=-120, out=-60] to (35, 0) [in=0, out=60] to (35, 10);
        % external lines
        \draw[thick,Latex-] (-35, 10) to (35, 10);
        \draw[thick,Latex-] (-35, -10) to (35, -10);
        \draw[black,fill=black] (5,0) circle (0.5);
        \draw[black,fill=black] (-5,0) circle (0.5);
        \draw[black,fill=black] (-15,0) circle (0.5);
        \draw[black,fill=black] (-25,0) circle (0.5);
        \draw[black,fill=black] (-35,0) circle (0.5);
    \end{knot}
    \end{tikzpicture}
\end{comment}
    \caption{Węzeł osobliwy $K[p, q]$ dla $p = 5, q = +3$.}
\end{figure}

\begin{proposition}
\index{genus}%
\index{indeks skrzyżowaniowy}%
\index{indeks warkoczowy}%
\index{liczba gordyjska}%
\index{liczba mostowa}%
\index{sygnatura}%
    Następujące funkcje: indeks skrzyżowaniowy $\crossing$, liczba gordyjska $\unknotting$, liczba mostowa $\bridge$, indeks warkoczowy $\braid$, genus $g$, sygnatura $\sigma$ nie są niezmiennikami Wasiljewa.
\end{proposition}

Wynik był znany już w latach 90., na przykład Birman, Lin \cite{birman1993} pokazali, że liczba gordyjska nie jest niezmiennikiem Wasiljewa.
\index[persons]{Birman, Joan}%
\index[persons]{Lin, Xiao-Song}%

\begin{proof}
    Dowód dla sygnatury jest w~podręczniku Murasugiego \cite[s. 312]{murasugi1996}.
    Natomiast żaden z~pozostałych niezmienników nie znika na osobliwym węźle $K[n+1, n]$ (ćwiczenie).
    % dowód dla reszty to ćwiczenie na następnej stronie, 313
\end{proof}

